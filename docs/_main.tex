% Options for packages loaded elsewhere
\PassOptionsToPackage{unicode}{hyperref}
\PassOptionsToPackage{hyphens}{url}
%
\documentclass[
]{book}
\usepackage{amsmath,amssymb}
\usepackage{lmodern}
\usepackage{iftex}
\ifPDFTeX
  \usepackage[T1]{fontenc}
  \usepackage[utf8]{inputenc}
  \usepackage{textcomp} % provide euro and other symbols
\else % if luatex or xetex
  \usepackage{unicode-math}
  \defaultfontfeatures{Scale=MatchLowercase}
  \defaultfontfeatures[\rmfamily]{Ligatures=TeX,Scale=1}
\fi
% Use upquote if available, for straight quotes in verbatim environments
\IfFileExists{upquote.sty}{\usepackage{upquote}}{}
\IfFileExists{microtype.sty}{% use microtype if available
  \usepackage[]{microtype}
  \UseMicrotypeSet[protrusion]{basicmath} % disable protrusion for tt fonts
}{}
\makeatletter
\@ifundefined{KOMAClassName}{% if non-KOMA class
  \IfFileExists{parskip.sty}{%
    \usepackage{parskip}
  }{% else
    \setlength{\parindent}{0pt}
    \setlength{\parskip}{6pt plus 2pt minus 1pt}}
}{% if KOMA class
  \KOMAoptions{parskip=half}}
\makeatother
\usepackage{xcolor}
\usepackage{longtable,booktabs,array}
\usepackage{calc} % for calculating minipage widths
% Correct order of tables after \paragraph or \subparagraph
\usepackage{etoolbox}
\makeatletter
\patchcmd\longtable{\par}{\if@noskipsec\mbox{}\fi\par}{}{}
\makeatother
% Allow footnotes in longtable head/foot
\IfFileExists{footnotehyper.sty}{\usepackage{footnotehyper}}{\usepackage{footnote}}
\makesavenoteenv{longtable}
\usepackage{graphicx}
\makeatletter
\def\maxwidth{\ifdim\Gin@nat@width>\linewidth\linewidth\else\Gin@nat@width\fi}
\def\maxheight{\ifdim\Gin@nat@height>\textheight\textheight\else\Gin@nat@height\fi}
\makeatother
% Scale images if necessary, so that they will not overflow the page
% margins by default, and it is still possible to overwrite the defaults
% using explicit options in \includegraphics[width, height, ...]{}
\setkeys{Gin}{width=\maxwidth,height=\maxheight,keepaspectratio}
% Set default figure placement to htbp
\makeatletter
\def\fps@figure{htbp}
\makeatother
\setlength{\emergencystretch}{3em} % prevent overfull lines
\providecommand{\tightlist}{%
  \setlength{\itemsep}{0pt}\setlength{\parskip}{0pt}}
\setcounter{secnumdepth}{5}
\usepackage{booktabs}
\ifLuaTeX
  \usepackage{selnolig}  % disable illegal ligatures
\fi
\usepackage[]{natbib}
\bibliographystyle{plainnat}
\IfFileExists{bookmark.sty}{\usepackage{bookmark}}{\usepackage{hyperref}}
\IfFileExists{xurl.sty}{\usepackage{xurl}}{} % add URL line breaks if available
\urlstyle{same} % disable monospaced font for URLs
\hypersetup{
  pdftitle={Бессонница},
  pdfauthor={Александра Булгакова},
  hidelinks,
  pdfcreator={LaTeX via pandoc}}

\title{Бессонница}
\author{Александра Булгакова}
\date{2023-02-11}

\begin{document}
\maketitle

{
\setcounter{tocdepth}{1}
\tableofcontents
}
\hypertarget{ux44dux43fux438ux433ux440ux430ux444}{%
\chapter*{Эпиграф}\label{ux44dux43fux438ux433ux440ux430ux444}}
\addcontentsline{toc}{chapter}{Эпиграф}

\emph{Эта история не о биполярном расстройстве личности, хотя оно здесь есть.}

\emph{Эта история не о любви, хотя она здесь есть.}

\emph{Эта история не о депрессии, хотя она здесь есть.}

\emph{Эта история не о счастье, хотя оно здесь есть.}

\emph{Эта история не о жизни, хотя она здесь есть.}

\emph{Эта история не о смерти, хотя она здесь есть.}

\hypertarget{ux447ux430ux441ux442ux44c-i.-ux431ux435ux441ux441ux43eux43dux43dux438ux446ux430}{%
\chapter*{Часть I. Бессонница}\label{ux447ux430ux441ux442ux44c-i.-ux431ux435ux441ux441ux43eux43dux43dux438ux446ux430}}
\addcontentsline{toc}{chapter}{Часть I. Бессонница}

\hypertarget{chapter-1}{%
\chapter{~}\label{chapter-1}}

Как правило, после внепланового завершения серьезных отношений даже сильные люди впадают в своего рода эмоциональную спячку. Так или иначе, приходится изолироваться от несостоявшейся второй половинки. Перебороть истерики и утихомириться в какой-нибудь тесной квартирке с мусорным ведром под раковиной, микроволновкой для полуфабрикатов, своими немногочисленными пожитками и бутылкой, о которой ты уже толком и не помнишь, что внутри, но к которой все равно продолжаешь периодически прикладываться с крайне отсутствующим видом.

Сколько бы лет тебе не было, -- двадцать пять или пятьдесят, неважно -- вдруг начинаешь вести жизнь завтрашнего пенсионера. Прекращаешь слушать музыку, за едой думаешь исключительно о еде и никогда не выключаешь телевизор. Последний ты даже не смотришь, но все равно оставляешь телек трещать на фоне. Чтобы чувствовать себя менее одиноко.

В моменты как этот всякая деятельность обращается бездействием. С каждым днем ты все больше и больше теряешь интерес к жизни, но все чаще начинаешь задумываться о смерти, которая становится чем-то ожидаемым. Логичным завершением одного тотального фиаско под названием твоя собственная жизнь.

Мать моего отца умерла за месяц до моего рождения. Дедушка же прожил еще полторы декады. Все эти годы он спал со включенным телевизором, который толком не смотрел. Еще дед никогда не смеялся и очень редко улыбался. Разве что мне. После его смерти я нашла коробку со старыми снимками. Практически на каждом из них лицо моего деда освещала широченная улыбка, но отец лишь пожал плечами. Сказал только, что со дня похорон своей жены дед так ни разу не засмеялся.

Он умер хмурым январским утром. Сгорел от внезапно давшего о себе знать рака. Дед пережил голодомор, войну, распад Советского Союза и трех украинских президентов, а потом почувствовал острую боль в желудке и умер спустя месяц. Это вообще легально?

--- Я люблю тебя --- сказала я дедушке, и вышла из палаты.

Знай я, что это будут последние слова, которые он от меня услышит, выбрала бы что-нибудь пооригинальней.

Тем не менее, я до последнего не верила, что рак заберет его так быстро. Мне только исполнилось пятнадцать, дедушке -- восемьдесят три. Он продолжал работать, бегал по утрам и выглядел лет на двадцать моложе своего возраста.

Затем болезнь впервые дала о себе знать. Какие-то две-три недели и недостающие года тут же отразились в чертах его лица.

Тем вечером за окном бушевал ветер. Выходя из палаты, я слышала, как по подоконнику забарабанили первые капли дождя.

--- Надеюсь, это случится не завтра, --- вдруг сказал дедушка. --- Мне бы очень не хотелось умереть в такую погоду.

--- Это не случится завтра, --- спокойно сказала я и наклонилась, чтобы поцеловать дедушкину щеку. --- Я люблю тебя.

И я ушла. Вышла под дождь и медленно зашагала в сторону дома.

Проснувшись следующим утром, я подошла к окну. За ним по-прежнему было темно, и я с удивлением осознала, что впервые в жизни встала раньше будильника. Фактически, утро еще не наступило. До школы оставалось несколько часов, а в душе засело ноющее чувство тревоги. Я взяла книгу -- кажется, это был Лавкрафт -- и устроилась на подоконнике.

Помнится, по радио то и дело передавали штормовое предупреждение, и по мере приближение рассвета мне открывались переполненные водой улицы. Вода и грязь, сопровождаемые возгласами ветра, заполняли собой все вокруг.

Дедушка не умер на следующий день. Он умер той ночью. Думаю, от этого я и проснулась.

Сейчас, спустя десять лет, я с ужасом осознаю, что не могу припомнить, как звучал голос моего деда. Я закрываю глаза и стараюсь расслабить сознание, вспоминая детство, проведенном в его доме. Вижу старый топчан, покрытые пылью ордена и желтую занавесь, что висела у входа в дедушкину спальню. За ними тянется старого образца гостиная. Она ничем не отличается от тех, что можно увидеть в домах других стариков из рабочего класса: раскладной диван, ковер на стене, кресла по обе стороны журнального столика, накрытого плетеной циновкой, и древний телевизор, который никогда не замолкает.

Мне вспоминается крохотная прихожая с обогревателем; я слышу яркий запах цитрусовых, чьи корки дедушка каждую зиму сушил для своей настойки. Вспоминается кухня, пиалка из хрусталя, неизменно полная конфет и гигантская копия наручных часов, висевшая в углу. На столе меня ждет чугунная сковорода времен Феликса Дзержинского, а в ней -- традиционная утренняя яичница с молодой картошкой, вкусней которой и быть не может.

По старой привычке, дедушка ест стоя. Он добавляет в картошку соли и с довольным видом наблюдает за тем, как я уплетаю завтрак. Чайник на плите уже начинает посвистывать, привлекая внимания котов, что всю ночь где-то пропадали, а теперь лениво дремлют на подоконнике.

Лежавший в тарелке завтрак вскоре исчез, как исчезнет и сковорода, и стол, на котором она стоит, кухня, и все остальные комнаты. На месте ветхого домика уже давно стоит другой, но память о нем никуда не делась. Я вспоминаю дедушку, заботливо перемешивающего сахар в моей чашке с чаем. Вижу, как он замечает пустую тарелку, и знаю, что будет дальше. Сейчас дед поставит передо мной чай и предложит добавки. Тогда я делаю глубокий вдох и стараюсь не думать ни о чем другом. Все жду, что голос сам всплывет в памяти.

Но он не всплывает, как не старайся. С пугающей точностью я помню слова, интонации, произношение. Короче говоря, что угодно, то только не то, что хочу вспомнить. И, все-таки, мне кажется, услышь я дедушкин голос хоть на мгновенье, -- случайно, и не подозревая, кому он принадлежит -- я бы обязательно его вспомнила.

Так вот, все эти годы дед жил со включенным телевизором. Прошло еще восемь лет, прежде чем я поняла, почему.

\hypertarget{chapter-2}{%
\chapter{~}\label{chapter-2}}

Приближался день рождения Адама, которого я не видела два с половиной года. Очередной день в баре. К тому времени я уже привыкла находиться по другую сторону стойки и читала что-то от Буковски, периодически поглядывая в сторону поддатых посетителей.

--- Освежи мне! --- заплетающимся языком произнес один из них.

После чего толкнул пивной бокал в обратную от меня сторону. Тот проехал пару метров вдоль барной стойки, ударился о стену и с характерным звоном рассыпался на тысячи блестящих стеклышек.

Мне вдруг очень захотелось стать этим бокалом.

На самом деле, мне даже нравилось работать за барной стойкой. Если не учитывать шестнадцатичасовой рабочий день, мизерную зарплату, постоянные недостачи и полное отсутствие чаевых, а также клиентов, в большинстве своем доводящих до исступления\ldots{} О чем я говорила? Ах, да все не так уж плохо.

Сложно сказать, почему, но, стоило мне надеть фартук и впервые перешагнуть эту заветную черту, отделяющую мир бармена от общества простых смертных, я моментально почувствовала себя в своей тарелке. Разбираться в винных сортах, изучать миксологию, работать над созданием собственных коктейлей и временами баловаться флейрингом -- было в этом что-то особенное. Я без труда могла рассказать о любом из напитков куда больше, чем указано на этикетке, посоветовать вино или удивить гостя чем-нибудь экзотичным. Такая работа и впрямь была мне по душе, но, быть может так скажет любой алкоголик.

Увы, несмотря на престижность заведения, где я работала, (а также на тот факт, что местная стойка считалась самой дорогой во всем городе) платили здесь плачевно мало. Настолько мало, что едва хватало на еду и коммуналку. С другой стороны, работа занимала у меня все время, так что, будь у меня лишние деньги, я все равно не успевала бы их потратить. Вероятно, в этом и был секрет выживания сотрудников «Штиля».

--- Плесни еще на посошок, --- на этот раз горе-метатель казенной посуды обошелся без спецэффектов.

Залпом расправился с последней порцией пива и вскоре исчез, оставив «Штиль» без единого посетителя.

-- Итак, на чем мы остановились\ldots{}

Десять страниц, пятнадцать, двадцать. Ничего не менялось. В зале по-прежнему было пусто.

Как и в моем сердце.

Не то, чтоб я ежеминутно думала о своем бывшем. Эти времена уже прошли. Я месяцев семь как выкарабкалась из затяжной депрессии и всячески старалась загрузить себя работой. Как я уже сказала, платили в «Штиле» смехотворно мало. Ну, кто будет так рвать задницу ради каких-то ста пятидесяти баксов в месяц?

К счастью, работа всецело меня выматывала, что оказалось главным из ее достоинств. После смены на всё про всё оставалось часов шесть, и это без учета дороги, а потом обратно за стойку. Знаете, нелегко быть в депрессии, когда на нее просто не остается времени. Мне чуть ли не заранее приходилось планировать свои нервные срывы.

Так вот, наступило двадцатое апреля две тысячи шестнадцатого года -- день толерантности марихуаны, а по совместительству и день рождения одного небезызвестного немецкого политика еврейского происхождения. День обещал быть куда более радужным, чем карьера последнего. Мне хотелось выглянуть на улицу, но за стойкой, само собой, окон не было.

Не было их и в зале-ресторане. Не было в боулинге, и на кухне тоже ни одного окошка не наблюдалось. Мне это всегда казалось странным. Будто у проектировщиков имелась какая-то тайная нелюбовь к сквозным отверстиям. С другой стороны, в зале ежедневно околачивались стриптизерши, проститутки, эскортницы и прочего рода содержанки, так что что-то здесь явно не вяжется.

В любом случае, за стойкой было темно как в причинном месте. Приходилось читать в свете неоновых вывесок с изображениями Моргана, Бушмилса, Дэниэлса и прочих хорошо мне знакомых парней. Если алое освещение надоест, можно чуток подвинуться влево. Пара шагов вдоль стойки, и алые страницы становятся зелеными в свете лозунгов Егеря, Бехеровки или Ксенты.

Говорят, человеку нужно всего три недели, чтобы привыкнуть к дискомфорту или избавиться от вредной привычки. Увы, я провела в «Штиле» не один год, но так и не смогла свыкнуться с отсутствием окон.

Единственным приятным моментом, помимо доступа к алкоголю, была входная дверь. Сплошь стеклянная, она находилась слева от моего рабочего места и открывала взгляду небольшой клочок улицы. Судя по всему, за пределами «Штиля» и впрямь стояла восхитительная погода. Сияло весеннее солнце; деревья уже позеленели и, движимые легким ветерком, отбрасывали причудливые тени на аллейку. Отсюда мне был виден каменный фонтан да пара скамеек. Брызги воды прямо-таки светились под лучами полуденного солнца. От этой радужной картины в душе сделалось как-то грустно, так что я отправилась варить кофе.

--- Ты будешь делать кофе? --- моментально среагировала одна из официанток, которых в «Штиле» звали просто фицами.

--- Кофе? --- эхом отозвались из зала.

Каким-то мистическим образом они всегда это чувствовали. Девушки по инерции стекались у барной стоило мне лишь подумать о том, чтобы варить кофе.

--- Буду.

--- И мне сделай, Лизочек, --- Вера сняла с полки пустую чашку, чмокнула меня в щеку и побежала обслуживать очередной столик.

Терпеть не могла, когда меня так называли.

За ней оживились и другие сотрудники развлекательного комплекса. Стало ясно, что одной чашкой кофе здесь не обойдется.

Вернувшись на прежнее место, я заметила, что у фонтана появился бездомный. Уже наполовину раздетый, он продолжал смущать прохожих и неспешно стягивал с себя одежду пока, наконец, не остался в чем мать родила. Затем товарищ без лишних раздумий погрузился в фонтан, лениво растянулся, опираясь на бортик, и уже беззаботно потягивал пивасик.

Вот, в принципе, и все, что вам нужно знать о городе, в котором я родилась.

***

Звонил телефон. Кассир сняла трубку и тут же передала ее мне со словами:

--- Это по твою душу.

Этажом ниже просили сделать пятнадцать детских мохито и четыре взрослых. Намечался очередной детский день рождения. С ума сойти. В мое время мы дарили друг другу наклейки и радовались мороженому из МакДональдс, а не арендовали целый этаж в ночном клубе вместе с аниматорами, лайт-джеями и диджеями. А ведь мне едва ли исполнилось двадцать два года.

--- И побыстрее! Мелюзга уже вовсю вопит! --- голос хриплый, чуть ли не потусторонний, и прямо-таки гавкает, а не говорит.

Почти два десятка мохито -- не самый удачный выбор, когда за стойкой нет крашмейкера. Единственный работающий аппарат находился в ночном клубе, так что мне постоянно приходилось бегать туда-сюда за колотым льдом.

Официантка уже стояла на раздаче, когда я положила трубку.

--- Я бы тебе помогла, --- сказала Вера, --- но начальство велело даже в туалет не выходить. Я в зале одна осталась.

--- Ничего, я схожу вниз\ldots{}

--- Пипец. Скоро обоссусь.

--- А ты пока принеси стаканы.

Вероника кивнула и уже собиралась уходить, когда я позвала ее по имени.

--- Слушай, а кто звонил?

--- Ну, Рыжая.

Я вскинула бровь.

-- Ты уверена?

-- Еще б мне не быть уверенной. Здесь только мы и есть. По одной на этаж. А чё такое?

-- Да голос, блин, какой-то жуткий. Как с того света, честное слово. Это точно Рыжая? Как-то на нее не похоже.

-- Чего не похоже? На Рыжую не похоже? Да она опять до шести утра в клубе бухала!

Вера закатила глаза и, прихватив поднос, ушла на мойку. Порой создавалось впечатление, что та завидует коллеге.

Как ни странно, внизу меня действительно ждала Тоня.

-- Привет, Рыжик, как твое похмелье?

Милая девушка, только временами врывается как гром среди ясного неба. Говорила она много, громко и очень быстро. Словом, Тоня с легкостью могла бы стать отличным примером экспрессии в бытовой речи.

-- ЛИЗА, ЛИЗА, ЛИЗА! -- закричала она, перекрывая шумы клубного подземелья. -- Сделай мне чего-нибудь этакого!

-- Этакакого?

-- Ну, чтоб получше стало, а то я щаз точно ласты откину! Вот увидишь, откину! -- она наполнила хайбол льдом и теперь жадно прижимала его к своему лицу. -- Так и шо? Ты сделаешь?

-- Сделаю. Как с заказом закончу.

-- Пожа-а-алуйста, сделай сейчас, а? А я тебе лёд пока замучу.

Ну, как тут откажешь?

До окончания дня оставалось еще немало. Я как раз колдовала над своим противопохмельным отваром, когда телефон в кармане моих джинс завибрировал. На экране меня ждало входящее сообщение от очередного бородатого незнакомца.

Его звали Вениамин.

«Как жизнь молодая?» -- спрашивал он.

Обычно я игнорирую подобные попытки знакомства, но\ldots{} То ли дело здесь было в четыре-двадцать, то ли в том, что мне осточертело стоять за стойкой, а, может, в том, что образ Адама то и дело упрямо выныривал из памяти -- понятия не имею. В общем, я ответила.

\hypertarget{chapter-3}{%
\chapter{~}\label{chapter-3}}

Все та же душная мгла повисла над городом, заботливо обволакивая каждую из его составляющих. До рассвета оставалось еще минимум шесть часов, что вовсе не мешало местной пьяни устраивать под окнами традиционные субботние концерты. Они орали и галдели так, словно эта ночь была последним шансом напиться.

Растянувшись поперек кровати, я пыталась сосредоточиться на книге. В ней говорилось о зимних лесах с их невероятными пейзажами, конных упряжках, размашистых балах и дворцах, одетых в блестящее убранство. А за окном лаяли псы, неслись машины и шумели районные алкоголики, напоминая мне, где на самом деле я нахожусь. Вероятно, дело было в моем душевном расстройстве: маниакальная фаза сменилась депрессивной. Мне вдруг стало неописуемо грустно и обидно за то, какой жизнью приходится жить. Еще и Адам постоянно лез в голову.

\emph{Не думай о нем.}

Чертов Адам с его обаятельной улыбкой да ямочками на щеках.

Захлопнув книгу, я отважилась закрыть окно

(я говорю «отважилась» потому, как термометр показывал под тридцать, а кондиционера в моей квартирке отродясь не водилось)

чтобы хоть немного приглушить раздражающие вопли. Подавляющему большинству людей всегда было плевать на беспредел, творящийся в этом районе. Несколько лет назад в парке под моим домом до смерти забили женщину. Другую изнасиловали и задушили чуть подальше, оставив валяться среди ив, кленов и прочей растительности, что окружала аллею. Нашли ее полностью голую около семи утра. Я прекрасно помню то утро потому, что именно в это время шла через парк поздравить своего отца с юбилеем, а еще потому, что я эйдетик.

Помнится, третью женщину убили как раз-таки на отрезке пути, что вел от одного из этих парков к другому. Она лежала в сквере у онкологического отделения. Снега той зимой были нешуточные, так что несчастную обнаружили лишь в апреле. Вид у нее был, прямо говоря, не очень, отчего распознать черты лица уже не представлялось возможным. Труп опознать не могли, и убитая номер три провела еще года полтора в морге той самой больницы пока дело, наконец, не закрыли.

Есть еще один небольшой парк за детским садом, где я когда-то училась падать с велосипеда. Там тоже кого-то отправили в мир иной. Еще один -- в пяти минутах ходьбы и в нем вообще регулярно находят трупы. Половина из них, конечно, местные торчки, которые не могут правильно рассчитать дозу, но есть и те, кто умер слегка менее приятной и чуть более насильственной смертью. К примеру, начальник районного отделения полиции.

Кстати говоря, по дороге от моего дома в сторону последней из описанных локаций тоже есть парк. Не знаю, что там у них насчет убийств, но однажды у меня закончились сигареты, и я вышла на их поиски. Как раз достигла середины этого крохотного, тянувшегося вдоль проезжей части сквера, когда незнакомый мужчина подбежал ко мне и со всей дури заехал в челюсть.

Я ошалело взглянула на незнакомца и расхохоталась.

Потное, осунувшееся лицо, зрачки как блюдца и челюсть пошла ходуном -- да это тип был явно под белым. Он, не моргая, таращился на меня, очевидно, прикидывая, стоит ли нанести повторный удар. Потом как будто о чем-то вспомнил. Схватился за голову и кинулся прочь.

-- Постой, паровоз! -- кулаки у меня чесались еще с вечера. -- Куда ты погнал?

-- А НУ СТОЯТЬ, ПЕДРИЛА! -- заорала моя мания.

Ей было все равно на то, что тот тип мог уложить меня одной правой.

-- Я не усну, пока не узнаю, зачем ты меня двинул!

Но мужик оказался бывалый. Он перепрыгнул через забор и довольно быстро исчез из поля зрения, так что мне пришлось возобновить поиски никотина.

Короче говоря, парки в этом районе так себе.

Так вот, обычно местный ночной беспредел меня едва ли волновал. Порой я даже сама являлась его частью. Но не сегодня. Этим вечером моя психика была особенно уязвима.

\emph{Не думай о нем.}

Биполярка вообще странная штука. То ты пляшешь, поешь, и трахаешься как под спидами, а затем на полставки подрабатываешь комиком и опять трахаешься. Не можешь спать (слишком перевозбужден) и не можешь есть потому, что чувство голода куда-то улетучивается. Тем не менее, окружающие считают тебя душой компании и вечно зовут на тусовки. Самые отважные умудряются еще и влюбиться.

Затем наступает депрессивный период, и ты опять не можешь спать, потому как слишком печален. Есть, кстати, не хочется по той же причине. Все самые страшные и болезненные воспоминания вырываются наружу, так что истерикам нет предела. Сначала это кратковременная злость, затем пара гордых слезинок, а потом истерика, сменяющаяся полноценной панической атакой. Длится она от пары минут до одного-двух часов, в зависимости от окружения и желания жить в целом. Зачастую, в борьбе с припадками, мне приходилось оставлять на конечностях порезы и ссадины различной глубины. Чтобы хоть как-то угомониться.

По весне завидев шрамы, люди почему-то думают, что я каждый день начинаю с того, что сажусь на край кровати с намерением вскрыть себе вены. Как будто я не в курсе, где эти самые вены находятся. Все дело в том, что временами острая физическая боль становится единственным способ отвлечь себя от печали. Порезы -- это не страшно. Разбитое сердце, душевная боль, одиночество и, в конечном итоге, немое отчаяние -- вот это куда страшнее, ведь именно они вызывают желание распрощаться с жизнью.

Такими днями общаться с людьми совершенно невозможно. Они мало что понимают. Вечно ждут от тебя забавных историй, обилия шуток и прочих признаков веселья. Люди привыкли видеть меня такой. Всегда и везде. Это смешно, но порой, соприкасаясь со мной во время депрессивной фазы, некоторые из них чуть ли не злятся на мою грусть и молчаливость. Капризничают и топают ножками, требуя веселья, к которому так привыкли. Разочаровываются, называют скучной и оскорбляются нежеланием вести бурную ораторскую деятельность. Проверку депрессией проходят единицы, но даже они не способны меня спасти.

В итоге внутри не остается ничего кроме боли. И скоро ты вновь призраком слоняешься по городу в кромешном одиночестве.

Биполярное аффективное расстройство -- вот официальное название того, чем я страдаю. Или наслаждаюсь. Этот вопрос никогда толком не проясняется. Впервые увидев список симптомов БАР-а, я, мягко говоря, охренела. Выглядело все так, словно кто-то просто наблюдал за мной на протяжении двадцати лишним лет, а затем собрал воедино все черты моего характера и назвал это биполярным расстройством. Отвратительное, скажу вам, чувство. Вот живешь ты себе, жалуешься на жару, пьешь вино и пишешь свои книги, а затем вдруг понимаешь, что никакая ты не творческая личность, а всего лишь очередная психбольная. Все твои достоинства, отрицательные качества, привычки и даже так называемые изюминки -- это все лишь признаки психического расстройства, которым болеет 2,4 \% земного населения. Как вам такая новость?

Кухонные часы показывали полночь.

Суббота закончилась. Пьяницы остались.

\hypertarget{chapter-4}{%
\chapter{~}\label{chapter-4}}

Тринадцатое утро мая две тысячи шестнадцатого года встретило меня умеренной погодой -- редкое явление в моих родных краях. За окном обнадеживающе светило солнце, но было довольно прохладно. Самое то для дальней поездки. Дни работы за стойкой приучили меня выползать из постели сразу же после пробуждения. Или вообще в нее не ложиться, если чувствуешь, что не сможешь проснуться вовремя.

Это работало, даже если я уснула с рассветом, так что я в кои-то веки не пропустила свой автобус и даже на него не опаздывала. Сказать больше, мне удалось сделать укладку и нарисовать шикарные стрелки. Вот они, прелести выходного дня.

Выпив кофе, я прихватила небольшой, но довольно увесистый чемодан, и поспешила покинуть пределы родного гетто. Уже на улице остановилась напротив сигаретного киоска, который, казалось, находился здесь всю жизнь.

Я позволила себе слегка забыть о времени, зачем-то разглядывая многочисленные ряды пестрых сигаретных пачек. К тому моменту я не курила уже практически пять месяцев и, как мне думалось, не собиралась. Но что-то заставило меня потянуться в карман за парочкой купюр. Наверное, все дело было в волнении, а точнее, в его отсутствии. Мне всегда думалось, что встречи с незнакомцами тем и прекрасны: ты ничего от них не ждешь. Как и внутри, внешне я оставалась абсолютно спокойной. Только вот внутренний голос откуда-то из глубин подсознания мягко напоминал, что обычно паника подкрадывается ко мне в последний момент. Меньше всего на свете мне хотелось обнаружить себя поддающейся нервному припадку где-нибудь на границе двух внезапно враждующих стран.

-- Пачку Винстона. Синего.

Я без особых зазрений совести расплатилась за сигареты. Так и закончилась моя борьба с никотиновой зависимостью.

Автобус отходил в четверть десятого, и теперь я таки на него опаздывала. В салон я вошла за секунду до отправления, соблюдая старые добрые традиции. Такой вот я человек: ну просто ненавижу ждать, а потому никогда не выхожу из дому заранее. На этот раз мне повезло. Дверь за моей спиной захлопнулась, автобус тронулся и спустя четверть часа я уже наблюдала за бесконечными полями да деревьями, что проносились за окном. Тем днем я читала «Бродяг Дхармы», что очень кстати описывало наступивший период моей жизни. Книга уже подходила к середине, а мы все ехали и ехали. Несколько раз проезжали какие-то населенные пункты, -- ПГТ или вроде того -- но было заметно, что до Крыма по-прежнему как до Луны.

Откровенно говоря, в Крыму я не была лет шесть, да и настоящих отношений у меня уже года три как не было. Словом, я понятия не имела, что меня ждет. Жизненный опыт подсказывал, что надежней всего будет не ждать ничего. В конце концов, рушащиеся надежды -- это всегда неприятно и лучший способ избежать подобной ситуации: попросту их не иметь.

Часов через шесть автобус, наконец, остановился. Водитель объявил, что через полчаса будем на границе, и отправился пить кофе. Внутри меня что-то ёкнуло. Впервые за много-много лет. Это меня обескуражило и я поспешила выйти на свежий воздух.

Одного быстрого взгляда хватило, дабы понять, что я нахожусь прямиком в центре какого-то украинского захолустья. Здесь люди курили под табличками «Не курить!», громко ругались матом при детишках и сидели на асфальте, когда рядом пустовала скамейка. В принципе, ничего необычного. Мы, южане, и не такое видали.

До отправления оставалось минут двадцать пять, так что я заглянула в одну из местных лавчонок. Это крохотное помещение едва превышало размеры туалета хрущевки. Все же, каким-то мистическим образом в него вместилась барная стойка, кофемашина, парочка стульев со столиком, телевизор и даже одно окно. Правда небольшое.

Еще внутри безымянного заведения обнаружились бармен, бариста, повар, продавец, кассир и фиц в одном лице. Девушка так резко выпрыгнула из-за стойки, что я бы обязательно подскочила, не будь я в тот момент где-то далеко в своих мыслях.

-- Посетитель! -- радостно воскликнула девушка. -- Вы -- наш первый посетитель за сегодня! Что будете?

Я слегка удивилась отсутствия акцента, характерного для подобных поселений. Затем взглянула на часы. Дело шло к вечеру. Поздновато как для первого посетителя.

-- Капучино.

-- Но у нас только латте и американо\ldots{}

-- Латте вполне подойдет.

Девушка кивнула и принялась возиться с кофемашиной. Та издавала предсмертные звуки, и я очень надеялась, что успею выпить кофе прежде чем она развалится.

-- Меня зовут Тома, -- представилась девушка. -- Вы недавно у нас?

По всем канонам современного бодипозитивного общества Тома выглядела вполне привлекательно. Это была высокая брюнетка с яркими губами и практически под ноль стриженными волосами. Не толстая и не худая. О таких обычно говорят «в теле». Черты лица -- очень выразительные и чувственные. Все это в совокупности с откровенно мужиковатой одеждой походило на стиль, популярный во времена Твигги.

-- Проездом, -- ответила я и расплатилась за кофе.

Тома окинула меня взглядом.

-- Красивые волосы, -- заметила она. -- Настоящие?

Что ж, дамы и господа, а вот и лидер среди наиболее часто задаваемых мне вопросов.

-- Более чем.

-- А далеко едешь-то?

-- В Севастополь.

-- Ух, далековато. И что там, в Севастополе?

Этот вопрос вызвал в моей голове небезызвестную цитату из «От заката до рассвета», но я как-то удержалась.

-- Мужчина.

-- Хороший?

-- Не исключено, что моей мечты, -- я пожала плечами. -- Но это не точно.

Протяжно вздохнув, Тома облокотилась о стойку.

-- Я тоже была здесь проездом, -- заметила она.

-- И что так?

-- И у меня тоже был мужчина. Как раз возвращалась к нему, когда парня встретила. Влюбилась. Вот разводимся с ним сейчас.

-- С которым из них? -- не поняла я.

Незнакомцы, отчего-то, так любят изливать мне душу. Возможно, как-то ощущают мою эмпатию, но в подобные моменты я всегда чувствую себя неловко, если у меня вдруг все в порядке.

Что ни говори, я в кои-то веки не страдала из-за разбитого сердца, не болталась на дне депрессии, куда обычно приводят творческие амбиции, и даже была немножечко счастлива. Мне сделалось неудобно. Пожалуй, никому не должно быть неловко за отсутствие жизненного дерьма, но мне почему-то стало. Я подумала, что, возможно, именно так себя и чувствовали мои друзья на протяжении парочки последних лет.

-- Да со вторым уже.

Я подняла стаканчик с кофе.

-- За несбывшиеся мечты.

Тома улыбнулась, но как-то грустно. Можно понять.

Я вспомнила о времени. Минута до отправления. Ну, все как обычно.

-- Мне пора. Спасибо за кофе.

-- Как тебя звать-то? -- вспомнила Тома.

-- Елизавета Васляева.

-- Ну, удачи тебе, Елизавета Васляева. Главное, замуж не выходи.

-- Постараюсь, -- уже в дверях ответила я.

\hypertarget{chapter-5}{%
\chapter{~}\label{chapter-5}}

Мы подъезжали к украинско-русской границе. Народ уже толпился у выхода из автобуса. Люди так настойчиво прокладывали себе путь, что я удивилась, как это еще никто не умер. Какая-то масштабная мадам правда завалилась на себе подобную и вскоре обе уже распластались вдоль салона.

Всего нас было тридцать три человека, не считая водителя. Прямо как годиков одному знаменитому божьему сыну. Тридцать человек спорили у входа, пытаясь первыми проникнуть к отделу с багажом. Еще две женщины поднимались с пола. Они выглядели сонными и обессиленными, но по инерции продолжали браниться. Завершая общую картину, я сидела в самом конце салона. Читала книгу и краем глаза поглядывала на происходящее вокруг. Багаж -- если таковым можно назвать чемоданчик размером с ноутбук -- и так был при мне.

Украинско-российская граница, как выяснилось, больше походила на декорации к первому сезоны «Ходячих мертвецов», нежели на хорошо охраняемое место переправы. То тут, то там виднелись полуметровые цементные блоки и целая куча хаотично разбросанной колючей проволоки. Еще одним фактором, производившим странное впечатление, выступали припаркованные повсюду автомобили. Большинство из них были с распахнутыми дверцами и опущенными стеклами, тогда как водителей внутри не наблюдалось. Те растерянно бегали от одной будки к другой, пытаясь правильно заполнить целую стопку бумажек, каждая из которых нужна лишь для получения следующей.

У импровизированного начала границы стояло несколько военных. Ребята были при параде, включая дубинки и огнестрельное. Фонари горели как-то вяло и, хотя солнце еще только подумывало садиться, становилось ясно, что освещаются лишь первые пару метров переправы. Кстати говоря, такой же участок пути имел более-менее вменяемое дорожное покрытие. За ним пешеходов ожидали овраги, трещины, камни, и прочие достопримечательности моего родного края.

Учитывая обстоятельства с ручной кладью, к пункту проверки документов я подошла первая. Вернее, поначалу мне так казалось. Но довольно возрастная женщина внезапно воткнулась прямо передо мной и уже самодовольно махала рукой какому-то не поспевающему за ней мужчинке. По ту сторону границы нас ждал другой автобус. До отправления оставалось больше часа, так что особой нужды в такой спешке не было. Как, видимо, и в вежливости.

-- Куда едете? -- спросил не поддающийся определению по половому признаку голос из будки.

-- Севастополь.

-- Причина визита?

-- Пока под вопросом.

Пауза.

Окошко распахнулось, и на свет божий выползло заспанное личико, немногим старше моего собственного.

-- Назовите причину визита, -- раздраженно повторила барышня.

-- Просто хочу кое-что проверить.

-- Нет, так нельзя.

-- А как можно?

-- Бизнес, туризм, или личная.

Так я совершенно неожиданным для себя образом узнала, как можно. Что бы это ни значило.

-- Что пишем? -- напомнил о себе голос из окошка.

Я призадумалась. Очевидно, что мой визит относился к категории личных. Однако, опыт минувших взаимоотношений с другими представителями земной расы показал, что ни в коем случае нельзя вешать ярлыки. Никакие. Даже если все кажется идеальным. Нет, особенно если все кажется идеальным! Более того, дело здесь не в возможной реакции других людей, а в твоем собственном подсознании. Я имею в виду, как только ты называешь мужчину своим молодым человеком, ты подсознательно начинаешь ждать от него определенную модель поведения. И чем больше его действия походят на те, что ты там себе успела придумать, тем сильнее ты расслабляешься. И тем болезненнее в дальнейшем будет принять суровую реальность.

Словом, в я только-только свыклась с мыслью об отсутствии разбитого сердца в моей грудной клетке и никак не была готова признать связь с Вениамином личной. Поэтому я ответила:

-- Бизнес.

-- Во-о-от как, -- работница таможни подняла бровь с той самой грациозностью, с какой бывалая проститутка поднимает юбку по окончанию тяжелого рабочего дня. Вроде как и лень, но по статусу положено. -- И кем вы работаете?

-- Я писатель.

-- Вот как, -- повторила она. -- И что, прямо пишите?

-- Преимущественно.

-- Вот как.

Мне вдруг стало очень смешно.

-- А работаете-то кем? -- не унималась мадам.

Я положительно не хотела слышать четвертое «вот как», поэтому решила сдаться.

-- Барменом.

-- Так и напишем, -- подытожив, она протянула мне небольшой бланк. -- Распишитесь внизу.

В графе причина въезда был подчеркнут «личный визит». Я выдохнула:

-- Вот как\ldots{}

***

Ох, чего только я не наслушалась перед выездом. Одни рассказывали мне душетрепательные истории и том, что на границе рвут документы. Вторые убеждали, что стоит мне запеть гимн Украины и меня тут же повалят на асфальт. Не то, чтобы я каждый день занималась подобными развлекательными программами, но кто это проверил и зачем? Была еще история про курицу.

Вообще, в моей жизни как-то подозрительно много бесполезных историй, что, так или иначе, связаны с курами. Понятия не имею, что это значит, но, пожалуй, что ничего. Так вот, подруга матери бывшего администратора «Штиля» решила поехать к племяннику, который остался жить на полуострове. Будучи коренной украинской бабулей, она прихватила с собой целую кучу еды, включая сырую курицу огромных размеров. Увы, правила запрещали перевозить продукты. В случае их обнаружения, нарушителя заставляли пройти весь путь обратно, -- от российской границы до украинской -- и оставить еду там, заполнив соответствующие бумаги. Если же ты спешишь на автобус, можно просто отдать еду пограничникам. К чему это я? Легенда гласит, что суровая бабуля съела свою злополучную курицу прямиком при служащих таможни. Сырой.

Стоя в очереди позади лишенной такта женщины я мысленно решила, что она вполне могла бы быть той самой бабулей, и теперь изо всех сил старалась не рассмеяться: перед глазами уже вовсю плясали живописные образы.

Проверка багажа не заняла много времени. Сникерс у меня не отняли и на откупоренную в дороге бутылку виски тоже никто не позарился. Подхватив платье свободной рукой, я шагала по широкой, отвратительно вымощенной дороге, которая с каждым новым метром становилась все хуже. Мелкие камешки то и дело норовили застрять в подошве моих ботинок. По обе стороны от этой недотрассы тянулись небольшие овраги. За ними -- высоченный забор из нержавеющей проволоки. Дальше виднелись бесхозные поля, кустарники и кое-какие деревья.

По мере моего приближения к русской границе забора становилось все меньше, а деревьев -- больше. Вскоре они уже занимали все пространство вдоль дороги. Благодаря размерам своего чемодана, я неплохо оторвалась от других пассажиров. Порой на пути мне встречались люди, идущие в противоположном направлении. Еще реже проезжали автомобили, но в основном я шла одна.

Раскинувшиеся по обе стороны дороги виды создавали иллюзию заброшенности. Казалось, я была единственной, чья нога ступала на эту землю на протяжении последних десятилетий. Тишина, покой и нетронутая природа. Все это напоминало мне пейзажи из документалок о Чикатило, которые я любила смотреть в детстве. Окончательно картину дополнил проржавевший автомобиль, наполовину погрязший в грязи. Стекол на нем не было, как и колес. Сквозь дверцы уже успели прорасти впечатляющих размеров сорняки. Как ни странно, машина удачно вписывалась в обстановку. Думаю, она и сейчас там стоит.

\emph{Самое то для внеплановой медитации}, решила я и забралась на крышу бесхозной машины.

Благодаря эйдетической памяти и своей творческой натуре я без труда могла переместиться куда угодно. Вот только не всегда могла это контролировать. Воображение, на этот раз слишком живое, унесло меня к глубоко похороненным страхам прошлого, так что от медитации пришлось отказаться.

В воздухе запахло морем. Наконец, вдалеке стали видны какие-то светящиеся вышки. Спустя минут семь мне преградил путь шлагбаум в компании молодого солдата. Парень попросил показать паспорт.

-- Наркотики, оружие, спиртное перевозите? -- серьезно спросил он.

-- Целую кучу, -- ответила я.

Чемодан открывать не стали.

-- Дальше по дороге идти нельзя, -- заметил солдат. Я вопросительно подняла брови. -- Такие правила. Пройдите во-о-он туда.

Он указал на начало очередного забора. Подойдя ближе, я ужаснулась. Нет, я, конечно, знала, что в России мало сторонников демократии, но не до такой же степени! В общем, вы в курсе, как выглядят наружные тюремные коридоры? Те, по которым зеков обычно ведут из одного крыла в другое? Такие узенькие площадки максимум метр в ширину, с двух сторон огражденные высоченным забором все с той же колючей проволокой на верхушке. Именно так выглядит начало российской границы.

Оценив ситуацию, я подкурила.

***

Спустя полчаса я, наконец, забралась во второй автобус. До отправления оставалось более двадцати минут, и я понятия не имела, как их провести. В другой ситуации я бы, наверное, уже давно умерла со скуки. Но не сегодня. Сегодня у меня был виски.

Конечно, я в сотый раз повторяла себе, что не стоит ждать ничего серьезного от предстоящей встречи. И, все-таки, я ждала. Вернее, та часть меня, которая заставляла мою внутреннюю феминистку забиться в угол. Предвкушения чего-то хорошего -- вот к чему я питала особую зависимость. Только в подобные волнительные моменты я могла почувствовать себя живой.

Откровенно говоря, и спустя годы, вспоминая вечер на таможне, я все так же чувствую послевкусие той необъяснимой эйфории, которая настигла меня за несколько часов до встречи с Вениамином. Автобус больше не казался мне душным, погода жаркой, а жизнь дерьмовой. Я сидела в конце салона, наблюдая за тем, как уже успевшая полюбиться мне лесополоса принимает солнечные ванны, и чувствовала себя как никогда восхитительно.

Это меня расслабило. Вскоре я прикрыла и позволила памяти ненадолго вернуть меня в прошлое.

\hypertarget{chapter-6}{%
\chapter{~}\label{chapter-6}}

За окном моросил октябрьский дождь две тысячи тринадцатого года выпуска, отчего внутри становилось крайне уютно. Обернувшись одеялом, я сонно смотрела в окно. Передо мной стояла чашка кофе, кола и наполовину пустая бутылка виски. По ту сторону стекла уже начинало темнеть, и я наблюдала, как в окнах соседних домов зажигаются огни. Одно за другим. Прямо как в детстве, но без виски, разумеется. До шестнадцати я пила только вино.

С минуту я разглядывала стоявшие на подоконнике жидкости, после чего потянулась кофе. Любовь пагубно влияла на мой алкоголизм.

За моей спиной скрипнула дверь. Обернувшись, я увидел, стоявшего в дверном проеме Адама. Дождь к тому моменту усилился, и мужчина уже стягивал с себя промокшую до нитки куртку. Несмотря на это, выглядел он как никогда очаровательно. На губах играла мечтательная улыбка, а карие глаза сияли, делая взгляд еще более выразительным. Не говоря ни слова, Адам подошел ко мне и, опустившись напротив, коснулся своими губами моей руки.

До этой минуты я и подумать не могла, что такой банальный и даже старомодный жест способен вызвать во мне целую бурю эмоций. Но он вызвал. Я смущенно улыбнулась, и Адам улыбнулся в ответ. Своей самой широкой улыбкой, открывающей ямочки на щеках. Забыв о кофе, я коснулась рукой темных волос, что вскоре начнут доставать до плеч, и вновь улыбнулась. На этот раз куда шире.

Адам поцеловал меня, и я вдруг осознала, что люблю его.

Той ночью ни один из нас не смог уснуть. Сначала мы были слишком заняты для такого скучного дела, а затем\ldots{}

-- О чем ты думаешь? -- спросил Адам, когда я собрала растрепавшиеся волосы и растянулась на подушке.

Я повременила с ответом.

-- Я думаю о том, что в бутылке подозрительно много виски как для пятничного вечера.

Адам вздохнул.

-- Не хочешь говорить?

-- Пожалуй, нет, -- медленно ответила я, ласково взглянув на своего собеседника.

Он обхватил меня за плечи и крепко поцеловал.

-- А теперь?

Я мужественно молчала.

-- Нет, ну так не честно! Я думал, мы договорились говорить друг другу всю правду, -- Адам обиженно отодвинулся от меня. -- Милая, ты заставляешь меня нервничать.

Так себе аргумент как для двадцатипятилетнего мужчины, но я и правда не знала, что ответить.

-- Не думаю, что сейчас подходящий момент.

Десятисекундная пауза.

-- Так о чем? -- вновь спросил Адам и я сдалась.

-- Думаю, я люблю тебя.

После этой фразы Адам молниеносно изменился в лице. Позже он утверждал, что знал, что я скажу именно это. Более того, он так настаивал именно потому, что думал, что сможет ответить мне тем же. Однако, в последний момент вдруг понял, что не сможет -- вся суть мужчин.

-- Мне нужна сигарета, -- отрезал Адам и покинул постель.

С этого момента наши мимолетные отношения круто переменились. Адам стал первым мужчиной, к которому я питала сильные чувства во взрослом возрасте. Конечно, мне было всего девятнадцать, но неплохой рабочий стаж, несколько написанных книг, вовремя уплаченные налоги и уйма забот позволяли мне считать себя взрослой.

Мы расстались спустя полтора месяца. Тем приятным погожим днем, когда я меньше всего этого ожидала. Без криков и прочих тягостных выяснений отношений. Он ушел, а я проплакала всю ночь. К утру замкнулась в себе и перекрасила волосы в алый.

-- Нельзя говорить человеку о любви на второй день знакомства! -- заявил Адам спустя дня три после этого самого второго дня знакомства. Кстати, формально мы были знакомы куда дольше, но никогда прежде не виделись. -- Это ненормально!

Никогда не забуду того, с каким чувством он произнес последнее слово. Ненормально. Словно это было не одно, а целых четыре отдельно существующих слова. Не-нор-маль-но. Так обычно разговаривают с детьми или умственно отсталыми.

Куда делся тот парень, который уговаривал меня дать шанс этим отношениям? В одной из моих книг персонаж по прозвищу Молния чуть не лишается своей возлюбленной из-за того, что признается ей в любви на второй день знакомства. Никто и не знает, насколько автобиографична эта сцена.

***

Теперь же меня занесло в Крым.

Поездка оказалась невыносимо долгой. В какой-то момент я даже уснула, что редко бывает со мной в общественном транспорте. Точнее, попросту отключилась. Сегодня ведь уже ни для кого не новость, что происходящее во сне проносится в мозге за долю секунды?

Что ж, тем вечером в моей голове пронеслось немало. Проснулась я в тревожном состоянии, и тут же потянулась к часам. Чувствовала себя так, словно проспала часов двадцать. На деле же -- тринадцать минут. Вот уж не думала, что освоенная в студенческие годы практика кратковременного сна однажды мне пригодится.

Так вот, проснулась я в тревожном состоянии. Мне снился Адам. Как всегда, сон оказался чересчур реалистичным, так что я не сразу пришла в себя. И пускай выглядела я абсолютно спокойной, внутри уже начинал мельтешить страх отношений. С одной стороны, я всем сердцем хотела полюбить кого-то сильнее, чем когда-то любила своего бывшего. С другой -- я боялась этого до умопомрачения. Именно поэтому на протяжении последних трех лет я никому не позволяла сблизиться со мной. Даже поцелуи во время секса были запрещены. Мне думалось, что они вызывают слишком интимную связь, как бы абсурдно это не звучало.

И вот я вновь куда-то ехала. На этот раз без планов о чем-то серьезном и с куда большими надеждами одновременно. От подобных мыслей сердце начинало колотиться так сильно, что становилось сложно дышать.

К тому моменту бутылка виски совсем опустела. Автобус проезжал Симферополь. Он показался мне до невозможности скучным и очень напоминал бесконечную улицу Чигрина -- не самый благополучный район моего родного города.

«Спокойно, Васляева, -- сказала я себе. -- Тебе уже не девятнадцать. Ни в кого ты поспешно не влюбишься.»

Вот она, самая распространенная из всех возможных человеческих глупостей: отчего-то полагать, что ты стал умнее.

\hypertarget{chapter-7}{%
\chapter{~}\label{chapter-7}}

Никто понятия не имел, когда должен прийти мой автобус. По предварительным расчетам, время прибытия -- около шести часов вечера. Но зная отечественную пунктуальность, я в этом очень сомневалась. Веня жил в пяти минутах езды от автовокзала, так что мы условились на том, что я поймаю вай-фай где-нибудь на подъезде к Севастополю.

Как ни странно, мы ехали без остановок, если не считать внезапной поломки автобуса. Следующий пришлось ждать посреди какого-то захолустья, так что вай-фаем там и не пахло. В итоге до пункта назначения я добралась лишь когда стрелка на часах стремительно приближалась в десяти вечера.

И вот автобус, наконец, остановился. Я выбралась на улицу и сладко потянулась. Первым, что привлекало внимание, был свежий морской воздух. Вдохнув его, я вдруг поняла, как же сильно истосковалась по бризу.

Солнце, конечно, уже давно село, прихватив с собой тепло весеннего дня. На мне было легкое платье в пол со спущенными плечами, которое теперь мелодично колыхалось в такт ветру. Становилось как-то прохладно. Вдоволь надышавшись морем, я осмотрелась в поисках подходящего заведения. По-прежнему никаких признаков беспроводного интернета. Естественно, местной сим-карты у меня тоже не было, а моя уже часа четыре как перестала быть активной.

В какой-то степени, тот факт, что на вокзале не было Вениамина, сыграл мне на руку. Таким образом, я могла еще раз все обдумать. И, все же, казалось, кто-то уже давно обдумал все за меня. Несмотря на бешеную усталость и практически полное отсутствие сна, я все еще чувствовала себя отлично. Вполне нормальный расклад. В рассвете бессонницы настоящая усталость обычно накрывает день на третий, а то и пятый, когда к ней подключается светомузыка.

К сожалению, здание вокзала не смогло вместить в себя что-либо кроме касс да комнаты ожидания. Пройдя сквозь нее, я вышла на улицу с обратной стороны и обнаружила, что пошел дождь. Стоило сделать пару шагов к проспекту, как ко мне тут же подбежала организованная группировка таксистов. Тут-то я и вспомнила о том, что адрес Вениамина остался в переписке, так что без интернета было не обойтись.

Как вы уже поняли, я никогда не отличалась привычкой планировать наперед. Если дело касалось работы, тут я стратег высшего разряда, но, когда речь заходит о личном комфорте -- только хардкор, только импровизация.

-- Подвести, красавица? -- хором спросило сразу несколько мужчин.

Я улыбнулась их синхронности. Города меняются, а таксисты везде одинаковые.

Вдалеке я заметила заправку. Она находилась в паре светофоров от меня и походила на те, в которых обычно продаются подозрительно дорогие хот-доги. И есть вай-фай. Таксисты еще что-то продолжали спрашивать, когда я ступила на первый пешеходный переход.

-- У вас виски есть?

-- Только кофе с виски.

-- Давайте самый большой.

Если верить табличке на входе, до закрытия оставалось всего-ничего. Но стоявшая за прилавком женщина никуда не спешила.

-- Паспорт с собой? -- спросила она.

-- Паспорт?

Я даже не сразу сообразила, в чем здесь дело.

-- Да, покажите паспорт. Лицам моложе восемнадцати выпивать не положено.

-- Да вы гоните!

Даже не знаю, что меня удивило больше: проверка документов на заправке или то, что кто-то назвал ирландский кофе
выпивкой. Я еще немного повыделывалась прежде, чем предъявить документы.

Здесь и впрямь был беспроводной интернет. Только вот доступ к нему ты мог получить лишь через местную сим-карту, которой у меня, как мы все уже поняли, не было. Еще и виски ненастоящий! С таким же успехом я обошла еще несколько заведений. Дождь заметно усилился и вот-вот обещал обратиться ливнем. Тогда-то на моем пути встретился какой-то местный фастфуд.

Внутри было пусто. Три столика и все свободны. За прилавком стояла полная девушка. Она красила ногти в желтый цвет и, кажется, была не слишком рада моему визиту.

-- Мы закрываемся через шесть минут.

-- Мне нужно иметь сим-карту, чтоб подключиться к вашему вай-фаю? -- на одном дыхании выпалила я.

Девушка взглянула на меня как на умалишенную. Мне, в общем-то, было не привыкать.

-- Нет. Но вам как бы надо бы чёта купить.

-- Латте, -- заказала я. -- Самый большой.

Людей на улицах было поразительно мало, что создавало особенно приятную, тихую атмосферу. Сидя за столиком у окна, я наблюдала за тем, как потоки дождевой воды текут вдоль улицы, и наслаждалась тремя минутами, оставшимися до закрытия заведения. Мне, наконец, удалось подключится к сети, и я ужаснулась количеству входящих сообщения. Кажется, Вениамин не на шутку разволновался. Я опоздала на четыре часа, а внезапно разыгравшийся шторм изрядно добавлял ситуации трагизма.

Успокоив Веню, я пообещала ждать в холле вокзала. Вечер обещал быть интересным.

\hypertarget{chapter-8}{%
\chapter{~}\label{chapter-8}}

Эта сцена не раз мелькала в моих мыслях до нашей первой встречи с Вениамином. Слушая его голос, собираясь на последний перед отпуском рабочий день, стоя за барной, отправляясь ко сну, сидя за рукописями и бегая за постоянно заканчивающимся льдом, я представляла, как впервые увижу его. Но еще чаще я думала о нашем с Веней знакомстве уже после того, как оно состоялось. Когда наполняла бокал вином, готовила ужин, гуляла около моря, ждала свой кофе в какой-нибудь забегаловке, смотрела, как он точит ножи или засыпает прямо посреди комнаты\ldots{}

И особенно сильно я думала о том вечере после очередной ссоры. Как бы ни обстояли дела, несмотря на всю горечь обиды, одно лишь это воспоминание было способно растопить лед, возникший в моем сердце.

Скажи мне кто-нибудь, что писать о собственной жизни будет так сложно -- ни за что бы не поверила. Казалось бы, ничего не нужно, ведь ты и так все знаешь и помнишь до мелочей. Ты часть истории, по отношению к которой твои собственные пожелания, по сути, не имеют никакого отношения. В этом-то и проблема. Произошло то, что произошло и все, что тебе остается -- проанализировать события и отойти в сторонку, притворившись обычным наблюдателем. Есть второй вариант, не подразумевающий сохранения ясности ума, так что мы его опустим.

Говорят, мне повезло с памятью. Я помню все имена, даты, события, диалоги и так далее до подробнейших деталей. Могу без труда вспомнить, во что был одет человек пять лет назад, что он говорил, с каким настроением и почему. Могу сказать, какая за окном стояла погода, что за музыка играла в проезжающей мимо машине и о чем я в тот момент думала. Эйдетизм подразумевает собой особый вид памяти. Совокупность зрительных образов с показаниями остальных счетчиков твоего организма: слуховых, вкусовых, тактильных, обонятельных, двигательных и даже интуитивных. При правильной концентрации можно перенестись в любое место, достаточно лишь разок побывать там, но есть у эйдетики один жирный минус. В справочниках, правда, он подается как большой плюс, но что они в этом смыслят: «эйдетические образы отличаются от образов восприятия тем, что человек как бы продолжает воспринимать предмет в момент его отсутствия».

Отсутствие -- ключевое слово.

***

Звуки и запахи -- это вообще отдельный разговор. Правильное сочетание этих элементов буквально открывает мне окно в прошлое. С такими манипуляциями памяти не мудрено влипнуть в депрессию.

Спустя год после болезненного расставания с Адамом мне нужно было съездить в Черниговскую область. Добиралась я, естественно, таким же разваливающимся поездом, каким когда-то добиралась к своему молодому человеку. Был прохладный осенний вечер сродни тем вечерам, когда я только раздумывала о перспективе отношений с Адамом. Тогда у меня были духи с запахом лаванды. Они пылились на полке с две тысячи тринадцатого, и черт меня дернул воспользоваться ими тогда, оправляясь на железнодорожный вокзал.

Запах парфюма и шум поезда буквально свели меня с ума. Картина перед глазами оставалась прежней, но в какой-то момент, я была уверена в том, что мне все еще девятнадцать, а поезд везет меня в Москву, и уже утром меня встретит туман Киевского вокзала. Из толпы на перроне покажется Адам -- в своей старой кожаной куртке -- и поспешит навстречу, чтобы помочь спустить с поезда тяжелый чемодан.

Не мудрено, что Шульц так ярко описывал приступы шизофрении с примесью самоубийств, возникшие в итоге настойчивых попыток вызвать из памяти эйдетические образы. Но что делать, когда они всплывают против твоей воли?

История закончилась тем, что я вылетела на улицу при первой же возможности, а затем стояла посреди какой-то неизведанной сельской местности и наблюдала за тем, как состав плавно удаляется в сторону заката.

Последнее, кстати, вовсе не метафора. В этот момент солнце действительно опускалось за горизонт, приглушая осенние оттенки. Я смотрела на исчезающий за небосводом алый диск, опавшие листья и разноцветные деревья, колышущиеся на ветру. Смотрела и думала, какая же я дура.

***

Вернемся к относительной реальности.

Первая половина мая шестнадцатого подходила к концу. Закинув ногу на ногу, я сидела в крохотном зале ожидания, который, по совместительству был холлом, а также единственной комнатой в здании севастопольского вокзала. Кроме меня здесь никого не было, так что я позволила себе дважды покурить. Честно говоря, я настолько волновалась, что курнула бы еще не раз и отнюдь не табака, но как-то не хотелось создавать вокруг себя газовую камеру. Под дождь выходить тоже не хотелось, и я просто продолжала сидеть, закинув ногу на ногу. Сперва мне думалось, что я читаю, но спустя минут пять пришлось признать, что я уже раз семнадцатый начинаю один и тот же абзац, но так и не добираюсь до середины.

\emph{Произносить в голове маячащий перед глазами текст и думать о чем-то на фоне -- поразительная, но такая бесполезная способность человеческого мозга}, подумала я, и оставила книгу в покое.

По обе стороны от меня находились большущие старые окна, в которые я то и дело поглядывала. К тому моменту мои каштановые волосы уже были достаточно длинными и объемными, дабы за ними спрятаться. При этом они были вьющимися, что позволяло мне откинуть парочку прядей и незаметно бросать пристальные взгляды то вправо, то влево. Стоит ли говорить, что сидела я при этом прямо и расслабленно, упорно соблюдая видимость чтения?

Вопреки всем моим усилиям, размытые потоки воды да отблески проезжающих по проспекту машин были всем, что я видела. В итоге я решила забить на это неблагодарное дело и попыталась еще раз погрузиться в книгу. К тому же, до окончания «Бродяг» оставалось всего ничего. Только вот я сама уже давно чувствовала себя той еще бродягой, но только сейчас впервые всерьез задалась вопросом: куда эта Дхарма меня приведет?

Я отбросила с лица надоедливую прядь волос -- прядь в действительности крайне надоедлива, так что вы еще не раз прочитаете о том, как я ее отбрасываю -- и склонилась над дорожным романом, когда дверь слева от меня распахнулась. Естественно, я знала, кто сейчас в нее войдет, но не стала поспешно отрываться от чтения.

Знаете, говорят, что женщина может думать миллион мыслей одновременно. Вообще-то, так и есть. За те несколько секунд, которые понадобились Вениамину, чтобы подойти к моему креслу, я задала себе как минимум десять вопросов.

\emph{Подправила ли я помаду? До конца ли высохли мои волосы? Где мой паспорт? Зачем мне сейчас паспорт? Успела ли я втянуть живот? Не прилипло ли платье к груди? А если прилипло, то насколько это сексуально? Достаточно ли непринужденно я сижу? Не слишком ли непринужденно я сижу? Пахну ли я также хорошо, как пять минут назад?}

И так далее, и тому подобное. Не поверите, но я даже успела запомнить номер страницы, на которой остановилась. Сейчас это кажется мне смешным. Как и многое другое.

Позже Веня говорил, что я держалась с ним так, словно мне было глубоко насрать, и это его окончательно очаровало. Такой вот у меня бесполезный талант: выглядеть хладнокровно, когда на самом деле внутри все сжимается от волнения и сердце скачет как курс доллара в Украине с наступлением военного положения. Видимо, по этой простой причине я неплохой игрок в покер.

Так вот, дверь распахнулась. Спустя пару мгновений я позволила себе поднять глаза. Передо мной стоял высокий блондин с зелеными глазами и хемингуэевский бородой. На нем была фланелевая клетчатая рубашка, которую затем я видела так часто, что та успела мне надоесть, и невероятно ласковая улыбка, которую я видела еще чаще. Она мне никогда не надоедала.

Откровенно говоря, я всегда думала, что предпочитаю длинноволосых брюнетов со смуглой кожей. Однако, увидев Вениамина, я поняла, что никогда не встречала мужчину прекраснее. У него была бледная кожа, длинный андеркат и профиль настоящего викинга. Волосы частично скрывала небольшая шляпа с полями.

Веня улыбнулся мне еще шире, и я вдруг осознала, что пялюсь на него уже с добрых полминуты. С другой стороны, он занимался тем же, так что все было в порядке. Когда наши взгляды встретились, Вениамин развел руки, предлагая мне окунуться в его объятья. Хотя, учитывая путь, который я проделала тем днем, правильней было бы сказать «упасть».

Он обнял меня, не говоря ни слова. Так мы и стояли в полнейшей тишине пустующего вокзала. Где-то очень далеко продолжали шуметь автомобили, а дождь играл ночную мелодию на оконных стеклах и тротуаре.

\hypertarget{chapter-9}{%
\chapter{~}\label{chapter-9}}

-- Твой самый большой недостаток? -- спросила я.

Мы шли вдоль стеллажей небольшой винной лавки.

-- Я алкоголик, -- спокойно ответил Веня. -- Алкоголь -- моя единственная проблема. Порой я не ведаю что творю, отчего друзья зовут меня Невероятный Алк. Но в остальном я идеален.

Он ухмыльнулся.

Я обвела взглядом стеклянные бутылки.

-- С вином тоже проблемы?

Он покачал головой.

-- В основном с водкой.

-- Отлично, -- я протянула руку к красному полусладкому. -- Тогда берем две.

Определившись с выпивкой, мы все так же продолжали прогуливаться вдоль винных полок. Веня держал меня за руку. Периодически я ловила на себе его смущенные взгляды. За прошедшие годы мужчины смотрели на меня по-разному. Большинство с желанием. Некоторые как на полнейшую стерву, но все с тем же желанием. Были даже те, в чьем взгляде читалось восхищение, но все это казалось мне поддельным. И вот, разгуливая по алкогольному кварталу, я осознала, что уже очень давно никто не смотрел на меня так.

-- А твой? -- поинтересовался мой спутник.

-- Мой самый большой недостаток?

-- Ага.

-- У меня биполярное расстройство.

-- Что это?

-- Маниакально-депрессивный психоз, -- объяснила я. -- Мне это название даже больше нравится. Оно звучит понятней, но
теперь считается оскорбительным, так что\ldots{}

-- Хорошо, давай поменяем правила, -- предложил Веня. -- Той самый большой плюс.

Теперь пришла моя очередь ухмыльнуться.

-- У меня биполярное расстройство.

-- И в чем здесь проблема?

-- Всегда по-разному.

-- Например? -- не унимался Веня.

Я вздохнула и закатила рукава, демонстрируя выдающуюся коллекцию шрамов с обеих сторон рук.

-- И всего-то? -- еще одна ласковая улыбка.

Я подняла рукав выше.

-- Еще бывает сигареты об себя тушу. Это не симптом, а, скорей, последствия, но, говорят, окружающих напрягает.

Мой спутник вручил мне бутылки и принялся закатывать рукава. Руки под рубашкой украшало куда большее количество шрамов. Они выглядели на удивление красиво, если не считать парочки партаков, одним из которых была эмблема супермена. Вот как нужно производить впечатление на женщину. Спрашиваешь о ее недостатке, а затем даешь понять, что для тебя это всего лишь пустяк.

Я весело рассмеялась. То ли шрамам, то ли татуировке.

-- Серьезно, -- спросила я, вскинув бровь. -- Супермен?

-- Как видишь. Терпеть его не могу.

-- Так что же не перебьешь?

-- Ну, так в этом-то и смысл, -- парировал Вениамин, направляясь в сторону кассы.

***

Звездное небо раскинулось над самым прекрасным из всех городов. Я видела в Севастополе уйму недостатков, но никогда не прекращала любить его. Думаю, с Вениамином было так же.

Очарованные вечером и компанией друг друга, мы сидели на парапете посреди пустой улицы. Несмотря на то, что дождь практически закончился, Вениамин держал надо мной зонтик. Все это время мы то без умолку болтали, то не могли прекратить целоваться. Обе винные бутылки стояли рядом. Никто к ним даже не притронулся.

Первым молчание нарушил Вениамин.

-- Твои волосы еще красивее, чем на фото.

-- А что насчет остальных частей меня? -- я повернулась к (своему?) мужчине, изображая пристальный взгляд.

-- Ты очень красивая, Васляева. Именно такой я тебя и представлял. И ты лучший в мире собеседник. Я окончательно очарован.

Я покраснела. Впервые за долгие годы существу мужского пола удалось меня смутить. Заметив это, Веня лишь крепче прижал меня к себе.

Море находилась совсем недалеко. Ночной ветер доносил до нас его ностальгические нотки, создавая иллюзию оазиса. Я набрала полные легкие свежего воздуха и медленно выдохнула.

-- Люблю я море.

-- А я -- тебя, -- отозвался Веня.

Здесь мне должен был вспомниться Адам, то, как я призналась ему в любви на второй день знакомства и все такое прочее. Но он мне почему-то не вспомнился. Я вообще напрочь забыла о его существовании.

-- Что ты сейчас сказал?

Я в удивлении отстранилась, на что Веня лишь развел руками.

-- Я люблю тебя, Васляева.

-- Секундочку. Ты признаешься мне в любви в первый день знакомства?

-- Так и есть. Ничего не могу с этим поделать. Да и формально мы знакомы гораздо дольше\ldots{}

\emph{Где-то я это уже слышала.}

Казалось, вся моя болтливость иссякла. Я просто сидела, переводя взгляд с Вени на сигарету в его руке и обратно. Затем пришла к самому мудрому в сложившихся обстоятельствах решению, и подкурила.

Тем временем Вениамин взглянул на часы.

-- Совсем забыл о времени, ты ведь устала! -- спохватился он. -- Пойдем.

Так мы стали жить вместе, фактически зная друг друга не больше трех часов.

\hypertarget{chapter-10}{%
\chapter{~}\label{chapter-10}}

Следующее утро мы как-то пропустили. Нежились в объятьях друг друга, пока не заметили, что дело близится к вечеру. Вино так и стояло закрытым.

-- Чем хочешь заняться? -- спросил Веня, наблюдая за тем, как я выбираюсь из ванной.

-- Хочу прогуляться.

-- Поехали в центр? -- предложил он.

И мы поехали. Несмотря на то, что солнце только-только село, снаружи отнюдь не было людно. Причем чем ближе мы подходили к центру, тем меньше пешеходов встречалось на нашем пути. Вероятно, это была старая часть города потому, как местами увиденное очень напоминало переулки Львова, что так дороги моему сердцу. Отсутствие публики создавало сказочное впечатление, словно не только этот вечер, но и все улочки, скамейки, парки -- все это принадлежало лишь нам двоим. Словно мы -- Маленький Принц, гуляющий по собственной планете.

-- Я покажу тебе свое новое любимое заведение, -- пообещал Вениамин.

-- Питейное?

-- Ну, а то! Тебе понравится.

Оказалось, что все горожане прятались по кафешкам. Мы проходили мимо множества баров, пабов, пиццерий и прочих угодий гедонизма, и в каждом из них виднелись переполненные залы.

Должна признать, «Фикус» и впрямь был миленьким местом. Снаружи он выглядел как крохотное ретро-кафе, но на деле оказался куда больше. Выполненный в моих излюбленных тонах интерьер не мог не ласкать глаз: черные диванчики и такие же черные столики с матовым покрытием. В этом радужном месте мы сделали наше первое совместное фото.

Шоты принесли на самом настоящем бревне, что было слегка необычно. И я говорю не обо всех этих выточенных подносах, что так популярны в социальных сетях, нет. Я говорю о самом настоящем бревне, которое можно встретить в лесу.

Вслед за шотами за наш столик пожаловало пиво для Вениамина и винишко для Елизаветы. Девица за соседним столиком бесстыдно пожирала моего спутника взглядом на протяжении всего вечера, тогда как взгляд Вениамина не отрывался от меня.

\emph{Барышня, вы меня сейчас выбесите}, сказал мой взгляд.

Но девица не отреагировала. Она постоянно поправляла волосы, закусывала губу и оттягивала майку как можно ниже.

Веня отлучился в уборную, и девица тут же подорвалась следом. Вероятно, она надеялась «случайно» столкнуться с ним в дверном проеме. Знаю я эти ваши женские штучки. Короче говоря, долго не думая, я запустила в уже начинавшую меня подбешивать барышню перечницей, после чего та села на место и тупо уставилась в пол. Как ни в чем не бывало. Неужели обязательно было до этого доводить?

Вскоре принесли ужин.

-- За мою ненаглядную деточку, -- произнес Вениамин, поднимая бокал.

-- За моего крымского дровосека, -- ответила я.

Разрезав пиццу, Веня потянулся за солью и\ldots{}

-- Милая, а куда делся перец? -- изумленно спросил он.

-- Понятия не имею, -- ответила я и приложилась к вину.

***

-- Что не так? -- спросил Веня, когда мы возвращались домой последним троллейбусом.

Время стояло позднее, и мы опять были совсем одни.

Я лишь отмахнулась.

-- Пожалуйста, скажи мне. В чем дело, любимая?

\emph{Любимая}. Это слово вызвало во мне ураган противоречивых чувств, в основном состоящий от всяческих производных счастья и страха.

-- Ты мне очень нравишься, -- наконец, ответила я, -- но люди слишком часто разбивают мне сердце.

-- Черт. Ты любишь кого-то другого?

-- Вовсе нет. Просто я боюсь открыться тебе. Боюсь своих чувств и того, какие у них могут быть последствия.

-- И это тебя печалит?

-- Не совсем. Мне кажется, что этот страх мешает мне почувствовать настоящую радость от того, что между сейчас нами происходит. Ты понимаешь?

-- Я понимаю, что кто-то очень сильно тебя обидел.

Я вздохнула.

-- Когда закончились твои последние отношения? -- я пристально посмотрела на Веню.

На этот раз по-настоящему.

-- Перед новым годом.

-- Но ты виделся с ней и после, не так ли?

Вениамин удивленно взглянул на меня.

-- Нет, не виделся.

-- Уверен?

Он слабо кивнул.

-- А я вот думаю, виделся.

-- С чего это ты взяла? -- осторожно спросил он.

-- С того, что большинство мужиков так делает. Спит со своими бывшими потому, что это удобно. Но только с теми, с кем они состояли в длительных отношениях. Когда секс уже входит в привычку.

-- Ладно. Ты права.

-- Так, когда ты в последний раз с ней виделся? -- спокойно спросила я.

-- В марте.

-- Значит, два месяца. И ты считаешь, что уже готов к новым отношениям?

-- Абсолютно, -- без малейшей заминки ответил Веня.

-- Ты ведь понимаешь, что это глупо вступать в новые отношения, когда ты еще не оправился от предыдущих?

-- Это не тот случай.

-- Почему?

-- Я ее не любил.

-- А почему так уверен, что меня любишь?

-- Твою мать! -- внезапно воскликнул Веня. -- Мы проехали остановку!

Я посмотрела в окно. Действительно, дом Вениамина остался позади, в свете ночных фонарей.

Пришлось возвращаться пешком.

-- Послушай, я несколько лет переживала болезненный разрыв отношений, которые и длились то не особо долго. А все потому, что и до этого со мной поступали не самым лучшим образом. Сейчас я в порядке. Не как другие люди, конечно, но в порядке по сравнению с тем, как было. Но я не чувствую, что могу справиться с подобной ситуацией еще раз. К тому же, с каждым разом чувства все сильнее, а значит, и боль становится глубже. Иногда мне кажется, что от моего сердца уже совсем ничего не осталось. Одни лишь осколки, но затем оп, и новое предательство. И вновь эта режущая боль. Но это, наверное, хорошо. Значит, есть чему биться.

Вениамин слушал меня с самым серьезным видом. Он молчал и лишь подкуривал одну за другой.

-- Я не обижу тебя, -- уверенно произнес он. -- Буду собирать твое разбитое сердце по кусочкам до тех пор, пока ты не почувствуешь себя в безопасности.

Его голос звучал настолько искренне, что я с трудом сдерживала слезы.

-- А потом?

-- А потом просто буду любить тебя. Хотя постой\ldots{} Я ведь уже люблю.

***

-- Ну, так что, Васляева, ты дашь мне шанс? -- спросил Вениамин.

Мы сидели на балконе. По правую руку расположилось море, по левую чернел лес, впереди мерцал ночной Севастополь.

Я выпила целую чашку кофе прежде чем ответить на этот вопрос.

-- И вот еще что, -- между тем произнесла я. -- Никогда не ври мне.

Он уже открыл рот, дабы что-то сказать, но я тут же продолжила.

-- Нет, я серьезно. Никогда не думай врать мне. Особенно по мелочам как сегодня. Меня достаточно обманывали, чтобы я могла распознать ложь. Я это ненавижу. На самом деле, я многое могу простить, но не обман.

-- Я тебя понял, -- виновато произнес Веня.

-- Так, когда ты в последний раз виделся со своей бывшей?

Он сделал небольшую паузу.

-- Три недели назад.

-- Хорошо.

-- Так вот\ldots{} Ты дашь мне шанс?

-- Да, кажется, уже дала.

Услышав это, Веня буквально засиял. Даже в полумраке была видна его довольная улыбка. Мы еще долго целовались, слушая, как где-то вдалеке волны бьются о берег.

\hypertarget{chapter-11}{%
\chapter{~}\label{chapter-11}}

Сколько бы воды не утекло, я по сей день считаю, что самые обворожительные вечера можно встретить лишь в Севастополе. В какой бы части города ты не находился, бриз найдет тебя. Морской воздух и волшебные закаты. Слушая меня, можно решить, что все дело в моей любви к Вениамину Ларионову. Что ж, первые припадки взрослых чувства я испытала задолго до описываемых событий. Было это на родине Булгакова, Пастернака, Маяковского, Мандельштама\ldots{} Список можно продолжать до следующей главы, но итог один: в Москву я так и не влюбилась.

Севастополь же мигом покорил мое сердце. Этот горный городок с его извилистыми дорогами, сумасшедшим трафиком, гранитными набережными и залитыми солнцем скверами вызывал у меня необъяснимое чувство ностальгии. Вы когда-нибудь окунались в прозу Брэдбери? Я имею в виду не сай-фай, а настоящие, автобиографические тексты вроде «Вина из одуванчиков» и «Лето, прощай!»? Душевная теплота и приятная меланхолия с едва заметной толикой грусти -- вот что несут в себе это книги. Вот что несет в себе Севастополь. Здесь солнце садится гораздо позже привычного времени, а море практически никогда не замолкает. И как же поразительно Севастопольское небо!

Тем не менее, дни в Крыму оказались куда более жаркими, чем мне представлялось. Будучи уроженкой юга, я все равно умирала от жары.

-- Зачем вообще строить города в месте, где настолько жарко? -- капризно спрашивала я, но Веня лишь рассмеялся.

Одним из таких убийственно жарких дней мы растянулись в постели. Веня лежал вдоль кровати, как и подобает нормальному человеку. Я же расположилась поперек, уложив голову на бедра своего благоверного и закинув ноги на стенку. Оба абсолютно голые. Я читала, Веня смотрел какое-то видео.

-- Сестра написала, -- сказал он.

-- И что там?

-- Говорит, бабушка уже обзвонила всех родственников. Рассказывает им, что я с невестой живу.

-- С невестой?

-- Ну!

-- А где она? Ты нас не представишь? -- я улыбнулась, не отрываясь от книги.

***

Знаете, Жорж Санд ежедневно трудилась над своими романами до половины одиннадцатого утра. Она не останавливалась, даже если заканчивала книгу, а лишь откладывала ее в сторонку и бралась за новую. Вот так просто, без набросков, творческих перерывов и традиционных писательских сумасшествий в честь законченного романа. С другой стороны, Хемингуэй писал по пятьсот слов в день. Минимум. Он брался за работу с первыми лучами солнца и, бывало, засиживался до полудня. Буковски тоже здоровски писал, от десяти до тридцати страниц за ночь, которые потом вычитывал в разгар полуденного похмелья. Ну, и, конечно, Стивен Кинг -- гуру писательского тайм-менеджмента. Пишет систематически да при любом удобном случае уже, должно быть, лет шестьдесят.

Подобных историй миллион и я всегда мысленно восхищалась такими людьми. Дело здесь не в организованности и воодушевлении, с которым каждый божий день они начинали творить. Я говорю как раз о том, что в назначенное время все эти авторы могли спокойно закончить работу и вернуться к жизни. Что же касается меня\ldots{}

Это, скорей напоминало распорядок дня Хантера Томпсона. Только на кокаин не всегда хватало, так что его приходилось заменять депрессией. Это, конечно, шутки, но! Томпсон вставал в 3 часа дня и активно начинал закидываться виски, коксом и табаком, изредка прерываясь на кофе, обед и курение травы, «чтобы снять напряжение дня». Затем снова кокаин, Чивас, трава и сигареты, к которым добавились капли кислоты и французского ликера, и вуаля:

\emph{12:00, полночь, Хантер С. Томпсон готов писать.}

Такая же хрень преследовала меня по жизни. Процесс переключения с официально признанной реальности на мою собственную требовал чуть ли не отшельничества, а потому в бытовых условиях работы, проблем и социума занимал немало времени. Но стоит мне переключиться -- к ужину не ждите и к завтраку тоже.

У меня есть отличная для писателя и ужасная для остальных аспектов моей жизни привычка. Я не могу оторваться от писанины даже если знаю, что ужасно опаздываю. Прекрасно вижу, который час, осознаю, что должна была выйти из дома еще двадцать минут назад, но упорно продолжаю сидеть за рукописью потому, что все остальное становится чертовки неважным. И дело здесь не в том, чтобы просто закончить предложение, диалог, или даже целую главу. На самом деле я попросту не могу остановиться, пока чувствую, что мне по-прежнему есть, что сказать.

Порой мне кажется, что ничего не изменится, случись хоть конец света. По сторонам будут раздаваться взрывы, мелькать вспышки и пролетать травмоопасные предметы, а я все так же преспокойненько продолжу сидеть, склонившись над текстом. Такой вот я человек.

Кстати, это касается не только письма. Чтение также частенько берет надо мной верх. С самого детства, о чем бы не оповестили меня родители, я автоматически отвечала:

-- Сейчас. Только главу дочитаю.

И не важно, что до конца этой главы может быть еще страниц пятьдесят.

Скажу больше, однажды я даже проехала нужную остановку. Было это не где-нибудь в городе, а за его чертой, в незыблемой глуши наполовину заброшенных дачных товариществ. Прекрасно помню этот вечер. На улице стояла зима, минусовая температура. Мне было двадцать, и я обещала Мишель приехать последним транспортом.

И вот я сижу у окна катившей прочь из города маршрутки. Вокруг все завалено снегом. На улице собачий холод и транспорт ходит ужасно. Мне повезло, встретить водителя, дом которого находился в нужной стороне, так что я сижу у окна и читаю мемуары вдовы Йена Кёртиса. Кто-то в салоне заказывает мою остановку, и я уже вижу, что до нее остается парочка метров. Маршрутка останавливается, люди начинают выходить.

\emph{Сейчас, только страницу дочитаю}, говорит внутренний голос.

Следующее, что я помню -- напрочь ошарашенный вид подруги, что ждала меня на остановке. Рот открыт, нижняя губа отвисла, глаза широко распахнуты. Ее лицо словно говорило: «Какого хуя?». Я пытаюсь подавить приступ подкатывающего смеха, но Мишель уже исчезает за горами снега, а маршрутное такси увозит меня в направлении загородного дурдома.

***

Спустя два года и одну незабываемую весну Веня задал мне какой-то вопрос. Случилось это как раз, когда я достигла середины свежего кинговского рассказа. И я ответила в лучших традициях Елизаветы Васляевой.

-- Сейчас. Только главу дочитаю.

Мой ответ почему-то развеселил Вениамина. Он заулыбался словно ребенок и принялся покрывать мое лицо поцелуями. До сих пор не знаю, о чем Веня спросил меня тем душным майским днем. Хотя кое-какие подозрения, все-таки, имеются

Наконец, на смену дню пришел вечер, которого -- я уверена -- с нетерпением ждали все четыреста шестнадцать тысяч двести шестьдесят три человека, проживающих в Севастополе весной две тысячи шестнадцатого. Жара отступила.

Вениамин недавно вышел из душа. Он курил, облокотившись на поручни балкона, а я сидела в тени, глядя на то, как его волосы цвета пшеницы, игриво переливаются в закатных лучах. Мы много говорили, рассказывая друг другу о прошлом. Делились душевными переживаниями, открывали свои маленькие тайны и отвечали на неловкие вопросы.

А еще мы много болтали. Щебетали обо всем подряд. Мы прожили вместе всего три дня, но, казалось, были знакомы вечность.

-- Я словно знаю тебя всю жизнь, -- сказал Веня.

Он определенно читал мои мысли.

-- Ты просто перевернула мое представление о женщинах. Я никогда не думал, что могу быть так счастлив. И я так сильно люблю тебя.

-- Ох, Веня\ldots{}

-- А ты со мной счастлива?

И я заплакала. От счастья. Впервые за миллионы световых лет.

***

Секс с Вениамином был потрясающим. Мы занимались им часами, прерываясь лишь на кофе и сигареты. До знакомства с Веней я особо не понимала, отчего вокруг куни создают так много шума. Оказалось, никто из моих бывших любовников просто не знал, как это делается. Но вот наступил тот день, когда миф о существовании оргазма от куннилингуса, наконец, перестал быть мифом.

О своей любви Вениамин говорил невероятно часто. Более того, он даже стал называть меня не иначе как «любовь моя», лишь изредка прерываясь на «деточку» и «любимую». Я же ни разу не сказала Вене о своих чувствах. Их становилось все больше, но у каждого из нас есть свои незавершенные гештальты, так что я просто не могла вымолвить эти три слова. Порой мне даже становилось от этого неловко.

-- Итадакимас! -- радостно крикнул Веня, доедая омлет, а затем добавил более спокойным тоном: -- Было очень вкусно.

-- Делов-то.

Он стал застегивать рубашку.

-- Неля поднесет прайсы к остановке. Заберу их и сразу к тебе, -- Веня уже натягивал джинсы.

В те дни Вениамин работал в одной относительно известной компании. Он был региональным супервайзером. Распространял продукцию для салонов красоты и барбершопов. Вопреки громкому названию, по Крыму у него было только двое подчиненных и большую часть работы Веня по-прежнему делал сам.

Прекрасно помню наш первый телефонный разговор. Он состоялся за несколько недель до встречи. Это был день рождения моего бывшего, и я чувствовала себя крайне паршиво. Узнав об этом, Вениамин решил меня подбодрить. Я была на работе, в «Штиле», и Вене пришлось ждать до трех часов ночи, пока моя смена, наконец, не закончилась, после чего он как ни в чем не бывало принялся рассказывать все, что знал об оттенке моих волос. Монолога вышло минут на тридцать.

Я крепко поцеловала его и чмокнула в щеку. Для уверенности.

-- Я люблю тебя, -- весело попрощался Вениамин.

Не дожидаясь пока входная дверь захлопнется, я проскочила за ним в коридор.

-- Венечка, -- начала я. -- Ты мне безумно нравишься, но ты ведь понимаешь, что сейчас я не могу ответить тебе тем же?

Из-за этого я чувствую себя неловко каждый раз, когда ты говоришь, что любишь меня. Не пойми меня неправильно, но мне очень сложно привыкнуть.

-- Привыкнуть к чему?

-- К тому, что обо мне заботятся. К тому, что я кому-то дорога. К тому, что меня любят, в конце концов.

У Вени зазвонил телефон. Видимо, Неля уже ждала его на остановке. Не обращая на это никакого внимания, Веня сжал мою руку.

-- Я буду ждать, -- пообещал он, глядя в мои глаза.

Ждать пришлось недолго.

\hypertarget{chapter-12}{%
\chapter{~}\label{chapter-12}}

После обеда мы решили отправиться в лес. В распоряжении Вениамина была дряхленькая рабочая «Таврия». Сперва машина меня ужаснула, (не столько потому, что я предпочитаю менее отечественного производителя, а лишь потому, что детство мое прошло в такой вот тарахтайке, а ребенком я и метра не могла проехать без того, чтоб меня не укачало) но вскоре эти поездки стали приносить нам немалую радость. В основном, благодаря реакции окружающих.

Просто представьте эту картину! С одной стороны, Вениамин, сидящий за рулем, в своей несменной пижонской шапочке, с хвостом золотистых волос, одетый в рваные джинсы, вечную клетчатую рубашку и грубые, незашнурованные ботинки; его руки в бесконечных шрамах и браслетах, а пальцы -- в кольцах -- весело стучат по баранке в такт музыке. С другой стороны, я, сидящая на пассажирском, с мастерски уложенной шевелюрой, неизменными алыми губами, в макияже, достойном ковровой дорожки, и с огромным стаканом кофе в руках. Оба с сигаретами. Оба в татуировках. Оба влюблены. Несемся куда-то в разваливающейся на ходу «Таврии» да распеваем странные песни.

\emph{When the truth is found to be lies}
\emph{And all the joy within you does\ldots{}}

Короче говоря, реакция была что надо.

\emph{Don't you want somebody to love?}

-- Давно ты водишь? -- спросила я.

-- Шесть лет, -- ответил Веня, лихо выворачивая баранку.

Прежде чем объединиться с природой, мы остановились на заправке. Прогулялись вдоль рядов с безделушками, пообедали горячими сосисками и пополнили запас кофе. Так начались наши заправочные путешествия. На протяжении последующих месяцев мы с Вениамином побывали на множестве заправок и стали экспертами в вопросах вроде: где выбор шоколада больше, а пенка в капучино выше.

Веня всегда любил эспрессо, максимум -- американо. Обычно вместе с ними давали крохотные шоколадки. Я же с подросткового возраста не признавала кофе без молока, но, в отличие от Вени, обожала шоколад. Порой у меня создавалось впечатление, словно он вовсе не хочет кофе и берет его лишь для того, чтобы порадовать меня очередной миниатюрной сладостью.

Нам нравилось завтракать на заправках. Была в этом какая-то особая романтика.

В итоге, мы отдалились от жилой зоны и теперь ехали вдоль деревьев с широкими кронами, что по обе стороны покрывали обочину. Вскоре мой Шумахер свернул на усыпанную песком тропинку. Вокруг все так и кипело зеленью. Приглушив музыку, мы полностью опустили стекла и какое-то время ехали молча, наслаждаясь умиротворяющей обстановкой.

Разодетые в зелень деревья ласково шептались, периодически касаясь друг друга ветвями. Птицы щебетали невозможно громко. Их голоса перекрывали предзакатные порывы ветра. В воздухе по-прежнему пахло морем, но теперь к этому добавились еще и запахи лесных деревьев. Опьяняющее сочетание.

Придерживая страницы пальцем, я цитировала Вене Керуака:

\emph{Так бывает в лесах, они всегда кажутся знакомыми, давно забытыми, как лицо умершего родственника, как давний сон, как принесенный волнами обрывок позабытой песни, и, больше всего -- как золотые вечности прошедшего детства или прошлой жизни, всего живущего и умирающего, миллион лет назад вот так же щемило сердце, и облака, проплывая над головой, подтверждают это чувство своей одинокой знакомостью.}

-- Красиво, -- заключил Веня.

Временами он бывал молчалив. Даже слишком. Я часами болтала, рассказывая о детстве, книгах, напитках, путешествиях и планах на будущее, а мой спутник лишь внимательно слушал. Однако, чем ближе подбирался вечер, тем оживленнее становился Вениамин. Порой я даже уставала от его внезапной активности.

Спустя несколько минут тропинка стала совсем узкой. Мы вышли из машины. Подкурив, Веня тут же уселся на капот, привычно скрестив ноги. Он смотрел куда-то вдаль и затягивался своей до невозможности крепкой сигаретой. Таким я видела его миллион раз: в парках, дворах, у кофеен, на парковке, около моря и на проспекте. Махатма Ганди двадцать первого века.

Пройдя сквозь сосны, мы вышли на холмистую поляну, со всех сторон окруженную деревьями. Впереди был неглубокий овраг, и мы теперь шли вдоль него, взявшись за руки. Помимо шляпы, на Вене были классические брюки, черная рубашка и серая старомодная жилетка. Картину довершал алый атласный галстук. Неожиданный наряд для вечерней прогулки по лесу.

Закат близился, так что Веня уже увлеченно рассказывал мне что-то о маленьких острых перцах под названием халапеньо. В отличие от меня, он обожал острую еду. И, говоря это, я не имею в виду васаби или зернистую горчицу.

-- Я люблю что-то менее острое, -- ответила я, когда Веня закончил свой монолог.

-- А я -- тебя.

Опять двадцать пять.

Сказав это, он продолжал весело шагать через поляну.

-- Правда?

-- Что?

Я остановилась и взглянула на Вениамина со всей серьезностью, которая только может быть присуща двадцати двух летней девушке.

-- Ты действительно любишь меня?

Он обхватил мое лицо руками и улыбнулся.

-- Ну, конечно, люблю, деточка.

Сложно объяснить, что я почувствовала в тот момент. Кажется, я впервые восприняла его слова серьезно и разрешила себе мысли о мелькающем впереди хэппи-энд. Несмотря на то, что пошел дождь, и на улице заметно похолодало, мне никогда в жизни не было так тепло.

Мы едва успели добежать до машины, когда дождь обратился ливнем.

***

Над городом бушевал настоящий шторм. Небо затянуло широкой серой пеленой, и только вдали виднелся крохотный просвет. Скопившиеся над Севастополем облака сочетали в себе все оттенки серого. Море превратилось в дребезжащее полотно; его синева сделалась почти черной. Одетые лишь наполовину, мы сидели, прислонившись к балконной стене, и наблюдали за разыгравшейся непогодой.

-- Очевидно, этим вечером придется остаться дома, -- заметила я.

Веня ответил не сразу. Глядя в окно, он витал где-то в своих собственных мыслях.

Тем временем я не отрывала глаз от Вениамина. Как и у меня, суммарно у Вени было три законченных татуировки. Не одна из них мне не пришлась по душе, но недавно он сделал четвертую -- античный профиль бородатого мужчины в венке, окутанный листьями. Татуировка мне нравилась. Она занимала целое предплечье, и временами, во время секса, я любила шутить о том, что из кустов за нами подсматривает какой-то мужик.

За окнами стремительно темнело. Дождь сменился градом. На балконе стало слишком шумно и ветрено. Вспомнив о вине, мы поспешили внутрь.

До встречи со мной Вениамин провел в Севастополе всего пару месяцев. Сам он был родом из Керчи и, закончив одни и без того обреченные, по его словам, отношения, захотел переехать. Начальство предложило ему работать в Севастополе, на что Веня согласился без лишних раздумий.

Теперь единственным интернетом, который оказался в распоряжении Вени, был мобильный, но и тот не особо гладко работал из-за расположенных неподалеку вышек. Мне нравилась такая изоляция. Никаких тебе обновлений, соцсетей и сомнительных сообщений от дальних родственников. Никаких сталкеров, хейтеров и прочих онлайн-личностей, о которых я даже не знала, чем заслужила такое внимание.

Вместо этого каждый день, в одно и то же время, мы ходили в какое-нибудь тихое заведение. Я ловила вай-фай и созванивалась с близкими. Правда денег не всегда хватало, так что иногда мы делали вид, что читаем меню. Изучали его так долго, что я могла полноценно поговорить с подругами, а затем громко решали, что пойдем в другое место. Помню, как-то раз в поисках вай-фая мы даже зашли в кинотеатр и притворились, что ждем свой сеанс.

Как я сказала, мне даже нравился подобного род интерактив. Но Веня уже успел истосковаться по прелестям беспроводной сети. В основном потому, что он уже очень давно не смотрел хороших фильмов. Благодаря открытому интернету близлежащего паба мы обзавелись несколькими, и теперь мы погрузились в один из них.

-- Как вино? -- поинтересовался Веня, не успела я сделать первый глоток.

-- Неплохо.

Вино оказалось паршивым и это еще очень мягко сказано, но мы пили его так, словно это было какое-нибудь Inglenook Cabernet Sauvignon. Выдержанно, с наслаждением в глазах. Светский прием продлился ровно пол бокала, после чего мы переглянулись и расхохотались.

-- Дерьмо редкостное, -- сквозь смех произнесла я.

-- За девять лет алкоголизма не встречал такого дерьмового вина! -- радостно согласился Веня.

Вениамину пришло сообщение. Очередной родственник удивлялся тому, что он женится.

Вениамин удивлялся не меньше.

-- Бабушка снова рассказывает, что я уже живу с невестой, -- объяснил он.

Мы опять рассмеялись.

-- Как вам такая новость, Елизавета Ларионова?

-- Блестяще, Вениамин Алексеевич!

К тому моменту фильм уже закончился и мы оба что-то читали.

-- Слушай, -- начал Веня спустя восемь страниц и один бокал. -- Так, а как ты на самом деле на это смотришь?

-- М? -- не отрываясь от книги, ответила я.

-- Выходи за меня, -- сказал Вениамин.

Книгу пришлось отложить.

\hypertarget{chapter-13}{%
\chapter{~}\label{chapter-13}}

-- Пойдем в ЗАГС? -- весело предложил Вениамин.

Я рассмеялась. Он тоже улыбался, но как-то спокойно. Словно и впрямь ожидал от меня ответа.

-- Постой. Ты это, что ли, серьезно?

-- Более чем, -- заверил Веня.

Я опешила настолько, что выпустила из руки бокал. Видимо, Веня был к этому готов потому, что успел его подхватить.

-- Мы знакомы всего три дня, Веня\ldots{}

-- \ldots{} и я за всю жизнь не был так счастлив, как за эти три дня.

Я молчала.

-- Я, наверное, не так начал.

Вениамин встал с кровати, на которой мы сидели в обнимку, и опустился передо мной на колени.

\emph{А, ну это, конечно, меняет дело}, прокомментировал внутренний голос. Тот самый, который всегда говорил с сарказмом.

\emph{Он встал на колени, и теперь это предложение совсем не выглядит странным.}

\emph{ОН ВСТАЛ НА КОЛЕНИ! ОН ВСТАЛ НА КОЛЕНИ!} -- заверещала в моей голове какая-то сумасшедшая.

Вениамин взял мою руку, поднес ее к губам и трепетно поцеловал.

-- Любовь моя, ты станешь моей женой? -- он снял одно из своих колец, готовясь надеть его на мой безымянный палец.

И я согласилась.

Чертов поцелуй руки. Всегда срабатывает.

***

Знаете эту старую песню, где девушка перечисляет вещи, о которых она думает, когда становится грустно? В моей вариации там были бы пёсики, лес, качественно приготовленный кофе, кладбища, бонги, горные просторы, пост-панк и, конечно же, литература. А еще в этот список непременно бы попал секс во время дождя. Ничего так не расслабляет, как монотонный звук барабанящих по подоконнику капель. Кроме, разве что, травки.

Следующая ночь показала, что секс после помолвки нравится мне куда больше. Просто, ввиду очевидных причин, я не имела об этом ни малейшего представления.

Наступившее утро встретило меня неожиданно солнечной погодой. Разбросав волосы на подушке, я сладко потягивалась, когда в спальню вошел Веня. Довольный и окрыленный.

-- Доброе утро, Венечка.

-- Доброе утро, любовь моя.

Вскоре мы переместились на балкон и разделили между собой последнюю сигарету. Погода стояла невообразимая. Кажется, вчерашний шторм открыл этот тонкий портал между весной и летом. Море вновь было тихим, ветер мягким, небо ясным, а я -- счастливой.

-- Как настроение? -- спросила я. -- Готов к рабочему дню?

Вениамин кивнул.

-- А ты готова? Посмотришь, как я работаю.

Я кинула в ответ. Вслед за этим в комнате прозвенел будильник.

-- Мы проснулись раньше будильника? -- удивленно спросила я.

-- Знаешь, сегодня я встал на рассвете. Вышел на балкон и просто наблюдал за тем, как над морем поднимается солнце. Я вижу этот пейзаж каждый день, но до сегодняшнего утра не понимал, насколько он прекрасен. Потом солнце встало. Я обернулся, и увидел, как ты сладко спишь в моей постели. Солнце падало на твои локоны, отчего они становились медными. И ты улыбалась. Даже во сне. Тогда я понял, что это еще прекраснее, чем все рассветы и закаты мира. Хотя они меня никогда не волновали. Я стоял в дверном проеме и наблюдал за тем, как ты спишь, когда услышал в голове собственный голос. Он сказал: «Вот, Веня, там спит твоя жена». После этого я ощутил такое счастье и спокойствие, каких не испытывал ни разу в жизни.

Во-первых, это была самая длинная лирическая реплика из всех, что я когда-либо слышала от Вениамина, а во-вторых\ldots{} Эти простые, но до невозможности греющие душу, слова, накрепко впечатались в мое подсознание. Словно кто-то взял волшебное перо, и написал их прямиком на моем сердце.

***

Наш автобус плавно приближался к противоположной части города. Обед, а по совместительству и час пик. Салон переполнен. Солнце беспощадно слепило пассажиров, что сидели или стояли, или впечатались в спины друг друга. Вениамин занял одиночное место, сразу возле выхода. Я умостилась у него на руках.

Соблюдая лучшие традиции пост-советского пространства, люди ссорились, обсуждали политику, проклинали погоду и выясняли, почем сайра на базаре. Большинство из них сидели, показывая сердитые, неудовлетворенные жизнью лица. Остальная часть пассажиров просто ехала с крайне отсутствующим видом.

Разделив наушники поровну, мы слушали старые рок-н-ролльные песни. И обнимались. Удивленные такой беспечностью, пассажиры то и дело бросали в нашу сторону не доверительные (а порой даже осуждающие) взгляды.

Реакция, конечно, не такая яркая как в случае с «Таврией», но тоже ничего.

-- Мы самые крутые в автобусе, -- шепнула я Вене.

В ответ на это он натянул на нос свои квадратные солнцезащитные очки, и артистично подмигнул мне.

***

Ехали мы долго. Наконец, автобус притормозил на нужной остановке. Еще один спальный район, очень похожий на то место, в котором я выросла: многочисленные бетонные здания без малейшего намека на культурное развитие, куча гастрономов, парикмахерских и парочка супермаркетов. Кое-где, правда, были признаки озелененной части пространства, чуть более воодушевляющие, чем пятнадцатиметровые николаевские скверы.

Веня мотался по салонам, принимал заявки и относил заказы. Раз или два я заходила с ним, но чаще всего ждала на улице со своим привычным парадно-выходным набором: книга, кофе и сигарета. Помню, как я сидела на парапете, углубившись в чтение, и чувствовала тяжесть кольца, появившегося на руке прошлой ночью. Оно было вылито из серебра, широченное и очень тяжелое. Но это была приятная тяжесть.

Так прошло больше пяти часов. Впереди оставался лишь один заказ, покончив с которым, можно было с чистой совестью отправляться на ужин. Видимо, из-за отсутствия машины, сегодня мы оделись попроще: оба в старых джинсах и толстовках. У Вени был черный спортивный рюкзак, который тот повсюду таскал с собой. Была еще огромная палитра -- глянцевая книженция с образцами красок его фирмы. Внутрь та не помещалась, так что Веня любил втискивать ее между спиной и рюкзаком. Я же шла налегке, если не считать дамской сумочки.

Решив, что бетонная зараза нам положительно надоела, Вениамин и я прошли несколько остановок через уютную лесопосадку. Высокие деревья с двух сторон отделяли ее от шума дороги и захудалых двориков, так что можно было представить, что ты находишься в лесу. Кстати говоря, деревьями этими были не какие-нибудь там тополя, а самые настоящие ели и сосны. Землю под ними усыпали сотни шишек и тысячи сухих иголок. Черт его знает где, Вениамин раздобыл здоровенную изогнутую палку, и теперь шел рядом, изображая друида.

-- До KFC парочка кварталов, -- сказал Веня.

Как выяснилось позже, где-то на этом отрезке пути я на него и запала. Окончательно и бесповоротно.

\hypertarget{chapter-14}{%
\chapter{~}\label{chapter-14}}

Севастопольская вариация KFC находилась около большого торгового центра, куда мы с Веней в дальнейшем не раз заглядывали. Это было милое местечко с разноцветными диванами, широкими столиками, которых было от силы штук пять, и огромными окнами, плавно переходящими в стеклянные двери.

Потягивая ледяное пиво, Веня пытался скачать какой-то фильм. Я расположилась справа от него и, закинув ноги на коленки своего спутника, размышляла над чувством внезапно охватившей меня эйфорией. Вернее, более углубленной ее формой. Теперь к списку моих ласкательно-уменьшительных имен добавилась еще и «невестушка». Воодушевленный изменениями в семейном статусе, Вениамин старался не упустить ни одной возможности называть меня так.

Втрескаться, втюриться, втюхаться, потерять голову, влюбиться по уши, отдать сердце, воспылать любовью, быть без ума, души не чаять, сохнуть, запасть и неровно дышать -- все это, в общем-то, лишь малая часть тех эмоций, которые я испытывала. Предмет моих воздыханий сидел напротив. Он уплетал третий бургер подряд, а я потягивала лимонад и думала о том, что это удивительное чувство внутри, видимо, и есть пресловутые бабочки.

Почему это произошло лишь спустя несколько дней? Можно много рассуждать о моем прошлом, разбитом сердце и защитном механизме, с которым явно что-то не то, раз он сдался так быстро, но правда в том, что я не знаю ответа. За что я его полюбила? Без понятия, но мне всегда казалось, что любят не за что-то, а кого-то и одним воодушевляющим весенним днем этим кем-то стал Вениамин Ларионов.

-- Нужно раздобыть тебе кольцо, -- деловито сказал Веня, покончив с бургером.

Я взглянула на свою руку. Нынешнее и правда было великовато. Вене пришлось согнуть его заднюю часть, чтобы кольцо хоть как-то держалось на моем пальце.

-- Да, наверное.

Вскоре к нам присоединилась Неля с ее прайсами. Это была весьма крупная женщина бальзаковского возраста. Она носила броский макияж из девяностых и короткую стрижку со странной треугольной челкой. Еще Неля была громкой и невыносимо много болтала. Ее историям конца и края не было видно, но имелось в них что-то по-домашнему приятное. Мне она сразу понравилась. Неля стала первым человеком, заказавшим при мне семь бургеров и что-то там еще, а затем употребившим все это прежде, чем я допила свой кофе.

За время выполнения заказа, Неля успела рассказать о своем сыне, долгах, кредитной истории, районе, в котором она живет, зарплате, каталогах, маникюре и, почему-то, Екатеринбурге.

-- Так вы двое и правда женитесь? -- внезапно спросила Неля.

За время ее монолога я слегка забыла о том, что умею разговаривать, так что ответила не сразу.

-- Правда, -- сказал Веня.

Он взял меня за руку.

-- Думаю да, -- добавила я.

Неля смотрела на нас с каким-то странным уважением. Затем объявили ее номер, и девушка, хотя, скорее, женщина поспешила забрать свой заказ. Позже она написала Вене о том, что в диком восторге от его выбора. Сказала, что со стороны мы смотримся как современные Курт и Кортни.

\emph{Что ж}, подумала я. \emph{Надеюсь, эти отношения не закончатся самоубийством одного из нас.}

На обратном пути мы заехали в несколько ювелирных. На самом деле я еще не пришла в себя от внезапно охвативших меня чувств, что уж говорить о перспективе внепланового замужества. Удивление прошло не сразу. Мне понадобилось месяца полтора, дабы свыкнуться с мыслью о том, что я чья-то невеста.

-- Какое ты хочешь? -- Вениамин увлеченно доставал кольца с подставок. -- Такое? Нет? Тогда может это? Или нет, подожди, вот это!

Говоря о кольце, мне не хотелось ничего особенного. Простая обручалка, тонкая, выполненная из серебра, меня вполне устраивала. Я никогда не любила золото. Увы, все, что встречалось на нашем пути, непременно содержало какие-то цветастые камни, стразы и бог его знает, что еще. Мы нашли парочку подходящих вариантов, но все кольца были либо малы, либо до безобразия велики. Вениамина это огорчило куда больше чем меня.

-- Оно того не стоит, -- ласково произнесла я, заметив грусть в глазах своего мужчины.

-- Деточка, -- вздохнул он, -- такое чувство, что сегодня все против нас.

Я обняла его за плечи.

-- Вовсе нет. Мы встретились четыре дня назад и уже сутки как помолвлены. По-твоему, все действительно выглядит настолько мрачно?

Веня избавился от сигареты и поцеловал меня.

-- А, знаешь, ты права.

-- К тому же, -- чуть погодя добавила я, -- вдруг это знак?

-- Знак?

-- Ну, да. Может быть, перед принятием самого важного решения в жизни, нам стоит пожить вместе чуть дольше? Ты сам-то не хочешь хорошенько обдумать этот вопрос? -- говоря это, я переживала, что Вениамин вновь затоскует, но он продолжал смотреть на меня спокойным, полным любви взглядом.

-- С ума сошла? Я думал об этом целых три дня! -- наконец, ответил Веня. -- Я вообще никогда прежде так долго ни о чем не думал.

Мы рассмеялись, хотя оба знали, что, в большинстве своем, так оно и было.

\hypertarget{chapter-15}{%
\chapter{~}\label{chapter-15}}

Очередной бар и мы в нем. Сидим, наслаждаясь прохладой кондиционера и потягивая напитки. Это была одна из многочисленных пивных, что разбросаны по всему Севастополю.

Вениамин зачем-то предлагает мне пиво. Я как всегда отказываюсь.

Совершенно не люблю пиво. И это вовсе не значит, что я не пила хорошего пива, как думают большинство тех, кто слышит это заявление. Я была на нескольких пивоварнях и долгое время работала за стойкой. Словом, перепробовала целую кучу сортов. Итог был один: пиво я на дух не переношу. Оно для меня как черный чай без сахара -- слишком горькое и бесполезное, чтобы пить.

Однажды в разгар маниакального эпизода я окажусь в Варшаве, где впервые добровольно закажу себе пива. Проделаю то же самое в Праге и Берлине, а потом еще в Бремене и парочке других немецких городов. Просто так. Потому, что могу. Я буду пить кофе без молока и чай без сахара с абсолютным безразличием, но\ldots{} Случится это спустя годы, а пока мне все еще двадцать два и пиво -- совсем не по душе этому персонажу.

По этой простой причине, тем вечером мне достался вишневый сидр. Вениамин же предпочел классику, и приканчивал уже второй бокал своего хмельного напитка. У нас был целый стол пивных закусок, разнообразию которых позавидовал бы сам Пантагрюэль. По телеку транслировали «Брюса Всемогущего». В детские годы этот фильм был одним из моих любимых фильмов, и это придавало особого уюта нашему вечеру.

-- Я сейчас так счастлив, -- с чувством сказал Вениамин. -- Никогда в жизни не был таким счастливым.

-- Это потому, что я не против каждый день ошиваться в барах? -- улыбаясь, уточнила я.

-- Вовсе нет, глупышка. Хотя, возможно, и не без этого, -- он подмигнул и тут же стал совсем серьезным. -- Я счастлив, потому что все вокруг кажется таким\ldots{} идеальным. Я сижу в пустом баре посреди ночи, смотрю старое кино. Нет никаких конфликтов и мордобоев. Просто тихий, приятный вечер. Рядом сидишь ты, вся такая идеальная. С моим кольцом на пальце. А я смотрю на все это, и не могу поверить в то, что мне и в самом деле так повезло.

Он протянул руку, чтобы погладить проходившего мимо старого кота.

Вениамин любил животных. Думаю, это и была одна из причин, по которым я решилась ему довериться. Он обожал котов, я же души не чаяла в собаках. Сколько себя помню, животные всегда были мне близки. Гораздо ближе людей, так что я никогда бы не связала свою жизнь с человеком, не любившим животных.

Каждый день, гуляя по городу, направляясь на рынок, или попивая пиво в каком-нибудь из севастопольских двориков, Веня останавливался, чтобы потискать очередную животинку. Таких остановок бывало по полсотни в день. Частенько мы покупали еду и отправлялись кормить уличных животных. Это качество я всегда ценила гораздо выше, чем физическую красоту или, скажем, умение управляться с клитором.

-- Расскажи мне о своих бывших, -- попросила я тем тихим вечером в пивной.

Вопреки тому, что большую часть вечера мой мужчина был разговорчив, -- даже болтлив -- он всячески избегал темы своих бывших отношений. В юношеские годы я слыла той еще ревнивицей и непременно оценила бы этот жест, но теперь такой расклад не мог не настораживать. В конце концов, я уже давно осознала, что ревновать к прошлому -- крайне глупо, и даже научилась уважать это самое прошлое. Проблема была в том, что Веня, естественно, не отрицал того, что не раз состоял в серьезных отношениях, но при этом, почему-то, вел себя так, словно этого не было.

-- Ее зовут Римма. Мы жили вместе полтора года. Я ее не любил, -- вот максимум выдаваемой Вениамином информации.

-- Зачем же ты тогда жил с ней так долго?

Обычно в ответ на этот, казалось бы, логичный вопрос, Веня просто пожимал плечами. Один раз, правда, заметил, что ему просто было одиноко. Что ж, на этот раз мне хотелось услышать более информативный ответ.

-- Расскажи мне о своих бывших?

-- Что ты хочешь знать? -- внезапно спросил Веня.

Я настолько изумилась этому вопросу, что как-то сразу и забыла, чего я там хочу знать. Почувствовала себя эдаким растерянным свидетелем Иеговы, которого впервые в жизни пустили на порог. Да так неожиданно, что бедняга и вовсе забыл, о чем положено говорить дальше.

Стоило собраться с мыслями прежде, чем болтливость моего жениха успеет улетучиться.

-- Хочу знать о той, кого любил, конечно же.

-- Ты первая женщина, которую я по-настоящему полюбил, -- просто ответил Веня.

Я не могла оставить это заявление без поцелуя.

-- В таком случае, -- продолжила я, оторвавшись от жениха, -- расскажи о той, о ком думал, что любил. Давай начнем с простого. Как она выглядела?

И он рассказал. Она была заметно старше и теперь преподавала в школе. Позже, взглянув на фото этой женщины, я в который раз убедилась в привычке человеческого мозга идеализировать прошлое. Вениамин описывал ее как натуральную блондинку с зелеными глазами и большой грудью. Со снимка же на меня смотрела более чем просто полная дамочка домашнего образца с какими-то серыми волосами. Она носила жуткую бабушкину одежду и выглядела куда старше своего возраста.

\emph{Баба как баба}, подумала я. \emph{Если уйдет, то точно не к ней.}

Не стану врать, все это как-то плохо вяжется с моей жизненной философией, но увиденное меня порадовало. Я была моложе, стройнее и привлекательней. И из моей постели еще никто не выходил полным сил.

Преимущество над бывшей любовью Вениамина казалось мне очевидным, и я тут же забыла об этой женщине, вычеркнув ее из мысленного списка своих потенциальных раздражителей. Сама мысль о том, что кто-то может изменить мне с ней казалась нелепой.

Паршиво так думать о людях, но мне было едва за двадцать и мне частенько срали в душу, так что чего вы теперь от меня хотите? Было непросто доверять людям, и, если осознание чьей-то серости на моем фоне могло предотвратить приступ ревности, этим стоило воспользоваться.

-- Когда я впервые привел Аллу домой, мать увела ее на балкон и на протяжении часа уговаривала бежать от меня куда подальше.

-- Как так? -- удивилась я, но Веня не ответил.

-- Я даже звал ее замуж, -- признался он. -- Договорился о том, чтобы устроить фаер-шоу около одного обрыва и сделал ей предложение.

С одной стороны, меня это очень удивило. Мои эгоцентричные женские нотки, все же, предпочли бы думать, что я была единственной, на ком Веня захотел жениться. С другой, он сделал мне предложение третьей ночью после нашей встречи. И почему меня должно было это удивить?

\emph{И где мое фаер-шоу?}

-- И что же произошло?

-- Сам не знаю, но вскоре я передумал жениться. Просто выносить ее не мог. Она, знаешь, была совершенно неинтереснейшим человеком и вела себя как моя мама, не как девушка. Понимаешь, о чем я?

-- Понимаю.

-- И что ты понимаешь?

-- Есть три типа отношений. Два из них провальные и твой как раз вписывается в их рамки.

-- А третий? -- спросил Веня.

-- А третий это то, что у нас с тобой, -- ответила я.

***

-- Еще была Римма, -- продолжал Вениамин. -- Она очень много врала. Без особых причин. Рассказывала, что над ней издеваются родители, которые оказались замечательными людьми. Рассказывала, что кто-то пытался изнасиловать ее в автобусе. Рассказывала, что ее постоянно хватают за задницу какие-то незнакомцы и что она все время падает, спотыкается, ударятся и все такое.

-- Сколько ей лет-то было когда вы сошлись?

-- Семнадцать.

-- Красивый пример юношеского максимализма. Кстати говоря, она красивая?

Веня отхлебнул пива.

-- Вообще нет. В очках еще более-менее.

Меня всегда настораживали подобного рода ответы.

-- Так почему ты с ней полтора года прожил?

Он традиционно пожал плечами.

-- Дальше стало хуже. Она начала рассказывать людям о том, что я ее избиваю. Как-то пыталась убедить моих друзей, что была беременна. Узнав об этом, я якобы избил ее прямо посреди улицы.

-- О, у меня тоже был один товарищ с явным дефицитом внимания. И тоже патологический лгун.

-- Тоже беременный?

-- Ну, почти. Умеем мы выбирать\ldots{} -- я подняла бокал. -- За умение выбирать спутников жизни.

-- А то! Даже не помню, когда в последний раз разговаривал вот так вот по душам, -- признался Веня. -- Если вообще разговаривал. Я имею в виду, с девушкой. А что насчет твоих бывших?

-- Ну, мне повезло больше, -- я ухмыльнулась. -- До встречи с тобой я практически три года провела вне отношений.

-- Серьезно?

-- Ага.

-- Почему? -- лицо моего возлюбленного выражало крайнее недоумение.

-- Да все очевидно. Я не могу размениваться на мимолетные отношение. Секс без любви -- это нормально, но вот отношения без любви меня пугают. Так что, да, пришлось немного побыть без них.

-- И каково это?

-- Ты действительно никогда не был один?

Веня отрицательно покачал головой. Он выглядел смущенным.

-- Нет. А как оно?

-- Преимущественно, печально, если поддаваться стереотипам системы, но есть свои плюсы. Например, поиски самости.

-- Какой такой самости?

-- Когда ты вне отношений не нужно циклиться на этих самых отношениях. Подстраиваться под кого-то, ждать одобрения и идти на постоянные компромиссы. Не нужно соответствовать чьей-то картине жизни. У тебя уйма времени, чтобы понять, кто ты, почему ты, и зачем ты. А стоит это осознать, начинаешь, к тому же, задумываться еще и о том, куда ты\ldots{}

Вениамин задумчиво почесал бороду.

-- Я не могу быть один.

-- Тогда ты никогда так и не узнаешь, кто ты.

-- Мне с собой скучно и грустно. Это я знаю наверняка. Терпеть не могу свою компанию, а ты говоришь, что мне это нужно. Вот предположим, гипотетически я и так знаю, что я мудак. Конченный человек. Тогда зачем мне поиски этой твоей самости?

-- Затем, что гипотетически ты такой именно из-за отсутствия поисков себя.

-- Господи, тебя не переубедить. Хорошо, а если я себе не понравлюсь?

-- Это не важно.

-- Почему?

-- Да потому, что к тому моменту ты уже будешь знать, кем ты хочешь быть, кем ты можешь стать, и что для этого нужно делать.

Жених взглянул на меня с недоверием, за которым, тем не менее, виднелись проблески интереса.

-- И что потом, Васляева?

-- А потом просто делать.

-- Что-о-о делать?!

-- Узнаешь, когда найдешь себя.

Веня откинулся на спинку кресла и задумчиво вздохнул.

-- Ты мне сейчас мозг сломаешь, -- признался он. -- Как мне понять, что делать, если я не знаю, кто я? И что значит, зачем я и почему я? Почему что? Нет, я точно не могу быть один. Но теперь мне интересно, что все это значит, деточка? Ты же знаешь, я у тебя любопытный.

Он посмотрел на меня с просьбой во взгляде.

-- Ну, хотя бы объясни, каково это жить без отношений? Тебя послушать, так это чуть ли не обязательно.

-- Вроде того.

-- Потому, что только так я смогу узнать, кто я и\ldots эм\ldots{} почему я?

-- Правильно.

-- Ну, так и что мне теперь делать, если я хочу это узнать, но не хочу и не могу быть один?

-- Да это не так уж и сложно, -- я улыбнулась. -- Попробуй при случае.

\hypertarget{chapter-16}{%
\chapter{~}\label{chapter-16}}

Ближе к полудню я нехотя выбралась из постели. Вене предстояло обойти долгий список салонов, и мой жених уже вовсю носился по квартире. Одной рукой он держал чашку со вчерашним кофе, второй сжимал не глаженный галстук. Вениамин бегал между ванной и балконом. Он что-то рассказывал, периодически размахивая своим алым галстуком. Да так, словно хотел привлечь внимание всех быков полуострова.

-- Вот поэтому я не люблю быть один. Мне с собой скучно, -- разобравшись с гардеробом, подытожил Веня. -- Невыносимо скучно. Как ты меня выносишь?

Он вернул свое внимание галстуку, повертел его в руках и со словами «Совсем мятый!» швырнул обратно в шкаф.

-- Можем идти! -- откуда-то из прихожей провозгласил Вениамин.

И я вдруг поняла, что по-прежнему сижу в кресле в одном белье.

Веня тоже это заметил и, по всей видимости, поспешил исправить ситуацию. Но что-то пошло не так, и вскоре мои трусики уже присоединились к галстуку. Склонившись над креслом, Вениамин опустил ладони на мои плечи. Его пальцы неспешно скользили по коже. Они коснулись шелковых бретелек моей майки. Оправили их, словно невзначай, и продолжили скользить вниз по телу, утягивая за собой остатки одежды.

От этих прикосновений голова шла кругом. При каждом вдохе тело покрывалось сладостным ознобом, и мне подумалось, что завтрак придется пропустить. Затем Вениамин принялся покрывать мою грудь поцелуями и думать мне больше не приходилось.

***

Большая часть салонов из списка находилась в районе Северной и еще парочка -- в центре. Вопреки тому, что это были два разных конца города, находились они напротив. Каждые полчаса к берегу подходил видавший виды паром, перевозивший горожан на соседний берег через Севастопольскую бухту. Назывался он «Адмирал Ларионов», что не могло не вызвать улыбки на моем лице.

И вот мы вновь бродим солнечными улочкам Севастополя. Конечно, Вене нужно было работать, но это никак не нарушало внезапной атмосферы праздника. На самом деле, мне нравилось просто гулять по городу. Плевать, куда ты идешь, когда рядом любимый человек.

-- Такое чувство, что у меня сегодня выходной, -- признался последний.

Покончив с центром, мы сделали короткую остановку в очередной кофейне, решив расположиться снаружи. Заведение было крохотным, его терраса -- еще меньше. Зато цены были довольно высокими, что вполне объясняло отсутствие посетителей. Заказав кофе, мы опустились в плетеные кресла и принялись обсуждать литературу.

Единственным жанром, который нравился Вениамину, было фэнтези. Я же предпочитала классику, философию, постмодернизм, гонзо, старые детективы и Стивена Кинга, чье творчество не поддается классификации. Еще я души не чаяла в поэзии, которую Веня совершенно не понимал. Словом, почва для дискуссии оказалась бесценной. Увлекшись разговором на любимую тему, я даже не заметила, как принесли кофе.

Более того, вскоре я осознала, что мы вдруг начали планировать свадьбу. Причем планировал, по большей части, Вениамин, а я лишь увлеченно слушала и периодически кивала. В отличие от предыдущего разговора, наши мнения касательно свадьбы более чем сходились. За исключением одного момента.

-- Никогда не хотела пышную свадьбу, -- сказала я, когда Веня, наконец, прервался на кофе. -- Я их попросту не понимаю. То есть, даже если опустить финансовый вопрос, на кой черт приглашать людей, которые и дня твоего рождения не вспомнят? Мне всегда казалось, что свадьба -- это очень личное. Если честно, глядя на пиршества своих бывших одноклассниц, у меня создается такое впечатление, словно они делают это вовсе не для себя.

-- Ты права, любовь моя, -- лучшая фраза из тех, что женщина может услышать от своего избранника. -- Я бы пригласил папу, сестру и, возможно, парочку старых приятелей.

Я подняла бровь.

-- Уверен?

-- Конечно, -- заверил меня Веня.

Он всегда как-то по-особенному произносил это слово. С придыханием и каким-то милым воодушевлением.

-- Эх, а я-то хотела предложить тебе взять кредит, который мы не сможем выплатить и спустя двадцать лет после развода\ldots{}

-- А что насчет места? Я бы предпочел природу.

-- Фиолент, -- внезапно для самой себя ответила я. -- И столик в каком-нибудь уютном ресторанчике.

-- Я поражен, - искренне произнес Вениамин. -- Боялся, ты захочешь что-то более публичное.

Обсудив еще несколько деталей, мой благоверный с радостью заявил:

-- Повторюсь, Васляева, ты -- моя идеальная женщина!

Как я уже говорила, в этой жизни мало что могло заставить меня залиться краской. Смущение -- явно не мой конек. Было странно смущаться из-за ласковых слов мужчины, с которым живешь с первого дня знакомства. Но я смутилась.

-- Тогда завтра можем отнести заявление в ЗАГС, -- довольно подытожил Веня.

Тут то и всплыл наш персональный камень преткновения. Учитывая то, что прежде я никогда не мечтала о замужестве, было довольно сложно свыкнуться с мыслями о предстоящей свадьбе. Еще сложнее мне было свыкнуться с мыслями о столь скорой свадьбе.

-- Ты сомневаешься, -- не спросил, я скорее сказал Веня, увидев мое лицо.

Конечно, я сомневалась. Не в своих чувствах к Вениамину и даже не в его чувствах ко мне. Я неподдельно верила в нас. Верила тому, что он любит меня и тому, что он действительно хочет быть со мной. И, все же, хотеть и быть -- немного разные вещи. А еще я не понаслышке знала, какими переменчивыми могут быть мужчины.

Нет, я любила Вениамина всем сердцем. Все, что когда-либо происходило в моей жизни, любые конфликты, трудности и прощания -- все это вдруг стало таким несущественным. Даже не выспавшись, будучи усталой, без стабильного финансового положения, простуженной, страдающей от сезонной аллергии или извечной мигрени, настигшей меня после сотрясения мозга, я постоянно находилась в прекрасном настроении. Пожалуй, так говорят о каждой новой пассии, но я в действительности никогда не чувствовала себя так, как чувствовала рядом с Веней. Я смотрела на него, сидящего в соломенном кресле, с этими волосами цвета золотистой пшеницы и едва различимой за ними одинокой серьгой, чуть более темной бородой, которую не мешало бы подстричь, с многочисленными браслетами, утратившими цвет татуировками, что, то тут, то там, выглядывали из-под футболки, в поношенных джинсах, на карманах которых виднелись контуры ножа и ручки, в дырявых ботинках, с которых все время свисали шнурки, в квадратных солнцезащитных очках, которые шли ему так, как не шли ни одному хипстеру, и с рубашкой, повязанной на бедрах, что так походила на куртку дровосека. Смотрела и понимала, что любовь -- слишком скудное слово, чтобы передать все то, что поселилось в моем сердце благодаря этому мужчине. Я любила его всей душой и тем, что осталось от моего сердца. Любила, и именно поэтому предпочла бы не сковывать себя узами брака. Слишком ценила эти отношения, чтобы добровольно делать столь амбициозные, необдуманные шаги.

-- Просто не хочу все испортить, -- наконец, ответила я. -- К тому же, я с рождения плохо переношу жару, -- и это было абсолютной правдой. -- Не люблю лето и люблю осень.

-- Как ты можешь что-то испортить? Своим согласием ты сделаешь меня самым счастливым мужчиной на свете.

-- Любимый, -- я взяла его за руку, -- ты и в самом деле удумал стать моим мужем?

-- Спрашиваешь!

-- В таком случае нам нужно провести какое-то время вместе. В конце концов, я вроде как не собираюсь выходить замуж дважды.

-- Переезжай ко мне, -- просто сказал Веня.

Я улыбнулась.

-- Но\ldots{} я и так живу с тобой.

-- Нет, осчастливь родителей, набей чемодан всякими своими Ремарками, Бродскими и Паланиками и переезжай насовсем!

Я не знала, что сказать. Замуж выходить -- так я первая, а вот переезд казался мне очень ответственным делом.

-- Надеюсь, ты не будешь тосковать по работе в баре?

-- По шестнадцатичасовому рабочему дню, вечным недостачам, уйме обязанностей, мизерной зарплате и полному отсутствию чаевых? Даже не знаю.

***

-- Может, все-таки, передумаешь? -- не унимался Веня. -- Отнесем заявление завтра, и у тебя еще будет целых два месяца убедиться в том, что я прав. Ты же еще не передумала стать Елизаветой Ларионовой?

Он опустил очки к носу и весело подмигнул мне.

-- Думаешь, два месяца -- достаточный промежуток времени для того, чтобы принять такое решение?

-- Я свое уже принял. Дело за тобой.

-- Что ж, Вениамин Алексеевич, в таком случае, если ты задашь мне этот вопрос через два месяца, мы пойдем в ЗАГС и напишем это чертово заявление.

На протяжении последних минут десяти, позади нас то и дело раздавался шорох, природу которого мне никак не удавалось распознать. Учитывая серьезность нашего разговора, никто и не думал оглядываться. Шорох затих так же неожиданно как появился. Затем в нашу сторону покатилось нечто, плавно тарахтевшее по деревянному полу террасы. Наконец обернувшись, я увидела у своих ног круглый коричневый предмет, впоследствии оказавшийся баночкой краски. Вслед за ней уже мчался средних лет мужчина, одновременно шумно ругаясь и извиняясь на ходу. Был он седовласый, жилистый, не особенно опрятный, с локонами, собранными в пучок, двухнедельной щетиной и весь перепачканный краской. Одним словом, на первый взгляд -- полубомж, который завтра мог бы оказаться в тренде.

Художник расположился за соседним и единственным, помимо нашего, занятым столиком. Там уже находились краски, палитры, мольберт, банка с кистями и прочие неотъемлемые атрибуты его ремесла. Писал мужчина очередные городские пейзажи вроде тенистых аллей и кафешек с цветочными клумбами. Такие обычно продают вдоль набережных. Подобные картины всегда привлекали меня в реальной жизни, но никак не на бумаге. Милые, идеально прорисованные картинки. И все одинаковые, словно их создатели целенаправленно избегали малейших проявлений собственного стиля. Или самих себя.

Да, пейзажи вроде этого меня не интересовали. Но вот сам мужчина, он оказался крайне интересным. Я имею в виду, то, как он нервно вышагивал вокруг мольберта, не прекращая бранить себя. Он буквально ругал себя за все на свете: от плохого цвета, света или тона, и заканчивая тем, что руки дрожат и краски вылетают из них на землю. Вот где и в самом деле экспрессии было побольше, чем во всех его вместе взятых картинах. Все это на миг перенесло меня в прошлое. Я вспомнила Эрика и то, с какой скрупулёзностью он писал музыку.

Вениамину позвонили из офиса, и он уже диктовал кому-то очередную заявку, а я допивала кофе и не сводила взгляд с этого удивительного полубомжа-полухудожника. Смотрела и думала великие думы о том, что порой вокруг готового продукта искусства устраивают слишком много шума. Иногда бывает и так, что этот самый предмет искусства ровным счетом не имеет никакого значения и, по сути, толком не является таковым, тогда как за творцом наблюдать куда приятнее. Выходит, что именно действие или бездействие в попытках создания и является настоящим искусством. Разве это новость? Мировое искусство кишит подобными сюжетами. Взять хотя бы Хантера Томпсона -- уникальный человек и тот еще нарколыга. Однажды, он поехал в Лас-Вегас чтобы освятить предстоящие гонки и случайно написал культовое и наиболее узнаваемое произведение в жанре гонзо. Говорят, что статью он так и не закончил. Есть еще буддисты. Они-то давненько в курсе такого расклада событий. Путь -- и есть счастье, и все такое прочее.

-- Что это за лицо на тебе такое? -- между тем поинтересовался Вениамин.

Я рассказала о том, что думаю. Постаралась, чтобы моя мысль звучала менее пафосно, и самонадеянно заявила, что пора бы ему начинать за мной записывать.

***

Покончив с делами, мы прошли вдоль причала Северной бухты, и уселись на самом краю. День выдался свежим, но тихим. Вокруг не было ни души. Перед нами простиралось море с едва уловимыми лазурными волнами. Они изредка мелькали на поверхности воды, напоминая складки на шелковом одеяле. Словно подобранное в тон морю, небо поражало глаза своей яркостью. Кое-где застыло несколько пушистых белоснежных облаков. На ближайшем берегу виднелись бесконечные обрывы, уйма зелени и парочка построек. Рядом со мной -- облаченный в капюшон и солнцезащитные очки -- сидел Вениамин. Мы курили по второй сигарете подряд. Время поджимало, но никому не хотелось покидать этот открывающий умиротворенную панораму уголок.

Внезапно из-за плеча моего суженого появился его водоплавающий однофамилец.

-- Прокатимся? -- игриво спросил Веня, указывая на медленно движущийся паром.

Я взглянула на эту сомнительную конструкцию, которая, казалось, внезапно для себя самой, вместила на борту добрую десятину города, переправляя ее в Артбухту -- еще одно место, в котором мы частенько гуляли. Кстати, официально, Вениамин сделал мне предложение на набережной Корнилова, что прилегает к этой самой бухте. Правда между неофициальным и официальным предложениями руки и сердца прошло менее десяти часов.

-- Сегодня я прокачусь только на одном Ларионове, и этот парнишка, -- я указала на паром, -- явно не мой тип.

\hypertarget{chapter-17}{%
\chapter{~}\label{chapter-17}}

Живя вместе, мы привыкли пропускать завтрак. Обычно просто просыпались, занимались сексом и выпивали по чашечке кофе. Затем отправлялись в город, чтобы выпить еще по одной. И лишь ближе к вечеру, когда с работой было покончено, и Вениамин с облегчением прятал блокнот для заявок, мы вдруг вспоминали, что неплохо бы было поесть.

Однако, этот день выдался не таким. Хотя бы потому, что сегодня Вене нужно было работать как раз в том районе, где мы жили. Сказать больше, все нужные ему салоны находились в десяти минутах ходьбы от дома. Над городом еще сияло полуденное солнце, когда возвращались домой. Веня купил несколько банок пива и теперь шел с довольным видом, приканчивая одну из них. Я разбавила свой кофе приличной порцией коньяка и поместила все это во флягу, с которой, казалось, меня связывало куда больше, чем с любым другим неодушевленным предметом.

В общем, прихватив с собой алкоголь, любовь и прочие атрибуты простого человеческого счастья, мы расположились в небольшом парке. Солнечные лучи норовили проскользнуть сквозь пышную крону дуба, под которым мы сидели; в ветвях щебетали птицы, яркая трава щекотала лодыжки, а из переносной колонки доносился голос Боба Марли. Здесь не было ни души. Вдали виднелся проспект, вдоль которого то и дело бегали озабоченные работой люди, а мы попивали свои напитки, дышали свежим майским воздухом и радовались тому, что не являемся частью этой суеты.

Одурманивающая атмосфера уединения продержалась около часа. Затем местные мамочки получили сигнал, который был слышен лишь им одним, и как по команде стали появляться со всех сторон парка. Мы ушли, не сговариваясь.

Через несколько остановок находился круглосуточный гастроном, куда мы еженощно захаживали. Сегодня же мы впервые появились там днем. Скорее по привычке и из желания пройтись, ведь в радиусе двадцати метров от сквера было множество работающих магазинов.

Внутри магазина можно было встретить огромного старого пса, одинокого кота и видавшую виды продавщицу. Она являла собой классический пример представительниц торговли девяностых годов: выжженные перекисью волосы, фиолетовые тени, которые тянулись до самых бровей и даже куда-то выше, и полнейшее отсутствие банальных знаний этикета сферы обслуживания. Если вы, как и я, росли в девяностые, один разговор с такой особью способен отворить вам удивительный портал в детство.

Не обращая никакого внимания на грубость продавщицы, Вениамин поглаживал кота. Тот сидел у прилавка и подобно хозяйке заведения излучал полнейшее безразличие ко всему, что происходило вокруг. Мне же всегда доставался пес. Хотя, скорее, я -- ему. Увидев меня, старый дворняга радостно вилял хвостом, и тут же падал на спину, ненавязчиво предлагая мне почесать его живот.

На пути домой нам всегда встречалась уйма уличных животных. И, все же, наиболее оживленное место представлял собой финальный отрезок пути -- широкая каменная лестница. Она имела несколько пролетов. На каждом обитало по семейству котиков. Подойдя ближе, Веня углубил руку в свой бесконечных размеров рюкзак, и достал огромную пачку сухого корма.

-- Зачем\ldots{}

«\ldots так много?» -- собиралась спросить я, но замолчала на полуслове.

Стоило моему мужчине высыпать его на землю, как из близрастущих кустов показалась еще пара кошек. А затем еще пятеро, еще трое и снова двое. В конечном счете, их оказалось не меньше двух десятков, если не считать тех, что к нашему появлению уже были на лестнице.

-- Такие дела, -- сказал Веня, видя мое удивление.

***

Лестничные кошки были третьей к концу остановкой. За ними шли дворовые, откормленные местными старушками животные. Последней станцией был Марсианин -- огромный камышовый кот, которые уже долгие годы жил на чердаке дома. Заходя в подъезд, мы пропускали его в лифт и подвозили на последний этаж. Тогда Марсик неспешно выходил из кабинки, оборачивался, словно хотел поблагодарить нас. Но потом вспоминал, что ему не до этого и поспешно поднимался по ведущим на чердак ступеням.

Словом, каждый совместный выход из дома оказывался небольшой экскурсией по местной фауне. Мне это нравилось.

Как я уже сказала, обычно до обеда мы о еде не вспоминали. Этот лишенный работы день стал исключением.

Чайник закипал, свиные ребрышки жарились, а картошка остывала на плите. Я услышала звук сработавшего фотоаппарата, и в замешательстве обернулась.

-- Елизавета Васляева стоит у плиты, -- прокомментировал Вениамин. -- Однажды репортеры неплохо наградят меня за такое фото.

Я рассмеялась, залившись румянцем.

-- Хорошо всмотрись в эту картину потому, что ты видишь ее в первый и последний раз, -- предупредила я.

И, как всегда, ошибалась.

***

-- Мы прямо-таки семья, -- заметила я, глядя за тем, с каким энтузиазмом Веня поедает приготовленный мною ужин. -- Уплетаем картошечку под покровом ночи и все такое.

Кстати говоря, его порции -- это было что-то с чем-то. Вкратце, она представляет собой дневной рацион питания какой-нибудь многодетной татарской семьи. К своим двадцати двум годам я видала разных людей, но ни одному из них не удавалось есть так много и выглядеть так подтянуто, не посещая при этом тренажерный зал. Веня же буквально ел суп кастрюлями, макароны пачками, а единственной приемлемой яичницей для него была та, в которую входило не менее шести яиц.

-- Я прямо-таки собираюсь на тебе жениться, -- напомнил Веня.

-- Прости. Все время об этом забываю.

И это было правдой. С непривычки я постоянно забывала о том, куда зашли наши отношения. Просто наслаждалась моментом до той самой минуты, которой Вениамин называл меня своей невестушкой. Тогда пространство вокруг меня замирало, и я начинала отчетливо осознавать ситуацию, слегка ее побаиваясь. Но, в основном, на душе становилось невероятно тепло. Я краснела от смущения и подолгу смотрела на Веню влюбленным взглядом.

-- Но ты права. Я тоже что-то почувствовал, -- отозвался Вениамин.

-- Наверное, это потому, что мы впервые едим дома.

И правда, все дело было в ночной картошечке. Ну, и в любви, которая то и дело витала в пропитанном специями воздухе.

-- Ну, ты же моя невестушка, помнишь?

-- Да как тут не помнить?

-- А мне вот, все-таки, кажется ты не всегда об этом помнишь\ldots{}

-- Ладно-ладно, ты меня раскусил, -- я убрала со стола пустые тарелки и улыбнулась своему возлюбленному виноватой улыбкой. -- Я правда не всегда об этом помню.

-- Почему? Я думал, это у меня проблемы с краткосрочной памятью.

-- Не в этом дело, Веня. Я просто никак не могу привыкнуть к мыслям о замужестве. Или о помолвке. Это как если бы ты однажды выкрасил волосы в черный\ldots{}

-- В черный?

Вениамин искренне изумился своему гипотетическому выбору нового цвета волос.

-- Да, именно. Ты бы стал брюнетом, но мысленно еще долго позиционировал себя как блондина. Потому, что всю свою жизнь ты был блондином и никогда не думал о себе, как о брюнете, понимаешь?

-- Допустим.

-- Вот и я никогда не думала о себе, как о жене или невесте.

-- Ну, так скажи это!

-- Что?

-- Я слышал, что в таком случае мысли надо озвучивать, -- объяснил Веня. -- Когда ты не можешь с чем-то свыкнуться, надо это просто произнести.

-- Это еще зачем?

\emph{Нельзя вешать ярлыки. Никакие. Даже если все кажется идеальным. Нет, особенно если все кажется идеальным!}

Я и так знала, зачем. Просто тянула время.

\emph{Как только ты называешь мужчину своим молодым человеком, ты подсознательно начинаешь ждать от него определенную модель поведения. И чем больше его действия походят на те, что ты там себе успела придумать, тем сильнее ты расслабляешься. }

\emph{И тем болезненнее в дальнейшем будет принять суровую реальность. }

Чьи это мысли? Я не могла вспомнить, хотя, казалось, слышала их совсем недавно.

Похоже, что мои.

-- Слова обретают смысл, когда произносишь их вслух, или что-то вроде того. Короче, если что-то кажется тебе нереальным, нужно это просто озвучить. Ты услышишь собственную мысль несколько раз и быстрее поймешь ее смысл. Давай же, попробуй! Скажи, что ты теперь невестушка!

Стоит ли говорить, что этот процесс не обошелся без заминок? Мне понадобилось пять попыток, две сигареты и новый слой укрывшего щеки румянца, чтобы на полном серьезе сказать вслух эти два несложных слова.

-- Я -- невеста, -- медленно произнесла я, едва ли веря собственному голосу.

Вениамин заулыбался от умиления тому, насколько по-детски это прозвучало.

Веня -- что за человек это был? Таинственный как ночь и предсказуемый как утренняя тошнота после знатной попойки. Он часто использовал в речи одни и те же фразы, произнося их с одинаковыми интонациями, так что вскоре предугадать следующее слово сделалось проще простого. Он не любил пиво, но потреблял его с завидной регулярностью, вероятно, бастуя против системы. Любил складные ножи и таскал с собой целую кучу холодного оружия, заботливо раскиданного по карманам джинс. При этом Вениамин имел крайне двойственную репутацию: малознакомые люди (в частности, женщины) его чуть ли не боготворили, тогда как в родном городе моего избранника буквально считали отбросом общества. Он несколько раз попадал в участок, (и каждый раз это происходило из-за неадекватного поведения в нетрезвом состоянии) убегал от приставов, пару раз просыпался рядом с совершенно незнакомыми ему людьми, а однажды пришел домой в одних штанах и понятия не имел, куда делось все остальное, включая документы и телефон. Веня всегда выходил покурить после третьей стопки, обожал коктейли с абсентом и домашние настойки. Не то чтоб его можно было назвать гурманом\ldots{} В отношении алкоголя мой мужчина руководствовался простым правилом: чем крепче -- тем лучше, благодаря чему состоял в черном списке многих питейных заведений разных городов Крыма.

Он носил шелковые рубашки, старомодную шляпу и не менее старомодные жилетки, не вылезая при этом из треснувших по швам ботинок. Последние, кстати, были демисезонными. По какой-то странной причине, Веня чрезмерно долго завязывал шнурки и делал это явно не так, как все нормальные люди. По этой же причине обычно он и вовсе не зашнуровывал свои ботинки. А еще Вениамин практически никогда не давал логически обоснованных ответов.

-- Зачем ты их носишь в такую жару? -- не прекращала спрашивать я, указывая на уже давно отжившие свое ботинки.

-- Они водонепроницаемые, -- отвечал Веня так, словно это многое объясняло.

Он обожал все острое и закидывался халапеньо как семечками, редко вылезал из любимой одежды и еще реже соглашался залезть под душ. Я никогда с точностью не могла понять, умный он или глупый. С одной стороны, Веня довольно много знал. Он мог рассказать мне кучу вещей о любой сфере деятельности. С другой -- мой жених имел крайне ограниченный круг интересов, сводящийся к алкоголю, фильмам, работе, котикам и мне. Правда, слегка в другом порядке.

Как и все мужчины, Вениамин был до невозможности противоречив. Забавней всего было то, что утром он мог выразить мне какую-то мысль, а к вечеру начинал активно ее оспаривать. Да еще и с таким рвением, словно и впрямь забывал, что мысль-то -- его. Видимо, противоречивость -- это обязательное качество каждого представителя сильного пола. Что-то вроде критических дней, которые нахрен никому не упали, но и без них тоже никуда. Только для мальчиков. Я знаю очень многих мужчин. Я с ними работала, дружила, встречалась, жила, имела родственные связи и все такое прочее. Но я не знаю ни одного мужчины, который бы, в конечном счете, не оказался противоречивым. Порой они сами себя так называют, а на следующий день отрицают это до последнего.

Наверное, в этом и есть смысл.

\hypertarget{chapter-18}{%
\chapter{~}\label{chapter-18}}

Есть у меня давняя подруга -- Марта Ульянова. Она познакомилась со своим мужем за год до нашей встречи с Вениамином. Хотя и повстречались они во Львове, Семен был родом из Беларуси. Он никогда толком путешествовал, имел суицидально-агрессивные порывы на почве алкоголизма, отсутствие жизненных перспектив, послужной список принятых внутривенно веществ и не менее послужной список вариаций на тему самоубийства. У него не было хоть какого-нибудь опыта работы, образования, выдержки и одного глаза. При всем при этом, человеком Семен оказался, в общем-то, неплохим. Позже мы с ним даже подружились.

Так вот, рассказав мне о своих первых внеплановых отношениях, (которые к тому же были единственными) Марта ожидала услышать в моем голосе нотки разочарования.

-- В глазах общества у меня самый хреновый мужик на свете, -- заявила она.

И вышла замуж, когда они были знакомы всего два месяца. Процесс исцеления был долгим, не спорю, но по истечении нескольких лет Семен избавился от каждой из перечисленных выше характеристик. Ну, кроме отсутствия глаза.

В общем, услышав о моем внезапном, но очень серьезном (что делало его еще более внезапным) романе с Вениамином, первым делом Марта спросила:

-- Вы принимали вместе ванну?

То есть, я рассказала ей о том, что Веня предложил мне руку и сердце и о том, что мне, по всей видимости, придется переехать в Крым. А ей были интересны совместные купания. Но я не слишком-то удивилась. Марта -- она такой человек, который оценивает любые ситуации, основываясь исключительно на своих действиях, вкусах, планах и воспоминаниях. Все дело в том, что одним из первых счастливых воспоминаний Марты и ее мужа стало принятие ванны.

Чтоб вы понимали, природа наградила Марту быстрым умом, способным усваивать, анализировать и вырабатывать тонны информации. Она была экспертом по таймменеджменту и гордостью всех учебных заведений, в которых когда-либо училась. Все это выработало в ней комплекс отличницы, из-за чего Марта отказывалась собирать волосы в косу, даже когда те ей мешали. А том, что она втихоря носит очки, я вообще узнала спустя десять лет после нашего знакомства.

Мы ходили в одну школу, имели общие интересы и общих друзей, но никогда толком не общались. Я постоянно слышала ее фамилию в списках претендентов на получение разного рода грамот и дипломов. Марта же знала меня визуально.

\emph{Такую девушку-вечеринку сложно не заметить}, позже сказала она.

Думаю, это продолжалось бы и дальше, если бы не события одного осеннего дня две тысячи шестого года.

В школе я всегда была чем-то средним между отличницей и троечницей. Как это вообще возможно? Секрет прост: я принципиально занималась лишь теми предметами, которые мне нравились и никогда ни у кого не списывала. Нравились мне языки и литература, искусство и многое другое. Зато математика во всех ее проявлениях стала мне более чем ненавистной. Это случилось благодаря экспериментальному классу, в который меня отдали в возрасте шести лет. Из-за этого последующие два года я имела по семь-восемь уроков математики каждый божий день, включая субботу. Таким образом, сменив школу, я читала книги на алгебре и писала собственные рассказы на геометрии.

Можно сказать, что эксперимент не удался. Если его целью не было пробуждение внутреннего гуманитария.

Так вот, по окончанию средней школы, мне нужно было написать несколько тестов по литературе и дополнить все это дело своим эссе. Это вроде как открывало передо мной перспективы бесплатного обучения в одном неплохом заведении, которое должно было дать мне корочку для поступления в ВУЗ -- так считала моя учительница литературы. И после долгих уговоров ей удалось убедить меня написать мой первый текст на заказ.

Каково же было мое удивление, когда я увидела финальные списки претендентов. Там было сказано, что помимо меня чести удостоилась еще и некая Марта из параллельного класса. Но место было всего одно, и я никак не ожидала встретить конкуренцию на последнем этапе отбора.

\emph{Нужно выяснить, что это за сучка}, подумала я.

Так я познакомилась со своей лучшей подругой, Мартой.

***

Близкое общение началось спустя четыре года, в институте, когда мне исполнилось шестнадцать. Тогда мы обе учились на факультете иностранной филологии. В те дни у Марты не было ни пирсинга, ни татуировок, ни черных волос, ни каре, ни, кстати говоря, сисек. Все это появилось спустя лет пять. Говоря же о наших студенческих деньках, Марта была очень стройной и носила длинные рыжие волосы. Невероятно красивые.

Расцвет этой дружбы как раз припал на пик моей необъяснимой тоски по шестидесятым. И глядя на то, как Марта -- в ее свободных, извечно цветастых одеждах -- дополняет что-то на лекциях и без конца тянет руку на семинарах, меня то и уносили флешбеки несуществующего прошлого. Какие-то отрывки воспоминаний о временах без Интернета, днях и ночах, которыми люди живут в настоящем и полностью поглощены этим процессом.

Танцы, палатки под звездным небом, песни у костра и очень много дороги -- все это молниеносными волнами пролетало сквозь меня пока я в очередной раз наблюдала за тем, как подруга рассказывает что-то скучающей аудитории. Погрузиться в картинку полноценно мне не удавалось, (или тогда я просто еще не умела этого делать) и я просто восторженно наблюдала за этими едва уловимыми всплесками сознания, стараясь хоть что-то сохранить в памяти.

Лишь один эпизод мне удалось запомнить более-менее детально. Дело было в каком-то лесу. Нас окружали высокие хвойные деревья, чьи верхушки, должно быть, касались неба. Где-то за ними скрывалось солнце, и мы долго шли по хрустящим веткам пока, наконец, не добрались до какого-то холма. Сперва до моих ушей еще доносился гул дороги, но вскоре он исчез. Здесь было пленительно спокойно. Даже птицы молчали. Тишину нарушало лишь парящее в воздухе пение.

\emph{Here comes the sun}

Я прекрасно знала слова потому, что сама их пела.

\emph{Here comes the sun}

\emph{And I say, it's all right}

На возвышении лежало старое дерево, а сквозь его ствол струились солнечные лучи, яркими стрелами пронзая завесу лесного дыма. Марта стремительно побежала вверх по склону.

-- Там внизу поляна! -- крикнула она.

Цветастые юбки мелькнули над упавшим деревом, и подруга исчезла из моего поля зрения.

Я поспешила за ней, но, поднявшись на холм, решила задержаться. Уселась на ствол и принялась внимательно разглядывать открывшийся отсюда вид: глухую лесную стену, ярко освещенную поляну и Марту, с широко разведенными руками кружащуюся на зеленой-зеленой траве. Рыжина ее волос подыгрывала в такт солнцу, но что-то в этой идеальной картинке показалось мне странным.

Ее волосы! Они были раза в три короче.

-- Когда ты подстригла волосы? -- спросила я на спуске с холма.

Должно быть, я выглядела странно. Подруга взглянула на меня как на лунатика и продолжила танцевать. Тогда я поймала ее руку и повторила свой вопрос.

-- О чем ты говоришь? -- с улыбкой спросила Марта. -- Они только-только отрасли. На лучше курни. Видок у тебя странный!

Она протянула мне трубку, но курнуть я не решилась. Голова раскалывалась, лесная картинка шла кругом, а девушка передо мной раздвоилась: одна Марта, с волосами ниже плеч, по-прежнему стояла посреди поляны, вторая, с волосами до талии, взялась непойми откуда. Обе подпирали руками бока и смотрели на меня с неприкрытой тревогой.

-- Лиза, с тобой все в порядке? -- спросила лесная Марта.

-- Лиза, с тобой все в порядке? -- спросила Марта с волосами до талии.

Голос последней звучал издали. Игнорируя головную боль, я сосредоточилась на второй Марте, и вскоре обнаружила, что та находится в пустой аудитории. Позади девушки виднелся пустой экран проектора, а передо мной лежала открытая тетрадь, но вместо конспекта под сегодняшней датой 26.09.2010 виднелась надпись:

Hic Sol Venit

Рядом был набросок крохотной елки.

-- Лиза?

Происходящее обескураживало, и мне понадобилось какое-то время, чтобы вспомнить, что Лиза -- это я.

-- Лиза, ты на связи? Прием.

-- Прием-прием, -- наконец, ответила я, и поднялась со своего места. -- Пара давно закончилась?

-- Минут пять назад. У тебя все в порядке?

-- Да, просто уснула.

-- Но я видела, ты сидела с открытыми глазами!

Я отмахнулась.

-- Мне такое уже говорили. Иногда я сплю с открытыми глазами, когда сильно устаю.

-- Ничего себе вот это у человека бессонница! -- восхитилась Марта.

Она продолжила укладывать канцелярию в сумку.

-- А что за пара хоть была? -- как бы невзначай поинтересовалась я прежде, чем последовать ее примеру.

-- Латынь, а что?

-- Да ниче. Одолжишь конспект на вечерок?

***

Марте о своих флешбеках я, конечно же, ничего не говорила. Пару раз та ловила мой взгляд и, видимо, решала, что я в восторге от ее знаний, потому как начинала говорить еще громче и серьезней.

Мне было все равно, что этих воспоминаний никогда не существовало в нашей Вселенной. Просто не могло существовать, ведь я родилась слишком поздно. Да и запомнить большинство таких флешбеков не делалось возможным, но их атмосфера -- ощущение покоя и счастья нахождения на своем месте -- никак не хотела забываться.

\emph{Я так когда-нибудь суициднусь}, подумалось мне после очередного припадка. \emph{Очнусь, обнаружу, что все это неправда и суициднусь. Точно-точно.}

И я решила, что однажды я обязательно верну шестидесятые. Ну, если не суициднусь.

***

Как-то в конце сентября мы возвращались домой. На остановку с нами шла еще одна девочка -- Алена. Это была школьная подруга Марты.

-- По ночам я тихонько подкрадываюсь к своему братику, -- говорила Алена, -- и дырявлю ему палец своим острым ногтем. Потом присасываюсь и пью кровь пока родители не придут с работы\ldots{}

Она еще долго рассказывала истории вроде этой, а я разрывалась между желанием включить диктофон и порывом орнуть в голосину. Ведомая этими эквивалентными чувствами, я посмотрела на Марту. Как и всегда, вид у нее был очень серьезный.

На протяжении всего пути Алена продолжала рассказывать свои неадекватные истории. Затем, наконец, увидела свою маршрутку и помчалась в ее сторону.

-- Как же мне осточертели эти люди.

Я едва сдерживала улыбку.

-- Мне тоже, -- поддержала меня Марта.

-- У тебя есть друзья?

Она призадумалась.

-- Нет. Думаю, нет.

-- Почему?

Марта пожала плечами, а затем указала на отдаляющуюся от нас Алену.

-- Вот, наверное, поэтому.

-- Ну, так давай дружить! -- я протянула ей руку.

-- А давай!

И мы действительно стали друзьями. Вот так вот просто, по обоюдному согласию. Как в каком-нибудь старом фильме. И вот, спустя шесть лет я позвонила в Волковыск, рассказала Марте о своем женихе, а та спросила:

-- Вы принимали вместе ванну?

К тому моменту она уже успела посетить уйму стран, переехать в Беларусь, обзавестись мужем, татуировками и обширным опытом принятия различных веществ. Кое в чем я тоже не отставала.

-- Так точно! -- отчиталась я.

В первые дни знакомства мы действительно купались вместе. Ванна была маленькой, а вот я -- не очень. Вениамин же был стройным, но высоким. Каким-то чудом мы, все-таки, поместились в крохотную ванную, да так, что в ней даже оставалось место.

-- Ты так приятно пахнешь, -- произнес мой мужчина, обнимая меня сзади.

-- Жаль, не могу сказать того же.

И мы рассмеялись.

Вся ванна была наполнена пеной. Кран оставался открытым, и скорее пена начала вываливаться за пределы ванны. Она напоминала огромные хлопья снега, которые как по волшебству перенеслись в этот жаркий предлетний день. У меня есть фотография этого феерического момента. К счастью, ее мало кто видел, но должна признать: я никогда прежде не была замечена с такой детской улыбкой.

Покончив с водными процедурами, мы принялись плескаться водой, бросая друг в друга остатки пены. Единственной вещью, которую Вениамин никогда не снимал, были деревянные бусы, купленные им у одного этнического старца, ежедневно расхаживающего по набережной Севастополя. Мой возлюбленный бесился с такой силой, что эти бусы то и дело подлетали, цепляясь за его нос. Низ его лица превратился в сплошную пену, благодаря чему Веня напоминал мальчишку, который измазался мылом, пытаясь сделать себе искусственную бороду. Думается, сейчас, до кончика волос измазанный пеной для ванны и улыбаясь во все зубы, он и был мальчишкой.

-- Странно\ldots{} -- начала я.

Мы уже выпустили воду из ванной. Веня смывал бальзам с моих волос. Я сидела под стенкой, обхватив руками колени.

-- Что?

-- Такое непривычное чувство.

-- Что за чувство?

-- Кажется, я счастлива. Даже припомнить не могу, когда я в последний раз была счастливой и чувствовала себя так\ldots{} -- я замолчала, пытаясь подобрать верное слово.

Перед глазами всплывали картины раннего детства.

-- Как?

-- Так спокойно.

Не сводя взгляда друг с друга, мы на какое-то время забыли о разговорах, наслаждаясь долгожданным спокойствием.

-- Ну, а что насчет тебя? -- наконец, спросила я. -- Ты-то когда в последний раз чувствовал себя счастливым.

-- До встречи с тобой?

Я кивнула.

-- Почти уверен, что никогда.

\hypertarget{chapter-19}{%
\chapter{~}\label{chapter-19}}

Однажды мы сидели в парке, попивали холодный кофе и разговаривая по душам. По сравнению с остальными днями, жара стояла умеренная. Вениамин растянутся поперек скамейки. Я же лежала, умостив голову ему на колени.

Вскоре к нам подбежала толпа местных бойскаутов. Во главе у них была стеснительная женщина с огромным фотоаппаратом.

-- Мы ищем парня с большой бородой, -- запыхавшись, произнес один из детей.

-- Кто ж не ищет? -- с улыбкой ответила я.

-- Правильно говорить `длинной', -- поправила его женщина.

Веня дружелюбно развел руками, как бы говоря: «Вот он я!».

Ко мне обратился другой ребёнок.

-- Вы можете заплести ему косу? -- увидев замешательство на моем лице, мальчик добавил: -- Пожалуйста! Мы очень отстали! До конца квеста аж целых два задания!

Я поднялась и взглянула на своего суженого.

-- Ты не против?

-- Вообще нет.

Веня снял шляпу, и я уже запустила пальцы в его спутавшиеся волосы, когда вновь заговорил первый ребёнок.

-- Вы не поняли. Коса нужна из большой бороды.

-- Длинной, -- снова напомнила женщина.

-- Ага. Длинной.

Дети в ожидании смотрели на нас. Я с трудом сдерживала желание расхохотаться, завидев выражение лица Вениамина. Какое-то время он пытался отказаться, но вскоре детишки его уговорили.

Я засучила рукава и принялась плести косу.

Из толпы вышла девочка. Она восторженно смотрела на мою левую руку.

-- Нам ещё нужна девушка с большой татуировкой!

-- Правильно говорить\ldots{} -- начала женщина, но, в итоге, подвисла.

Вот это детишкам повезло. Они сфотографировались с нами и умчались в сторону фонтана, переполненные той особенной, искренней радостью к ничего не значащим вещам, которую мы можем испытывать лишь в детстве.

***

-- Ты никогда не рассказывал мне о своих родителях, -- вдруг поняла я, глядя за тем, как бойскауты покидают парк.

-- У меня не очень благополучная семья.

-- Не хочешь говорить об этом? Я не обижусь.

-- А вот и нет. Сегодня мне впервые в жизни хочется об этом поговорить, -- внезапно ответил Веня.

И он начал свой долгий рассказ. Как оказалось, отец Вениамина на протяжении многих лет изменял своей жене. Он был не прочь выпить и, кажется, даже пару раз поднимал на нее руку. При этом Веня продолжал считать отца своим лучшим другим. Он говорил о нем с уважением и каким-то странным для меня благоговением. С другой стороны, мать Вениамина была отнюдь не подарком. Со слов мужа, за ней имелся какой-то пунктик помешательства на деньгах, что очень напрягало всех членов семьи. Со временем это переросло в настоящую манию. Она принялась скупать кучу золота и укладывать его в большую хрустальную чашу. К тому же, она постоянно выдумывала какие-то гадости, (сначала о своём муже, а затем и о сыне) которые затем рассказывала большому количеству посторонних людей.

-- Было много неприятных моментов, -- грустно произнес Веня.

-- Например, каких?

-- Ну, например, в детстве мне приходилось самому готовить. Когда мне было лет десять, мама вдруг стала готовить отдельно для себя и моей сестры. Просто ставила перед ней еду, а остальное уносила к себе.

Я в ужасе посмотрела на жениха.

-- И это далеко не самое худшее, -- заверил он.

Однажды отец Вени -- Алексей Вениаминович -- устал от такой обстановки, и объявил жене, что хочет развода. В ответ та схватила дочь, все деньги и драгоценности, которые смогла найти, и испарилась в неизвестном направлении.

-- Она меня бросила. За все эти годы ни разу не приехала ко мне. Отцовская любовница и то проявила ко мне больше заботы.

Его голос звучал отрешенно. Так, словно Вениамин рассказывал мне услышанную где-то историю. То, что произошло с кем угодно, только не с ним самим.

-- Ты до сих пор с ней не общаешься? -- спросила я.

-- Она вернулась год назад. Сказала, что снова будет жить в нашей квартире потому, что закончились деньги. А отец переехал на дачу.

Затем Веня рассказал ещё море не менее ужасных историй, после чего меня нежданно-негаданно стало одолевать желание встречать его вкусными ужинами, будить пахнущими завтраками и все такое прочее. В общем, мне внезапно захотелось стать примерной женой образца пятидесятых и потратить каждую минуту своей жизни исключительно на заботу об этом мужчине.

Слава богу, это был всего лишь мимолётный порыв.

-- Она отвратительная женщина.

-- Ты ее простил?

-- Мне абсолютно наплевать. Уже давно, -- отчеканил Веня.

Я ему не поверила.

\hypertarget{chapter-20}{%
\chapter{~}\label{chapter-20}}

Весна стремительно близилась к своему завершению. С каждым мгновением солнечные лучи делались все настойчивей, асфальт плавился, а воздух становился горячее. Тем не менее, мы все так же проводили дни за долгими разговорами, распитием спиртного и прогулками через весь город. Еще мы подолгу валялись в постели, предаваясь любви друг к другу. Откровенно говоря, секса было так много, что по приезду домой я увидела, что моя талия заметно уменьшилась в размерах.

Но и до моего отъезда успело произойти еще немало чудесных вещей.

В конце рабочей недели мы решили прогуляться по слегка необычному маршруту. Напротив остановки около нашего дома находился супермаркет, где мы часто делали покупки, по-детски радуясь этому семейному процессу. Помню, первым что мы купили, стала пара бокалов и формочки для льда -- ну, кто бы сомневался. По традиции, около входа в магазин находилась крохотная кофейня с хорошим кофе и плохой баристой. Приходя туда, я каждый раз взбивала молоко вместо нее. С правой стороны от маркета виднелся ряд магазинов, плавно перетекающий в толпу баров и кафешек.

Так вот, пройдя несколько метров между этими строениями, ты попадаешь на идеально прямую дорогу. С краю стоит заправочная станция, а за ней находится просторный путь, уводящий в соседний район. Первую половину дороги понемногу спускаешься вниз. Затем спуск становится резче, а на пути встречается старый пешеходный мост, перекинутый через некогда существующую реку, на месте которой теперь нет ничего, кроме пустоши. Второй отрезок пути -- зеркальное отражение своего предшественника. Сначала тебя ждет резкий, в после -- более плавный подъем. Никаких развилок и поворотов. Таким образом, маршрут представляет собой букву С, лежащую на спине.

Еще стоя в самом начале пути, мы прекрасно видели все, что находилось на другом конце. В основном это были высотки вперемешку с привычными двориками, полными растительности, лавочек и качелей.

Такой вот маршрут мы выбрали. Вернее, нам пришлось это сделать потому, что на другом конце Веню ждала клиентка, припозднившаяся на склад. Она должна была отдать деньги за купленные в начале недели краски для волос, или что-то вроде того. Сумма была неплохой, Вене причиталось получить свои пятнадцать процентов.

Мы вышли вечером, около половины восьмого. На удивление, нужная нам улица оказалась пустынной. За все время пути нам не встретилось ни души.

-- Какой он, твой родной город? -- спросила я, подкуривая первую за вечер сигарету.

-- Красивый и маленький, -- после небольших раздумий ответил Вениамин. -- Прямо как мой член.

И пускай мы оба знали о том, что его агрегат нельзя назвать маленьким, Веня никак не мог прекратить шутить на эту тему.

-- И что в нем красивого? Я имею в виду\ldots{}

Он не дал мне закончить.

-- Не знаю, но женщинам нравится.

-- \ldots Керчь.

Наконец-таки осознав, что в двадцать пятый раз шутки про пенис уже не кажутся смешными, Вениамин ответил:

-- Поехали, сама увидишь.

\emph{О, мой многословный мужчина!}

-- Когда?

Я обожала поездки всех видов и направлений. Частенько ездила куда-то в моменты печали, творческого упадка или душевных терзаний. Бывало, даже не знала конечного пункта назначения, но все равно ехала потому, что в итоге всегда чувствовала себя лучше. Как ни крути, а в дороге одиночество ощущается куда меньше. Наверное, дело здесь в том, что любое движение подразумевает наличие цели. Отправляясь в путь, ты временно обеспечиваешь себя каким-никаким, а, все-таки, смыслом жизни. Это иллюзорное чувство спокойствия не раз позволяло мне отвлечься от тоски и сосредоточиться на решении других, более материальных проблем.

В общем, к тому времени я уже была специалистом по всякого рода одиноким поездкам. Думаю, моих знаний вполне хватило бы на то, чтоб распределить их на категории и создать собственную программу семинаров. Что-то вроде «Куда поехать, если с вами расстались в скайпе?», «Как преодолеть потерю работы с помощью билета в один конец?» и так далее и тому подобное.

Чего я, собственно, не знала, так это, каково это -- путешествовать с любимым человеком. Несмотря на наличие бывших возлюбленных, с этим пунктом у меня как-то не сложилось. Но мне всегда представлялось, что совместные поездки -- это что-то удивительное. Пожалуй, именно поэтому я так быстро соглашалась со всеми бредовыми идеями жениха.

Мои размышления прервал голос Вени.

-- Поехали в следующем месяце? Мне как раз нужно будет съездить в Керчь.

Я вздохнула.

-- Ты же знаешь, что к тому моменту я уже уеду.

-- А ты возвращайся.

Он сказал это спокойно и ласково. Так, словно мы уже миллион лет были в месте. Словно не боялся моего отказа и точно знал, что все у нас будет хорошо. Вениамин вел себя так на протяжении всего времени, что мы были знакомы. Спокойствие и уверенность, безусловно, входили в список вещей, за которые я его обожала. Если я ломала ноготь, слышала какую-то грубость, не находила своего размера в магазине одежды, получала очередной отказ от издательства, встречала резкую критику или попадала в любую другую неприятную ситуацию, стоило мне посмотреть на Вениамина, и раздражение проходило. Казалось, ничто не могло заставить его унывать.

Вот и сейчас, глядя на него, я не могла спрятать улыбку.

-- Я бы взглянула на Керчь.

-- Познакомлю тебя с папой, покажу каменоломни и свой любимый бар.

-- Куда же без этого.

Он развел руками.

-- И на пляж сходим?

-- И на пляж сходим.

-- Есть одно но\ldots{} -- сказал Веня.

Видно было, что жениху неловко, и он не особо хочет заканчивать эту фразу.

-- Ну что там еще такое?

-- В Керчи обо мне ходит не самая лучшая слава.

Как будто я этого не знала.

-- Ты хочешь сказать, на меня все будут смотреть с сожалением?

-- Ну, вроде того.

-- А потом отводить в сторонку и уговаривать бежать от тебя, куда глаза глядят?

Он улыбнулся.

-- Именно так все и будет.

Я поцеловала своего мужчину.

Веня был значительно выше меня. Закинув руку мне на плечо и обняв за шею, он спокойно мог продолжить свой путь, не чувствуя при этом дискомфорта. Так мы и шли, приближаясь к пешеходному мостику.

-- Пускай болтают. А я буду улыбаться и потягивать винишко, -- ответила я. -- Расскажу им о нашем медовом месяце в Шотландии и о том, что мы сделаем одинаковые футболки со своим фото.

-- И о двух прелестных детишках, -- подхватил Веня. -- Белокурой девочке и кареглазом мальчике.

-- Скажу, что в свои девять Йозеф в идеале знает четыре языка. А потом извинюсь, согнусь в реверансе и объявлю, что мне пора идти. Забрать Аглаю Вениаминовну из музыкальной школы, которую она заканчивает в следующем году.

-- Она играет на рояле, флейте и кларнете.

-- Одновременно.

После этой реплики мы разразились звонким смехом.

-- Аглая Вениаминовна\ldots{} -- сквозь смех произнес Веня. -- Где ты вообще это взяла?

-- Подсмотрела у Достоевского.

Мы шагнули на мост, по-прежнему оставаясь единственными пешеходами во всей округе. С правой стороны уже запоздало садилось солнце. Его оранжевые лучи заливали отрезок, по которому много лет назад протекала то ли река, то ли ручей. Купающееся в золоте, это место больше не выглядело таким забытым. Мне даже стало немного жаль того, что солнце скоро сядет, и местность у моих ног вновь обратится никому не нужной пустошью. Так ведь и с людьми бывает.

Я озвучила свою мысль. Веня ее одобрил.

-- Запиши это как-нибудь, -- сказал он.

Я так и сделала. Пускай и с заметной задержкой.

***

Обсуждая призрачное совместное будущее, мы, сами того не замечая, подошли к высоткам. Вениамин отыскал нужный дом и отзвонился клиентке. Та сказала, что сейчас подойдет, и жених исчез в уютной мгле дворика, последовавшей за закатом.

Я осталась ждать у дороги, разглядывая местность. Вокруг были невысокие кустарники, детские площадки, магазин, аптека, неработающий сигаретный киоск и ряд придорожных фонарей, которые как раз начинали загораться. В какой бы город я не приезжала, все спальные районы бывших советских стран всегда выглядели одинаково. При этом, несмотря на серую монотонность, было в них что-то приятно печальное. Как воспоминания о детских днях, которым никогда не суждено вернуться.

Несколько минут прошли в полнейшем одиночестве. Я все так же стояла, думая свои прозаические думы и всматриваясь в темноту двора, когда краем глаза заметила какое-то движение. Откуда-то из-за угла появилась женщина лет на пятнадцать старше меня. На ней была пижама и что-то очень странное, при ближайшем рассмотрении оказавшееся термобигудями в виде плюшевых пенисов. Я прямо-таки обалдела.

-- Вы -- Вениамин? -- громко спросила женщина.

Я прямо-таки обалдела во второй раз, и даже обернулась по сторонам, дабы убедиться, что позади меня не стоит Веня. Само собой, никакого Вени около меня не наблюдалось. Должна признать, на протяжении жизни я встречала разных людей (по большей части, не очень хороших) и как они только меня не называли! Но Вениамином -- никогда.

Даже не стану акцентировать внимание на том, что тем днем на мне было надето платье, а волосы как обычно волнами ниспадали на плечи. В общем, мне понадобилось секунд пять для того, чтобы собраться с мыслями.

\emph{Она назвала меня Вениамином?}

\emph{Какая разница, когда у нее члены в волосах! }

И еще десять на то, чтоб не расхохотаться.

-- Нет, -- ответила я, стараясь звучать максимально нейтрально. -- Вениамин -- мой жених.

Не вдаваясь в расспросы, женщина

\emph{(ЧЛЕНЫ В ВОЛОСАХ!)}

изъяла стопку денег откуда-то из-за пазухи и протянула ее мне.

-- Ну, по-всякому в жизни бывает, -- сказала она и вскорости исчезла за углом дома.

\emph{Не то слово, мамзеля}, подумала я.

Как только эта внезапная воительница за права трансгендеров скрылась с горизонта, из злополучного дворика появился мой благоверный. Растерянный и одинокий он медленно подошел ко мне, подкуривая очередную сигарету.

-- Нашел свою клиентку? -- с улыбкой спросила я.

-- Да.

-- Забрал деньги?

Он помедлил.

-- Нет. Она сказала, что отдала их\ldots{} Вениамину.

Больше не в силах скрывать улыбку, я отдала ему конверт.

-- Вообще-то я подумывала однажды взять твою фамилию, -- сказала я. -- Но вышло немного иначе.

Оставшуюся часть вечера мы провели, раскачиваясь на качелях.

\end{document}
