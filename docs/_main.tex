% Options for packages loaded elsewhere
\PassOptionsToPackage{unicode}{hyperref}
\PassOptionsToPackage{hyphens}{url}
%
\documentclass[
]{book}
\usepackage{amsmath,amssymb}
\usepackage{lmodern}
\usepackage{iftex}
\ifPDFTeX
  \usepackage[T1]{fontenc}
  \usepackage[utf8]{inputenc}
  \usepackage{textcomp} % provide euro and other symbols
\else % if luatex or xetex
  \usepackage{unicode-math}
  \defaultfontfeatures{Scale=MatchLowercase}
  \defaultfontfeatures[\rmfamily]{Ligatures=TeX,Scale=1}
\fi
% Use upquote if available, for straight quotes in verbatim environments
\IfFileExists{upquote.sty}{\usepackage{upquote}}{}
\IfFileExists{microtype.sty}{% use microtype if available
  \usepackage[]{microtype}
  \UseMicrotypeSet[protrusion]{basicmath} % disable protrusion for tt fonts
}{}
\makeatletter
\@ifundefined{KOMAClassName}{% if non-KOMA class
  \IfFileExists{parskip.sty}{%
    \usepackage{parskip}
  }{% else
    \setlength{\parindent}{0pt}
    \setlength{\parskip}{6pt plus 2pt minus 1pt}}
}{% if KOMA class
  \KOMAoptions{parskip=half}}
\makeatother
\usepackage{xcolor}
\usepackage{longtable,booktabs,array}
\usepackage{calc} % for calculating minipage widths
% Correct order of tables after \paragraph or \subparagraph
\usepackage{etoolbox}
\makeatletter
\patchcmd\longtable{\par}{\if@noskipsec\mbox{}\fi\par}{}{}
\makeatother
% Allow footnotes in longtable head/foot
\IfFileExists{footnotehyper.sty}{\usepackage{footnotehyper}}{\usepackage{footnote}}
\makesavenoteenv{longtable}
\usepackage{graphicx}
\makeatletter
\def\maxwidth{\ifdim\Gin@nat@width>\linewidth\linewidth\else\Gin@nat@width\fi}
\def\maxheight{\ifdim\Gin@nat@height>\textheight\textheight\else\Gin@nat@height\fi}
\makeatother
% Scale images if necessary, so that they will not overflow the page
% margins by default, and it is still possible to overwrite the defaults
% using explicit options in \includegraphics[width, height, ...]{}
\setkeys{Gin}{width=\maxwidth,height=\maxheight,keepaspectratio}
% Set default figure placement to htbp
\makeatletter
\def\fps@figure{htbp}
\makeatother
\setlength{\emergencystretch}{3em} % prevent overfull lines
\providecommand{\tightlist}{%
  \setlength{\itemsep}{0pt}\setlength{\parskip}{0pt}}
\setcounter{secnumdepth}{5}
\usepackage{booktabs}
\ifLuaTeX
  \usepackage{selnolig}  % disable illegal ligatures
\fi
\usepackage[]{natbib}
\bibliographystyle{plainnat}
\IfFileExists{bookmark.sty}{\usepackage{bookmark}}{\usepackage{hyperref}}
\IfFileExists{xurl.sty}{\usepackage{xurl}}{} % add URL line breaks if available
\urlstyle{same} % disable monospaced font for URLs
\hypersetup{
  pdftitle={Бессонница},
  pdfauthor={Александра Булгакова},
  hidelinks,
  pdfcreator={LaTeX via pandoc}}

\title{Бессонница}
\author{Александра Булгакова}
\date{2023-02-12}

\begin{document}
\maketitle

{
\setcounter{tocdepth}{1}
\tableofcontents
}
\hypertarget{ux44dux43fux438ux433ux440ux430ux444}{%
\chapter*{Эпиграф}\label{ux44dux43fux438ux433ux440ux430ux444}}
\addcontentsline{toc}{chapter}{Эпиграф}

\emph{Эта история не о биполярном расстройстве личности, хотя оно здесь есть.}

\emph{Эта история не о любви, хотя она здесь есть.}

\emph{Эта история не о депрессии, хотя она здесь есть.}

\emph{Эта история не о счастье, хотя оно здесь есть.}

\emph{Эта история не о жизни, хотя она здесь есть.}

\emph{Эта история не о смерти, хотя она здесь есть.}

\hypertarget{ux447ux430ux441ux442ux44c-i.-ux431ux435ux441ux441ux43eux43dux43dux438ux446ux430}{%
\chapter*{Часть I. Бессонница}\label{ux447ux430ux441ux442ux44c-i.-ux431ux435ux441ux441ux43eux43dux43dux438ux446ux430}}
\addcontentsline{toc}{chapter}{Часть I. Бессонница}

\hypertarget{chapter-1}{%
\chapter{~}\label{chapter-1}}

Как правило, после внепланового завершения серьезных отношений даже сильные люди впадают в своего рода эмоциональную спячку. Так или иначе, приходится изолироваться от несостоявшейся второй половинки. Перебороть истерики и утихомириться в какой-нибудь тесной квартирке с мусорным ведром под раковиной, микроволновкой для полуфабрикатов, своими немногочисленными пожитками и бутылкой, о которой ты уже толком и не помнишь, что внутри, но к которой все равно продолжаешь периодически прикладываться с крайне отсутствующим видом.

Сколько бы лет тебе не было, -- двадцать пять или пятьдесят, неважно -- вдруг начинаешь вести жизнь завтрашнего пенсионера. Прекращаешь слушать музыку, за едой думаешь исключительно о еде и никогда не выключаешь телевизор. Последний ты даже не смотришь, но все равно оставляешь телек трещать на фоне. Чтобы чувствовать себя менее одиноко.

В моменты как этот всякая деятельность обращается бездействием. С каждым днем ты все больше и больше теряешь интерес к жизни, но все чаще начинаешь задумываться о смерти, которая становится чем-то ожидаемым. Логичным завершением одного тотального фиаско под названием твоя собственная жизнь.

Мать моего отца умерла за месяц до моего рождения. Дедушка же прожил еще полторы декады. Все эти годы он спал со включенным телевизором, который толком не смотрел. Еще дед никогда не смеялся и очень редко улыбался. Разве что мне. После его смерти я нашла коробку со старыми снимками. Практически на каждом из них лицо моего деда освещала широченная улыбка, но отец лишь пожал плечами. Сказал только, что со дня похорон своей жены дед так ни разу не засмеялся.

Он умер хмурым январским утром. Сгорел от внезапно давшего о себе знать рака. Дед пережил голодомор, войну, распад Советского Союза и трех украинских президентов, а потом почувствовал острую боль в желудке и умер спустя месяц. Это вообще легально?

--- Я люблю тебя --- сказала я дедушке, и вышла из палаты.

Знай я, что это будут последние слова, которые он от меня услышит, выбрала бы что-нибудь пооригинальней.

Тем не менее, я до последнего не верила, что рак заберет его так быстро. Мне только исполнилось пятнадцать, дедушке -- восемьдесят три. Он продолжал работать, бегал по утрам и выглядел лет на двадцать моложе своего возраста.

Затем болезнь впервые дала о себе знать. Какие-то две-три недели и недостающие года тут же отразились в чертах его лица.

Тем вечером за окном бушевал ветер. Выходя из палаты, я слышала, как по подоконнику забарабанили первые капли дождя.

--- Надеюсь, это случится не завтра, --- вдруг сказал дедушка. --- Мне бы очень не хотелось умереть в такую погоду.

--- Это не случится завтра, --- спокойно сказала я и наклонилась, чтобы поцеловать дедушкину щеку. --- Я люблю тебя.

И я ушла. Вышла под дождь и медленно зашагала в сторону дома.

Проснувшись следующим утром, я подошла к окну. За ним по-прежнему было темно, и я с удивлением осознала, что впервые в жизни встала раньше будильника. Фактически, утро еще не наступило. До школы оставалось несколько часов, а в душе засело ноющее чувство тревоги. Я взяла книгу -- кажется, это был Лавкрафт -- и устроилась на подоконнике.

Помнится, по радио то и дело передавали штормовое предупреждение, и по мере приближение рассвета мне открывались переполненные водой улицы. Вода и грязь, сопровождаемые возгласами ветра, заполняли собой все вокруг.

Дедушка не умер на следующий день. Он умер той ночью. Думаю, от этого я и проснулась.

Сейчас, спустя десять лет, я с ужасом осознаю, что не могу припомнить, как звучал голос моего деда. Я закрываю глаза и стараюсь расслабить сознание, вспоминая детство, проведенном в его доме. Вижу старый топчан, покрытые пылью ордена и желтую занавесь, что висела у входа в дедушкину спальню. За ними тянется старого образца гостиная. Она ничем не отличается от тех, что можно увидеть в домах других стариков из рабочего класса: раскладной диван, ковер на стене, кресла по обе стороны журнального столика, накрытого плетеной циновкой, и древний телевизор, который никогда не замолкает.

Мне вспоминается крохотная прихожая с обогревателем; я слышу яркий запах цитрусовых, чьи корки дедушка каждую зиму сушил для своей настойки. Вспоминается кухня, пиалка из хрусталя, неизменно полная конфет и гигантская копия наручных часов, висевшая в углу. На столе меня ждет чугунная сковорода времен Феликса Дзержинского, а в ней -- традиционная утренняя яичница с молодой картошкой, вкусней которой и быть не может.

По старой привычке, дедушка ест стоя. Он добавляет в картошку соли и с довольным видом наблюдает за тем, как я уплетаю завтрак. Чайник на плите уже начинает посвистывать, привлекая внимания котов, что всю ночь где-то пропадали, а теперь лениво дремлют на подоконнике.

Лежавший в тарелке завтрак вскоре исчез, как исчезнет и сковорода, и стол, на котором она стоит, кухня, и все остальные комнаты. На месте ветхого домика уже давно стоит другой, но память о нем никуда не делась. Я вспоминаю дедушку, заботливо перемешивающего сахар в моей чашке с чаем. Вижу, как он замечает пустую тарелку, и знаю, что будет дальше. Сейчас дед поставит передо мной чай и предложит добавки. Тогда я делаю глубокий вдох и стараюсь не думать ни о чем другом. Все жду, что голос сам всплывет в памяти.

Но он не всплывает, как не старайся. С пугающей точностью я помню слова, интонации, произношение. Короче говоря, что угодно, то только не то, что хочу вспомнить. И, все-таки, мне кажется, услышь я дедушкин голос хоть на мгновенье, -- случайно, и не подозревая, кому он принадлежит -- я бы обязательно его вспомнила.

Так вот, все эти годы дед жил со включенным телевизором. Прошло еще восемь лет, прежде чем я поняла, почему.

\hypertarget{chapter-2}{%
\chapter{~}\label{chapter-2}}

Приближался день рождения Адама, которого я не видела два с половиной года. Очередной день в баре. К тому времени я уже привыкла находиться по другую сторону стойки и читала что-то от Буковски, периодически поглядывая в сторону поддатых посетителей.

--- Освежи мне! --- заплетающимся языком произнес один из них.

После чего толкнул пивной бокал в обратную от меня сторону. Тот проехал пару метров вдоль барной стойки, ударился о стену и с характерным звоном рассыпался на тысячи блестящих стеклышек.

Мне вдруг очень захотелось стать этим бокалом.

На самом деле, мне даже нравилось работать за барной стойкой. Если не учитывать шестнадцатичасовой рабочий день, мизерную зарплату, постоянные недостачи и полное отсутствие чаевых, а также клиентов, в большинстве своем доводящих до исступления\ldots{} О чем я говорила? Ах, да все не так уж плохо.

Сложно сказать, почему, но, стоило мне надеть фартук и впервые перешагнуть эту заветную черту, отделяющую мир бармена от общества простых смертных, я моментально почувствовала себя в своей тарелке. Разбираться в винных сортах, изучать миксологию, работать над созданием собственных коктейлей и временами баловаться флейрингом -- было в этом что-то особенное. Я без труда могла рассказать о любом из напитков куда больше, чем указано на этикетке, посоветовать вино или удивить гостя чем-нибудь экзотичным. Такая работа и впрямь была мне по душе, но, быть может так скажет любой алкоголик.

Увы, несмотря на престижность заведения, где я работала, (а также на тот факт, что местная стойка считалась самой дорогой во всем городе) платили здесь плачевно мало. Настолько мало, что едва хватало на еду и коммуналку. С другой стороны, работа занимала у меня все время, так что, будь у меня лишние деньги, я все равно не успевала бы их потратить. Вероятно, в этом и был секрет выживания сотрудников «Штиля».

--- Плесни еще на посошок, --- на этот раз горе-метатель казенной посуды обошелся без спецэффектов.

Залпом расправился с последней порцией пива и вскоре исчез, оставив «Штиль» без единого посетителя.

-- Итак, на чем мы остановились\ldots{}

Десять страниц, пятнадцать, двадцать. Ничего не менялось. В зале по-прежнему было пусто.

Как и в моем сердце.

Не то, чтоб я ежеминутно думала о своем бывшем. Эти времена уже прошли. Я месяцев семь как выкарабкалась из затяжной депрессии и всячески старалась загрузить себя работой. Как я уже сказала, платили в «Штиле» смехотворно мало. Ну, кто будет так рвать задницу ради каких-то ста пятидесяти баксов в месяц?

К счастью, работа всецело меня выматывала, что оказалось главным из ее достоинств. После смены на всё про всё оставалось часов шесть, и это без учета дороги, а потом обратно за стойку. Знаете, нелегко быть в депрессии, когда на нее просто не остается времени. Мне чуть ли не заранее приходилось планировать свои нервные срывы.

Так вот, наступило двадцатое апреля две тысячи шестнадцатого года -- день толерантности марихуаны, а по совместительству и день рождения одного небезызвестного немецкого политика еврейского происхождения. День обещал быть куда более радужным, чем карьера последнего. Мне хотелось выглянуть на улицу, но за стойкой, само собой, окон не было.

Не было их и в зале-ресторане. Не было в боулинге, и на кухне тоже ни одного окошка не наблюдалось. Мне это всегда казалось странным. Будто у проектировщиков имелась какая-то тайная нелюбовь к сквозным отверстиям. С другой стороны, в зале ежедневно околачивались стриптизерши, проститутки, эскортницы и прочего рода содержанки, так что что-то здесь явно не вяжется.

В любом случае, за стойкой было темно как в причинном месте. Приходилось читать в свете неоновых вывесок с изображениями Моргана, Бушмилса, Дэниэлса и прочих хорошо мне знакомых парней. Если алое освещение надоест, можно чуток подвинуться влево. Пара шагов вдоль стойки, и алые страницы становятся зелеными в свете лозунгов Егеря, Бехеровки или Ксенты.

Говорят, человеку нужно всего три недели, чтобы привыкнуть к дискомфорту или избавиться от вредной привычки. Увы, я провела в «Штиле» не один год, но так и не смогла свыкнуться с отсутствием окон.

Единственным приятным моментом, помимо доступа к алкоголю, была входная дверь. Сплошь стеклянная, она находилась слева от моего рабочего места и открывала взгляду небольшой клочок улицы. Судя по всему, за пределами «Штиля» и впрямь стояла восхитительная погода. Сияло весеннее солнце; деревья уже позеленели и, движимые легким ветерком, отбрасывали причудливые тени на аллейку. Отсюда мне был виден каменный фонтан да пара скамеек. Брызги воды прямо-таки светились под лучами полуденного солнца. От этой радужной картины в душе сделалось как-то грустно, так что я отправилась варить кофе.

--- Ты будешь делать кофе? --- моментально среагировала одна из официанток, которых в «Штиле» звали просто фицами.

--- Кофе? --- эхом отозвались из зала.

Каким-то мистическим образом они всегда это чувствовали. Девушки по инерции стекались у барной стоило мне лишь подумать о том, чтобы варить кофе.

--- Буду.

--- И мне сделай, Лизочек, --- Вера сняла с полки пустую чашку, чмокнула меня в щеку и побежала обслуживать очередной столик.

Терпеть не могла, когда меня так называли.

За ней оживились и другие сотрудники развлекательного комплекса. Стало ясно, что одной чашкой кофе здесь не обойдется.

Вернувшись на прежнее место, я заметила, что у фонтана появился бездомный. Уже наполовину раздетый, он продолжал смущать прохожих и неспешно стягивал с себя одежду пока, наконец, не остался в чем мать родила. Затем товарищ без лишних раздумий погрузился в фонтан, лениво растянулся, опираясь на бортик, и уже беззаботно потягивал пивасик.

Вот, в принципе, и все, что вам нужно знать о городе, в котором я родилась.

***

Звонил телефон. Кассир сняла трубку и тут же передала ее мне со словами:

--- Это по твою душу.

Этажом ниже просили сделать пятнадцать детских мохито и четыре взрослых. Намечался очередной детский день рождения. С ума сойти. В мое время мы дарили друг другу наклейки и радовались мороженому из МакДональдс, а не арендовали целый этаж в ночном клубе вместе с аниматорами, лайт-джеями и диджеями. А ведь мне едва ли исполнилось двадцать два года.

--- И побыстрее! Мелюзга уже вовсю вопит! --- голос хриплый, чуть ли не потусторонний, и прямо-таки гавкает, а не говорит.

Почти два десятка мохито -- не самый удачный выбор, когда за стойкой нет крашмейкера. Единственный работающий аппарат находился в ночном клубе, так что мне постоянно приходилось бегать туда-сюда за колотым льдом.

Официантка уже стояла на раздаче, когда я положила трубку.

--- Я бы тебе помогла, --- сказала Вера, --- но начальство велело даже в туалет не выходить. Я в зале одна осталась.

--- Ничего, я схожу вниз\ldots{}

--- Пипец. Скоро обоссусь.

--- А ты пока принеси стаканы.

Вероника кивнула и уже собиралась уходить, когда я позвала ее по имени.

--- Слушай, а кто звонил?

--- Ну, Рыжая.

Я вскинула бровь.

-- Ты уверена?

-- Еще б мне не быть уверенной. Здесь только мы и есть. По одной на этаж. А чё такое?

-- Да голос, блин, какой-то жуткий. Как с того света, честное слово. Это точно Рыжая? Как-то на нее не похоже.

-- Чего не похоже? На Рыжую не похоже? Да она опять до шести утра в клубе бухала!

Вера закатила глаза и, прихватив поднос, ушла на мойку. Порой создавалось впечатление, что та завидует коллеге.

Как ни странно, внизу меня действительно ждала Тоня.

-- Привет, Рыжик, как твое похмелье?

Милая девушка, только временами врывается как гром среди ясного неба. Говорила она много, громко и очень быстро. Словом, Тоня с легкостью могла бы стать отличным примером экспрессии в бытовой речи.

-- ЛИЗА, ЛИЗА, ЛИЗА! -- закричала она, перекрывая шумы клубного подземелья. -- Сделай мне чего-нибудь этакого!

-- Этакакого?

-- Ну, чтоб получше стало, а то я щаз точно ласты откину! Вот увидишь, откину! -- она наполнила хайбол льдом и теперь жадно прижимала его к своему лицу. -- Так и шо? Ты сделаешь?

-- Сделаю. Как с заказом закончу.

-- Пожа-а-алуйста, сделай сейчас, а? А я тебе лёд пока замучу.

Ну, как тут откажешь?

До окончания дня оставалось еще немало. Я как раз колдовала над своим противопохмельным отваром, когда телефон в кармане моих джинс завибрировал. На экране меня ждало входящее сообщение от очередного бородатого незнакомца.

Его звали Вениамин.

«Как жизнь молодая?» -- спрашивал он.

Обычно я игнорирую подобные попытки знакомства, но\ldots{} То ли дело здесь было в четыре-двадцать, то ли в том, что мне осточертело стоять за стойкой, а, может, в том, что образ Адама то и дело упрямо выныривал из памяти -- понятия не имею. В общем, я ответила.

\hypertarget{chapter-3}{%
\chapter{~}\label{chapter-3}}

Все та же душная мгла повисла над городом, заботливо обволакивая каждую из его составляющих. До рассвета оставалось еще минимум шесть часов, что вовсе не мешало местной пьяни устраивать под окнами традиционные субботние концерты. Они орали и галдели так, словно эта ночь была последним шансом напиться.

Растянувшись поперек кровати, я пыталась сосредоточиться на книге. В ней говорилось о зимних лесах с их невероятными пейзажами, конных упряжках, размашистых балах и дворцах, одетых в блестящее убранство. А за окном лаяли псы, неслись машины и шумели районные алкоголики, напоминая мне, где на самом деле я нахожусь. Вероятно, дело было в моем душевном расстройстве: маниакальная фаза сменилась депрессивной. Мне вдруг стало неописуемо грустно и обидно за то, какой жизнью приходится жить. Еще и Адам постоянно лез в голову.

\emph{Не думай о нем.}

Чертов Адам с его обаятельной улыбкой да ямочками на щеках.

Захлопнув книгу, я отважилась закрыть окно

(я говорю «отважилась» потому, как термометр показывал под тридцать, а кондиционера в моей квартирке отродясь не водилось)

чтобы хоть немного приглушить раздражающие вопли. Подавляющему большинству людей всегда было плевать на беспредел, творящийся в этом районе. Несколько лет назад в парке под моим домом до смерти забили женщину. Другую изнасиловали и задушили чуть подальше, оставив валяться среди ив, кленов и прочей растительности, что окружала аллею. Нашли ее полностью голую около семи утра. Я прекрасно помню то утро потому, что именно в это время шла через парк поздравить своего отца с юбилеем, а еще потому, что я эйдетик.

Помнится, третью женщину убили как раз-таки на отрезке пути, что вел от одного из этих парков к другому. Она лежала в сквере у онкологического отделения. Снега той зимой были нешуточные, так что несчастную обнаружили лишь в апреле. Вид у нее был, прямо говоря, не очень, отчего распознать черты лица уже не представлялось возможным. Труп опознать не могли, и убитая номер три провела еще года полтора в морге той самой больницы пока дело, наконец, не закрыли.

Есть еще один небольшой парк за детским садом, где я когда-то училась падать с велосипеда. Там тоже кого-то отправили в мир иной. Еще один -- в пяти минутах ходьбы и в нем вообще регулярно находят трупы. Половина из них, конечно, местные торчки, которые не могут правильно рассчитать дозу, но есть и те, кто умер слегка менее приятной и чуть более насильственной смертью. К примеру, начальник районного отделения полиции.

Кстати говоря, по дороге от моего дома в сторону последней из описанных локаций тоже есть парк. Не знаю, что там у них насчет убийств, но однажды у меня закончились сигареты, и я вышла на их поиски. Как раз достигла середины этого крохотного, тянувшегося вдоль проезжей части сквера, когда незнакомый мужчина подбежал ко мне и со всей дури заехал в челюсть.

Я ошалело взглянула на незнакомца и расхохоталась.

Потное, осунувшееся лицо, зрачки как блюдца и челюсть пошла ходуном -- да это тип был явно под белым. Он, не моргая, таращился на меня, очевидно, прикидывая, стоит ли нанести повторный удар. Потом как будто о чем-то вспомнил. Схватился за голову и кинулся прочь.

-- Постой, паровоз! -- кулаки у меня чесались еще с вечера. -- Куда ты погнал?

-- А НУ СТОЯТЬ, ПЕДРИЛА! -- заорала моя мания.

Ей было все равно на то, что тот тип мог уложить меня одной правой.

-- Я не усну, пока не узнаю, зачем ты меня двинул!

Но мужик оказался бывалый. Он перепрыгнул через забор и довольно быстро исчез из поля зрения, так что мне пришлось возобновить поиски никотина.

Короче говоря, парки в этом районе так себе.

Так вот, обычно местный ночной беспредел меня едва ли волновал. Порой я даже сама являлась его частью. Но не сегодня. Этим вечером моя психика была особенно уязвима.

\emph{Не думай о нем.}

Биполярка вообще странная штука. То ты пляшешь, поешь, и трахаешься как под спидами, а затем на полставки подрабатываешь комиком и опять трахаешься. Не можешь спать (слишком перевозбужден) и не можешь есть потому, что чувство голода куда-то улетучивается. Тем не менее, окружающие считают тебя душой компании и вечно зовут на тусовки. Самые отважные умудряются еще и влюбиться.

Затем наступает депрессивный период, и ты опять не можешь спать, потому как слишком печален. Есть, кстати, не хочется по той же причине. Все самые страшные и болезненные воспоминания вырываются наружу, так что истерикам нет предела. Сначала это кратковременная злость, затем пара гордых слезинок, а потом истерика, сменяющаяся полноценной панической атакой. Длится она от пары минут до одного-двух часов, в зависимости от окружения и желания жить в целом. Зачастую, в борьбе с припадками, мне приходилось оставлять на конечностях порезы и ссадины различной глубины. Чтобы хоть как-то угомониться.

По весне завидев шрамы, люди почему-то думают, что я каждый день начинаю с того, что сажусь на край кровати с намерением вскрыть себе вены. Как будто я не в курсе, где эти самые вены находятся. Все дело в том, что временами острая физическая боль становится единственным способ отвлечь себя от печали. Порезы -- это не страшно. Разбитое сердце, душевная боль, одиночество и, в конечном итоге, немое отчаяние -- вот это куда страшнее, ведь именно они вызывают желание распрощаться с жизнью.

Такими днями общаться с людьми совершенно невозможно. Они мало что понимают. Вечно ждут от тебя забавных историй, обилия шуток и прочих признаков веселья. Люди привыкли видеть меня такой. Всегда и везде. Это смешно, но порой, соприкасаясь со мной во время депрессивной фазы, некоторые из них чуть ли не злятся на мою грусть и молчаливость. Капризничают и топают ножками, требуя веселья, к которому так привыкли. Разочаровываются, называют скучной и оскорбляются нежеланием вести бурную ораторскую деятельность. Проверку депрессией проходят единицы, но даже они не способны меня спасти.

В итоге внутри не остается ничего кроме боли. И скоро ты вновь призраком слоняешься по городу в кромешном одиночестве.

Биполярное аффективное расстройство -- вот официальное название того, чем я страдаю. Или наслаждаюсь. Этот вопрос никогда толком не проясняется. Впервые увидев список симптомов БАР-а, я, мягко говоря, охренела. Выглядело все так, словно кто-то просто наблюдал за мной на протяжении двадцати лишним лет, а затем собрал воедино все черты моего характера и назвал это биполярным расстройством. Отвратительное, скажу вам, чувство. Вот живешь ты себе, жалуешься на жару, пьешь вино и пишешь свои книги, а затем вдруг понимаешь, что никакая ты не творческая личность, а всего лишь очередная психбольная. Все твои достоинства, отрицательные качества, привычки и даже так называемые изюминки -- это все лишь признаки психического расстройства, которым болеет 2,4 \% земного населения. Как вам такая новость?

Кухонные часы показывали полночь.

Суббота закончилась. Пьяницы остались.

\hypertarget{chapter-4}{%
\chapter{~}\label{chapter-4}}

Тринадцатое утро мая две тысячи шестнадцатого года встретило меня умеренной погодой -- редкое явление в моих родных краях. За окном обнадеживающе светило солнце, но было довольно прохладно. Самое то для дальней поездки. Дни работы за стойкой приучили меня выползать из постели сразу же после пробуждения. Или вообще в нее не ложиться, если чувствуешь, что не сможешь проснуться вовремя.

Это работало, даже если я уснула с рассветом, так что я в кои-то веки не пропустила свой автобус и даже на него не опаздывала. Сказать больше, мне удалось сделать укладку и нарисовать шикарные стрелки. Вот они, прелести выходного дня.

Выпив кофе, я прихватила небольшой, но довольно увесистый чемодан, и поспешила покинуть пределы родного гетто. Уже на улице остановилась напротив сигаретного киоска, который, казалось, находился здесь всю жизнь.

Я позволила себе слегка забыть о времени, зачем-то разглядывая многочисленные ряды пестрых сигаретных пачек. К тому моменту я не курила уже практически пять месяцев и, как мне думалось, не собиралась. Но что-то заставило меня потянуться в карман за парочкой купюр. Наверное, все дело было в волнении, а точнее, в его отсутствии. Мне всегда думалось, что встречи с незнакомцами тем и прекрасны: ты ничего от них не ждешь. Как и внутри, внешне я оставалась абсолютно спокойной. Только вот внутренний голос откуда-то из глубин подсознания мягко напоминал, что обычно паника подкрадывается ко мне в последний момент. Меньше всего на свете мне хотелось обнаружить себя поддающейся нервному припадку где-нибудь на границе двух внезапно враждующих стран.

-- Пачку Винстона. Синего.

Я без особых зазрений совести расплатилась за сигареты. Так и закончилась моя борьба с никотиновой зависимостью.

Автобус отходил в четверть десятого, и теперь я таки на него опаздывала. В салон я вошла за секунду до отправления, соблюдая старые добрые традиции. Такой вот я человек: ну просто ненавижу ждать, а потому никогда не выхожу из дому заранее. На этот раз мне повезло. Дверь за моей спиной захлопнулась, автобус тронулся и спустя четверть часа я уже наблюдала за бесконечными полями да деревьями, что проносились за окном. Тем днем я читала «Бродяг Дхармы», что очень кстати описывало наступивший период моей жизни. Книга уже подходила к середине, а мы все ехали и ехали. Несколько раз проезжали какие-то населенные пункты, -- ПГТ или вроде того -- но было заметно, что до Крыма по-прежнему как до Луны.

Откровенно говоря, в Крыму я не была лет шесть, да и настоящих отношений у меня уже года три как не было. Словом, я понятия не имела, что меня ждет. Жизненный опыт подсказывал, что надежней всего будет не ждать ничего. В конце концов, рушащиеся надежды -- это всегда неприятно и лучший способ избежать подобной ситуации: попросту их не иметь.

Часов через шесть автобус, наконец, остановился. Водитель объявил, что через полчаса будем на границе, и отправился пить кофе. Внутри меня что-то ёкнуло. Впервые за много-много лет. Это меня обескуражило и я поспешила выйти на свежий воздух.

Одного быстрого взгляда хватило, дабы понять, что я нахожусь прямиком в центре какого-то украинского захолустья. Здесь люди курили под табличками «Не курить!», громко ругались матом при детишках и сидели на асфальте, когда рядом пустовала скамейка. В принципе, ничего необычного. Мы, южане, и не такое видали.

До отправления оставалось минут двадцать пять, так что я заглянула в одну из местных лавчонок. Это крохотное помещение едва превышало размеры туалета хрущевки. Все же, каким-то мистическим образом в него вместилась барная стойка, кофемашина, парочка стульев со столиком, телевизор и даже одно окно. Правда небольшое.

Еще внутри безымянного заведения обнаружились бармен, бариста, повар, продавец, кассир и фиц в одном лице. Девушка так резко выпрыгнула из-за стойки, что я бы обязательно подскочила, не будь я в тот момент где-то далеко в своих мыслях.

-- Посетитель! -- радостно воскликнула девушка. -- Вы -- наш первый посетитель за сегодня! Что будете?

Я слегка удивилась отсутствия акцента, характерного для подобных поселений. Затем взглянула на часы. Дело шло к вечеру. Поздновато как для первого посетителя.

-- Капучино.

-- Но у нас только латте и американо\ldots{}

-- Латте вполне подойдет.

Девушка кивнула и принялась возиться с кофемашиной. Та издавала предсмертные звуки, и я очень надеялась, что успею выпить кофе прежде чем она развалится.

-- Меня зовут Тома, -- представилась девушка. -- Вы недавно у нас?

По всем канонам современного бодипозитивного общества Тома выглядела вполне привлекательно. Это была высокая брюнетка с яркими губами и практически под ноль стриженными волосами. Не толстая и не худая. О таких обычно говорят «в теле». Черты лица -- очень выразительные и чувственные. Все это в совокупности с откровенно мужиковатой одеждой походило на стиль, популярный во времена Твигги.

-- Проездом, -- ответила я и расплатилась за кофе.

Тома окинула меня взглядом.

-- Красивые волосы, -- заметила она. -- Настоящие?

Что ж, дамы и господа, а вот и лидер среди наиболее часто задаваемых мне вопросов.

-- Более чем.

-- А далеко едешь-то?

-- В Севастополь.

-- Ух, далековато. И что там, в Севастополе?

Этот вопрос вызвал в моей голове небезызвестную цитату из «От заката до рассвета», но я как-то удержалась.

-- Мужчина.

-- Хороший?

-- Не исключено, что моей мечты, -- я пожала плечами. -- Но это не точно.

Протяжно вздохнув, Тома облокотилась о стойку.

-- Я тоже была здесь проездом, -- заметила она.

-- И что так?

-- И у меня тоже был мужчина. Как раз возвращалась к нему, когда парня встретила. Влюбилась. Вот разводимся с ним сейчас.

-- С которым из них? -- не поняла я.

Незнакомцы, отчего-то, так любят изливать мне душу. Возможно, как-то ощущают мою эмпатию, но в подобные моменты я всегда чувствую себя неловко, если у меня вдруг все в порядке.

Что ни говори, я в кои-то веки не страдала из-за разбитого сердца, не болталась на дне депрессии, куда обычно приводят творческие амбиции, и даже была немножечко счастлива. Мне сделалось неудобно. Пожалуй, никому не должно быть неловко за отсутствие жизненного дерьма, но мне почему-то стало. Я подумала, что, возможно, именно так себя и чувствовали мои друзья на протяжении парочки последних лет.

-- Да со вторым уже.

Я подняла стаканчик с кофе.

-- За несбывшиеся мечты.

Тома улыбнулась, но как-то грустно. Можно понять.

Я вспомнила о времени. Минута до отправления. Ну, все как обычно.

-- Мне пора. Спасибо за кофе.

-- Как тебя звать-то? -- вспомнила Тома.

-- Елизавета Васляева.

-- Ну, удачи тебе, Елизавета Васляева. Главное, замуж не выходи.

-- Постараюсь, -- уже в дверях ответила я.

\hypertarget{chapter-5}{%
\chapter{~}\label{chapter-5}}

Мы подъезжали к украинско-русской границе. Народ уже толпился у выхода из автобуса. Люди так настойчиво прокладывали себе путь, что я удивилась, как это еще никто не умер. Какая-то масштабная мадам правда завалилась на себе подобную и вскоре обе уже распластались вдоль салона.

Всего нас было тридцать три человека, не считая водителя. Прямо как годиков одному знаменитому божьему сыну. Тридцать человек спорили у входа, пытаясь первыми проникнуть к отделу с багажом. Еще две женщины поднимались с пола. Они выглядели сонными и обессиленными, но по инерции продолжали браниться. Завершая общую картину, я сидела в самом конце салона. Читала книгу и краем глаза поглядывала на происходящее вокруг. Багаж -- если таковым можно назвать чемоданчик размером с ноутбук -- и так был при мне.

Украинско-российская граница, как выяснилось, больше походила на декорации к первому сезоны «Ходячих мертвецов», нежели на хорошо охраняемое место переправы. То тут, то там виднелись полуметровые цементные блоки и целая куча хаотично разбросанной колючей проволоки. Еще одним фактором, производившим странное впечатление, выступали припаркованные повсюду автомобили. Большинство из них были с распахнутыми дверцами и опущенными стеклами, тогда как водителей внутри не наблюдалось. Те растерянно бегали от одной будки к другой, пытаясь правильно заполнить целую стопку бумажек, каждая из которых нужна лишь для получения следующей.

У импровизированного начала границы стояло несколько военных. Ребята были при параде, включая дубинки и огнестрельное. Фонари горели как-то вяло и, хотя солнце еще только подумывало садиться, становилось ясно, что освещаются лишь первые пару метров переправы. Кстати говоря, такой же участок пути имел более-менее вменяемое дорожное покрытие. За ним пешеходов ожидали овраги, трещины, камни, и прочие достопримечательности моего родного края.

Учитывая обстоятельства с ручной кладью, к пункту проверки документов я подошла первая. Вернее, поначалу мне так казалось. Но довольно возрастная женщина внезапно воткнулась прямо передо мной и уже самодовольно махала рукой какому-то не поспевающему за ней мужчинке. По ту сторону границы нас ждал другой автобус. До отправления оставалось больше часа, так что особой нужды в такой спешке не было. Как, видимо, и в вежливости.

-- Куда едете? -- спросил не поддающийся определению по половому признаку голос из будки.

-- Севастополь.

-- Причина визита?

-- Пока под вопросом.

Пауза.

Окошко распахнулось, и на свет божий выползло заспанное личико, немногим старше моего собственного.

-- Назовите причину визита, -- раздраженно повторила барышня.

-- Просто хочу кое-что проверить.

-- Нет, так нельзя.

-- А как можно?

-- Бизнес, туризм, или личная.

Так я совершенно неожиданным для себя образом узнала, как можно. Что бы это ни значило.

-- Что пишем? -- напомнил о себе голос из окошка.

Я призадумалась. Очевидно, что мой визит относился к категории личных. Однако, опыт минувших взаимоотношений с другими представителями земной расы показал, что ни в коем случае нельзя вешать ярлыки. Никакие. Даже если все кажется идеальным. Нет, особенно если все кажется идеальным! Более того, дело здесь не в возможной реакции других людей, а в твоем собственном подсознании. Я имею в виду, как только ты называешь мужчину своим молодым человеком, ты подсознательно начинаешь ждать от него определенную модель поведения. И чем больше его действия походят на те, что ты там себе успела придумать, тем сильнее ты расслабляешься. И тем болезненнее в дальнейшем будет принять суровую реальность.

Словом, в я только-только свыклась с мыслью об отсутствии разбитого сердца в моей грудной клетке и никак не была готова признать связь с Вениамином личной. Поэтому я ответила:

-- Бизнес.

-- Во-о-от как, -- работница таможни подняла бровь с той самой грациозностью, с какой бывалая проститутка поднимает юбку по окончанию тяжелого рабочего дня. Вроде как и лень, но по статусу положено. -- И кем вы работаете?

-- Я писатель.

-- Вот как, -- повторила она. -- И что, прямо пишите?

-- Преимущественно.

-- Вот как.

Мне вдруг стало очень смешно.

-- А работаете-то кем? -- не унималась мадам.

Я положительно не хотела слышать четвертое «вот как», поэтому решила сдаться.

-- Барменом.

-- Так и напишем, -- подытожив, она протянула мне небольшой бланк. -- Распишитесь внизу.

В графе причина въезда был подчеркнут «личный визит». Я выдохнула:

-- Вот как\ldots{}

***

Ох, чего только я не наслушалась перед выездом. Одни рассказывали мне душетрепательные истории и том, что на границе рвут документы. Вторые убеждали, что стоит мне запеть гимн Украины и меня тут же повалят на асфальт. Не то, чтобы я каждый день занималась подобными развлекательными программами, но кто это проверил и зачем? Была еще история про курицу.

Вообще, в моей жизни как-то подозрительно много бесполезных историй, что, так или иначе, связаны с курами. Понятия не имею, что это значит, но, пожалуй, что ничего. Так вот, подруга матери бывшего администратора «Штиля» решила поехать к племяннику, который остался жить на полуострове. Будучи коренной украинской бабулей, она прихватила с собой целую кучу еды, включая сырую курицу огромных размеров. Увы, правила запрещали перевозить продукты. В случае их обнаружения, нарушителя заставляли пройти весь путь обратно, -- от российской границы до украинской -- и оставить еду там, заполнив соответствующие бумаги. Если же ты спешишь на автобус, можно просто отдать еду пограничникам. К чему это я? Легенда гласит, что суровая бабуля съела свою злополучную курицу прямиком при служащих таможни. Сырой.

Стоя в очереди позади лишенной такта женщины я мысленно решила, что она вполне могла бы быть той самой бабулей, и теперь изо всех сил старалась не рассмеяться: перед глазами уже вовсю плясали живописные образы.

Проверка багажа не заняла много времени. Сникерс у меня не отняли и на откупоренную в дороге бутылку виски тоже никто не позарился. Подхватив платье свободной рукой, я шагала по широкой, отвратительно вымощенной дороге, которая с каждым новым метром становилась все хуже. Мелкие камешки то и дело норовили застрять в подошве моих ботинок. По обе стороны от этой недотрассы тянулись небольшие овраги. За ними -- высоченный забор из нержавеющей проволоки. Дальше виднелись бесхозные поля, кустарники и кое-какие деревья.

По мере моего приближения к русской границе забора становилось все меньше, а деревьев -- больше. Вскоре они уже занимали все пространство вдоль дороги. Благодаря размерам своего чемодана, я неплохо оторвалась от других пассажиров. Порой на пути мне встречались люди, идущие в противоположном направлении. Еще реже проезжали автомобили, но в основном я шла одна.

Раскинувшиеся по обе стороны дороги виды создавали иллюзию заброшенности. Казалось, я была единственной, чья нога ступала на эту землю на протяжении последних десятилетий. Тишина, покой и нетронутая природа. Все это напоминало мне пейзажи из документалок о Чикатило, которые я любила смотреть в детстве. Окончательно картину дополнил проржавевший автомобиль, наполовину погрязший в грязи. Стекол на нем не было, как и колес. Сквозь дверцы уже успели прорасти впечатляющих размеров сорняки. Как ни странно, машина удачно вписывалась в обстановку. Думаю, она и сейчас там стоит.

\emph{Самое то для внеплановой медитации}, решила я и забралась на крышу бесхозной машины.

Благодаря эйдетической памяти и своей творческой натуре я без труда могла переместиться куда угодно. Вот только не всегда могла это контролировать. Воображение, на этот раз слишком живое, унесло меня к глубоко похороненным страхам прошлого, так что от медитации пришлось отказаться.

В воздухе запахло морем. Наконец, вдалеке стали видны какие-то светящиеся вышки. Спустя минут семь мне преградил путь шлагбаум в компании молодого солдата. Парень попросил показать паспорт.

-- Наркотики, оружие, спиртное перевозите? -- серьезно спросил он.

-- Целую кучу, -- ответила я.

Чемодан открывать не стали.

-- Дальше по дороге идти нельзя, -- заметил солдат. Я вопросительно подняла брови. -- Такие правила. Пройдите во-о-он туда.

Он указал на начало очередного забора. Подойдя ближе, я ужаснулась. Нет, я, конечно, знала, что в России мало сторонников демократии, но не до такой же степени! В общем, вы в курсе, как выглядят наружные тюремные коридоры? Те, по которым зеков обычно ведут из одного крыла в другое? Такие узенькие площадки максимум метр в ширину, с двух сторон огражденные высоченным забором все с той же колючей проволокой на верхушке. Именно так выглядит начало российской границы.

Оценив ситуацию, я подкурила.

***

Спустя полчаса я, наконец, забралась во второй автобус. До отправления оставалось более двадцати минут, и я понятия не имела, как их провести. В другой ситуации я бы, наверное, уже давно умерла со скуки. Но не сегодня. Сегодня у меня был виски.

Конечно, я в сотый раз повторяла себе, что не стоит ждать ничего серьезного от предстоящей встречи. И, все-таки, я ждала. Вернее, та часть меня, которая заставляла мою внутреннюю феминистку забиться в угол. Предвкушения чего-то хорошего -- вот к чему я питала особую зависимость. Только в подобные волнительные моменты я могла почувствовать себя живой.

Откровенно говоря, и спустя годы, вспоминая вечер на таможне, я все так же чувствую послевкусие той необъяснимой эйфории, которая настигла меня за несколько часов до встречи с Вениамином. Автобус больше не казался мне душным, погода жаркой, а жизнь дерьмовой. Я сидела в конце салона, наблюдая за тем, как уже успевшая полюбиться мне лесополоса принимает солнечные ванны, и чувствовала себя как никогда восхитительно.

Это меня расслабило. Вскоре я прикрыла и позволила памяти ненадолго вернуть меня в прошлое.

\hypertarget{chapter-6}{%
\chapter{~}\label{chapter-6}}

За окном моросил октябрьский дождь две тысячи тринадцатого года выпуска, отчего внутри становилось крайне уютно. Обернувшись одеялом, я сонно смотрела в окно. Передо мной стояла чашка кофе, кола и наполовину пустая бутылка виски. По ту сторону стекла уже начинало темнеть, и я наблюдала, как в окнах соседних домов зажигаются огни. Одно за другим. Прямо как в детстве, но без виски, разумеется. До шестнадцати я пила только вино.

С минуту я разглядывала стоявшие на подоконнике жидкости, после чего потянулась кофе. Любовь пагубно влияла на мой алкоголизм.

За моей спиной скрипнула дверь. Обернувшись, я увидел, стоявшего в дверном проеме Адама. Дождь к тому моменту усилился, и мужчина уже стягивал с себя промокшую до нитки куртку. Несмотря на это, выглядел он как никогда очаровательно. На губах играла мечтательная улыбка, а карие глаза сияли, делая взгляд еще более выразительным. Не говоря ни слова, Адам подошел ко мне и, опустившись напротив, коснулся своими губами моей руки.

До этой минуты я и подумать не могла, что такой банальный и даже старомодный жест способен вызвать во мне целую бурю эмоций. Но он вызвал. Я смущенно улыбнулась, и Адам улыбнулся в ответ. Своей самой широкой улыбкой, открывающей ямочки на щеках. Забыв о кофе, я коснулась рукой темных волос, что вскоре начнут доставать до плеч, и вновь улыбнулась. На этот раз куда шире.

Адам поцеловал меня, и я вдруг осознала, что люблю его.

Той ночью ни один из нас не смог уснуть. Сначала мы были слишком заняты для такого скучного дела, а затем\ldots{}

-- О чем ты думаешь? -- спросил Адам, когда я собрала растрепавшиеся волосы и растянулась на подушке.

Я повременила с ответом.

-- Я думаю о том, что в бутылке подозрительно много виски как для пятничного вечера.

Адам вздохнул.

-- Не хочешь говорить?

-- Пожалуй, нет, -- медленно ответила я, ласково взглянув на своего собеседника.

Он обхватил меня за плечи и крепко поцеловал.

-- А теперь?

Я мужественно молчала.

-- Нет, ну так не честно! Я думал, мы договорились говорить друг другу всю правду, -- Адам обиженно отодвинулся от меня. -- Милая, ты заставляешь меня нервничать.

Так себе аргумент как для двадцатипятилетнего мужчины, но я и правда не знала, что ответить.

-- Не думаю, что сейчас подходящий момент.

Десятисекундная пауза.

-- Так о чем? -- вновь спросил Адам и я сдалась.

-- Думаю, я люблю тебя.

После этой фразы Адам молниеносно изменился в лице. Позже он утверждал, что знал, что я скажу именно это. Более того, он так настаивал именно потому, что думал, что сможет ответить мне тем же. Однако, в последний момент вдруг понял, что не сможет -- вся суть мужчин.

-- Мне нужна сигарета, -- отрезал Адам и покинул постель.

С этого момента наши мимолетные отношения круто переменились. Адам стал первым мужчиной, к которому я питала сильные чувства во взрослом возрасте. Конечно, мне было всего девятнадцать, но неплохой рабочий стаж, несколько написанных книг, вовремя уплаченные налоги и уйма забот позволяли мне считать себя взрослой.

Мы расстались спустя полтора месяца. Тем приятным погожим днем, когда я меньше всего этого ожидала. Без криков и прочих тягостных выяснений отношений. Он ушел, а я проплакала всю ночь. К утру замкнулась в себе и перекрасила волосы в алый.

-- Нельзя говорить человеку о любви на второй день знакомства! -- заявил Адам спустя дня три после этого самого второго дня знакомства. Кстати, формально мы были знакомы куда дольше, но никогда прежде не виделись. -- Это ненормально!

Никогда не забуду того, с каким чувством он произнес последнее слово. Ненормально. Словно это было не одно, а целых четыре отдельно существующих слова. Не-нор-маль-но. Так обычно разговаривают с детьми или умственно отсталыми.

Куда делся тот парень, который уговаривал меня дать шанс этим отношениям? В одной из моих книг персонаж по прозвищу Молния чуть не лишается своей возлюбленной из-за того, что признается ей в любви на второй день знакомства. Никто и не знает, насколько автобиографична эта сцена.

***

Теперь же меня занесло в Крым.

Поездка оказалась невыносимо долгой. В какой-то момент я даже уснула, что редко бывает со мной в общественном транспорте. Точнее, попросту отключилась. Сегодня ведь уже ни для кого не новость, что происходящее во сне проносится в мозге за долю секунды?

Что ж, тем вечером в моей голове пронеслось немало. Проснулась я в тревожном состоянии, и тут же потянулась к часам. Чувствовала себя так, словно проспала часов двадцать. На деле же -- тринадцать минут. Вот уж не думала, что освоенная в студенческие годы практика кратковременного сна однажды мне пригодится.

Так вот, проснулась я в тревожном состоянии. Мне снился Адам. Как всегда, сон оказался чересчур реалистичным, так что я не сразу пришла в себя. И пускай выглядела я абсолютно спокойной, внутри уже начинал мельтешить страх отношений. С одной стороны, я всем сердцем хотела полюбить кого-то сильнее, чем когда-то любила своего бывшего. С другой -- я боялась этого до умопомрачения. Именно поэтому на протяжении последних трех лет я никому не позволяла сблизиться со мной. Даже поцелуи во время секса были запрещены. Мне думалось, что они вызывают слишком интимную связь, как бы абсурдно это не звучало.

И вот я вновь куда-то ехала. На этот раз без планов о чем-то серьезном и с куда большими надеждами одновременно. От подобных мыслей сердце начинало колотиться так сильно, что становилось сложно дышать.

К тому моменту бутылка виски совсем опустела. Автобус проезжал Симферополь. Он показался мне до невозможности скучным и очень напоминал бесконечную улицу Чигрина -- не самый благополучный район моего родного города.

«Спокойно, Васляева, -- сказала я себе. -- Тебе уже не девятнадцать. Ни в кого ты поспешно не влюбишься.»

Вот она, самая распространенная из всех возможных человеческих глупостей: отчего-то полагать, что ты стал умнее.

\hypertarget{chapter-7}{%
\chapter{~}\label{chapter-7}}

Никто понятия не имел, когда должен прийти мой автобус. По предварительным расчетам, время прибытия -- около шести часов вечера. Но зная отечественную пунктуальность, я в этом очень сомневалась. Веня жил в пяти минутах езды от автовокзала, так что мы условились на том, что я поймаю вай-фай где-нибудь на подъезде к Севастополю.

Как ни странно, мы ехали без остановок, если не считать внезапной поломки автобуса. Следующий пришлось ждать посреди какого-то захолустья, так что вай-фаем там и не пахло. В итоге до пункта назначения я добралась лишь когда стрелка на часах стремительно приближалась в десяти вечера.

И вот автобус, наконец, остановился. Я выбралась на улицу и сладко потянулась. Первым, что привлекало внимание, был свежий морской воздух. Вдохнув его, я вдруг поняла, как же сильно истосковалась по бризу.

Солнце, конечно, уже давно село, прихватив с собой тепло весеннего дня. На мне было легкое платье в пол со спущенными плечами, которое теперь мелодично колыхалось в такт ветру. Становилось как-то прохладно. Вдоволь надышавшись морем, я осмотрелась в поисках подходящего заведения. По-прежнему никаких признаков беспроводного интернета. Естественно, местной сим-карты у меня тоже не было, а моя уже часа четыре как перестала быть активной.

В какой-то степени, тот факт, что на вокзале не было Вениамина, сыграл мне на руку. Таким образом, я могла еще раз все обдумать. И, все же, казалось, кто-то уже давно обдумал все за меня. Несмотря на бешеную усталость и практически полное отсутствие сна, я все еще чувствовала себя отлично. Вполне нормальный расклад. В рассвете бессонницы настоящая усталость обычно накрывает день на третий, а то и пятый, когда к ней подключается светомузыка.

К сожалению, здание вокзала не смогло вместить в себя что-либо кроме касс да комнаты ожидания. Пройдя сквозь нее, я вышла на улицу с обратной стороны и обнаружила, что пошел дождь. Стоило сделать пару шагов к проспекту, как ко мне тут же подбежала организованная группировка таксистов. Тут-то я и вспомнила о том, что адрес Вениамина остался в переписке, так что без интернета было не обойтись.

Как вы уже поняли, я никогда не отличалась привычкой планировать наперед. Если дело касалось работы, тут я стратег высшего разряда, но, когда речь заходит о личном комфорте -- только хардкор, только импровизация.

-- Подвести, красавица? -- хором спросило сразу несколько мужчин.

Я улыбнулась их синхронности. Города меняются, а таксисты везде одинаковые.

Вдалеке я заметила заправку. Она находилась в паре светофоров от меня и походила на те, в которых обычно продаются подозрительно дорогие хот-доги. И есть вай-фай. Таксисты еще что-то продолжали спрашивать, когда я ступила на первый пешеходный переход.

-- У вас виски есть?

-- Только кофе с виски.

-- Давайте самый большой.

Если верить табличке на входе, до закрытия оставалось всего-ничего. Но стоявшая за прилавком женщина никуда не спешила.

-- Паспорт с собой? -- спросила она.

-- Паспорт?

Я даже не сразу сообразила, в чем здесь дело.

-- Да, покажите паспорт. Лицам моложе восемнадцати выпивать не положено.

-- Да вы гоните!

Даже не знаю, что меня удивило больше: проверка документов на заправке или то, что кто-то назвал ирландский кофе
выпивкой. Я еще немного повыделывалась прежде, чем предъявить документы.

Здесь и впрямь был беспроводной интернет. Только вот доступ к нему ты мог получить лишь через местную сим-карту, которой у меня, как мы все уже поняли, не было. Еще и виски ненастоящий! С таким же успехом я обошла еще несколько заведений. Дождь заметно усилился и вот-вот обещал обратиться ливнем. Тогда-то на моем пути встретился какой-то местный фастфуд.

Внутри было пусто. Три столика и все свободны. За прилавком стояла полная девушка. Она красила ногти в желтый цвет и, кажется, была не слишком рада моему визиту.

-- Мы закрываемся через шесть минут.

-- Мне нужно иметь сим-карту, чтоб подключиться к вашему вай-фаю? -- на одном дыхании выпалила я.

Девушка взглянула на меня как на умалишенную. Мне, в общем-то, было не привыкать.

-- Нет. Но вам как бы надо бы чёта купить.

-- Латте, -- заказала я. -- Самый большой.

Людей на улицах было поразительно мало, что создавало особенно приятную, тихую атмосферу. Сидя за столиком у окна, я наблюдала за тем, как потоки дождевой воды текут вдоль улицы, и наслаждалась тремя минутами, оставшимися до закрытия заведения. Мне, наконец, удалось подключится к сети, и я ужаснулась количеству входящих сообщения. Кажется, Вениамин не на шутку разволновался. Я опоздала на четыре часа, а внезапно разыгравшийся шторм изрядно добавлял ситуации трагизма.

Успокоив Веню, я пообещала ждать в холле вокзала. Вечер обещал быть интересным.

\hypertarget{chapter-8}{%
\chapter{~}\label{chapter-8}}

Эта сцена не раз мелькала в моих мыслях до нашей первой встречи с Вениамином. Слушая его голос, собираясь на последний перед отпуском рабочий день, стоя за барной, отправляясь ко сну, сидя за рукописями и бегая за постоянно заканчивающимся льдом, я представляла, как впервые увижу его. Но еще чаще я думала о нашем с Веней знакомстве уже после того, как оно состоялось. Когда наполняла бокал вином, готовила ужин, гуляла около моря, ждала свой кофе в какой-нибудь забегаловке, смотрела, как он точит ножи или засыпает прямо посреди комнаты\ldots{}

И особенно сильно я думала о том вечере после очередной ссоры. Как бы ни обстояли дела, несмотря на всю горечь обиды, одно лишь это воспоминание было способно растопить лед, возникший в моем сердце.

Скажи мне кто-нибудь, что писать о собственной жизни будет так сложно -- ни за что бы не поверила. Казалось бы, ничего не нужно, ведь ты и так все знаешь и помнишь до мелочей. Ты часть истории, по отношению к которой твои собственные пожелания, по сути, не имеют никакого отношения. В этом-то и проблема. Произошло то, что произошло и все, что тебе остается -- проанализировать события и отойти в сторонку, притворившись обычным наблюдателем. Есть второй вариант, не подразумевающий сохранения ясности ума, так что мы его опустим.

Говорят, мне повезло с памятью. Я помню все имена, даты, события, диалоги и так далее до подробнейших деталей. Могу без труда вспомнить, во что был одет человек пять лет назад, что он говорил, с каким настроением и почему. Могу сказать, какая за окном стояла погода, что за музыка играла в проезжающей мимо машине и о чем я в тот момент думала. Эйдетизм подразумевает собой особый вид памяти. Совокупность зрительных образов с показаниями остальных счетчиков твоего организма: слуховых, вкусовых, тактильных, обонятельных, двигательных и даже интуитивных. При правильной концентрации можно перенестись в любое место, достаточно лишь разок побывать там, но есть у эйдетики один жирный минус. В справочниках, правда, он подается как большой плюс, но что они в этом смыслят: «эйдетические образы отличаются от образов восприятия тем, что человек как бы продолжает воспринимать предмет в момент его отсутствия».

Отсутствие -- ключевое слово.

***

Звуки и запахи -- это вообще отдельный разговор. Правильное сочетание этих элементов буквально открывает мне окно в прошлое. С такими манипуляциями памяти не мудрено влипнуть в депрессию.

Спустя год после болезненного расставания с Адамом мне нужно было съездить в Черниговскую область. Добиралась я, естественно, таким же разваливающимся поездом, каким когда-то добиралась к своему молодому человеку. Был прохладный осенний вечер сродни тем вечерам, когда я только раздумывала о перспективе отношений с Адамом. Тогда у меня были духи с запахом лаванды. Они пылились на полке с две тысячи тринадцатого, и черт меня дернул воспользоваться ими тогда, оправляясь на железнодорожный вокзал.

Запах парфюма и шум поезда буквально свели меня с ума. Картина перед глазами оставалась прежней, но в какой-то момент, я была уверена в том, что мне все еще девятнадцать, а поезд везет меня в Москву, и уже утром меня встретит туман Киевского вокзала. Из толпы на перроне покажется Адам -- в своей старой кожаной куртке -- и поспешит навстречу, чтобы помочь спустить с поезда тяжелый чемодан.

Не мудрено, что Шульц так ярко описывал приступы шизофрении с примесью самоубийств, возникшие в итоге настойчивых попыток вызвать из памяти эйдетические образы. Но что делать, когда они всплывают против твоей воли?

История закончилась тем, что я вылетела на улицу при первой же возможности, а затем стояла посреди какой-то неизведанной сельской местности и наблюдала за тем, как состав плавно удаляется в сторону заката.

Последнее, кстати, вовсе не метафора. В этот момент солнце действительно опускалось за горизонт, приглушая осенние оттенки. Я смотрела на исчезающий за небосводом алый диск, опавшие листья и разноцветные деревья, колышущиеся на ветру. Смотрела и думала, какая же я дура.

***

Вернемся к относительной реальности.

Первая половина мая шестнадцатого подходила к концу. Закинув ногу на ногу, я сидела в крохотном зале ожидания, который, по совместительству был холлом, а также единственной комнатой в здании севастопольского вокзала. Кроме меня здесь никого не было, так что я позволила себе дважды покурить. Честно говоря, я настолько волновалась, что курнула бы еще не раз и отнюдь не табака, но как-то не хотелось создавать вокруг себя газовую камеру. Под дождь выходить тоже не хотелось, и я просто продолжала сидеть, закинув ногу на ногу. Сперва мне думалось, что я читаю, но спустя минут пять пришлось признать, что я уже раз семнадцатый начинаю один и тот же абзац, но так и не добираюсь до середины.

\emph{Произносить в голове маячащий перед глазами текст и думать о чем-то на фоне -- поразительная, но такая бесполезная способность человеческого мозга}, подумала я, и оставила книгу в покое.

По обе стороны от меня находились большущие старые окна, в которые я то и дело поглядывала. К тому моменту мои каштановые волосы уже были достаточно длинными и объемными, дабы за ними спрятаться. При этом они были вьющимися, что позволяло мне откинуть парочку прядей и незаметно бросать пристальные взгляды то вправо, то влево. Стоит ли говорить, что сидела я при этом прямо и расслабленно, упорно соблюдая видимость чтения?

Вопреки всем моим усилиям, размытые потоки воды да отблески проезжающих по проспекту машин были всем, что я видела. В итоге я решила забить на это неблагодарное дело и попыталась еще раз погрузиться в книгу. К тому же, до окончания «Бродяг» оставалось всего ничего. Только вот я сама уже давно чувствовала себя той еще бродягой, но только сейчас впервые всерьез задалась вопросом: куда эта Дхарма меня приведет?

Я отбросила с лица надоедливую прядь волос -- прядь в действительности крайне надоедлива, так что вы еще не раз прочитаете о том, как я ее отбрасываю -- и склонилась над дорожным романом, когда дверь слева от меня распахнулась. Естественно, я знала, кто сейчас в нее войдет, но не стала поспешно отрываться от чтения.

Знаете, говорят, что женщина может думать миллион мыслей одновременно. Вообще-то, так и есть. За те несколько секунд, которые понадобились Вениамину, чтобы подойти к моему креслу, я задала себе как минимум десять вопросов.

\emph{Подправила ли я помаду? До конца ли высохли мои волосы? Где мой паспорт? Зачем мне сейчас паспорт? Успела ли я втянуть живот? Не прилипло ли платье к груди? А если прилипло, то насколько это сексуально? Достаточно ли непринужденно я сижу? Не слишком ли непринужденно я сижу? Пахну ли я также хорошо, как пять минут назад?}

И так далее, и тому подобное. Не поверите, но я даже успела запомнить номер страницы, на которой остановилась. Сейчас это кажется мне смешным. Как и многое другое.

Позже Веня говорил, что я держалась с ним так, словно мне было глубоко насрать, и это его окончательно очаровало. Такой вот у меня бесполезный талант: выглядеть хладнокровно, когда на самом деле внутри все сжимается от волнения и сердце скачет как курс доллара в Украине с наступлением военного положения. Видимо, по этой простой причине я неплохой игрок в покер.

Так вот, дверь распахнулась. Спустя пару мгновений я позволила себе поднять глаза. Передо мной стоял высокий блондин с зелеными глазами и хемингуэевский бородой. На нем была фланелевая клетчатая рубашка, которую затем я видела так часто, что та успела мне надоесть, и невероятно ласковая улыбка, которую я видела еще чаще. Она мне никогда не надоедала.

Откровенно говоря, я всегда думала, что предпочитаю длинноволосых брюнетов со смуглой кожей. Однако, увидев Вениамина, я поняла, что никогда не встречала мужчину прекраснее. У него была бледная кожа, длинный андеркат и профиль настоящего викинга. Волосы частично скрывала небольшая шляпа с полями.

Веня улыбнулся мне еще шире, и я вдруг осознала, что пялюсь на него уже с добрых полминуты. С другой стороны, он занимался тем же, так что все было в порядке. Когда наши взгляды встретились, Вениамин развел руки, предлагая мне окунуться в его объятья. Хотя, учитывая путь, который я проделала тем днем, правильней было бы сказать «упасть».

Он обнял меня, не говоря ни слова. Так мы и стояли в полнейшей тишине пустующего вокзала. Где-то очень далеко продолжали шуметь автомобили, а дождь играл ночную мелодию на оконных стеклах и тротуаре.

\hypertarget{chapter-9}{%
\chapter{~}\label{chapter-9}}

-- Твой самый большой недостаток? -- спросила я.

Мы шли вдоль стеллажей небольшой винной лавки.

-- Я алкоголик, -- спокойно ответил Веня. -- Алкоголь -- моя единственная проблема. Порой я не ведаю что творю, отчего друзья зовут меня Невероятный Алк. Но в остальном я идеален.

Он ухмыльнулся.

Я обвела взглядом стеклянные бутылки.

-- С вином тоже проблемы?

Он покачал головой.

-- В основном с водкой.

-- Отлично, -- я протянула руку к красному полусладкому. -- Тогда берем две.

Определившись с выпивкой, мы все так же продолжали прогуливаться вдоль винных полок. Веня держал меня за руку. Периодически я ловила на себе его смущенные взгляды. За прошедшие годы мужчины смотрели на меня по-разному. Большинство с желанием. Некоторые как на полнейшую стерву, но все с тем же желанием. Были даже те, в чьем взгляде читалось восхищение, но все это казалось мне поддельным. И вот, разгуливая по алкогольному кварталу, я осознала, что уже очень давно никто не смотрел на меня так.

-- А твой? -- поинтересовался мой спутник.

-- Мой самый большой недостаток?

-- Ага.

-- У меня биполярное расстройство.

-- Что это?

-- Маниакально-депрессивный психоз, -- объяснила я. -- Мне это название даже больше нравится. Оно звучит понятней, но
теперь считается оскорбительным, так что\ldots{}

-- Хорошо, давай поменяем правила, -- предложил Веня. -- Той самый большой плюс.

Теперь пришла моя очередь ухмыльнуться.

-- У меня биполярное расстройство.

-- И в чем здесь проблема?

-- Всегда по-разному.

-- Например? -- не унимался Веня.

Я вздохнула и закатила рукава, демонстрируя выдающуюся коллекцию шрамов с обеих сторон рук.

-- И всего-то? -- еще одна ласковая улыбка.

Я подняла рукав выше.

-- Еще бывает сигареты об себя тушу. Это не симптом, а, скорей, последствия, но, говорят, окружающих напрягает.

Мой спутник вручил мне бутылки и принялся закатывать рукава. Руки под рубашкой украшало куда большее количество шрамов. Они выглядели на удивление красиво, если не считать парочки партаков, одним из которых была эмблема супермена. Вот как нужно производить впечатление на женщину. Спрашиваешь о ее недостатке, а затем даешь понять, что для тебя это всего лишь пустяк.

Я весело рассмеялась. То ли шрамам, то ли татуировке.

-- Серьезно, -- спросила я, вскинув бровь. -- Супермен?

-- Как видишь. Терпеть его не могу.

-- Так что же не перебьешь?

-- Ну, так в этом-то и смысл, -- парировал Вениамин, направляясь в сторону кассы.

***

Звездное небо раскинулось над самым прекрасным из всех городов. Я видела в Севастополе уйму недостатков, но никогда не прекращала любить его. Думаю, с Вениамином было так же.

Очарованные вечером и компанией друг друга, мы сидели на парапете посреди пустой улицы. Несмотря на то, что дождь практически закончился, Вениамин держал надо мной зонтик. Все это время мы то без умолку болтали, то не могли прекратить целоваться. Обе винные бутылки стояли рядом. Никто к ним даже не притронулся.

Первым молчание нарушил Вениамин.

-- Твои волосы еще красивее, чем на фото.

-- А что насчет остальных частей меня? -- я повернулась к (своему?) мужчине, изображая пристальный взгляд.

-- Ты очень красивая, Васляева. Именно такой я тебя и представлял. И ты лучший в мире собеседник. Я окончательно очарован.

Я покраснела. Впервые за долгие годы существу мужского пола удалось меня смутить. Заметив это, Веня лишь крепче прижал меня к себе.

Море находилась совсем недалеко. Ночной ветер доносил до нас его ностальгические нотки, создавая иллюзию оазиса. Я набрала полные легкие свежего воздуха и медленно выдохнула.

-- Люблю я море.

-- А я -- тебя, -- отозвался Веня.

Здесь мне должен был вспомниться Адам, то, как я призналась ему в любви на второй день знакомства и все такое прочее. Но он мне почему-то не вспомнился. Я вообще напрочь забыла о его существовании.

-- Что ты сейчас сказал?

Я в удивлении отстранилась, на что Веня лишь развел руками.

-- Я люблю тебя, Васляева.

-- Секундочку. Ты признаешься мне в любви в первый день знакомства?

-- Так и есть. Ничего не могу с этим поделать. Да и формально мы знакомы гораздо дольше\ldots{}

\emph{Где-то я это уже слышала.}

Казалось, вся моя болтливость иссякла. Я просто сидела, переводя взгляд с Вени на сигарету в его руке и обратно. Затем пришла к самому мудрому в сложившихся обстоятельствах решению, и подкурила.

Тем временем Вениамин взглянул на часы.

-- Совсем забыл о времени, ты ведь устала! -- спохватился он. -- Пойдем.

Так мы стали жить вместе, фактически зная друг друга не больше трех часов.

\hypertarget{chapter-10}{%
\chapter{~}\label{chapter-10}}

Следующее утро мы как-то пропустили. Нежились в объятьях друг друга, пока не заметили, что дело близится к вечеру. Вино так и стояло закрытым.

-- Чем хочешь заняться? -- спросил Веня, наблюдая за тем, как я выбираюсь из ванной.

-- Хочу прогуляться.

-- Поехали в центр? -- предложил он.

И мы поехали. Несмотря на то, что солнце только-только село, снаружи отнюдь не было людно. Причем чем ближе мы подходили к центру, тем меньше пешеходов встречалось на нашем пути. Вероятно, это была старая часть города потому, как местами увиденное очень напоминало переулки Львова, что так дороги моему сердцу. Отсутствие публики создавало сказочное впечатление, словно не только этот вечер, но и все улочки, скамейки, парки -- все это принадлежало лишь нам двоим. Словно мы -- Маленький Принц, гуляющий по собственной планете.

-- Я покажу тебе свое новое любимое заведение, -- пообещал Вениамин.

-- Питейное?

-- Ну, а то! Тебе понравится.

Оказалось, что все горожане прятались по кафешкам. Мы проходили мимо множества баров, пабов, пиццерий и прочих угодий гедонизма, и в каждом из них виднелись переполненные залы.

Должна признать, «Фикус» и впрямь был миленьким местом. Снаружи он выглядел как крохотное ретро-кафе, но на деле оказался куда больше. Выполненный в моих излюбленных тонах интерьер не мог не ласкать глаз: черные диванчики и такие же черные столики с матовым покрытием. В этом радужном месте мы сделали наше первое совместное фото.

Шоты принесли на самом настоящем бревне, что было слегка необычно. И я говорю не обо всех этих выточенных подносах, что так популярны в социальных сетях, нет. Я говорю о самом настоящем бревне, которое можно встретить в лесу.

Вслед за шотами за наш столик пожаловало пиво для Вениамина и винишко для Елизаветы. Девица за соседним столиком бесстыдно пожирала моего спутника взглядом на протяжении всего вечера, тогда как взгляд Вениамина не отрывался от меня.

\emph{Барышня, вы меня сейчас выбесите}, сказал мой взгляд.

Но девица не отреагировала. Она постоянно поправляла волосы, закусывала губу и оттягивала майку как можно ниже.

Веня отлучился в уборную, и девица тут же подорвалась следом. Вероятно, она надеялась «случайно» столкнуться с ним в дверном проеме. Знаю я эти ваши женские штучки. Короче говоря, долго не думая, я запустила в уже начинавшую меня подбешивать барышню перечницей, после чего та села на место и тупо уставилась в пол. Как ни в чем не бывало. Неужели обязательно было до этого доводить?

Вскоре принесли ужин.

-- За мою ненаглядную деточку, -- произнес Вениамин, поднимая бокал.

-- За моего крымского дровосека, -- ответила я.

Разрезав пиццу, Веня потянулся за солью и\ldots{}

-- Милая, а куда делся перец? -- изумленно спросил он.

-- Понятия не имею, -- ответила я и приложилась к вину.

***

-- Что не так? -- спросил Веня, когда мы возвращались домой последним троллейбусом.

Время стояло позднее, и мы опять были совсем одни.

Я лишь отмахнулась.

-- Пожалуйста, скажи мне. В чем дело, любимая?

\emph{Любимая}. Это слово вызвало во мне ураган противоречивых чувств, в основном состоящий от всяческих производных счастья и страха.

-- Ты мне очень нравишься, -- наконец, ответила я, -- но люди слишком часто разбивают мне сердце.

-- Черт. Ты любишь кого-то другого?

-- Вовсе нет. Просто я боюсь открыться тебе. Боюсь своих чувств и того, какие у них могут быть последствия.

-- И это тебя печалит?

-- Не совсем. Мне кажется, что этот страх мешает мне почувствовать настоящую радость от того, что между сейчас нами происходит. Ты понимаешь?

-- Я понимаю, что кто-то очень сильно тебя обидел.

Я вздохнула.

-- Когда закончились твои последние отношения? -- я пристально посмотрела на Веню.

На этот раз по-настоящему.

-- Перед новым годом.

-- Но ты виделся с ней и после, не так ли?

Вениамин удивленно взглянул на меня.

-- Нет, не виделся.

-- Уверен?

Он слабо кивнул.

-- А я вот думаю, виделся.

-- С чего это ты взяла? -- осторожно спросил он.

-- С того, что большинство мужиков так делает. Спит со своими бывшими потому, что это удобно. Но только с теми, с кем они состояли в длительных отношениях. Когда секс уже входит в привычку.

-- Ладно. Ты права.

-- Так, когда ты в последний раз с ней виделся? -- спокойно спросила я.

-- В марте.

-- Значит, два месяца. И ты считаешь, что уже готов к новым отношениям?

-- Абсолютно, -- без малейшей заминки ответил Веня.

-- Ты ведь понимаешь, что это глупо вступать в новые отношения, когда ты еще не оправился от предыдущих?

-- Это не тот случай.

-- Почему?

-- Я ее не любил.

-- А почему так уверен, что меня любишь?

-- Твою мать! -- внезапно воскликнул Веня. -- Мы проехали остановку!

Я посмотрела в окно. Действительно, дом Вениамина остался позади, в свете ночных фонарей.

Пришлось возвращаться пешком.

-- Послушай, я несколько лет переживала болезненный разрыв отношений, которые и длились то не особо долго. А все потому, что и до этого со мной поступали не самым лучшим образом. Сейчас я в порядке. Не как другие люди, конечно, но в порядке по сравнению с тем, как было. Но я не чувствую, что могу справиться с подобной ситуацией еще раз. К тому же, с каждым разом чувства все сильнее, а значит, и боль становится глубже. Иногда мне кажется, что от моего сердца уже совсем ничего не осталось. Одни лишь осколки, но затем оп, и новое предательство. И вновь эта режущая боль. Но это, наверное, хорошо. Значит, есть чему биться.

Вениамин слушал меня с самым серьезным видом. Он молчал и лишь подкуривал одну за другой.

-- Я не обижу тебя, -- уверенно произнес он. -- Буду собирать твое разбитое сердце по кусочкам до тех пор, пока ты не почувствуешь себя в безопасности.

Его голос звучал настолько искренне, что я с трудом сдерживала слезы.

-- А потом?

-- А потом просто буду любить тебя. Хотя постой\ldots{} Я ведь уже люблю.

***

-- Ну, так что, Васляева, ты дашь мне шанс? -- спросил Вениамин.

Мы сидели на балконе. По правую руку расположилось море, по левую чернел лес, впереди мерцал ночной Севастополь.

Я выпила целую чашку кофе прежде чем ответить на этот вопрос.

-- И вот еще что, -- между тем произнесла я. -- Никогда не ври мне.

Он уже открыл рот, дабы что-то сказать, но я тут же продолжила.

-- Нет, я серьезно. Никогда не думай врать мне. Особенно по мелочам как сегодня. Меня достаточно обманывали, чтобы я могла распознать ложь. Я это ненавижу. На самом деле, я многое могу простить, но не обман.

-- Я тебя понял, -- виновато произнес Веня.

-- Так, когда ты в последний раз виделся со своей бывшей?

Он сделал небольшую паузу.

-- Три недели назад.

-- Хорошо.

-- Так вот\ldots{} Ты дашь мне шанс?

-- Да, кажется, уже дала.

Услышав это, Веня буквально засиял. Даже в полумраке была видна его довольная улыбка. Мы еще долго целовались, слушая, как где-то вдалеке волны бьются о берег.

\hypertarget{chapter-11}{%
\chapter{~}\label{chapter-11}}

Сколько бы воды не утекло, я по сей день считаю, что самые обворожительные вечера можно встретить лишь в Севастополе. В какой бы части города ты не находился, бриз найдет тебя. Морской воздух и волшебные закаты. Слушая меня, можно решить, что все дело в моей любви к Вениамину Ларионову. Что ж, первые припадки взрослых чувства я испытала задолго до описываемых событий. Было это на родине Булгакова, Пастернака, Маяковского, Мандельштама\ldots{} Список можно продолжать до следующей главы, но итог один: в Москву я так и не влюбилась.

Севастополь же мигом покорил мое сердце. Этот горный городок с его извилистыми дорогами, сумасшедшим трафиком, гранитными набережными и залитыми солнцем скверами вызывал у меня необъяснимое чувство ностальгии. Вы когда-нибудь окунались в прозу Брэдбери? Я имею в виду не сай-фай, а настоящие, автобиографические тексты вроде «Вина из одуванчиков» и «Лето, прощай!»? Душевная теплота и приятная меланхолия с едва заметной толикой грусти -- вот что несут в себе это книги. Вот что несет в себе Севастополь. Здесь солнце садится гораздо позже привычного времени, а море практически никогда не замолкает. И как же поразительно Севастопольское небо!

Тем не менее, дни в Крыму оказались куда более жаркими, чем мне представлялось. Будучи уроженкой юга, я все равно умирала от жары.

-- Зачем вообще строить города в месте, где настолько жарко? -- капризно спрашивала я, но Веня лишь рассмеялся.

Одним из таких убийственно жарких дней мы растянулись в постели. Веня лежал вдоль кровати, как и подобает нормальному человеку. Я же расположилась поперек, уложив голову на бедра своего благоверного и закинув ноги на стенку. Оба абсолютно голые. Я читала, Веня смотрел какое-то видео.

-- Сестра написала, -- сказал он.

-- И что там?

-- Говорит, бабушка уже обзвонила всех родственников. Рассказывает им, что я с невестой живу.

-- С невестой?

-- Ну!

-- А где она? Ты нас не представишь? -- я улыбнулась, не отрываясь от книги.

***

Знаете, Жорж Санд ежедневно трудилась над своими романами до половины одиннадцатого утра. Она не останавливалась, даже если заканчивала книгу, а лишь откладывала ее в сторонку и бралась за новую. Вот так просто, без набросков, творческих перерывов и традиционных писательских сумасшествий в честь законченного романа. С другой стороны, Хемингуэй писал по пятьсот слов в день. Минимум. Он брался за работу с первыми лучами солнца и, бывало, засиживался до полудня. Буковски тоже здоровски писал, от десяти до тридцати страниц за ночь, которые потом вычитывал в разгар полуденного похмелья. Ну, и, конечно, Стивен Кинг -- гуру писательского тайм-менеджмента. Пишет систематически да при любом удобном случае уже, должно быть, лет шестьдесят.

Подобных историй миллион и я всегда мысленно восхищалась такими людьми. Дело здесь не в организованности и воодушевлении, с которым каждый божий день они начинали творить. Я говорю как раз о том, что в назначенное время все эти авторы могли спокойно закончить работу и вернуться к жизни. Что же касается меня\ldots{}

Это, скорей напоминало распорядок дня Хантера Томпсона. Только на кокаин не всегда хватало, так что его приходилось заменять депрессией. Это, конечно, шутки, но! Томпсон вставал в 3 часа дня и активно начинал закидываться виски, коксом и табаком, изредка прерываясь на кофе, обед и курение травы, «чтобы снять напряжение дня». Затем снова кокаин, Чивас, трава и сигареты, к которым добавились капли кислоты и французского ликера, и вуаля:

\emph{12:00, полночь, Хантер С. Томпсон готов писать.}

Такая же хрень преследовала меня по жизни. Процесс переключения с официально признанной реальности на мою собственную требовал чуть ли не отшельничества, а потому в бытовых условиях работы, проблем и социума занимал немало времени. Но стоит мне переключиться -- к ужину не ждите и к завтраку тоже.

У меня есть отличная для писателя и ужасная для остальных аспектов моей жизни привычка. Я не могу оторваться от писанины даже если знаю, что ужасно опаздываю. Прекрасно вижу, который час, осознаю, что должна была выйти из дома еще двадцать минут назад, но упорно продолжаю сидеть за рукописью потому, что все остальное становится чертовки неважным. И дело здесь не в том, чтобы просто закончить предложение, диалог, или даже целую главу. На самом деле я попросту не могу остановиться, пока чувствую, что мне по-прежнему есть, что сказать.

Порой мне кажется, что ничего не изменится, случись хоть конец света. По сторонам будут раздаваться взрывы, мелькать вспышки и пролетать травмоопасные предметы, а я все так же преспокойненько продолжу сидеть, склонившись над текстом. Такой вот я человек.

Кстати, это касается не только письма. Чтение также частенько берет надо мной верх. С самого детства, о чем бы не оповестили меня родители, я автоматически отвечала:

-- Сейчас. Только главу дочитаю.

И не важно, что до конца этой главы может быть еще страниц пятьдесят.

Скажу больше, однажды я даже проехала нужную остановку. Было это не где-нибудь в городе, а за его чертой, в незыблемой глуши наполовину заброшенных дачных товариществ. Прекрасно помню этот вечер. На улице стояла зима, минусовая температура. Мне было двадцать, и я обещала Мишель приехать последним транспортом.

И вот я сижу у окна катившей прочь из города маршрутки. Вокруг все завалено снегом. На улице собачий холод и транспорт ходит ужасно. Мне повезло, встретить водителя, дом которого находился в нужной стороне, так что я сижу у окна и читаю мемуары вдовы Йена Кёртиса. Кто-то в салоне заказывает мою остановку, и я уже вижу, что до нее остается парочка метров. Маршрутка останавливается, люди начинают выходить.

\emph{Сейчас, только страницу дочитаю}, говорит внутренний голос.

Следующее, что я помню -- напрочь ошарашенный вид подруги, что ждала меня на остановке. Рот открыт, нижняя губа отвисла, глаза широко распахнуты. Ее лицо словно говорило: «Какого хуя?». Я пытаюсь подавить приступ подкатывающего смеха, но Мишель уже исчезает за горами снега, а маршрутное такси увозит меня в направлении загородного дурдома.

***

Спустя два года и одну незабываемую весну Веня задал мне какой-то вопрос. Случилось это как раз, когда я достигла середины свежего кинговского рассказа. И я ответила в лучших традициях Елизаветы Васляевой.

-- Сейчас. Только главу дочитаю.

Мой ответ почему-то развеселил Вениамина. Он заулыбался словно ребенок и принялся покрывать мое лицо поцелуями. До сих пор не знаю, о чем Веня спросил меня тем душным майским днем. Хотя кое-какие подозрения, все-таки, имеются

Наконец, на смену дню пришел вечер, которого -- я уверена -- с нетерпением ждали все четыреста шестнадцать тысяч двести шестьдесят три человека, проживающих в Севастополе весной две тысячи шестнадцатого. Жара отступила.

Вениамин недавно вышел из душа. Он курил, облокотившись на поручни балкона, а я сидела в тени, глядя на то, как его волосы цвета пшеницы, игриво переливаются в закатных лучах. Мы много говорили, рассказывая друг другу о прошлом. Делились душевными переживаниями, открывали свои маленькие тайны и отвечали на неловкие вопросы.

А еще мы много болтали. Щебетали обо всем подряд. Мы прожили вместе всего три дня, но, казалось, были знакомы вечность.

-- Я словно знаю тебя всю жизнь, -- сказал Веня.

Он определенно читал мои мысли.

-- Ты просто перевернула мое представление о женщинах. Я никогда не думал, что могу быть так счастлив. И я так сильно люблю тебя.

-- Ох, Веня\ldots{}

-- А ты со мной счастлива?

И я заплакала. От счастья. Впервые за миллионы световых лет.

***

Секс с Вениамином был потрясающим. Мы занимались им часами, прерываясь лишь на кофе и сигареты. До знакомства с Веней я особо не понимала, отчего вокруг куни создают так много шума. Оказалось, никто из моих бывших любовников просто не знал, как это делается. Но вот наступил тот день, когда миф о существовании оргазма от куннилингуса, наконец, перестал быть мифом.

О своей любви Вениамин говорил невероятно часто. Более того, он даже стал называть меня не иначе как «любовь моя», лишь изредка прерываясь на «деточку» и «любимую». Я же ни разу не сказала Вене о своих чувствах. Их становилось все больше, но у каждого из нас есть свои незавершенные гештальты, так что я просто не могла вымолвить эти три слова. Порой мне даже становилось от этого неловко.

-- Итадакимас! -- радостно крикнул Веня, доедая омлет, а затем добавил более спокойным тоном: -- Было очень вкусно.

-- Делов-то.

Он стал застегивать рубашку.

-- Неля поднесет прайсы к остановке. Заберу их и сразу к тебе, -- Веня уже натягивал джинсы.

В те дни Вениамин работал в одной относительно известной компании. Он был региональным супервайзером. Распространял продукцию для салонов красоты и барбершопов. Вопреки громкому названию, по Крыму у него было только двое подчиненных и большую часть работы Веня по-прежнему делал сам.

Прекрасно помню наш первый телефонный разговор. Он состоялся за несколько недель до встречи. Это был день рождения моего бывшего, и я чувствовала себя крайне паршиво. Узнав об этом, Вениамин решил меня подбодрить. Я была на работе, в «Штиле», и Вене пришлось ждать до трех часов ночи, пока моя смена, наконец, не закончилась, после чего он как ни в чем не бывало принялся рассказывать все, что знал об оттенке моих волос. Монолога вышло минут на тридцать.

Я крепко поцеловала его и чмокнула в щеку. Для уверенности.

-- Я люблю тебя, -- весело попрощался Вениамин.

Не дожидаясь пока входная дверь захлопнется, я проскочила за ним в коридор.

-- Венечка, -- начала я. -- Ты мне безумно нравишься, но ты ведь понимаешь, что сейчас я не могу ответить тебе тем же?

Из-за этого я чувствую себя неловко каждый раз, когда ты говоришь, что любишь меня. Не пойми меня неправильно, но мне очень сложно привыкнуть.

-- Привыкнуть к чему?

-- К тому, что обо мне заботятся. К тому, что я кому-то дорога. К тому, что меня любят, в конце концов.

У Вени зазвонил телефон. Видимо, Неля уже ждала его на остановке. Не обращая на это никакого внимания, Веня сжал мою руку.

-- Я буду ждать, -- пообещал он, глядя в мои глаза.

Ждать пришлось недолго.

\hypertarget{chapter-12}{%
\chapter{~}\label{chapter-12}}

После обеда мы решили отправиться в лес. В распоряжении Вениамина была дряхленькая рабочая «Таврия». Сперва машина меня ужаснула, (не столько потому, что я предпочитаю менее отечественного производителя, а лишь потому, что детство мое прошло в такой вот тарахтайке, а ребенком я и метра не могла проехать без того, чтоб меня не укачало) но вскоре эти поездки стали приносить нам немалую радость. В основном, благодаря реакции окружающих.

Просто представьте эту картину! С одной стороны, Вениамин, сидящий за рулем, в своей несменной пижонской шапочке, с хвостом золотистых волос, одетый в рваные джинсы, вечную клетчатую рубашку и грубые, незашнурованные ботинки; его руки в бесконечных шрамах и браслетах, а пальцы -- в кольцах -- весело стучат по баранке в такт музыке. С другой стороны, я, сидящая на пассажирском, с мастерски уложенной шевелюрой, неизменными алыми губами, в макияже, достойном ковровой дорожки, и с огромным стаканом кофе в руках. Оба с сигаретами. Оба в татуировках. Оба влюблены. Несемся куда-то в разваливающейся на ходу «Таврии» да распеваем странные песни.

\emph{When the truth is found to be lies}
\emph{And all the joy within you does\ldots{}}

Короче говоря, реакция была что надо.

\emph{Don't you want somebody to love?}

-- Давно ты водишь? -- спросила я.

-- Шесть лет, -- ответил Веня, лихо выворачивая баранку.

Прежде чем объединиться с природой, мы остановились на заправке. Прогулялись вдоль рядов с безделушками, пообедали горячими сосисками и пополнили запас кофе. Так начались наши заправочные путешествия. На протяжении последующих месяцев мы с Вениамином побывали на множестве заправок и стали экспертами в вопросах вроде: где выбор шоколада больше, а пенка в капучино выше.

Веня всегда любил эспрессо, максимум -- американо. Обычно вместе с ними давали крохотные шоколадки. Я же с подросткового возраста не признавала кофе без молока, но, в отличие от Вени, обожала шоколад. Порой у меня создавалось впечатление, словно он вовсе не хочет кофе и берет его лишь для того, чтобы порадовать меня очередной миниатюрной сладостью.

Нам нравилось завтракать на заправках. Была в этом какая-то особая романтика.

В итоге, мы отдалились от жилой зоны и теперь ехали вдоль деревьев с широкими кронами, что по обе стороны покрывали обочину. Вскоре мой Шумахер свернул на усыпанную песком тропинку. Вокруг все так и кипело зеленью. Приглушив музыку, мы полностью опустили стекла и какое-то время ехали молча, наслаждаясь умиротворяющей обстановкой.

Разодетые в зелень деревья ласково шептались, периодически касаясь друг друга ветвями. Птицы щебетали невозможно громко. Их голоса перекрывали предзакатные порывы ветра. В воздухе по-прежнему пахло морем, но теперь к этому добавились еще и запахи лесных деревьев. Опьяняющее сочетание.

Придерживая страницы пальцем, я цитировала Вене Керуака:

\emph{Так бывает в лесах, они всегда кажутся знакомыми, давно забытыми, как лицо умершего родственника, как давний сон, как принесенный волнами обрывок позабытой песни, и, больше всего -- как золотые вечности прошедшего детства или прошлой жизни, всего живущего и умирающего, миллион лет назад вот так же щемило сердце, и облака, проплывая над головой, подтверждают это чувство своей одинокой знакомостью.}

-- Красиво, -- заключил Веня.

Временами он бывал молчалив. Даже слишком. Я часами болтала, рассказывая о детстве, книгах, напитках, путешествиях и планах на будущее, а мой спутник лишь внимательно слушал. Однако, чем ближе подбирался вечер, тем оживленнее становился Вениамин. Порой я даже уставала от его внезапной активности.

Спустя несколько минут тропинка стала совсем узкой. Мы вышли из машины. Подкурив, Веня тут же уселся на капот, привычно скрестив ноги. Он смотрел куда-то вдаль и затягивался своей до невозможности крепкой сигаретой. Таким я видела его миллион раз: в парках, дворах, у кофеен, на парковке, около моря и на проспекте. Махатма Ганди двадцать первого века.

Пройдя сквозь сосны, мы вышли на холмистую поляну, со всех сторон окруженную деревьями. Впереди был неглубокий овраг, и мы теперь шли вдоль него, взявшись за руки. Помимо шляпы, на Вене были классические брюки, черная рубашка и серая старомодная жилетка. Картину довершал алый атласный галстук. Неожиданный наряд для вечерней прогулки по лесу.

Закат близился, так что Веня уже увлеченно рассказывал мне что-то о маленьких острых перцах под названием халапеньо. В отличие от меня, он обожал острую еду. И, говоря это, я не имею в виду васаби или зернистую горчицу.

-- Я люблю что-то менее острое, -- ответила я, когда Веня закончил свой монолог.

-- А я -- тебя.

Опять двадцать пять.

Сказав это, он продолжал весело шагать через поляну.

-- Правда?

-- Что?

Я остановилась и взглянула на Вениамина со всей серьезностью, которая только может быть присуща двадцати двух летней девушке.

-- Ты действительно любишь меня?

Он обхватил мое лицо руками и улыбнулся.

-- Ну, конечно, люблю, деточка.

Сложно объяснить, что я почувствовала в тот момент. Кажется, я впервые восприняла его слова серьезно и разрешила себе мысли о мелькающем впереди хэппи-энд. Несмотря на то, что пошел дождь, и на улице заметно похолодало, мне никогда в жизни не было так тепло.

Мы едва успели добежать до машины, когда дождь обратился ливнем.

***

Над городом бушевал настоящий шторм. Небо затянуло широкой серой пеленой, и только вдали виднелся крохотный просвет. Скопившиеся над Севастополем облака сочетали в себе все оттенки серого. Море превратилось в дребезжащее полотно; его синева сделалась почти черной. Одетые лишь наполовину, мы сидели, прислонившись к балконной стене, и наблюдали за разыгравшейся непогодой.

-- Очевидно, этим вечером придется остаться дома, -- заметила я.

Веня ответил не сразу. Глядя в окно, он витал где-то в своих собственных мыслях.

Тем временем я не отрывала глаз от Вениамина. Как и у меня, суммарно у Вени было три законченных татуировки. Не одна из них мне не пришлась по душе, но недавно он сделал четвертую -- античный профиль бородатого мужчины в венке, окутанный листьями. Татуировка мне нравилась. Она занимала целое предплечье, и временами, во время секса, я любила шутить о том, что из кустов за нами подсматривает какой-то мужик.

За окнами стремительно темнело. Дождь сменился градом. На балконе стало слишком шумно и ветрено. Вспомнив о вине, мы поспешили внутрь.

До встречи со мной Вениамин провел в Севастополе всего пару месяцев. Сам он был родом из Керчи и, закончив одни и без того обреченные, по его словам, отношения, захотел переехать. Начальство предложило ему работать в Севастополе, на что Веня согласился без лишних раздумий.

Теперь единственным интернетом, который оказался в распоряжении Вени, был мобильный, но и тот не особо гладко работал из-за расположенных неподалеку вышек. Мне нравилась такая изоляция. Никаких тебе обновлений, соцсетей и сомнительных сообщений от дальних родственников. Никаких сталкеров, хейтеров и прочих онлайн-личностей, о которых я даже не знала, чем заслужила такое внимание.

Вместо этого каждый день, в одно и то же время, мы ходили в какое-нибудь тихое заведение. Я ловила вай-фай и созванивалась с близкими. Правда денег не всегда хватало, так что иногда мы делали вид, что читаем меню. Изучали его так долго, что я могла полноценно поговорить с подругами, а затем громко решали, что пойдем в другое место. Помню, как-то раз в поисках вай-фая мы даже зашли в кинотеатр и притворились, что ждем свой сеанс.

Как я сказала, мне даже нравился подобного род интерактив. Но Веня уже успел истосковаться по прелестям беспроводной сети. В основном потому, что он уже очень давно не смотрел хороших фильмов. Благодаря открытому интернету близлежащего паба мы обзавелись несколькими, и теперь мы погрузились в один из них.

-- Как вино? -- поинтересовался Веня, не успела я сделать первый глоток.

-- Неплохо.

Вино оказалось паршивым и это еще очень мягко сказано, но мы пили его так, словно это было какое-нибудь Inglenook Cabernet Sauvignon. Выдержанно, с наслаждением в глазах. Светский прием продлился ровно пол бокала, после чего мы переглянулись и расхохотались.

-- Дерьмо редкостное, -- сквозь смех произнесла я.

-- За девять лет алкоголизма не встречал такого дерьмового вина! -- радостно согласился Веня.

Вениамину пришло сообщение. Очередной родственник удивлялся тому, что он женится.

Вениамин удивлялся не меньше.

-- Бабушка снова рассказывает, что я уже живу с невестой, -- объяснил он.

Мы опять рассмеялись.

-- Как вам такая новость, Елизавета Ларионова?

-- Блестяще, Вениамин Алексеевич!

К тому моменту фильм уже закончился и мы оба что-то читали.

-- Слушай, -- начал Веня спустя восемь страниц и один бокал. -- Так, а как ты на самом деле на это смотришь?

-- М? -- не отрываясь от книги, ответила я.

-- Выходи за меня, -- сказал Вениамин.

Книгу пришлось отложить.

\hypertarget{chapter-13}{%
\chapter{~}\label{chapter-13}}

-- Пойдем в ЗАГС? -- весело предложил Вениамин.

Я рассмеялась. Он тоже улыбался, но как-то спокойно. Словно и впрямь ожидал от меня ответа.

-- Постой. Ты это, что ли, серьезно?

-- Более чем, -- заверил Веня.

Я опешила настолько, что выпустила из руки бокал. Видимо, Веня был к этому готов потому, что успел его подхватить.

-- Мы знакомы всего три дня, Веня\ldots{}

-- \ldots{} и я за всю жизнь не был так счастлив, как за эти три дня.

Я молчала.

-- Я, наверное, не так начал.

Вениамин встал с кровати, на которой мы сидели в обнимку, и опустился передо мной на колени.

\emph{А, ну это, конечно, меняет дело}, прокомментировал внутренний голос. Тот самый, который всегда говорил с сарказмом.

\emph{Он встал на колени, и теперь это предложение совсем не выглядит странным.}

\emph{ОН ВСТАЛ НА КОЛЕНИ! ОН ВСТАЛ НА КОЛЕНИ!} -- заверещала в моей голове какая-то сумасшедшая.

Вениамин взял мою руку, поднес ее к губам и трепетно поцеловал.

-- Любовь моя, ты станешь моей женой? -- он снял одно из своих колец, готовясь надеть его на мой безымянный палец.

И я согласилась.

Чертов поцелуй руки. Всегда срабатывает.

***

Знаете эту старую песню, где девушка перечисляет вещи, о которых она думает, когда становится грустно? В моей вариации там были бы пёсики, лес, качественно приготовленный кофе, кладбища, бонги, горные просторы, пост-панк и, конечно же, литература. А еще в этот список непременно бы попал секс во время дождя. Ничего так не расслабляет, как монотонный звук барабанящих по подоконнику капель. Кроме, разве что, травки.

Следующая ночь показала, что секс после помолвки нравится мне куда больше. Просто, ввиду очевидных причин, я не имела об этом ни малейшего представления.

Наступившее утро встретило меня неожиданно солнечной погодой. Разбросав волосы на подушке, я сладко потягивалась, когда в спальню вошел Веня. Довольный и окрыленный.

-- Доброе утро, Венечка.

-- Доброе утро, любовь моя.

Вскоре мы переместились на балкон и разделили между собой последнюю сигарету. Погода стояла невообразимая. Кажется, вчерашний шторм открыл этот тонкий портал между весной и летом. Море вновь было тихим, ветер мягким, небо ясным, а я -- счастливой.

-- Как настроение? -- спросила я. -- Готов к рабочему дню?

Вениамин кивнул.

-- А ты готова? Посмотришь, как я работаю.

Я кинула в ответ. Вслед за этим в комнате прозвенел будильник.

-- Мы проснулись раньше будильника? -- удивленно спросила я.

-- Знаешь, сегодня я встал на рассвете. Вышел на балкон и просто наблюдал за тем, как над морем поднимается солнце. Я вижу этот пейзаж каждый день, но до сегодняшнего утра не понимал, насколько он прекрасен. Потом солнце встало. Я обернулся, и увидел, как ты сладко спишь в моей постели. Солнце падало на твои локоны, отчего они становились медными. И ты улыбалась. Даже во сне. Тогда я понял, что это еще прекраснее, чем все рассветы и закаты мира. Хотя они меня никогда не волновали. Я стоял в дверном проеме и наблюдал за тем, как ты спишь, когда услышал в голове собственный голос. Он сказал: «Вот, Веня, там спит твоя жена». После этого я ощутил такое счастье и спокойствие, каких не испытывал ни разу в жизни.

Во-первых, это была самая длинная лирическая реплика из всех, что я когда-либо слышала от Вениамина, а во-вторых\ldots{} Эти простые, но до невозможности греющие душу, слова, накрепко впечатались в мое подсознание. Словно кто-то взял волшебное перо, и написал их прямиком на моем сердце.

***

Наш автобус плавно приближался к противоположной части города. Обед, а по совместительству и час пик. Салон переполнен. Солнце беспощадно слепило пассажиров, что сидели или стояли, или впечатались в спины друг друга. Вениамин занял одиночное место, сразу возле выхода. Я умостилась у него на руках.

Соблюдая лучшие традиции пост-советского пространства, люди ссорились, обсуждали политику, проклинали погоду и выясняли, почем сайра на базаре. Большинство из них сидели, показывая сердитые, неудовлетворенные жизнью лица. Остальная часть пассажиров просто ехала с крайне отсутствующим видом.

Разделив наушники поровну, мы слушали старые рок-н-ролльные песни. И обнимались. Удивленные такой беспечностью, пассажиры то и дело бросали в нашу сторону не доверительные (а порой даже осуждающие) взгляды.

Реакция, конечно, не такая яркая как в случае с «Таврией», но тоже ничего.

-- Мы самые крутые в автобусе, -- шепнула я Вене.

В ответ на это он натянул на нос свои квадратные солнцезащитные очки, и артистично подмигнул мне.

***

Ехали мы долго. Наконец, автобус притормозил на нужной остановке. Еще один спальный район, очень похожий на то место, в котором я выросла: многочисленные бетонные здания без малейшего намека на культурное развитие, куча гастрономов, парикмахерских и парочка супермаркетов. Кое-где, правда, были признаки озелененной части пространства, чуть более воодушевляющие, чем пятнадцатиметровые николаевские скверы.

Веня мотался по салонам, принимал заявки и относил заказы. Раз или два я заходила с ним, но чаще всего ждала на улице со своим привычным парадно-выходным набором: книга, кофе и сигарета. Помню, как я сидела на парапете, углубившись в чтение, и чувствовала тяжесть кольца, появившегося на руке прошлой ночью. Оно было вылито из серебра, широченное и очень тяжелое. Но это была приятная тяжесть.

Так прошло больше пяти часов. Впереди оставался лишь один заказ, покончив с которым, можно было с чистой совестью отправляться на ужин. Видимо, из-за отсутствия машины, сегодня мы оделись попроще: оба в старых джинсах и толстовках. У Вени был черный спортивный рюкзак, который тот повсюду таскал с собой. Была еще огромная палитра -- глянцевая книженция с образцами красок его фирмы. Внутрь та не помещалась, так что Веня любил втискивать ее между спиной и рюкзаком. Я же шла налегке, если не считать дамской сумочки.

Решив, что бетонная зараза нам положительно надоела, Вениамин и я прошли несколько остановок через уютную лесопосадку. Высокие деревья с двух сторон отделяли ее от шума дороги и захудалых двориков, так что можно было представить, что ты находишься в лесу. Кстати говоря, деревьями этими были не какие-нибудь там тополя, а самые настоящие ели и сосны. Землю под ними усыпали сотни шишек и тысячи сухих иголок. Черт его знает где, Вениамин раздобыл здоровенную изогнутую палку, и теперь шел рядом, изображая друида.

-- До KFC парочка кварталов, -- сказал Веня.

Как выяснилось позже, где-то на этом отрезке пути я на него и запала. Окончательно и бесповоротно.

\hypertarget{chapter-14}{%
\chapter{~}\label{chapter-14}}

Севастопольская вариация KFC находилась около большого торгового центра, куда мы с Веней в дальнейшем не раз заглядывали. Это было милое местечко с разноцветными диванами, широкими столиками, которых было от силы штук пять, и огромными окнами, плавно переходящими в стеклянные двери.

Потягивая ледяное пиво, Веня пытался скачать какой-то фильм. Я расположилась справа от него и, закинув ноги на коленки своего спутника, размышляла над чувством внезапно охватившей меня эйфорией. Вернее, более углубленной ее формой. Теперь к списку моих ласкательно-уменьшительных имен добавилась еще и «невестушка». Воодушевленный изменениями в семейном статусе, Вениамин старался не упустить ни одной возможности называть меня так.

Втрескаться, втюриться, втюхаться, потерять голову, влюбиться по уши, отдать сердце, воспылать любовью, быть без ума, души не чаять, сохнуть, запасть и неровно дышать -- все это, в общем-то, лишь малая часть тех эмоций, которые я испытывала. Предмет моих воздыханий сидел напротив. Он уплетал третий бургер подряд, а я потягивала лимонад и думала о том, что это удивительное чувство внутри, видимо, и есть пресловутые бабочки.

Почему это произошло лишь спустя несколько дней? Можно много рассуждать о моем прошлом, разбитом сердце и защитном механизме, с которым явно что-то не то, раз он сдался так быстро, но правда в том, что я не знаю ответа. За что я его полюбила? Без понятия, но мне всегда казалось, что любят не за что-то, а кого-то и одним воодушевляющим весенним днем этим кем-то стал Вениамин Ларионов.

-- Нужно раздобыть тебе кольцо, -- деловито сказал Веня, покончив с бургером.

Я взглянула на свою руку. Нынешнее и правда было великовато. Вене пришлось согнуть его заднюю часть, чтобы кольцо хоть как-то держалось на моем пальце.

-- Да, наверное.

Вскоре к нам присоединилась Неля с ее прайсами. Это была весьма крупная женщина бальзаковского возраста. Она носила броский макияж из девяностых и короткую стрижку со странной треугольной челкой. Еще Неля была громкой и невыносимо много болтала. Ее историям конца и края не было видно, но имелось в них что-то по-домашнему приятное. Мне она сразу понравилась. Неля стала первым человеком, заказавшим при мне семь бургеров и что-то там еще, а затем употребившим все это прежде, чем я допила свой кофе.

За время выполнения заказа, Неля успела рассказать о своем сыне, долгах, кредитной истории, районе, в котором она живет, зарплате, каталогах, маникюре и, почему-то, Екатеринбурге.

-- Так вы двое и правда женитесь? -- внезапно спросила Неля.

За время ее монолога я слегка забыла о том, что умею разговаривать, так что ответила не сразу.

-- Правда, -- сказал Веня.

Он взял меня за руку.

-- Думаю да, -- добавила я.

Неля смотрела на нас с каким-то странным уважением. Затем объявили ее номер, и девушка, хотя, скорее, женщина поспешила забрать свой заказ. Позже она написала Вене о том, что в диком восторге от его выбора. Сказала, что со стороны мы смотримся как современные Курт и Кортни.

\emph{Что ж}, подумала я. \emph{Надеюсь, эти отношения не закончатся самоубийством одного из нас.}

На обратном пути мы заехали в несколько ювелирных. На самом деле я еще не пришла в себя от внезапно охвативших меня чувств, что уж говорить о перспективе внепланового замужества. Удивление прошло не сразу. Мне понадобилось месяца полтора, дабы свыкнуться с мыслью о том, что я чья-то невеста.

-- Какое ты хочешь? -- Вениамин увлеченно доставал кольца с подставок. -- Такое? Нет? Тогда может это? Или нет, подожди, вот это!

Говоря о кольце, мне не хотелось ничего особенного. Простая обручалка, тонкая, выполненная из серебра, меня вполне устраивала. Я никогда не любила золото. Увы, все, что встречалось на нашем пути, непременно содержало какие-то цветастые камни, стразы и бог его знает, что еще. Мы нашли парочку подходящих вариантов, но все кольца были либо малы, либо до безобразия велики. Вениамина это огорчило куда больше чем меня.

-- Оно того не стоит, -- ласково произнесла я, заметив грусть в глазах своего мужчины.

-- Деточка, -- вздохнул он, -- такое чувство, что сегодня все против нас.

Я обняла его за плечи.

-- Вовсе нет. Мы встретились четыре дня назад и уже сутки как помолвлены. По-твоему, все действительно выглядит настолько мрачно?

Веня избавился от сигареты и поцеловал меня.

-- А, знаешь, ты права.

-- К тому же, -- чуть погодя добавила я, -- вдруг это знак?

-- Знак?

-- Ну, да. Может быть, перед принятием самого важного решения в жизни, нам стоит пожить вместе чуть дольше? Ты сам-то не хочешь хорошенько обдумать этот вопрос? -- говоря это, я переживала, что Вениамин вновь затоскует, но он продолжал смотреть на меня спокойным, полным любви взглядом.

-- С ума сошла? Я думал об этом целых три дня! -- наконец, ответил Веня. -- Я вообще никогда прежде так долго ни о чем не думал.

Мы рассмеялись, хотя оба знали, что, в большинстве своем, так оно и было.

\hypertarget{chapter-15}{%
\chapter{~}\label{chapter-15}}

Очередной бар и мы в нем. Сидим, наслаждаясь прохладой кондиционера и потягивая напитки. Это была одна из многочисленных пивных, что разбросаны по всему Севастополю.

Вениамин зачем-то предлагает мне пиво. Я как всегда отказываюсь.

Совершенно не люблю пиво. И это вовсе не значит, что я не пила хорошего пива, как думают большинство тех, кто слышит это заявление. Я была на нескольких пивоварнях и долгое время работала за стойкой. Словом, перепробовала целую кучу сортов. Итог был один: пиво я на дух не переношу. Оно для меня как черный чай без сахара -- слишком горькое и бесполезное, чтобы пить.

Однажды в разгар маниакального эпизода я окажусь в Варшаве, где впервые добровольно закажу себе пива. Проделаю то же самое в Праге и Берлине, а потом еще в Бремене и парочке других немецких городов. Просто так. Потому, что могу. Я буду пить кофе без молока и чай без сахара с абсолютным безразличием, но\ldots{} Случится это спустя годы, а пока мне все еще двадцать два и пиво -- совсем не по душе этому персонажу.

По этой простой причине, тем вечером мне достался вишневый сидр. Вениамин же предпочел классику, и приканчивал уже второй бокал своего хмельного напитка. У нас был целый стол пивных закусок, разнообразию которых позавидовал бы сам Пантагрюэль. По телеку транслировали «Брюса Всемогущего». В детские годы этот фильм был одним из моих любимых фильмов, и это придавало особого уюта нашему вечеру.

-- Я сейчас так счастлив, -- с чувством сказал Вениамин. -- Никогда в жизни не был таким счастливым.

-- Это потому, что я не против каждый день ошиваться в барах? -- улыбаясь, уточнила я.

-- Вовсе нет, глупышка. Хотя, возможно, и не без этого, -- он подмигнул и тут же стал совсем серьезным. -- Я счастлив, потому что все вокруг кажется таким\ldots{} идеальным. Я сижу в пустом баре посреди ночи, смотрю старое кино. Нет никаких конфликтов и мордобоев. Просто тихий, приятный вечер. Рядом сидишь ты, вся такая идеальная. С моим кольцом на пальце. А я смотрю на все это, и не могу поверить в то, что мне и в самом деле так повезло.

Он протянул руку, чтобы погладить проходившего мимо старого кота.

Вениамин любил животных. Думаю, это и была одна из причин, по которым я решилась ему довериться. Он обожал котов, я же души не чаяла в собаках. Сколько себя помню, животные всегда были мне близки. Гораздо ближе людей, так что я никогда бы не связала свою жизнь с человеком, не любившим животных.

Каждый день, гуляя по городу, направляясь на рынок, или попивая пиво в каком-нибудь из севастопольских двориков, Веня останавливался, чтобы потискать очередную животинку. Таких остановок бывало по полсотни в день. Частенько мы покупали еду и отправлялись кормить уличных животных. Это качество я всегда ценила гораздо выше, чем физическую красоту или, скажем, умение управляться с клитором.

-- Расскажи мне о своих бывших, -- попросила я тем тихим вечером в пивной.

Вопреки тому, что большую часть вечера мой мужчина был разговорчив, -- даже болтлив -- он всячески избегал темы своих бывших отношений. В юношеские годы я слыла той еще ревнивицей и непременно оценила бы этот жест, но теперь такой расклад не мог не настораживать. В конце концов, я уже давно осознала, что ревновать к прошлому -- крайне глупо, и даже научилась уважать это самое прошлое. Проблема была в том, что Веня, естественно, не отрицал того, что не раз состоял в серьезных отношениях, но при этом, почему-то, вел себя так, словно этого не было.

-- Ее зовут Римма. Мы жили вместе полтора года. Я ее не любил, -- вот максимум выдаваемой Вениамином информации.

-- Зачем же ты тогда жил с ней так долго?

Обычно в ответ на этот, казалось бы, логичный вопрос, Веня просто пожимал плечами. Один раз, правда, заметил, что ему просто было одиноко. Что ж, на этот раз мне хотелось услышать более информативный ответ.

-- Расскажи мне о своих бывших?

-- Что ты хочешь знать? -- внезапно спросил Веня.

Я настолько изумилась этому вопросу, что как-то сразу и забыла, чего я там хочу знать. Почувствовала себя эдаким растерянным свидетелем Иеговы, которого впервые в жизни пустили на порог. Да так неожиданно, что бедняга и вовсе забыл, о чем положено говорить дальше.

Стоило собраться с мыслями прежде, чем болтливость моего жениха успеет улетучиться.

-- Хочу знать о той, кого любил, конечно же.

-- Ты первая женщина, которую я по-настоящему полюбил, -- просто ответил Веня.

Я не могла оставить это заявление без поцелуя.

-- В таком случае, -- продолжила я, оторвавшись от жениха, -- расскажи о той, о ком думал, что любил. Давай начнем с простого. Как она выглядела?

И он рассказал. Она была заметно старше и теперь преподавала в школе. Позже, взглянув на фото этой женщины, я в который раз убедилась в привычке человеческого мозга идеализировать прошлое. Вениамин описывал ее как натуральную блондинку с зелеными глазами и большой грудью. Со снимка же на меня смотрела более чем просто полная дамочка домашнего образца с какими-то серыми волосами. Она носила жуткую бабушкину одежду и выглядела куда старше своего возраста.

\emph{Баба как баба}, подумала я. \emph{Если уйдет, то точно не к ней.}

Не стану врать, все это как-то плохо вяжется с моей жизненной философией, но увиденное меня порадовало. Я была моложе, стройнее и привлекательней. И из моей постели еще никто не выходил полным сил.

Преимущество над бывшей любовью Вениамина казалось мне очевидным, и я тут же забыла об этой женщине, вычеркнув ее из мысленного списка своих потенциальных раздражителей. Сама мысль о том, что кто-то может изменить мне с ней казалась нелепой.

Паршиво так думать о людях, но мне было едва за двадцать и мне частенько срали в душу, так что чего вы теперь от меня хотите? Было непросто доверять людям, и, если осознание чьей-то серости на моем фоне могло предотвратить приступ ревности, этим стоило воспользоваться.

-- Когда я впервые привел Аллу домой, мать увела ее на балкон и на протяжении часа уговаривала бежать от меня куда подальше.

-- Как так? -- удивилась я, но Веня не ответил.

-- Я даже звал ее замуж, -- признался он. -- Договорился о том, чтобы устроить фаер-шоу около одного обрыва и сделал ей предложение.

С одной стороны, меня это очень удивило. Мои эгоцентричные женские нотки, все же, предпочли бы думать, что я была единственной, на ком Веня захотел жениться. С другой, он сделал мне предложение третьей ночью после нашей встречи. И почему меня должно было это удивить?

\emph{И где мое фаер-шоу?}

-- И что же произошло?

-- Сам не знаю, но вскоре я передумал жениться. Просто выносить ее не мог. Она, знаешь, была совершенно неинтереснейшим человеком и вела себя как моя мама, не как девушка. Понимаешь, о чем я?

-- Понимаю.

-- И что ты понимаешь?

-- Есть три типа отношений. Два из них провальные и твой как раз вписывается в их рамки.

-- А третий? -- спросил Веня.

-- А третий это то, что у нас с тобой, -- ответила я.

***

-- Еще была Римма, -- продолжал Вениамин. -- Она очень много врала. Без особых причин. Рассказывала, что над ней издеваются родители, которые оказались замечательными людьми. Рассказывала, что кто-то пытался изнасиловать ее в автобусе. Рассказывала, что ее постоянно хватают за задницу какие-то незнакомцы и что она все время падает, спотыкается, ударятся и все такое.

-- Сколько ей лет-то было когда вы сошлись?

-- Семнадцать.

-- Красивый пример юношеского максимализма. Кстати говоря, она красивая?

Веня отхлебнул пива.

-- Вообще нет. В очках еще более-менее.

Меня всегда настораживали подобного рода ответы.

-- Так почему ты с ней полтора года прожил?

Он традиционно пожал плечами.

-- Дальше стало хуже. Она начала рассказывать людям о том, что я ее избиваю. Как-то пыталась убедить моих друзей, что была беременна. Узнав об этом, я якобы избил ее прямо посреди улицы.

-- О, у меня тоже был один товарищ с явным дефицитом внимания. И тоже патологический лгун.

-- Тоже беременный?

-- Ну, почти. Умеем мы выбирать\ldots{} -- я подняла бокал. -- За умение выбирать спутников жизни.

-- А то! Даже не помню, когда в последний раз разговаривал вот так вот по душам, -- признался Веня. -- Если вообще разговаривал. Я имею в виду, с девушкой. А что насчет твоих бывших?

-- Ну, мне повезло больше, -- я ухмыльнулась. -- До встречи с тобой я практически три года провела вне отношений.

-- Серьезно?

-- Ага.

-- Почему? -- лицо моего возлюбленного выражало крайнее недоумение.

-- Да все очевидно. Я не могу размениваться на мимолетные отношение. Секс без любви -- это нормально, но вот отношения без любви меня пугают. Так что, да, пришлось немного побыть без них.

-- И каково это?

-- Ты действительно никогда не был один?

Веня отрицательно покачал головой. Он выглядел смущенным.

-- Нет. А как оно?

-- Преимущественно, печально, если поддаваться стереотипам системы, но есть свои плюсы. Например, поиски самости.

-- Какой такой самости?

-- Когда ты вне отношений не нужно циклиться на этих самых отношениях. Подстраиваться под кого-то, ждать одобрения и идти на постоянные компромиссы. Не нужно соответствовать чьей-то картине жизни. У тебя уйма времени, чтобы понять, кто ты, почему ты, и зачем ты. А стоит это осознать, начинаешь, к тому же, задумываться еще и о том, куда ты\ldots{}

Вениамин задумчиво почесал бороду.

-- Я не могу быть один.

-- Тогда ты никогда так и не узнаешь, кто ты.

-- Мне с собой скучно и грустно. Это я знаю наверняка. Терпеть не могу свою компанию, а ты говоришь, что мне это нужно. Вот предположим, гипотетически я и так знаю, что я мудак. Конченный человек. Тогда зачем мне поиски этой твоей самости?

-- Затем, что гипотетически ты такой именно из-за отсутствия поисков себя.

-- Господи, тебя не переубедить. Хорошо, а если я себе не понравлюсь?

-- Это не важно.

-- Почему?

-- Да потому, что к тому моменту ты уже будешь знать, кем ты хочешь быть, кем ты можешь стать, и что для этого нужно делать.

Жених взглянул на меня с недоверием, за которым, тем не менее, виднелись проблески интереса.

-- И что потом, Васляева?

-- А потом просто делать.

-- Что-о-о делать?!

-- Узнаешь, когда найдешь себя.

Веня откинулся на спинку кресла и задумчиво вздохнул.

-- Ты мне сейчас мозг сломаешь, -- признался он. -- Как мне понять, что делать, если я не знаю, кто я? И что значит, зачем я и почему я? Почему что? Нет, я точно не могу быть один. Но теперь мне интересно, что все это значит, деточка? Ты же знаешь, я у тебя любопытный.

Он посмотрел на меня с просьбой во взгляде.

-- Ну, хотя бы объясни, каково это жить без отношений? Тебя послушать, так это чуть ли не обязательно.

-- Вроде того.

-- Потому, что только так я смогу узнать, кто я и\ldots эм\ldots{} почему я?

-- Правильно.

-- Ну, так и что мне теперь делать, если я хочу это узнать, но не хочу и не могу быть один?

-- Да это не так уж и сложно, -- я улыбнулась. -- Попробуй при случае.

\hypertarget{chapter-16}{%
\chapter{~}\label{chapter-16}}

Ближе к полудню я нехотя выбралась из постели. Вене предстояло обойти долгий список салонов, и мой жених уже вовсю носился по квартире. Одной рукой он держал чашку со вчерашним кофе, второй сжимал не глаженный галстук. Вениамин бегал между ванной и балконом. Он что-то рассказывал, периодически размахивая своим алым галстуком. Да так, словно хотел привлечь внимание всех быков полуострова.

-- Вот поэтому я не люблю быть один. Мне с собой скучно, -- разобравшись с гардеробом, подытожил Веня. -- Невыносимо скучно. Как ты меня выносишь?

Он вернул свое внимание галстуку, повертел его в руках и со словами «Совсем мятый!» швырнул обратно в шкаф.

-- Можем идти! -- откуда-то из прихожей провозгласил Вениамин.

И я вдруг поняла, что по-прежнему сижу в кресле в одном белье.

Веня тоже это заметил и, по всей видимости, поспешил исправить ситуацию. Но что-то пошло не так, и вскоре мои трусики уже присоединились к галстуку. Склонившись над креслом, Вениамин опустил ладони на мои плечи. Его пальцы неспешно скользили по коже. Они коснулись шелковых бретелек моей майки. Оправили их, словно невзначай, и продолжили скользить вниз по телу, утягивая за собой остатки одежды.

От этих прикосновений голова шла кругом. При каждом вдохе тело покрывалось сладостным ознобом, и мне подумалось, что завтрак придется пропустить. Затем Вениамин принялся покрывать мою грудь поцелуями и думать мне больше не приходилось.

***

Большая часть салонов из списка находилась в районе Северной и еще парочка -- в центре. Вопреки тому, что это были два разных конца города, находились они напротив. Каждые полчаса к берегу подходил видавший виды паром, перевозивший горожан на соседний берег через Севастопольскую бухту. Назывался он «Адмирал Ларионов», что не могло не вызвать улыбки на моем лице.

И вот мы вновь бродим солнечными улочкам Севастополя. Конечно, Вене нужно было работать, но это никак не нарушало внезапной атмосферы праздника. На самом деле, мне нравилось просто гулять по городу. Плевать, куда ты идешь, когда рядом любимый человек.

-- Такое чувство, что у меня сегодня выходной, -- признался последний.

Покончив с центром, мы сделали короткую остановку в очередной кофейне, решив расположиться снаружи. Заведение было крохотным, его терраса -- еще меньше. Зато цены были довольно высокими, что вполне объясняло отсутствие посетителей. Заказав кофе, мы опустились в плетеные кресла и принялись обсуждать литературу.

Единственным жанром, который нравился Вениамину, было фэнтези. Я же предпочитала классику, философию, постмодернизм, гонзо, старые детективы и Стивена Кинга, чье творчество не поддается классификации. Еще я души не чаяла в поэзии, которую Веня совершенно не понимал. Словом, почва для дискуссии оказалась бесценной. Увлекшись разговором на любимую тему, я даже не заметила, как принесли кофе.

Более того, вскоре я осознала, что мы вдруг начали планировать свадьбу. Причем планировал, по большей части, Вениамин, а я лишь увлеченно слушала и периодически кивала. В отличие от предыдущего разговора, наши мнения касательно свадьбы более чем сходились. За исключением одного момента.

-- Никогда не хотела пышную свадьбу, -- сказала я, когда Веня, наконец, прервался на кофе. -- Я их попросту не понимаю. То есть, даже если опустить финансовый вопрос, на кой черт приглашать людей, которые и дня твоего рождения не вспомнят? Мне всегда казалось, что свадьба -- это очень личное. Если честно, глядя на пиршества своих бывших одноклассниц, у меня создается такое впечатление, словно они делают это вовсе не для себя.

-- Ты права, любовь моя, -- лучшая фраза из тех, что женщина может услышать от своего избранника. -- Я бы пригласил папу, сестру и, возможно, парочку старых приятелей.

Я подняла бровь.

-- Уверен?

-- Конечно, -- заверил меня Веня.

Он всегда как-то по-особенному произносил это слово. С придыханием и каким-то милым воодушевлением.

-- Эх, а я-то хотела предложить тебе взять кредит, который мы не сможем выплатить и спустя двадцать лет после развода\ldots{}

-- А что насчет места? Я бы предпочел природу.

-- Фиолент, -- внезапно для самой себя ответила я. -- И столик в каком-нибудь уютном ресторанчике.

-- Я поражен, - искренне произнес Вениамин. -- Боялся, ты захочешь что-то более публичное.

Обсудив еще несколько деталей, мой благоверный с радостью заявил:

-- Повторюсь, Васляева, ты -- моя идеальная женщина!

Как я уже говорила, в этой жизни мало что могло заставить меня залиться краской. Смущение -- явно не мой конек. Было странно смущаться из-за ласковых слов мужчины, с которым живешь с первого дня знакомства. Но я смутилась.

-- Тогда завтра можем отнести заявление в ЗАГС, -- довольно подытожил Веня.

Тут то и всплыл наш персональный камень преткновения. Учитывая то, что прежде я никогда не мечтала о замужестве, было довольно сложно свыкнуться с мыслями о предстоящей свадьбе. Еще сложнее мне было свыкнуться с мыслями о столь скорой свадьбе.

-- Ты сомневаешься, -- не спросил, я скорее сказал Веня, увидев мое лицо.

Конечно, я сомневалась. Не в своих чувствах к Вениамину и даже не в его чувствах ко мне. Я неподдельно верила в нас. Верила тому, что он любит меня и тому, что он действительно хочет быть со мной. И, все же, хотеть и быть -- немного разные вещи. А еще я не понаслышке знала, какими переменчивыми могут быть мужчины.

Нет, я любила Вениамина всем сердцем. Все, что когда-либо происходило в моей жизни, любые конфликты, трудности и прощания -- все это вдруг стало таким несущественным. Даже не выспавшись, будучи усталой, без стабильного финансового положения, простуженной, страдающей от сезонной аллергии или извечной мигрени, настигшей меня после сотрясения мозга, я постоянно находилась в прекрасном настроении. Пожалуй, так говорят о каждой новой пассии, но я в действительности никогда не чувствовала себя так, как чувствовала рядом с Веней. Я смотрела на него, сидящего в соломенном кресле, с этими волосами цвета золотистой пшеницы и едва различимой за ними одинокой серьгой, чуть более темной бородой, которую не мешало бы подстричь, с многочисленными браслетами, утратившими цвет татуировками, что, то тут, то там, выглядывали из-под футболки, в поношенных джинсах, на карманах которых виднелись контуры ножа и ручки, в дырявых ботинках, с которых все время свисали шнурки, в квадратных солнцезащитных очках, которые шли ему так, как не шли ни одному хипстеру, и с рубашкой, повязанной на бедрах, что так походила на куртку дровосека. Смотрела и понимала, что любовь -- слишком скудное слово, чтобы передать все то, что поселилось в моем сердце благодаря этому мужчине. Я любила его всей душой и тем, что осталось от моего сердца. Любила, и именно поэтому предпочла бы не сковывать себя узами брака. Слишком ценила эти отношения, чтобы добровольно делать столь амбициозные, необдуманные шаги.

-- Просто не хочу все испортить, -- наконец, ответила я. -- К тому же, я с рождения плохо переношу жару, -- и это было абсолютной правдой. -- Не люблю лето и люблю осень.

-- Как ты можешь что-то испортить? Своим согласием ты сделаешь меня самым счастливым мужчиной на свете.

-- Любимый, -- я взяла его за руку, -- ты и в самом деле удумал стать моим мужем?

-- Спрашиваешь!

-- В таком случае нам нужно провести какое-то время вместе. В конце концов, я вроде как не собираюсь выходить замуж дважды.

-- Переезжай ко мне, -- просто сказал Веня.

Я улыбнулась.

-- Но\ldots{} я и так живу с тобой.

-- Нет, осчастливь родителей, набей чемодан всякими своими Ремарками, Бродскими и Паланиками и переезжай насовсем!

Я не знала, что сказать. Замуж выходить -- так я первая, а вот переезд казался мне очень ответственным делом.

-- Надеюсь, ты не будешь тосковать по работе в баре?

-- По шестнадцатичасовому рабочему дню, вечным недостачам, уйме обязанностей, мизерной зарплате и полному отсутствию чаевых? Даже не знаю.

***

-- Может, все-таки, передумаешь? -- не унимался Веня. -- Отнесем заявление завтра, и у тебя еще будет целых два месяца убедиться в том, что я прав. Ты же еще не передумала стать Елизаветой Ларионовой?

Он опустил очки к носу и весело подмигнул мне.

-- Думаешь, два месяца -- достаточный промежуток времени для того, чтобы принять такое решение?

-- Я свое уже принял. Дело за тобой.

-- Что ж, Вениамин Алексеевич, в таком случае, если ты задашь мне этот вопрос через два месяца, мы пойдем в ЗАГС и напишем это чертово заявление.

На протяжении последних минут десяти, позади нас то и дело раздавался шорох, природу которого мне никак не удавалось распознать. Учитывая серьезность нашего разговора, никто и не думал оглядываться. Шорох затих так же неожиданно как появился. Затем в нашу сторону покатилось нечто, плавно тарахтевшее по деревянному полу террасы. Наконец обернувшись, я увидела у своих ног круглый коричневый предмет, впоследствии оказавшийся баночкой краски. Вслед за ней уже мчался средних лет мужчина, одновременно шумно ругаясь и извиняясь на ходу. Был он седовласый, жилистый, не особенно опрятный, с локонами, собранными в пучок, двухнедельной щетиной и весь перепачканный краской. Одним словом, на первый взгляд -- полубомж, который завтра мог бы оказаться в тренде.

Художник расположился за соседним и единственным, помимо нашего, занятым столиком. Там уже находились краски, палитры, мольберт, банка с кистями и прочие неотъемлемые атрибуты его ремесла. Писал мужчина очередные городские пейзажи вроде тенистых аллей и кафешек с цветочными клумбами. Такие обычно продают вдоль набережных. Подобные картины всегда привлекали меня в реальной жизни, но никак не на бумаге. Милые, идеально прорисованные картинки. И все одинаковые, словно их создатели целенаправленно избегали малейших проявлений собственного стиля. Или самих себя.

Да, пейзажи вроде этого меня не интересовали. Но вот сам мужчина, он оказался крайне интересным. Я имею в виду, то, как он нервно вышагивал вокруг мольберта, не прекращая бранить себя. Он буквально ругал себя за все на свете: от плохого цвета, света или тона, и заканчивая тем, что руки дрожат и краски вылетают из них на землю. Вот где и в самом деле экспрессии было побольше, чем во всех его вместе взятых картинах. Все это на миг перенесло меня в прошлое. Я вспомнила Эрика и то, с какой скрупулёзностью он писал музыку.

Вениамину позвонили из офиса, и он уже диктовал кому-то очередную заявку, а я допивала кофе и не сводила взгляд с этого удивительного полубомжа-полухудожника. Смотрела и думала великие думы о том, что порой вокруг готового продукта искусства устраивают слишком много шума. Иногда бывает и так, что этот самый предмет искусства ровным счетом не имеет никакого значения и, по сути, толком не является таковым, тогда как за творцом наблюдать куда приятнее. Выходит, что именно действие или бездействие в попытках создания и является настоящим искусством. Разве это новость? Мировое искусство кишит подобными сюжетами. Взять хотя бы Хантера Томпсона -- уникальный человек и тот еще нарколыга. Однажды, он поехал в Лас-Вегас чтобы освятить предстоящие гонки и случайно написал культовое и наиболее узнаваемое произведение в жанре гонзо. Говорят, что статью он так и не закончил. Есть еще буддисты. Они-то давненько в курсе такого расклада событий. Путь -- и есть счастье, и все такое прочее.

-- Что это за лицо на тебе такое? -- между тем поинтересовался Вениамин.

Я рассказала о том, что думаю. Постаралась, чтобы моя мысль звучала менее пафосно, и самонадеянно заявила, что пора бы ему начинать за мной записывать.

***

Покончив с делами, мы прошли вдоль причала Северной бухты, и уселись на самом краю. День выдался свежим, но тихим. Вокруг не было ни души. Перед нами простиралось море с едва уловимыми лазурными волнами. Они изредка мелькали на поверхности воды, напоминая складки на шелковом одеяле. Словно подобранное в тон морю, небо поражало глаза своей яркостью. Кое-где застыло несколько пушистых белоснежных облаков. На ближайшем берегу виднелись бесконечные обрывы, уйма зелени и парочка построек. Рядом со мной -- облаченный в капюшон и солнцезащитные очки -- сидел Вениамин. Мы курили по второй сигарете подряд. Время поджимало, но никому не хотелось покидать этот открывающий умиротворенную панораму уголок.

Внезапно из-за плеча моего суженого появился его водоплавающий однофамилец.

-- Прокатимся? -- игриво спросил Веня, указывая на медленно движущийся паром.

Я взглянула на эту сомнительную конструкцию, которая, казалось, внезапно для себя самой, вместила на борту добрую десятину города, переправляя ее в Артбухту -- еще одно место, в котором мы частенько гуляли. Кстати, официально, Вениамин сделал мне предложение на набережной Корнилова, что прилегает к этой самой бухте. Правда между неофициальным и официальным предложениями руки и сердца прошло менее десяти часов.

-- Сегодня я прокачусь только на одном Ларионове, и этот парнишка, -- я указала на паром, -- явно не мой тип.

\hypertarget{chapter-17}{%
\chapter{~}\label{chapter-17}}

Живя вместе, мы привыкли пропускать завтрак. Обычно просто просыпались, занимались сексом и выпивали по чашечке кофе. Затем отправлялись в город, чтобы выпить еще по одной. И лишь ближе к вечеру, когда с работой было покончено, и Вениамин с облегчением прятал блокнот для заявок, мы вдруг вспоминали, что неплохо бы было поесть.

Однако, этот день выдался не таким. Хотя бы потому, что сегодня Вене нужно было работать как раз в том районе, где мы жили. Сказать больше, все нужные ему салоны находились в десяти минутах ходьбы от дома. Над городом еще сияло полуденное солнце, когда возвращались домой. Веня купил несколько банок пива и теперь шел с довольным видом, приканчивая одну из них. Я разбавила свой кофе приличной порцией коньяка и поместила все это во флягу, с которой, казалось, меня связывало куда больше, чем с любым другим неодушевленным предметом.

В общем, прихватив с собой алкоголь, любовь и прочие атрибуты простого человеческого счастья, мы расположились в небольшом парке. Солнечные лучи норовили проскользнуть сквозь пышную крону дуба, под которым мы сидели; в ветвях щебетали птицы, яркая трава щекотала лодыжки, а из переносной колонки доносился голос Боба Марли. Здесь не было ни души. Вдали виднелся проспект, вдоль которого то и дело бегали озабоченные работой люди, а мы попивали свои напитки, дышали свежим майским воздухом и радовались тому, что не являемся частью этой суеты.

Одурманивающая атмосфера уединения продержалась около часа. Затем местные мамочки получили сигнал, который был слышен лишь им одним, и как по команде стали появляться со всех сторон парка. Мы ушли, не сговариваясь.

Через несколько остановок находился круглосуточный гастроном, куда мы еженощно захаживали. Сегодня же мы впервые появились там днем. Скорее по привычке и из желания пройтись, ведь в радиусе двадцати метров от сквера было множество работающих магазинов.

Внутри магазина можно было встретить огромного старого пса, одинокого кота и видавшую виды продавщицу. Она являла собой классический пример представительниц торговли девяностых годов: выжженные перекисью волосы, фиолетовые тени, которые тянулись до самых бровей и даже куда-то выше, и полнейшее отсутствие банальных знаний этикета сферы обслуживания. Если вы, как и я, росли в девяностые, один разговор с такой особью способен отворить вам удивительный портал в детство.

Не обращая никакого внимания на грубость продавщицы, Вениамин поглаживал кота. Тот сидел у прилавка и подобно хозяйке заведения излучал полнейшее безразличие ко всему, что происходило вокруг. Мне же всегда доставался пес. Хотя, скорее, я -- ему. Увидев меня, старый дворняга радостно вилял хвостом, и тут же падал на спину, ненавязчиво предлагая мне почесать его живот.

На пути домой нам всегда встречалась уйма уличных животных. И, все же, наиболее оживленное место представлял собой финальный отрезок пути -- широкая каменная лестница. Она имела несколько пролетов. На каждом обитало по семейству котиков. Подойдя ближе, Веня углубил руку в свой бесконечных размеров рюкзак, и достал огромную пачку сухого корма.

-- Зачем\ldots{}

«\ldots так много?» -- собиралась спросить я, но замолчала на полуслове.

Стоило моему мужчине высыпать его на землю, как из близрастущих кустов показалась еще пара кошек. А затем еще пятеро, еще трое и снова двое. В конечном счете, их оказалось не меньше двух десятков, если не считать тех, что к нашему появлению уже были на лестнице.

-- Такие дела, -- сказал Веня, видя мое удивление.

***

Лестничные кошки были третьей к концу остановкой. За ними шли дворовые, откормленные местными старушками животные. Последней станцией был Марсианин -- огромный камышовый кот, которые уже долгие годы жил на чердаке дома. Заходя в подъезд, мы пропускали его в лифт и подвозили на последний этаж. Тогда Марсик неспешно выходил из кабинки, оборачивался, словно хотел поблагодарить нас. Но потом вспоминал, что ему не до этого и поспешно поднимался по ведущим на чердак ступеням.

Словом, каждый совместный выход из дома оказывался небольшой экскурсией по местной фауне. Мне это нравилось.

Как я уже сказала, обычно до обеда мы о еде не вспоминали. Этот лишенный работы день стал исключением.

Чайник закипал, свиные ребрышки жарились, а картошка остывала на плите. Я услышала звук сработавшего фотоаппарата, и в замешательстве обернулась.

-- Елизавета Васляева стоит у плиты, -- прокомментировал Вениамин. -- Однажды репортеры неплохо наградят меня за такое фото.

Я рассмеялась, залившись румянцем.

-- Хорошо всмотрись в эту картину потому, что ты видишь ее в первый и последний раз, -- предупредила я.

И, как всегда, ошибалась.

***

-- Мы прямо-таки семья, -- заметила я, глядя за тем, с каким энтузиазмом Веня поедает приготовленный мною ужин. -- Уплетаем картошечку под покровом ночи и все такое.

Кстати говоря, его порции -- это было что-то с чем-то. Вкратце, она представляет собой дневной рацион питания какой-нибудь многодетной татарской семьи. К своим двадцати двум годам я видала разных людей, но ни одному из них не удавалось есть так много и выглядеть так подтянуто, не посещая при этом тренажерный зал. Веня же буквально ел суп кастрюлями, макароны пачками, а единственной приемлемой яичницей для него была та, в которую входило не менее шести яиц.

-- Я прямо-таки собираюсь на тебе жениться, -- напомнил Веня.

-- Прости. Все время об этом забываю.

И это было правдой. С непривычки я постоянно забывала о том, куда зашли наши отношения. Просто наслаждалась моментом до той самой минуты, которой Вениамин называл меня своей невестушкой. Тогда пространство вокруг меня замирало, и я начинала отчетливо осознавать ситуацию, слегка ее побаиваясь. Но, в основном, на душе становилось невероятно тепло. Я краснела от смущения и подолгу смотрела на Веню влюбленным взглядом.

-- Но ты права. Я тоже что-то почувствовал, -- отозвался Вениамин.

-- Наверное, это потому, что мы впервые едим дома.

И правда, все дело было в ночной картошечке. Ну, и в любви, которая то и дело витала в пропитанном специями воздухе.

-- Ну, ты же моя невестушка, помнишь?

-- Да как тут не помнить?

-- А мне вот, все-таки, кажется ты не всегда об этом помнишь\ldots{}

-- Ладно-ладно, ты меня раскусил, -- я убрала со стола пустые тарелки и улыбнулась своему возлюбленному виноватой улыбкой. -- Я правда не всегда об этом помню.

-- Почему? Я думал, это у меня проблемы с краткосрочной памятью.

-- Не в этом дело, Веня. Я просто никак не могу привыкнуть к мыслям о замужестве. Или о помолвке. Это как если бы ты однажды выкрасил волосы в черный\ldots{}

-- В черный?

Вениамин искренне изумился своему гипотетическому выбору нового цвета волос.

-- Да, именно. Ты бы стал брюнетом, но мысленно еще долго позиционировал себя как блондина. Потому, что всю свою жизнь ты был блондином и никогда не думал о себе, как о брюнете, понимаешь?

-- Допустим.

-- Вот и я никогда не думала о себе, как о жене или невесте.

-- Ну, так скажи это!

-- Что?

-- Я слышал, что в таком случае мысли надо озвучивать, -- объяснил Веня. -- Когда ты не можешь с чем-то свыкнуться, надо это просто произнести.

-- Это еще зачем?

\emph{Нельзя вешать ярлыки. Никакие. Даже если все кажется идеальным. Нет, особенно если все кажется идеальным!}

Я и так знала, зачем. Просто тянула время.

\emph{Как только ты называешь мужчину своим молодым человеком, ты подсознательно начинаешь ждать от него определенную модель поведения. И чем больше его действия походят на те, что ты там себе успела придумать, тем сильнее ты расслабляешься. }

\emph{И тем болезненнее в дальнейшем будет принять суровую реальность. }

Чьи это мысли? Я не могла вспомнить, хотя, казалось, слышала их совсем недавно.

Похоже, что мои.

-- Слова обретают смысл, когда произносишь их вслух, или что-то вроде того. Короче, если что-то кажется тебе нереальным, нужно это просто озвучить. Ты услышишь собственную мысль несколько раз и быстрее поймешь ее смысл. Давай же, попробуй! Скажи, что ты теперь невестушка!

Стоит ли говорить, что этот процесс не обошелся без заминок? Мне понадобилось пять попыток, две сигареты и новый слой укрывшего щеки румянца, чтобы на полном серьезе сказать вслух эти два несложных слова.

-- Я -- невеста, -- медленно произнесла я, едва ли веря собственному голосу.

Вениамин заулыбался от умиления тому, насколько по-детски это прозвучало.

Веня -- что за человек это был? Таинственный как ночь и предсказуемый как утренняя тошнота после знатной попойки. Он часто использовал в речи одни и те же фразы, произнося их с одинаковыми интонациями, так что вскоре предугадать следующее слово сделалось проще простого. Он не любил пиво, но потреблял его с завидной регулярностью, вероятно, бастуя против системы. Любил складные ножи и таскал с собой целую кучу холодного оружия, заботливо раскиданного по карманам джинс. При этом Вениамин имел крайне двойственную репутацию: малознакомые люди (в частности, женщины) его чуть ли не боготворили, тогда как в родном городе моего избранника буквально считали отбросом общества. Он несколько раз попадал в участок, (и каждый раз это происходило из-за неадекватного поведения в нетрезвом состоянии) убегал от приставов, пару раз просыпался рядом с совершенно незнакомыми ему людьми, а однажды пришел домой в одних штанах и понятия не имел, куда делось все остальное, включая документы и телефон. Веня всегда выходил покурить после третьей стопки, обожал коктейли с абсентом и домашние настойки. Не то чтоб его можно было назвать гурманом\ldots{} В отношении алкоголя мой мужчина руководствовался простым правилом: чем крепче -- тем лучше, благодаря чему состоял в черном списке многих питейных заведений разных городов Крыма.

Он носил шелковые рубашки, старомодную шляпу и не менее старомодные жилетки, не вылезая при этом из треснувших по швам ботинок. Последние, кстати, были демисезонными. По какой-то странной причине, Веня чрезмерно долго завязывал шнурки и делал это явно не так, как все нормальные люди. По этой же причине обычно он и вовсе не зашнуровывал свои ботинки. А еще Вениамин практически никогда не давал логически обоснованных ответов.

-- Зачем ты их носишь в такую жару? -- не прекращала спрашивать я, указывая на уже давно отжившие свое ботинки.

-- Они водонепроницаемые, -- отвечал Веня так, словно это многое объясняло.

Он обожал все острое и закидывался халапеньо как семечками, редко вылезал из любимой одежды и еще реже соглашался залезть под душ. Я никогда с точностью не могла понять, умный он или глупый. С одной стороны, Веня довольно много знал. Он мог рассказать мне кучу вещей о любой сфере деятельности. С другой -- мой жених имел крайне ограниченный круг интересов, сводящийся к алкоголю, фильмам, работе, котикам и мне. Правда, слегка в другом порядке.

Как и все мужчины, Вениамин был до невозможности противоречив. Забавней всего было то, что утром он мог выразить мне какую-то мысль, а к вечеру начинал активно ее оспаривать. Да еще и с таким рвением, словно и впрямь забывал, что мысль-то -- его. Видимо, противоречивость -- это обязательное качество каждого представителя сильного пола. Что-то вроде критических дней, которые нахрен никому не упали, но и без них тоже никуда. Только для мальчиков. Я знаю очень многих мужчин. Я с ними работала, дружила, встречалась, жила, имела родственные связи и все такое прочее. Но я не знаю ни одного мужчины, который бы, в конечном счете, не оказался противоречивым. Порой они сами себя так называют, а на следующий день отрицают это до последнего.

Наверное, в этом и есть смысл.

\hypertarget{chapter-18}{%
\chapter{~}\label{chapter-18}}

Есть у меня давняя подруга -- Марта Ульянова. Она познакомилась со своим мужем за год до нашей встречи с Вениамином. Хотя и повстречались они во Львове, Семен был родом из Беларуси. Он никогда толком путешествовал, имел суицидально-агрессивные порывы на почве алкоголизма, отсутствие жизненных перспектив, послужной список принятых внутривенно веществ и не менее послужной список вариаций на тему самоубийства. У него не было хоть какого-нибудь опыта работы, образования, выдержки и одного глаза. При всем при этом, человеком Семен оказался, в общем-то, неплохим. Позже мы с ним даже подружились.

Так вот, рассказав мне о своих первых внеплановых отношениях, (которые к тому же были единственными) Марта ожидала услышать в моем голосе нотки разочарования.

-- В глазах общества у меня самый хреновый мужик на свете, -- заявила она.

И вышла замуж, когда они были знакомы всего два месяца. Процесс исцеления был долгим, не спорю, но по истечении нескольких лет Семен избавился от каждой из перечисленных выше характеристик. Ну, кроме отсутствия глаза.

В общем, услышав о моем внезапном, но очень серьезном (что делало его еще более внезапным) романе с Вениамином, первым делом Марта спросила:

-- Вы принимали вместе ванну?

То есть, я рассказала ей о том, что Веня предложил мне руку и сердце и о том, что мне, по всей видимости, придется переехать в Крым. А ей были интересны совместные купания. Но я не слишком-то удивилась. Марта -- она такой человек, который оценивает любые ситуации, основываясь исключительно на своих действиях, вкусах, планах и воспоминаниях. Все дело в том, что одним из первых счастливых воспоминаний Марты и ее мужа стало принятие ванны.

Чтоб вы понимали, природа наградила Марту быстрым умом, способным усваивать, анализировать и вырабатывать тонны информации. Она была экспертом по таймменеджменту и гордостью всех учебных заведений, в которых когда-либо училась. Все это выработало в ней комплекс отличницы, из-за чего Марта отказывалась собирать волосы в косу, даже когда те ей мешали. А том, что она втихоря носит очки, я вообще узнала спустя десять лет после нашего знакомства.

Мы ходили в одну школу, имели общие интересы и общих друзей, но никогда толком не общались. Я постоянно слышала ее фамилию в списках претендентов на получение разного рода грамот и дипломов. Марта же знала меня визуально.

\emph{Такую девушку-вечеринку сложно не заметить}, позже сказала она.

Думаю, это продолжалось бы и дальше, если бы не события одного осеннего дня две тысячи шестого года.

В школе я всегда была чем-то средним между отличницей и троечницей. Как это вообще возможно? Секрет прост: я принципиально занималась лишь теми предметами, которые мне нравились и никогда ни у кого не списывала. Нравились мне языки и литература, искусство и многое другое. Зато математика во всех ее проявлениях стала мне более чем ненавистной. Это случилось благодаря экспериментальному классу, в который меня отдали в возрасте шести лет. Из-за этого последующие два года я имела по семь-восемь уроков математики каждый божий день, включая субботу. Таким образом, сменив школу, я читала книги на алгебре и писала собственные рассказы на геометрии.

Можно сказать, что эксперимент не удался. Если его целью не было пробуждение внутреннего гуманитария.

Так вот, по окончанию средней школы, мне нужно было написать несколько тестов по литературе и дополнить все это дело своим эссе. Это вроде как открывало передо мной перспективы бесплатного обучения в одном неплохом заведении, которое должно было дать мне корочку для поступления в ВУЗ -- так считала моя учительница литературы. И после долгих уговоров ей удалось убедить меня написать мой первый текст на заказ.

Каково же было мое удивление, когда я увидела финальные списки претендентов. Там было сказано, что помимо меня чести удостоилась еще и некая Марта из параллельного класса. Но место было всего одно, и я никак не ожидала встретить конкуренцию на последнем этапе отбора.

\emph{Нужно выяснить, что это за сучка}, подумала я.

Так я познакомилась со своей лучшей подругой, Мартой.

***

Близкое общение началось спустя четыре года, в институте, когда мне исполнилось шестнадцать. Тогда мы обе учились на факультете иностранной филологии. В те дни у Марты не было ни пирсинга, ни татуировок, ни черных волос, ни каре, ни, кстати говоря, сисек. Все это появилось спустя лет пять. Говоря же о наших студенческих деньках, Марта была очень стройной и носила длинные рыжие волосы. Невероятно красивые.

Расцвет этой дружбы как раз припал на пик моей необъяснимой тоски по шестидесятым. И глядя на то, как Марта -- в ее свободных, извечно цветастых одеждах -- дополняет что-то на лекциях и без конца тянет руку на семинарах, меня то и уносили флешбеки несуществующего прошлого. Какие-то отрывки воспоминаний о временах без Интернета, днях и ночах, которыми люди живут в настоящем и полностью поглощены этим процессом.

Танцы, палатки под звездным небом, песни у костра и очень много дороги -- все это молниеносными волнами пролетало сквозь меня пока я в очередной раз наблюдала за тем, как подруга рассказывает что-то скучающей аудитории. Погрузиться в картинку полноценно мне не удавалось, (или тогда я просто еще не умела этого делать) и я просто восторженно наблюдала за этими едва уловимыми всплесками сознания, стараясь хоть что-то сохранить в памяти.

Лишь один эпизод мне удалось запомнить более-менее детально. Дело было в каком-то лесу. Нас окружали высокие хвойные деревья, чьи верхушки, должно быть, касались неба. Где-то за ними скрывалось солнце, и мы долго шли по хрустящим веткам пока, наконец, не добрались до какого-то холма. Сперва до моих ушей еще доносился гул дороги, но вскоре он исчез. Здесь было пленительно спокойно. Даже птицы молчали. Тишину нарушало лишь парящее в воздухе пение.

\emph{Here comes the sun}

Я прекрасно знала слова потому, что сама их пела.

\emph{Here comes the sun}

\emph{And I say, it's all right}

На возвышении лежало старое дерево, а сквозь его ствол струились солнечные лучи, яркими стрелами пронзая завесу лесного дыма. Марта стремительно побежала вверх по склону.

-- Там внизу поляна! -- крикнула она.

Цветастые юбки мелькнули над упавшим деревом, и подруга исчезла из моего поля зрения.

Я поспешила за ней, но, поднявшись на холм, решила задержаться. Уселась на ствол и принялась внимательно разглядывать открывшийся отсюда вид: глухую лесную стену, ярко освещенную поляну и Марту, с широко разведенными руками кружащуюся на зеленой-зеленой траве. Рыжина ее волос подыгрывала в такт солнцу, но что-то в этой идеальной картинке показалось мне странным.

Ее волосы! Они были раза в три короче.

-- Когда ты подстригла волосы? -- спросила я на спуске с холма.

Должно быть, я выглядела странно. Подруга взглянула на меня как на лунатика и продолжила танцевать. Тогда я поймала ее руку и повторила свой вопрос.

-- О чем ты говоришь? -- с улыбкой спросила Марта. -- Они только-только отрасли. На лучше курни. Видок у тебя странный!

Она протянула мне трубку, но курнуть я не решилась. Голова раскалывалась, лесная картинка шла кругом, а девушка передо мной раздвоилась: одна Марта, с волосами ниже плеч, по-прежнему стояла посреди поляны, вторая, с волосами до талии, взялась непойми откуда. Обе подпирали руками бока и смотрели на меня с неприкрытой тревогой.

-- Лиза, с тобой все в порядке? -- спросила лесная Марта.

-- Лиза, с тобой все в порядке? -- спросила Марта с волосами до талии.

Голос последней звучал издали. Игнорируя головную боль, я сосредоточилась на второй Марте, и вскоре обнаружила, что та находится в пустой аудитории. Позади девушки виднелся пустой экран проектора, а передо мной лежала открытая тетрадь, но вместо конспекта под сегодняшней датой 26.09.2010 виднелась надпись:

Hic Sol Venit

Рядом был набросок крохотной елки.

-- Лиза?

Происходящее обескураживало, и мне понадобилось какое-то время, чтобы вспомнить, что Лиза -- это я.

-- Лиза, ты на связи? Прием.

-- Прием-прием, -- наконец, ответила я, и поднялась со своего места. -- Пара давно закончилась?

-- Минут пять назад. У тебя все в порядке?

-- Да, просто уснула.

-- Но я видела, ты сидела с открытыми глазами!

Я отмахнулась.

-- Мне такое уже говорили. Иногда я сплю с открытыми глазами, когда сильно устаю.

-- Ничего себе вот это у человека бессонница! -- восхитилась Марта.

Она продолжила укладывать канцелярию в сумку.

-- А что за пара хоть была? -- как бы невзначай поинтересовалась я прежде, чем последовать ее примеру.

-- Латынь, а что?

-- Да ниче. Одолжишь конспект на вечерок?

***

Марте о своих флешбеках я, конечно же, ничего не говорила. Пару раз та ловила мой взгляд и, видимо, решала, что я в восторге от ее знаний, потому как начинала говорить еще громче и серьезней.

Мне было все равно, что этих воспоминаний никогда не существовало в нашей Вселенной. Просто не могло существовать, ведь я родилась слишком поздно. Да и запомнить большинство таких флешбеков не делалось возможным, но их атмосфера -- ощущение покоя и счастья нахождения на своем месте -- никак не хотела забываться.

\emph{Я так когда-нибудь суициднусь}, подумалось мне после очередного припадка. \emph{Очнусь, обнаружу, что все это неправда и суициднусь. Точно-точно.}

И я решила, что однажды я обязательно верну шестидесятые. Ну, если не суициднусь.

***

Как-то в конце сентября мы возвращались домой. На остановку с нами шла еще одна девочка -- Алена. Это была школьная подруга Марты.

-- По ночам я тихонько подкрадываюсь к своему братику, -- говорила Алена, -- и дырявлю ему палец своим острым ногтем. Потом присасываюсь и пью кровь пока родители не придут с работы\ldots{}

Она еще долго рассказывала истории вроде этой, а я разрывалась между желанием включить диктофон и порывом орнуть в голосину. Ведомая этими эквивалентными чувствами, я посмотрела на Марту. Как и всегда, вид у нее был очень серьезный.

На протяжении всего пути Алена продолжала рассказывать свои неадекватные истории. Затем, наконец, увидела свою маршрутку и помчалась в ее сторону.

-- Как же мне осточертели эти люди.

Я едва сдерживала улыбку.

-- Мне тоже, -- поддержала меня Марта.

-- У тебя есть друзья?

Она призадумалась.

-- Нет. Думаю, нет.

-- Почему?

Марта пожала плечами, а затем указала на отдаляющуюся от нас Алену.

-- Вот, наверное, поэтому.

-- Ну, так давай дружить! -- я протянула ей руку.

-- А давай!

И мы действительно стали друзьями. Вот так вот просто, по обоюдному согласию. Как в каком-нибудь старом фильме. И вот, спустя шесть лет я позвонила в Волковыск, рассказала Марте о своем женихе, а та спросила:

-- Вы принимали вместе ванну?

К тому моменту она уже успела посетить уйму стран, переехать в Беларусь, обзавестись мужем, татуировками и обширным опытом принятия различных веществ. Кое в чем я тоже не отставала.

-- Так точно! -- отчиталась я.

В первые дни знакомства мы действительно купались вместе. Ванна была маленькой, а вот я -- не очень. Вениамин же был стройным, но высоким. Каким-то чудом мы, все-таки, поместились в крохотную ванную, да так, что в ней даже оставалось место.

-- Ты так приятно пахнешь, -- произнес мой мужчина, обнимая меня сзади.

-- Жаль, не могу сказать того же.

И мы рассмеялись.

Вся ванна была наполнена пеной. Кран оставался открытым, и скорее пена начала вываливаться за пределы ванны. Она напоминала огромные хлопья снега, которые как по волшебству перенеслись в этот жаркий предлетний день. У меня есть фотография этого феерического момента. К счастью, ее мало кто видел, но должна признать: я никогда прежде не была замечена с такой детской улыбкой.

Покончив с водными процедурами, мы принялись плескаться водой, бросая друг в друга остатки пены. Единственной вещью, которую Вениамин никогда не снимал, были деревянные бусы, купленные им у одного этнического старца, ежедневно расхаживающего по набережной Севастополя. Мой возлюбленный бесился с такой силой, что эти бусы то и дело подлетали, цепляясь за его нос. Низ его лица превратился в сплошную пену, благодаря чему Веня напоминал мальчишку, который измазался мылом, пытаясь сделать себе искусственную бороду. Думается, сейчас, до кончика волос измазанный пеной для ванны и улыбаясь во все зубы, он и был мальчишкой.

-- Странно\ldots{} -- начала я.

Мы уже выпустили воду из ванной. Веня смывал бальзам с моих волос. Я сидела под стенкой, обхватив руками колени.

-- Что?

-- Такое непривычное чувство.

-- Что за чувство?

-- Кажется, я счастлива. Даже припомнить не могу, когда я в последний раз была счастливой и чувствовала себя так\ldots{} -- я замолчала, пытаясь подобрать верное слово.

Перед глазами всплывали картины раннего детства.

-- Как?

-- Так спокойно.

Не сводя взгляда друг с друга, мы на какое-то время забыли о разговорах, наслаждаясь долгожданным спокойствием.

-- Ну, а что насчет тебя? -- наконец, спросила я. -- Ты-то когда в последний раз чувствовал себя счастливым.

-- До встречи с тобой?

Я кивнула.

-- Почти уверен, что никогда.

\hypertarget{chapter-19}{%
\chapter{~}\label{chapter-19}}

Однажды мы сидели в парке, попивали холодный кофе и разговаривая по душам. По сравнению с остальными днями, жара стояла умеренная. Вениамин растянутся поперек скамейки. Я же лежала, умостив голову ему на колени.

Вскоре к нам подбежала толпа местных бойскаутов. Во главе у них была стеснительная женщина с огромным фотоаппаратом.

-- Мы ищем парня с большой бородой, -- запыхавшись, произнес один из детей.

-- Кто ж не ищет? -- с улыбкой ответила я.

-- Правильно говорить `длинной', -- поправила его женщина.

Веня дружелюбно развел руками, как бы говоря: «Вот он я!».

Ко мне обратился другой ребёнок.

-- Вы можете заплести ему косу? -- увидев замешательство на моем лице, мальчик добавил: -- Пожалуйста! Мы очень отстали! До конца квеста аж целых два задания!

Я поднялась и взглянула на своего суженого.

-- Ты не против?

-- Вообще нет.

Веня снял шляпу, и я уже запустила пальцы в его спутавшиеся волосы, когда вновь заговорил первый ребёнок.

-- Вы не поняли. Коса нужна из большой бороды.

-- Длинной, -- снова напомнила женщина.

-- Ага. Длинной.

Дети в ожидании смотрели на нас. Я с трудом сдерживала желание расхохотаться, завидев выражение лица Вениамина. Какое-то время он пытался отказаться, но вскоре детишки его уговорили.

Я засучила рукава и принялась плести косу.

Из толпы вышла девочка. Она восторженно смотрела на мою левую руку.

-- Нам ещё нужна девушка с большой татуировкой!

-- Правильно говорить\ldots{} -- начала женщина, но, в итоге, подвисла.

Вот это детишкам повезло. Они сфотографировались с нами и умчались в сторону фонтана, переполненные той особенной, искренней радостью к ничего не значащим вещам, которую мы можем испытывать лишь в детстве.

***

-- Ты никогда не рассказывал мне о своих родителях, -- вдруг поняла я, глядя за тем, как бойскауты покидают парк.

-- У меня не очень благополучная семья.

-- Не хочешь говорить об этом? Я не обижусь.

-- А вот и нет. Сегодня мне впервые в жизни хочется об этом поговорить, -- внезапно ответил Веня.

И он начал свой долгий рассказ. Как оказалось, отец Вениамина на протяжении многих лет изменял своей жене. Он был не прочь выпить и, кажется, даже пару раз поднимал на нее руку. При этом Веня продолжал считать отца своим лучшим другим. Он говорил о нем с уважением и каким-то странным для меня благоговением. С другой стороны, мать Вениамина была отнюдь не подарком. Со слов мужа, за ней имелся какой-то пунктик помешательства на деньгах, что очень напрягало всех членов семьи. Со временем это переросло в настоящую манию. Она принялась скупать кучу золота и укладывать его в большую хрустальную чашу. К тому же, она постоянно выдумывала какие-то гадости, (сначала о своём муже, а затем и о сыне) которые затем рассказывала большому количеству посторонних людей.

-- Было много неприятных моментов, -- грустно произнес Веня.

-- Например, каких?

-- Ну, например, в детстве мне приходилось самому готовить. Когда мне было лет десять, мама вдруг стала готовить отдельно для себя и моей сестры. Просто ставила перед ней еду, а остальное уносила к себе.

Я в ужасе посмотрела на жениха.

-- И это далеко не самое худшее, -- заверил он.

Однажды отец Вени -- Алексей Вениаминович -- устал от такой обстановки, и объявил жене, что хочет развода. В ответ та схватила дочь, все деньги и драгоценности, которые смогла найти, и испарилась в неизвестном направлении.

-- Она меня бросила. За все эти годы ни разу не приехала ко мне. Отцовская любовница и то проявила ко мне больше заботы.

Его голос звучал отрешенно. Так, словно Вениамин рассказывал мне услышанную где-то историю. То, что произошло с кем угодно, только не с ним самим.

-- Ты до сих пор с ней не общаешься? -- спросила я.

-- Она вернулась год назад. Сказала, что снова будет жить в нашей квартире потому, что закончились деньги. А отец переехал на дачу.

Затем Веня рассказал ещё море не менее ужасных историй, после чего меня нежданно-негаданно стало одолевать желание встречать его вкусными ужинами, будить пахнущими завтраками и все такое прочее. В общем, мне внезапно захотелось стать примерной женой образца пятидесятых и потратить каждую минуту своей жизни исключительно на заботу об этом мужчине.

Слава богу, это был всего лишь мимолётный порыв.

-- Она отвратительная женщина.

-- Ты ее простил?

-- Мне абсолютно наплевать. Уже давно, -- отчеканил Веня.

Я ему не поверила.

\hypertarget{chapter-20}{%
\chapter{~}\label{chapter-20}}

Весна стремительно близилась к своему завершению. С каждым мгновением солнечные лучи делались все настойчивей, асфальт плавился, а воздух становился горячее. Тем не менее, мы все так же проводили дни за долгими разговорами, распитием спиртного и прогулками через весь город. Еще мы подолгу валялись в постели, предаваясь любви друг к другу. Откровенно говоря, секса было так много, что по приезду домой я увидела, что моя талия заметно уменьшилась в размерах.

Но и до моего отъезда успело произойти еще немало чудесных вещей.

В конце рабочей недели мы решили прогуляться по слегка необычному маршруту. Напротив остановки около нашего дома находился супермаркет, где мы часто делали покупки, по-детски радуясь этому семейному процессу. Помню, первым что мы купили, стала пара бокалов и формочки для льда -- ну, кто бы сомневался. По традиции, около входа в магазин находилась крохотная кофейня с хорошим кофе и плохой баристой. Приходя туда, я каждый раз взбивала молоко вместо нее. С правой стороны от маркета виднелся ряд магазинов, плавно перетекающий в толпу баров и кафешек.

Так вот, пройдя несколько метров между этими строениями, ты попадаешь на идеально прямую дорогу. С краю стоит заправочная станция, а за ней находится просторный путь, уводящий в соседний район. Первую половину дороги понемногу спускаешься вниз. Затем спуск становится резче, а на пути встречается старый пешеходный мост, перекинутый через некогда существующую реку, на месте которой теперь нет ничего, кроме пустоши. Второй отрезок пути -- зеркальное отражение своего предшественника. Сначала тебя ждет резкий, в после -- более плавный подъем. Никаких развилок и поворотов. Таким образом, маршрут представляет собой букву С, лежащую на спине.

Еще стоя в самом начале пути, мы прекрасно видели все, что находилось на другом конце. В основном это были высотки вперемешку с привычными двориками, полными растительности, лавочек и качелей.

Такой вот маршрут мы выбрали. Вернее, нам пришлось это сделать потому, что на другом конце Веню ждала клиентка, припозднившаяся на склад. Она должна была отдать деньги за купленные в начале недели краски для волос, или что-то вроде того. Сумма была неплохой, Вене причиталось получить свои пятнадцать процентов.

Мы вышли вечером, около половины восьмого. На удивление, нужная нам улица оказалась пустынной. За все время пути нам не встретилось ни души.

-- Какой он, твой родной город? -- спросила я, подкуривая первую за вечер сигарету.

-- Красивый и маленький, -- после небольших раздумий ответил Вениамин. -- Прямо как мой член.

И пускай мы оба знали о том, что его агрегат нельзя назвать маленьким, Веня никак не мог прекратить шутить на эту тему.

-- И что в нем красивого? Я имею в виду\ldots{}

Он не дал мне закончить.

-- Не знаю, но женщинам нравится.

-- \ldots Керчь.

Наконец-таки осознав, что в двадцать пятый раз шутки про пенис уже не кажутся смешными, Вениамин ответил:

-- Поехали, сама увидишь.

\emph{О, мой многословный мужчина!}

-- Когда?

Я обожала поездки всех видов и направлений. Частенько ездила куда-то в моменты печали, творческого упадка или душевных терзаний. Бывало, даже не знала конечного пункта назначения, но все равно ехала потому, что в итоге всегда чувствовала себя лучше. Как ни крути, а в дороге одиночество ощущается куда меньше. Наверное, дело здесь в том, что любое движение подразумевает наличие цели. Отправляясь в путь, ты временно обеспечиваешь себя каким-никаким, а, все-таки, смыслом жизни. Это иллюзорное чувство спокойствия не раз позволяло мне отвлечься от тоски и сосредоточиться на решении других, более материальных проблем.

В общем, к тому времени я уже была специалистом по всякого рода одиноким поездкам. Думаю, моих знаний вполне хватило бы на то, чтоб распределить их на категории и создать собственную программу семинаров. Что-то вроде «Куда поехать, если с вами расстались в скайпе?», «Как преодолеть потерю работы с помощью билета в один конец?» и так далее и тому подобное.

Чего я, собственно, не знала, так это, каково это -- путешествовать с любимым человеком. Несмотря на наличие бывших возлюбленных, с этим пунктом у меня как-то не сложилось. Но мне всегда представлялось, что совместные поездки -- это что-то удивительное. Пожалуй, именно поэтому я так быстро соглашалась со всеми бредовыми идеями жениха.

Мои размышления прервал голос Вени.

-- Поехали в следующем месяце? Мне как раз нужно будет съездить в Керчь.

Я вздохнула.

-- Ты же знаешь, что к тому моменту я уже уеду.

-- А ты возвращайся.

Он сказал это спокойно и ласково. Так, словно мы уже миллион лет были в месте. Словно не боялся моего отказа и точно знал, что все у нас будет хорошо. Вениамин вел себя так на протяжении всего времени, что мы были знакомы. Спокойствие и уверенность, безусловно, входили в список вещей, за которые я его обожала. Если я ломала ноготь, слышала какую-то грубость, не находила своего размера в магазине одежды, получала очередной отказ от издательства, встречала резкую критику или попадала в любую другую неприятную ситуацию, стоило мне посмотреть на Вениамина, и раздражение проходило. Казалось, ничто не могло заставить его унывать.

Вот и сейчас, глядя на него, я не могла спрятать улыбку.

-- Я бы взглянула на Керчь.

-- Познакомлю тебя с папой, покажу каменоломни и свой любимый бар.

-- Куда же без этого.

Он развел руками.

-- И на пляж сходим?

-- И на пляж сходим.

-- Есть одно но\ldots{} -- сказал Веня.

Видно было, что жениху неловко, и он не особо хочет заканчивать эту фразу.

-- Ну что там еще такое?

-- В Керчи обо мне ходит не самая лучшая слава.

Как будто я этого не знала.

-- Ты хочешь сказать, на меня все будут смотреть с сожалением?

-- Ну, вроде того.

-- А потом отводить в сторонку и уговаривать бежать от тебя, куда глаза глядят?

Он улыбнулся.

-- Именно так все и будет.

Я поцеловала своего мужчину.

Веня был значительно выше меня. Закинув руку мне на плечо и обняв за шею, он спокойно мог продолжить свой путь, не чувствуя при этом дискомфорта. Так мы и шли, приближаясь к пешеходному мостику.

-- Пускай болтают. А я буду улыбаться и потягивать винишко, -- ответила я. -- Расскажу им о нашем медовом месяце в Шотландии и о том, что мы сделаем одинаковые футболки со своим фото.

-- И о двух прелестных детишках, -- подхватил Веня. -- Белокурой девочке и кареглазом мальчике.

-- Скажу, что в свои девять Йозеф в идеале знает четыре языка. А потом извинюсь, согнусь в реверансе и объявлю, что мне пора идти. Забрать Аглаю Вениаминовну из музыкальной школы, которую она заканчивает в следующем году.

-- Она играет на рояле, флейте и кларнете.

-- Одновременно.

После этой реплики мы разразились звонким смехом.

-- Аглая Вениаминовна\ldots{} -- сквозь смех произнес Веня. -- Где ты вообще это взяла?

-- Подсмотрела у Достоевского.

Мы шагнули на мост, по-прежнему оставаясь единственными пешеходами во всей округе. С правой стороны уже запоздало садилось солнце. Его оранжевые лучи заливали отрезок, по которому много лет назад протекала то ли река, то ли ручей. Купающееся в золоте, это место больше не выглядело таким забытым. Мне даже стало немного жаль того, что солнце скоро сядет, и местность у моих ног вновь обратится никому не нужной пустошью. Так ведь и с людьми бывает.

Я озвучила свою мысль. Веня ее одобрил.

-- Запиши это как-нибудь, -- сказал он.

Я так и сделала. Пускай и с заметной задержкой.

***

Обсуждая призрачное совместное будущее, мы, сами того не замечая, подошли к высоткам. Вениамин отыскал нужный дом и отзвонился клиентке. Та сказала, что сейчас подойдет, и жених исчез в уютной мгле дворика, последовавшей за закатом.

Я осталась ждать у дороги, разглядывая местность. Вокруг были невысокие кустарники, детские площадки, магазин, аптека, неработающий сигаретный киоск и ряд придорожных фонарей, которые как раз начинали загораться. В какой бы город я не приезжала, все спальные районы бывших советских стран всегда выглядели одинаково. При этом, несмотря на серую монотонность, было в них что-то приятно печальное. Как воспоминания о детских днях, которым никогда не суждено вернуться.

Несколько минут прошли в полнейшем одиночестве. Я все так же стояла, думая свои прозаические думы и всматриваясь в темноту двора, когда краем глаза заметила какое-то движение. Откуда-то из-за угла появилась женщина лет на пятнадцать старше меня. На ней была пижама и что-то очень странное, при ближайшем рассмотрении оказавшееся термобигудями в виде плюшевых пенисов. Я прямо-таки обалдела.

-- Вы -- Вениамин? -- громко спросила женщина.

Я прямо-таки обалдела во второй раз, и даже обернулась по сторонам, дабы убедиться, что позади меня не стоит Веня. Само собой, никакого Вени около меня не наблюдалось. Должна признать, на протяжении жизни я встречала разных людей (по большей части, не очень хороших) и как они только меня не называли! Но Вениамином -- никогда.

Даже не стану акцентировать внимание на том, что тем днем на мне было надето платье, а волосы как обычно волнами ниспадали на плечи. В общем, мне понадобилось секунд пять для того, чтобы собраться с мыслями.

\emph{Она назвала меня Вениамином?}

\emph{Какая разница, когда у нее члены в волосах! }

И еще десять на то, чтоб не расхохотаться.

-- Нет, -- ответила я, стараясь звучать максимально нейтрально. -- Вениамин -- мой жених.

Не вдаваясь в расспросы, женщина

\emph{(ЧЛЕНЫ В ВОЛОСАХ!)}

изъяла стопку денег откуда-то из-за пазухи и протянула ее мне.

-- Ну, по-всякому в жизни бывает, -- сказала она и вскорости исчезла за углом дома.

\emph{Не то слово, мамзеля}, подумала я.

Как только эта внезапная воительница за права трансгендеров скрылась с горизонта, из злополучного дворика появился мой благоверный. Растерянный и одинокий он медленно подошел ко мне, подкуривая очередную сигарету.

-- Нашел свою клиентку? -- с улыбкой спросила я.

-- Да.

-- Забрал деньги?

Он помедлил.

-- Нет. Она сказала, что отдала их\ldots{} Вениамину.

Больше не в силах скрывать улыбку, я отдала ему конверт.

-- Вообще-то я подумывала однажды взять твою фамилию, -- сказала я. -- Но вышло немного иначе.

Оставшуюся часть вечера мы провели, раскачиваясь на качелях.

\hypertarget{chapter-21}{%
\chapter{~}\label{chapter-21}}

Временами я смотрела на Веню абсолютно восторженным взглядом. Особенно привлекательным он казался мне, занимаясь своими делами. Будь это работа, чистка ботинок или вождение. Словом, любая сосредоточенность добавляла мужчине сразу несколько баллов по шкале очарования. Самым поразительным в этой ситуации было то, что Веня, казалось, и вовсе не догадывался о существовании собственной харизмы.

-- Ты у меня такой красивый, -- говорила я в такие моменты.

-- Все ты неправильно говоришь. -- поправлял Веня. -- правильно говорить: мы с тобой такая красивая пара.

Он любил, когда я носила очки, пускай мы оба знали о том, что со зрением у меня все не так плохо. А ещё Вениамин обожал, когда я подбирала волосы, закалывая их на макушке. Все, кому не лень, говорили, что так я выгляжу лучше. Знаете, открытое лицо и все такое, но мне это вообще не нравилось. Честно говоря, я по сей день терпеть не могу то, как смотрюсь с собранными волосами, так что если я это и делала, то исключительно для Вени. Или потому, что этого требовал этикет работы за барной стойкой.

Помнится, однажды, в первые дни моего приезда, мы как обычно околачивались в центре. Мне полюбилась одна придорожная пекарня, которую окружали салоны, парикмахерские, магазины и прочие пристанища клиентов Вениамина. Как и все мои любимые заведения, пекарня была совсем небольшой, если не сказать маленькой. Треть зала занимали высокие стойки и витрины, на которых красовалась выложенная в ряды выпечка. Всевозможные кексы, пончики, пирожные, рогалики, ватрушки, пирожки и печенье наполняли воздух праздничным ароматом. В углу стояла кофе-машина, а вдоль стеклянных стен тянулась стойка, предназначенная для клиентов. Здесь я проводила по несколько часов в день. В большинстве своём за кофе и чтением.

Был у пекарни и задний дворик. А ещё там был открытый вай-фай, так что порой, когда в карманах не наблюдалось лишних денег, мы с Вениамином устраивались на заднем крылечке и подключались к сети. Напротив были захудалые дворики в вперемешку с не менее захудалыми гаражами, а около ступеней каждый день стоял чей-то ржавеющий москвич.

Одним из таких дней я ждала прихода Вениамина. Тот отвез товар и теперь стоял около гаражей. На нем были все те же рваные джинсы, чёрная футболка, которая выигрышно обтягивала торс, и огромные солнцезащитные очки. Позади золотились высокие купола стоящей неподалёку церкви. Волосы Веня перекинул на правую сторону, и они сияли на солнце в тон куполам. Ещё не полностью отделавшись от рабочего духа, Вениамин прикуривал правой рукой. В левой он держал прайсы, которые внимательно просматривал. Вид у него был до ужаса деловой. И до ужаса обворожительный.

Желая поймать момент, я тихонько сфотографировала своего жениха. Позже, глядя на этот снимок, я не прекращала поражаться очарованию Вениамина, лучшей деталью которого было то, что вид этот Веня принимал вполне естественно. И знать об этом не знал.

***

-- Мое любимое мороженное, -- гордо заявил Веня, протягивая мне внушительных видов рожок. Тот был двухцветным и сочетал в себе ягодное и медовое мороженое. -- Я за ним ездил сюда из самой Керчи. Все детство.

А это было без малого шесть часов пути, если не учитывать пробки и остановку в столице. Я терпеть не могла плодово-ягодное мороженное, но так и не решилась сказать об этом Вене. Уж слишком довольное и по-детски радостное у него было лицо при виде этой сладости.

Взяв любимого под руку, я шла вдоль Приморского бульвара. Справа от нас виднелся памятник затопленным кораблями. Это был утес из гранитных глыб, искусственно выложенный на воде. Выше -- светлая колонна, на капители которой восседал двуглавый орел. Присмотревшись, на головах птицы можно было увидеть императорскую корону, а в клювах -- внушительных масштабов венок из желудей, лавровых и дубовых листьев. Еще одной примечательной деталью памятника затопленным кораблям был якорь. Утонченный, тот крепился к венку и ниспадал на цепи.

Около монумента толпились туристы. Некоторые бродили сами по себе, периодически делая селфи. Другие были в сопровождении гидов и внимательно следили за их рассказами. Со всех сторон до моих ушей доносились обрывки фраз, взятых из путеводителя.

-- \ldots культурное наследие Крыма\ldots{}

-- \ldots проект скульптора академика\ldots{}

-- \ldots в тысяча девятьсот пятом\ldots{}

-- \ldots эмблема Севастополя\ldots{}

-- \ldots в честь пятидесятилетия обороны города\ldots{}

И так далее, до бесконечности.

Плавной дугой бульвар сворачивал направо. Подле него покоились отточенные водой каменные глыбы. То и дело дул бодрящий ветер. Волны игриво касались языками цементных глыб, омывая бульвар. Со всех сторон показывалась морская пена. Она ударялась о крохотные скалы и тут же исчезала, чтобы через мгновенье вновь предстать нашим взорам.

И, все же, казалось, все было наоборот, и это бульвар омывал могущественное синее море.

Вдоволь налюбовавшись туристами, мы поднялись по закругляющейся лестнице и вскоре оказались на смотровой площадке. Отсюда бульвар, памятник, да и линия воды казались какими-то сказочными. Никогда не перестану изумляться красоте моря. Где бы я не находилась, и какие бы мысли не тяготили мне душу, шум прибоя способен унести за собой любые тревоги. Всему виной море с его неизмеримыми широтами, игривыми волнами, пеной, что томно остается на берегу как легкое воспоминание о прошедшем шторме. Море с его волнительными закатами и упоительными рассветами, лазурной гладью, парящими чайками, и теплыми огнями, украдкой зовущими тебя в далекие края. Море с его бескрайними берегами, оживленными причалами, старыми портами и тихими бухтами. И прежде всего -- с его манящей таинственностью. Никогда не перестану изумляться красоте моря. Глядя на него, я словно заглядываю в свою душу, позволяя сознанию многое переосмыслить.

Конечно, Бродский как всегда был прав, советуя каждому страждущему сесть в поезд и высадиться у моря. Эти строки знают все, однако мало кого интересует их окончание.

\emph{Оно обширнее. Оно}

\emph{и глубже. Это превосходство}

\emph{не слишком радостное. Но}

\emph{уж если чувствовать сиротство, }

\emph{то лучше в тех местах, чей вид }

\emph{волнует, нежели язвит.}

-- так заканчивается его «С видом на море».

Сколько себя помню, меня всегда снедало одиночество. В большинстве своем, это было вызвано отсутствием родственной души, которую я, неистово исповедуя культ любви, искала в будущем избраннике. Стряслось это со мной годиков в тринадцать, и чем больше лет проходило, тем более одинокой я себя чувствовала. Вокруг всегда было много людей разных интересов и возрастных категорий. Друзья, приятели, знакомые, парни\ldots{} К семнадцати годам я окончательно осознала, что мне плевать на их внимание. Пару раз я встречала мужчин, которые завладевали моим сердцем, не прикладывая при этом особых усилий. Первым оказался Эрик. Ему тогда было семнадцать, мне -- пятнадцать. Мы общались несколько лет, и я не могла не замечать, что Эрика ко мне постоянно тянуло. Но что-то вечно мешало ему разобраться в своих чувствах. Так у нас ничего и не сложилось, а я все ждала и ждала, когда же он одумается. Меня нисколько не интересовали другие ухажеры, число которых увеличивалось с каждым годом.

Думаю, еще в те дни в моей голове ясно укрепилась идея о том, что в одиночестве счастлив не будешь. Печальный настрой, на самом деле. Я любила путешествовать, но не видела в этом смысла, если рядом нет любимого человека. С кем тогда я буду делить свою радость? Ведомая немалыми амбициями, я почему-то отказывала себе во многом, считая это пустым. При этом я никогда не мечтала стать чьей-нибудь женой и ни разу не думала о свадьбе. Мысль о том, что счастье должно вытекать из самодостаточности и оставаться в тебе вне зависимости от окружения или его отсутствия пару раз приходила мне в голову, но я настойчиво ее прогоняла. Все, чего мне хотелось -- быть с близким по духу человеком, а Эрик мне виделся именно таким.

Прошло несколько лет прежде, чем я смогла окончательно забыть об Эрике. Расставшись с очередной девушкой, он тут же искал со мной общения. Затем находил новую пассию и все повторялось. Перешагнув восемнадцатилетний рубеж, я, наконец, осознала свою глупость. Так я распрощалась со своей школьной любовью, чтобы в девятнадцать вновь влюбиться как последняя школьница. На этот раз в Адама. И теперь чувства были сильнее. К тому же, они оказались взаимными, и я с трепетом размышляла о нашем будущем. Мне хотелось совместных поездок, душевных разговоров и, конечно же, обсуждения творчества друг друга. Как и моя первая любовь, Адам был музыкантом. Я уже говорила, что он ушел от меня за неделю до моего двадцатилетия и это действительно расстроило меня до глубины души. Лишь по прошествии времени я поняла, что три года -- слишком долгий срок для того, чтобы убиваться по мужчине. Но так было с Эриком. Так случилось и с Адамом.

Не поймите меня неправильно, я никогда толком не бегала за мужчиной. Не кричала о своей любви после разрыва отношений. Увы, это вовсе не значит, что она исчезала. Любовь была для меня неотъемлемой частичкой гармонии не только с внешним миром, но и с самой собой. Пожалуй, именно в этом и крылась проблема, но я была так верна идее вечной любви, что просто не замечала таких очевидностей. Будучи по природе своей склонной к самоанализу, я, тем не менее, не так давно осознала причины собственных сердечных трагедий.

\hypertarget{chapter-22}{%
\chapter{~}\label{chapter-22}}

Случилось это за полгода до нашего знакомства с Вениамином. Сама того не желая, я внезапно уличила в предательстве близкую подругу. Ту самую, чье имя уже давно было вытатуировано на моей руке. Уставшая от постоянной лжи, слез и боли я, наконец-таки, явно почувствовала, как же сильно мне осточертели эти депрессии. Почувствовала и буквально заставила себя пойти на постоянную работу, которая отнимает все время.

\emph{Депрессия? У меня нет на нее времени.}

Что бы там не говорили о трудоголизме, но он явно помог мне начать ценить свободные минутки. Это привело к тому, что я и впрямь начала осознавать ценность таких простых, но приятных вещей как вечерние чтение и утренняя чашка кофе. В один из редких выходных две тысячи пятнадцатого я сидела, раскурившись марихуаной, и наблюдала за зимним рассветом. За окнами стоял глухой февраль. Я сделала глубокую тягу и, задержав дым в легких, продолжила размышлять о смысле жизни. Тогда я лишь начинала всерьез увлекаться буддизмом и с радостью штудировала все его источники. Особенно сильно я ценила многовековые книги.

Было раннее воскресное утро, но солнце все никак не хотело показываться за горизонтом, как всегда бывает в феврале. Стоявшие неподалеку дома продолжали дремать под снежным одеялом, и лишь кое-где понемногу начинал зажигаться свет. Пустынные улицы также покрывал толстый шар снега. Снег шел всю ночь. Никем еще не тронутый, он блестел и переливался в свете уличных фонарей. Временами дул легкий ветер, с нежностью поднимал крошечные снежинки, разнося их по округе.

Укутавшись в плед, я сидела на крыше дома, что был соседним от моего. В отличии от остальных домов этого района, здесь вход на крышу всегда оставался незапертым. К тому же, с нее открывался неплохой вид на все те же парки, аптеки, дороги и игровые площадки. Подо мной было девять этажей, если не считать чердака, а на коленях торжественно покоился недокуренный косяк.

Хотя на улице по-прежнему было мрачно и явно стоял минус, холода практически не ощущался. Мыслями я находилась где-то очень далеко, но подсознательно поглядывала на восток в ожидании того, что солнце покажет свои первые лучи.

Само собой, я думала об Адаме,

\emph{Чертов Адам с его ореховыми глазами и очаровательными ямочками на щеках!}

а вместе с ним и обо всех своих вместе взятых несчастных отношениях. О людях и о местах, которые когда-то были так дороги моему сердцу.

\emph{There are places I remember}

\emph{All my life though some have changed\ldots{}}

Меня всегда поражало то, с какой легкостью люди заводят новые отношения, будь то любовные или дружеские связи. Ведь если я люблю, то всегда самоотверженно, всей душой. Иначе я попросту не вижу смысла в отношениях. Все или ничего, смекаете? Не удивительно, что при таком раскладе у меня уходили годы на восстановление после разрыва с близким человеком. И даже поле того, как я о нем напрочь забуду, порой какой-то предмет, звук или запах вызывает во мне старые воспоминания. Убитая любовь вылезает из могилы, и вот что-то уже тихонько отзывается в моей груди. Ноет, не позволяя забыть о прошлом.

Сидя на крыше и то и дело поглядывая на восток, я вспоминала нашу первую встречу.

***

Октябрь только начинал одевать деревья в золото, когда я вышла в коридор поезда, что направлялся в Москву. Дорога заняла сутки. То были часы, наполненные всеми ужасами, какие только может предоставить украинская железная дорога. Естественно, я их не заметила. Я была влюблена и счастлива, так что даже самая неудобная полка в поезде казалась мне мягким диваном, а самое грязное стекло поражало прекрасными видами осенних пейзажей. Считая себя сильным человеком, я всегда подсознательно тянулась к тем, кто морально был еще сильнее меня. Сила духа, творческая жилка и преданность своему делу -- вот первые признаки, которые я искала в своем спутнике жизни.

Рядом со мной ехала светловолосая миниатюрная девушка по имени Анастасия. Вероятно, ей было очень скучно, потому как Настя всячески пыталась завести со мной диалог. Помимо невероятно узкой талии таинственно сочетающейся с бюстом пятого размера, я не заметила в ней ничего необычного. Знакомиться с кем-то совсем не хотелось, но мой центральный орган кровообращения томился в сладостном предвкушении грядущей встречи. Словом, настроение у меня было чудесное, так что я машинально поддержала диалог.

-- Никак к парню едешь? -- спросила Настя, видя мои горящие глаза.

Что ж, на протяжении всей взрослой жизни я старательно пыталась замаскировать свои чувства. Не подавать признаков влюбленности хотя бы до того момента, пока она не окажется взаимной. Наверное, мне просто не хотелось выглядеть дурой в глазах окружающих, если что-то пойдет не так. В сравнении с другими девушками, мне это удавалось на ура. Я размышляла спокойно, мало жестикулировала и говорила без резких, будоражащих слух интонаций, которые так часто выдают влюбленных. Помнила куда иду и зачем и никогда не терялась на средине мысли. И, все же, я ничего не могла поделать со своими глазами, которые то и дело светились радостью предвкушения грядущей встречи.

-- К знакомому, -- уклончиво ответила я. -- И к подруге.

Последнее, кстати, было правдой.

-- А я же от разводиться еду, -- удрученно вздохнула Настя.

Она захлопнула свой «Космополитан» и принялась в деталях пересказывать мне последние два года своей жизни. За это время я успела трижды выпить чай, выкурить пять сигарет и придумать несколько имен для персонажей будущей книги. Односторонний разговор моей новой знакомой был таким долгим и беспощадным, что я даже не сразу поняла, что Настя его закончила. Временами она делала большие паузы. То ли для пущего трагизма, то ли для того, чтоб скушать очередную котлету. Ими же, кстати, она настойчиво пыталась накормить и меня, а я никогда не могла есть перед важным событием. Дальше в ход пошла картошка, яйца и шоколад. В отличии от Насти, я не была обладательницей крохотной талии, так что я каждый раз отказывалась под предлогом диеты.

\emph{Я не могу есть при посторонних. Не могу есть, когда волнуюсь. Не могу есть, пока не разрешу важный вопрос} -- настоящий аргумент звучал бы стремно, согласитесь.

-- А ты собсно чим занимаешься? -- беспечно поинтересовалась Настя.

Так, словно ее полуторачасового монолога и вовсе не было.

К слову, у Насти был очень странный говор. Местами она говорила четко, как бы стараясь воспроизвести московское произношение с рвением прилежного студента, а затем забывалась и в рассказах девушки начинали проскакивать деревенские словечки. В итоге выходила лютая смесь русского и украинского, которая не поддавалась правилам грамматики ни одного из этих языков. В моих краях такую речь называли суржиком.

-- Я пишу книги.

-- Так ты отой, писатель?

-- С чего ты взяла? -- не удержалась я, и спутница посмотрела на меня в глубоком замешательстве.

С нечеловеческим усилием воли и божьей помощью мне удалось подавить ухмылку.

-- Всего лишь шучу, -- поспешно объяснила я. -- Да, я писатель.

Настя призналась, что за свою жизнь не прочитала ни одной книги, если не считать кратких пересказов, которые она просматривала вместо чтения школьной программы. При этом, ее почему-то жутко заинтересовали мои опусы. Ведомая этим непонятным порывом, собеседница принялась засыпать меня уймой вопросов.

Как часто я пишу? -- стоит придти вдохновению, и я хватаюсь за книгу, где бы ни была.

Как я пишу? -- от руки; в бесконечных блокнотах хорошо заточенным карандашом.

Как я знаю, что надо сделать перерыв? -- когда понимаю, что закончила каждую из появившихся мыслей -- все до единой -- и мне просто больше нечего сказать; писатель всегда сам чувствует, когда нужно отложить работы и подзарядиться идеями от реальной жизни.

Как я придумываю имя для основных персонажей? -- обычно если герой является моим альтер-эго, его или ее имя будет начинаться заглавную букву моего имени, а фамилия должна непременно начинаться с Р.

-- Почему Р?

-- Чтоб никто не догадался.

-- Смысле?

-- Потому, что я ее не выговариваю, - спокойно ответила я, и улыбнулась.

Шутка, которую понимают только в узких кругах.

-- Правда что ле? А я и не замитыла.

-- Никто не замечает, пока я сама не скажу.

-- Как так?

-- Это потому, что я не картавлю. Я просто ее игнорирую, заменяя каким-нибудь похожим звуком. -- сказала я уже, наверное, в пятисотый раз за последние девятнадцать лет. -- Хотя, местами бывает сложно. Особенно если после «р» идет мягкий знак и они находятся в конце слова. Тогда меня просто никто не понимает.

Спустя еще одну чашку чая с расспросами было покончено. Настя вернулась к своему журналу, и я не без облегчения откинулась на стену и поспешила надеть наушники на случай если девушка передумает. Уши обволакивали шуршащие нотки Интерпола. Наконец, можно вернуться к греющим душу думам: о наступающем утре, об Адаме, нашей первой встрече и том, как он меня поцелует.

\emph{Интересно, сколько он продержится без поцелуя?} -- подумалось мне.

Я взглянула в собственное отражение, улыбающееся мне со стального поручня, и поняла, что недолго.

\hypertarget{chapter-23}{%
\chapter{~}\label{chapter-23}}

Он был тем самым молодым человеком, подходящим моему правилу трех «п» -- преданный своему делу, полон жизни и похотливый. Признаться, я влюбилась в его голос еще до того, как увидела самого Адама.

Миновав поле, поезд заехал на акведук. По обе стороны простирался Днепр. Я мечтательно любовалась извилистыми линиями на воде, оставленными после небольших волн и поражалась тому, с какой красотой в них играют солнечные блики.

Отдавала ли я себе отчёт в том, что бывала счастлива лишь когда мое сердце переполняла любовью и любая из ее производных? Думаю нет, но я искренне верила, что это и есть настоящее счастье. Вдоволь насмотревшись на проносящиеся за окном поля, леса, долины и озера, я взгромоздила на колени огромную записную книгу и принялась писать. Вместо ожидаемого очерка из-под моей руки вышло сентиментальное письмо, занявшее три с половиной листа. Его я вручила своей подруге и попросила показать мне лишь в том случае, если я сама об этом попрошу.

\emph{Здравствуй, милая моя!}

\emph{Этой ночью я практически не спала, волнение было столь сильным, что у меня (впервые в жизни!) тряслись коленки. Однако, утром меня охватило странное спокойствие и сейчас, сидя за этим письмом в поезде, я все так же спокойна.}

\emph{Только что осознала, что, зайдя в вагон, почувствовала, что вернулась домой. Все поезда для меня одинаковы и в каждом я ощущаю себя дома. Подумать только, сколько эмоций я пережила за эти двадцать лет, находясь в дороге. У меня так много воспоминаний, но я знаю одно: я никогда еще не была так спокойна и счастлива как сейчас.}

\emph{В моем городе невыносимый холод и совершенно отсутствует золотая осень, чего не скажешь о местах, которые видны из окна. Слева замерзшее поле, а по правую сторону раскинулся лес. Он невероятно прекрасен. Кое-где мелькают небольшие домики -- это то, о чем я мечтаю, милая. Все эти осенние цвета, я так влюблена в них, приятный холод и шуршание листьев\ldots{} вся моя жизнь прошла бы осенью, будь это возможно. }

\emph{Снова-таки, впервые в жизнь, я чувствую, что встретила того самого и это изменило меня, сделала той, кем мне стоило бы быть. }

\emph{Странно, что мне пришло это в голову, но мне весьма нравятся мои короткие волосы (хотя, кто-то бы их таковыми не назвал), это так удобно, а главное -- они такие живые, ведь я уже почти избавилась от всех испорченных волос. }

\emph{Как видишь, мои мысли -- бесконечно хаотичны. Пожалуй, это единственное, что меня огорчает, ведь в наушниках любимая музыка и я думаю о тебе, о нашей дружбе и о том, что, наконец, увижу тебя, ведь ты мой дорогой друг и я безумно тебя люблю, не забывай об этом.}

\emph{Только что выяснила, что курить в поезде уже совсем нельзя. Очень жаль, ведь, зная свою удачу, я обещала Адаму не сходить на перрон до Москвы. Увидев мой трагический взгляд, проводник дала мне вафель. Черт возьми, я перечитала предыдущее предложение и просто не могу остановить смех, шикарно. Я даже не буду его убирать, надеюсь, ты улыбнешься, прочитав это. Хорошее предложение, такое можно диктовать деткам в школе, как пример сложноподчиненного, чертовски смешно. }

\emph{Очень хочется курить, но еще больше мне хочется обнять своего избранника, поэтому я здесь.}

\emph{Знаю, тебе сложно в это поверить, но я ни разу не чувствовала того, что чувствую сейчас к нему. И это чудесно. }

\emph{P.S. Ох, я, все же, не удержалась и пару часов назад вышла прогуляться. Удивительное место, названия которого я не запомнила, станция находится прямо около леса и невероятная осенняя погода: свежий ветер, нежаркое солнце, а вокруг -- сплошная осень. Сейчас я проезжаю места, подобные тому, но погода здесь уже не так дружелюбна. Представляю, что ждет меня в Москве, но главное ведь, кто ждет меня там, правда? }

\emph{Бывает ведь так, я еще не разу не видела его, но уже безумно влюблена и очень тоскую хотя бы по тому, что не могу слышать его голос. }

\emph{B вагоне шумно, а я так устала, чувствую себя невероятно вымотанной, это и не удивительно. И все же, меня весьма удивляет, что я совсем не волнуюсь, думаю, я начну переживать в самый последний момент, хотя я знаю, все в порядке, все в абсолютном полном порядке. }

\emph{Возвращаю свое внимание книге. Не стану окончательно с тобой прощаться, ведь мы с тобой совсем скоро встретимся. Подумать только, спустя несколько часов я его увижу\ldots{} и это делает меня счастливой. }

\emph{Всего хорошего, милая.}

\emph{Твой верный друг, }

\emph{Васляева}

***

Наконец, день сменился вечером, а за ним подкралась долгожданная ночь. Пассажиры уже давно отдыхали на своих полках, наполняя плацкарт храпом всех видов и размеров. Мы как раз подъезжали к белорусской границе. За стеклом проносились черные силуэты деревьев растущих здесь лесов, напоминая о детстве и о традиционных летних поездках в Черниговскую область. О лесе и прилегающему к нему хутору, о ежедневных походах за грибами, о дальней родственнице-тезке, которая была мне как родная бабушка, о дедушке, который на самом деле не был мне родственником. Но все же, о дедушке с его самокрутками, печкой-лежанкой и мастерски написанными стихотворениями, их собаках, самодельном пруде, деревянной беседке посреди дикого озера, огромном дубе, растущим за участком, песчаных карьерах, вафельных тортах и термосе с кофе, которые мы с кузеном утаскивали на плот прежде, чем уплыть вглубь озера.

\emph{С кофой} -- так говорила бабушка Лиза.

Воспоминания одно за одним проносились перед моими глазами. Подложив руки под голову, я укуталась в одеяло и прикипела глазами к незашторенному окну, мечтательно глядя на звёзды.

В январе в моей голове зародилась идея нового романа. Мысль о нем я вывешивала без малого девять месяцев и лишь в сентябре начала делать первые заметки. Книга была о Луне и относилась к научной фантастике, чего нельзя было сказать ни об одной из моих прошлых работ, ни о моих литературных вкусах в целом. И, все-таки, мне она нравилась. Я даже как-то по-матерински полюбила эту книгу и теперь, разглядывая звезды, думала о том, как расскажу о ней Адаму. Этот монолог раз десять успел пронестись в моих мыслях прежде чем я, наконец, уснула.

Наступило утро. Взволнованная я проснулась часа на полтора раньше будильника. До Москвы оставалось более двух часов езды, и я знала, что сейчас Адам по-прежнему дрыхнет в своей постели. Мысль об этом на мгновение наполнила меня пока ещё ничем не обоснованной радостью, как всегда происходило со мной в моменты предвкушения чего-то хорошего, а главное -- долгожданного.

***

-- Я устала от отношений, -- с уверенностью заявляла я в свои неполные двадцать лет. -- Просто хочу чего-то приятного. Совместных прогулок, вечернего распития вина, походов в кино и поцелуев в пустом переулке. Положительных эмоций мне! И душевного тепла подайте. Как в книгах Ремарка и Хемингуэя, что тут сложного? Хочу всепоглощающего романа, который не продлится дольше двух недель. Чтобы потом не нужно расхлебывать все это пост-любовное дерьмо. Пускай будет волшебство, а потом пускай его не будет. Мне бы только нужных эмоций да вдохновения\ldots{}

Говоря это, я действительно верила, что желания мои были далеки от долгосрочных отношений. Все, что я делала прежде, так это необдуманно скакала из одних длительных отношений в другие, стараясь забыть первую школьную любовь. В какой-то момент мне это даже удалось, но я почему-то продолжила встречаться с теми, кого не любила. Наверное, по инерции. Думаю, таким образом мое подсознание пыталось защитить меня от перспективы разбитого сердца, ведь я ничего не чувствовала по отношению к тем парням, а значит они не могли причинить мне боль. К тому же, все это создавало иллюзию отсутствия одиночества. Хотя что может быть хуже, чем отдавать себя не тем людям?

Так длилось около четырех лет. Когда мне стукнуло девятнадцать, я поняла, что окончательно излечилась от Эрика головного мозга. Ещё я поняла, что была глубоко несчастна. Конечно, есть вещи и похуже того, чтобы просыпаться рядом с нелюбимым человеком, но и в этом приятного мало. Словом, обнаружив себя на тонущем корабле односторонних отношений, я их прекратила.

И осталась одна. Дышать стало неописуемо легче. Тут же отпала нужда в том, чтобы посвящать время тому, кого не очень-то горишь желанием видеть. Рядом с собой или вообще. Отпала потребность идти на уступки другому человеку и заниматься тем, чем заниматься вовсе не хочешь. Наконец-то, я могла спокойно писать. Тогда я поняла, что больше никогда не стану строить отношения с нелюбимым человеком. Одиночество меня совсем не пугало. Я даже успела по нему соскучиться и решила, что лучше буду одна, нежели опять стану тратить своё время и энергию на неправильного человека. О таких обычно говорят, что он герой не твоего романа. Это выражение можно было применить к каждому из моих бывших. К тому же, даже в отношениях можно быть самым одиноким.

Распрощавшись со странными, ненужными и непонятными отношениями, я углубилась в работу. Помимо писательства у меня было ещё два менее любимых, но более оплачиваемых дела -- я работала IT-шником и давала уроки английского. Большая часть весны и целое лето прошли в работе и раздумьях. Я старалась разобраться в себе пока последней ночью августа две тысячи тринадцатого не поняла природу всех (ну или хотя бы большинства) своих жизненных ошибок и осознала, что готова идти дальше.

Той же ночью Адам признался, что запал на меня. Мы вели переписку около года и он никогда прежде не видел меня, если не считать фотографий и видео-звонков. Мне нравилась его музыка и постоянное желание дискуссий. Зачастую это была лишь приятная болтовня о погоде, искусстве и том как у кого дела пока однажды, последней ночью лета того года, мы не завели разговор по душам.

Адаму было двадцать пять и незадолго до этого он потерял отца.

-- Я устала от серьёзных отношений. Эта фигня себя не оправдывает. Теперь мне просто хочется отдохнуть и посвятить время поискам самости.

-- А я хочу серьёзных отношений, -- не прекращая повторял Адам. -- Хочу найти девушку на который однажды решу жениться. Можно сказать, я вообще не ищу девушку. Я ищу сразу жену.

Он говорил, что устал от одиночества и считал, что готов завести семью. Наверное, как и каждый одинокий человек, которому стукнула четверть века.

-- Как знаешь, -- ответила я, когда монолог Адама о радостях семейной жизни был окончен.

Мне никогда прежде не хотелось выйти замуж и я никогда не воображала деталей своей свадьбы или того, как я буду идти по церковному проходу с букетом в руках.

-- Я все равно предпочту остаться в стороне.

Адам, конечно же, возражал. Между нами были две тысячи километров и парочка разногласий, которые каждый пытался решить своим путём. Адам рассказывал мне о своей музыке и много спрашивал о моих книгах. Я слушала его и незаметно влюблялась, наивно продолжая полагать, что не хочу ничего кроме коротенького романа.

Месяц и неделя прошли как-то странно. Они тянулись вечность, а вместе с тем умудрились пролететь за один день. И вот я стояла у наглухо закрытого окна в конце вагона и смотрела на стремительно приближающуюся ко мне столицу соседней страны.

\hypertarget{chapter-24}{%
\chapter{~}\label{chapter-24}}

Вы когда-нибудь просыпались в переполненном купе от криков незнакомого младенца, от неожиданности решив, что он ваш, проспав при этом часа три, а затем пытались принять душ в отечественном поезде, (где ванной и вовсе нет) движущемся по еще более отечественным дорогам, уложить волосы и нанести первоклассный макияж? Если так, вы должны понимать, на что мне пришлось пойти ради этого мужчины.

Искусно скрыв хронические мешки под глазами, и сделав вид, что именно такой цветущей и посвежевшей я сегодня и проснулась, я подхватила свой чемодан и неспешно зашагала к выходу. В те дни я носила чёрные волосы с холодным, синим отливом. Они спускались чуть ниже плеч и игриво завивались на концах. Я всегда предпочитала пробор не по центру, так что передние локоны слегка закрывали левую половину лица. Одежду я носила в тон волосам -- черное платье, черное пальто, черное белье, черные чулки, шляпа и, конечно же, черные сапоги. Еще у меня был парфюм с запахом лаванды и пирсинг над верхней губой. Благодаря постоянным отбеливающим маскам от природы смуглая кожа к тому времени стала почти фарфоровой, и я чувствовала себя прекрасно.

Единственное недоразумение представлял собой чемодан, что по незамысловатым обстоятельствам оказался ярко розовым. Мой (чёрный, понятное дело) сломался незадолго до поездки, а единственным который Марта смогла мне одолжить, был этот. Что ж, розовый чемодан вызывал во мне некую неуверенность, но я поспешила принять вид, словно это абсолютно меня не смущало. Удивительно, насколько важной кажется подобная херня в юном возрасте, правда? Спустя пару лет я буду спокойно выходить в магазин в берцах и халате.

Дверь поезда распахнулось. В лицо ударил приятный утренний холод. Свежий осенний воздух приятно обволакивал, навевая целую стопку воспоминаний.

Я выросла около железной дороги. Засыпала, слушая шум отбывающих поездов и голос диктора, объявляющего новые. Толком не знаю почему, но я всегда любила железнодорожные вокзалы по утрам. Вероятно, это было связано с моими детскими разъездами по стране и тем, какие эмоции охватывали меня в предвкушении этих поездок. В такую рань большинство детей спали на руках родителей. Я же энергично прогуливалось вдоль перрона, жадно впитывая царившую на вокзале атмосферу. Вдыхала запахи железной дороги, наслаждалась тишиной и ожидала прибытия поезда.

Лишь с лихвой вдохнув утренний воздух, я взглянула на перрон. Адам стоял в толпе встречающих. Теперь между нами были от силы пять метров. Как я и, он был одет неожиданно легко для такой погоды. И, все же, он был здесь. Стоял совсем близко и улыбался мне счастливой, немного смущенной улыбкой, открывающей ямочки на щеках.

Не отрывая взглядов, мы смотрели друг на друга секунд десять, прежде чем я вспомнила, что преграждаю путь другим пассажирам. Опомнилась, опустила глаза, и заметила, что последняя ступенька подо мной слишком высокая, дабы с неё можно было спокойно сойти. В руках у меня был огромный чемодан,

\emph{Сраный розовый чемодан!}

а на ногах красовались каблуки. Зная о прелестях местного транспорта, я, все же, как-то упустила этот момент и теперь с тревогой, но не без улыбки, вообразила, каким эффектным может стать мое появление.

В конце концов, в узких кругах я была известна именно благодаря таким представлениям. От этих мыслей мне вдруг захотелось громко рассмеяться. И застрелиться.

Тем временем Адам уже шёл навстречу. Он мужественно пытался пробраться сквозь толпу встречавших, но по-прежнему был недостаточно близко, чтобы мне помочь. Пассажиры за моей спиной уже начинали рассерженно толкать друг друга, когда произошло чудо. Пожилой проводник, стоявший около поезда, внезапно взял мой багаж одной рукой. Второй он подхватил меня за талию и буквально спустил на землю вместе с чемоданом.

-- Спасибо, -- удивленно произнесла я.

Проводник молча подмигнул мне и вернулся на исходную.

Я сделала несколько шагов в обход толпы, стараясь принять безразличный вид, когда лицом к лицу столкнулась с Адамом.

-- Ну, здравствуй, -- улыбнулся он.

-- Здравствуй, -- ответила я, уткнувшись носом в плечо Адама потому, что к тому моменту он уже крепко обнимал меня.

В отличии от меня, Адам и не пытался скрывать свои чувства. Он заметно покраснел и неловко выпустил меня из объятий. Зрачки его расширились, что только придавало красоты этим карим глазам.

-- Привет, -- ещё раз поздоровался Адам.

-- Привет, -- глупо повторила я.

Молодой человек предложил взять его под руку, что я и сделала, вручив ему свой чемодан. Мы поспешно зашагали прочь от перрона.

Киевский вокзал оказался чересчур оживленным местом. Почти таким же оживленным, как и сама Москва. Чем дальше мы отдалялись от железной дороги, тем больше становились толпы окружавших нас людей. Павильон кишел суетливыми, спешившими во все стороны гражданами.

Мы остановились на перекур, хотя скорее это был лишь предлог для того, чтобы лучше рассмотреть друг друга. Поднеся к губам сигарету, я с облегчением облокотилась на стену, впервые заметив какие высокие здесь потолки.

Увлеченные собственным обществом, мы поговорили о погоде, о том, как плохо этой ночью спалось каждому из нас и еще какой-то ерунде.

-- Я часа три не мог уснуть, -- признался Адам.

-- А я и не пыталась.

-- Испугался даже, что просплю. А потом встал раньше будильника.

У него был этот быстрый московский акцент, который так портит женщин и так идет мужчинам. Как и все москвичи, Адам, конечно же, полагал, что разговаривает с единственно правильным произношением. При этом он искренне умилялся моему южному акценту.

Сворачивая в сторону парковки, я впервые с удивлением осознала, что вскоре останусь наедине с по сути незнакомым мне человеком. Буду не один час, а вероятно и не один день находится рядом с мужчиной, которого я никогда прежде не видела. Странно, но эта мысль ничуть не пугала.

Мой спутник настоял на том, чтобы я пристегнулась. Он вел свой черный «Мерс» конца 90-х спокойно и сдержанно, не отрывая взгляда от дороги. Ехали мы часа полтора и это с учетом того, что тем утром практически не было пробок. Лишь останавливаясь на светофоре, Адам отводил взгляд от дороги, чтобы одарить меня полной нежности улыбкой. Где-то на половине пути мы угодили в небольшую пробку, и я поймала Адама на том, что он пялится на мои ноги. Я, и обе моих нижних конечности были мысленно польщены этим вниманием, но, приличия ради, я, все-таки, одарила Адама скептическим взглядом. Мужчина тут же покраснел и виновато улыбнулся, явив миру те самые очаровательные ямочки на щеках.

\hypertarget{chapter-25}{%
\chapter{~}\label{chapter-25}}

За окнами моросил легкий дождик, оставшийся с ночи, но погода стояла ясная. Солнце ненавязчиво освещало залитые осенними красками деревья, вдоль которых мы проезжали. Оставался последний отрезок пути. Теперь светофоров не было, как не было и пешеходных переходов. Адам оказался добросовестным водителем. Он пристально следил за дорогой и изредка спрашивал меня о чем-то. Я же, само собой, воспользовалась возможностью внимательнее рассмотреть своего спутника. Он носил короткую бороду и имел каштановые волосы, что чуть касались плеч. Фигуру Адама едва ли можно было назвать стройной, но это ничуть его не портило. Широкие плечи и жилистые руки придавали ему особое мужество. Говоря о чертах лица, мы с Адамом были очень похожи и вполне могли бы сойти за родственников, если бы не его высокий лоб.

Сменив за стеклом несколько улиц, машина повернула в очередной раз и остановилась около небольшого магазина. Вместо того, чтоб выйти на улицу, мы вдруг начали болтать. Тогда я впервые со школьных лет осознала, что нахожусь рядом с мужчиной, к которому, кажется, начинаю что-то чувствовать. Мы говорили, наверное, минут сорок, которые пролетели как четверть часа. Было что-то по-своему очаровательное в этих разговорах на парковке. Позже это превратилось в своего рода традицию и, когда Адам останавливал где-нибудь свой автомобиль, никто не спешил его покидать. Обнявшись или взявшись за руки, мы сидели на передних сиденьях и делились друг с другом сокровенными мыслями. Я обожала эти разговоры почти так же сильно как обожала своего собеседника. В постели мы были увлечены друг другом, а любые разговоры дома все равно заканчивались постелью. На улице же мы гуляли, болтая об искусстве во всех его проявлениях, окружающих нас вещах и чем-то там еще. И только приехав в какой-нибудь пункт назначения и, оставаясь в машине, мы решались открыть друг другу душу. Здесь, в тишине припаркованного автомобиля, ничего не могло помешать нам. И мы по-настоящему говорили. О детстве, друзьях, жизненных целях и о том, что было в наших сердцах. Так происходило каждый день того короткого срока, который продлились наши отношения.

Первый из таких разговоров произошел спустя пару часов после нашей встречи. Закончился он тем, что я не смогла отстегнуть ремень безопасности. Погрузилась в беседу и инстинктивно нажимала на кнопку, но ничего не происходило. Ремень попросту заело.

-- Теперь ты точно от меня не сбежишь, -- смеялся Адам, наблюдая за тем, как я вожусь с ремнем.

-- Еще раз это повторишь, и я откушу тебе нос, -- клятвенно пообещала я.

-- Точно не сбежишь!

-- Не беси меня, мужчина!

-- Тебе придется остаться со мной на веки вечные\ldots{}

-- Я серьезно, откушу только так!

Ситуация делалась комичной, но на каком-то уровне подсознания чертов ремень и впрямь начинал меня раздражать.

-- Ну, так кусай уже, господи боже!

После этой фразы Адам бесстрашно повернулся в мою сторону и наклонился.

Лишь на мгновенье я представила, что и впрямь укушу его. Укушу с той силой, с какой Энди Дюфрейн угрожал откусить член одному очень настойчивому зеку. Так бы и закончились наши не успевшие начаться отношения. Сама мысль об этом рассмешила меня -- мастера визуализации -- до глубины души. Глядя на Адама, чье лицо теперь было в паре сантиметров от моего, я так и застыла с широкой улыбкой. Уж не знаю, какие мысли в этот момент посетили голову моего спутника, но явно не о Дюфрейне, потому как он тут же поцеловал меня. Одной рукой Адам коснулся моего подбородка, другой -- осторожно отстегнул ремень безопасности.

Признаться, я никак не ожидала этого поцелуя, пускай и думала о нем на протяжении всего нашего двухчасового пути. И на протяжении минувших суток, проведенных в поезде. И, возможно, -- только возможно -- что и на протяжении последних шести недель, предшествующих нашей первой встрече.

Оторвавшись от меня, Адам заглушил автомобиль. Я в смущении опустила глаза. До этого момента я и не замечала, как вкусно он пахнет. Пожалуй, это был первый зафиксированный случай в истории, когда мужчине удалось меня смутить. То ли еще будет!

Спустя десять минут мы уже стояли в супермаркете, но все так же не могли оторваться друг от друга. В одной руке Адам держал корзину с покупками, второй сжимал мое запястье. Выглядел он при этом абсолютно счастливым. Стоя между рядами с алкоголем, шоколадом и кофе, мы целовались как последние школьники за гаражами, не обращая ни малейшего внимание на проходивших мимо покупателей.

-- И часто ты целуешься в публичных местах? -- спросила я, радуясь тому, что этим утром решила отказаться от яркой помады.

-- Вообще нет, -- ответил Адам и вновь поцеловал меня.

Глаза его непрестанно сияли, и я уже успела полюбить этот блеск. Стоит ли говорить, что прошедшие часы заставили меня напрочь позабыть о нежелании заводить серьезные отношения? Нет, я не захотела скоропостижно выйти замуж и нарожать Адаму кареглазых детишек. Попросту не думала о том, как далеко может зайти это увлечение.

Из окна его спальни был виден лес, который я зачарованно разглядывала, потягивая виски. Мы снова курили, и Адам не менее зачарованно разглядывал меня. Наконец, одним движением руки Адам подхватил меня за талию и, усадил на диван рядом с собой. На мне было платье, что расстегивалось на спине. Оно покупалось специально для этого случая и уже нетерпеливо ждало возможности быть сброшенным.

-- Как мало человеку нужно для счастья, -- мечтательно произнёс Адам.

Улыбка не покидала его губ.

До этого момента я и не представляла, что мое сердце способно биться с такой скоростью. Буквально чувствовала, как Адама тянет ко мне и не могла отрицать, что ощущаю то же самое. И, все же, мы пролежали в объятьях с добрых полчаса прежде, чем перешли к делу. Перебросив мои волосы на одну сторону, Адам принялся целовать освободившееся от них плечо. Его губы плавно переместились к моей шее, и вскоре я услышала звук расстегивающейся змейки. Негромко шурша, платье медленно сползло на диван, оголяя плечи. Я осталась в одном белье, да чулках с поясом для подвязок. До сих пор чувствовала себя смущенной и ощущала, как кровь приливает к щекам. Несмотря на десяток отношений, до этого дня у меня было всего два любовника, и, что самое смешное, ни один из них в это не верил. Более того, все, кому я об этом говорила, окидывали меня недоверчивым взглядом.

-- Не то, чтобы я осуждала тех, кто меняет мужчин как перчатки\ldots{} -- обычно говорили мне женщины. -- Я считаю, что сексу нет места вне отношений. У меня было только пять парней и я считаю, что каждый из них стал важной частью моей жизни.

Я отвечала, что у меня было всего два партнера, с каждым из которых меня связывали годы отношений. Не думаю, что хоть одна из этих будущих мамочек в декрете и специалистов по наращиванию ресничек мне поверила. На самом деле, я никогда толком и декольте то не носила. Мне казалось смешным, что общество почему-то делает из меня куда более искушенную деву, нежели человек, которым я являлась. В какой-то момент мне даже понравился этот образ иронично настроенной, опытной и не особенно доброй женщины. Пускай мне еще не было двадцати, я прекрасно знала, чем порой оборачивается неприкрытая доброта и помнила, как больно бывает после. В общем, новый исторически появившийся у меня образ, скрывал эту самую доброту, отлично защищал от вероятности того, что очередной приятель -- или, что еще хуже, друг -- смешает меня с дерьмом.

Теперь же я находилась рядом с Адамом в кромешной тишине стен его спальни. Позже он признался, что первый год нашего общения был уверен, что я та еще стерва, безжалостно манипулирующая людьми, и лишь во время нашего первого душевного разговора последним летним днем понял, что это не так. Каким-то загадочным образом, ему удалось заглянуть в мой внутренний мир и увидеть все имеющиеся в нем слабости, которые я так старательно скрывала от общества. А, может, мне просто надоело прятаться.

Повинуясь какому-то внезапному порыву, я обернулась и взглянула прямо в глаза Адаму. Стоит отдать ему должное, в это мгновение Адам даже мельком не взглянул на грудь или другие полуобнаженные части моего тела. Он не сводил взгляда с моих глаз и спокойно ждал того, что будет дальше.

-- Даже не верится, что мы знакомы от силы часа три, -- сказала я, стараясь придать безмятежность своему голосу. -- Вот это мы шлюхи.

И с этими словами я принялась расстегивать пуговицы на его рубашке.

\hypertarget{chapter-26}{%
\chapter{~}\label{chapter-26}}

Пожалуй, нужно и впрямь быть осторожнее со своими желаниями. Так или иначе, той осенью я получила свой двухнедельный роман, который не могла забыть последующие годы. Вернемся же к зиме пятнадцатого и ко мне, беспечно покуривающей косяк на крыше соседней многоэтажки.

Со дня встречи с Адамом прошло два года и четыре месяца. На улице стояло тихое, снежное утро февраля. Как и все мои феврали, этот был мрачным и одиноким. Сидя на крыше и потягивая принесенный из дома в термосе глинтвейн, я вдруг спросила себя:

\emph{Будут ли мои феврали такими же одинокими, когда я отыщу свою родственную душу? }

Первые полгода после расставания с Адамом я еще как-то держалась. Окрашивание волос в алый цвет стало первым, что я сделала после его прощального звонка. Зачем? Сама не знаю, но мне понравилось. Я накупила себе целую кучу новых туфель и появлялась на людях исключительно при полном параде. Мои губы всегда были подведены алой помадой, а на глазах виднелись осторожно выведенные стрелки. Словом, меня внезапно настиг новый образ, очень походивший на современный пин-ап. Мало кто знал о том, что я продолжала переживать по поводу Адама. В те дни это было известно разве что Мишель -- моей ближайшей, пускай в дальнейшем и не очень галантно поведшей себя подруге да парочке старых знакомых, которым я проговорилась по пьяни. В общем и целом, ни мое окружение, ни, конечно же, сам предмет моих воздыханий не догадывались о том, что я чувствую. При этом какая-то часть моего сознаний по-детски продолжала верить в то, что нас еще ждет счастливый конец.

Большой вклад в попытки восстановить мое душевное здоровье внесла литература. Без сомнений, ее позиция в рейтинге вещей, имеющих смысл в моей жизни, укрепилась еще сильнее. Оставшись одна, я ни дня не проводила без писательства и поражалась тому, что еще в начале осени жаловалась на отсутствие вдохновения. Ложечка душевной боли, щепотка разбитых сердец, приправить суицидиком, и я становлюсь самым продуктивным автором на свете. Независимо от моего желание, главный мужской персонаж романа постепенно приобретал черты былого возлюбленного, и чем больше я этому противилась, тем сильней мой герой походил на Адама.

-- Что бы там ни произошло, я никогда этого не забуду, -- дрожащим голосом произнес Адам за ночь до того, как мне нужно было покинуть Москву.

-- Я тоже. Ни за что не забуду, -- пообещала я, чувствуя, как к глазам подступают слезы.

Последние я сдерживала изо всех сил, но одинокая слеза таки набралась смелости скатиться по моей щеке, и мужчина тут же смахнул ее.

-- Пришлешь мне экземпляр своей книги, когда закончишь? -- спросил Адам.

Я покачала головой.

-- Посмотрим, понравится ли она мне.

-- А может не понравится? -- изумился Адам.

-- Конечно.

-- Но ты все равно пришли. Даже если это будет худшей из твоих идей. Я хочу ее увидеть.

-- Почему?

-- Потому, что она развивалась в твоей голове в то время, как ты была со мной, -- просто объяснил Адам.

Я улыбнулась.

-- Какое самолюбование!

-- Так что, пришлешь?

-- Давай так, -- предложила я. -- Я не стану обещать, что покажу тебе эту книгу, но могу пообещать, что посвящу ее тебе. И если у книги дела пойдут лучше, чем у наших отношений, ты узнаешь о ней. Откроешь, и поймешь, что она для тебя.

Так уж сложилось, что у книги дела и впрямь пошли лучше, чем у наших отношений. Правда я так и не решилась ее издать, но в отличии от нашего с Адамом романом, мое творение прожило гораздо дольше и обрело нуждающийся в продолжении конец. Как я уже говорила, все мои персонажи так или иначе отображали меня и мое окружение. И хотя книга была вовсе не о любви, без влюбленной парочки здесь не обошлось. Думаю, и так понятно, кто был прототипами этих ребят.

\emph{Пускай хотя бы у них все будет в порядке!} -- думала я каждый раз, которым бралась за книгу.

***

Так пронеслась зима, а вслед за ней и несколько недель весны. Я опомнилась лишь в апреле четырнадцатого, когда книга была дописана. Это вынудило меня вернуться к безжалостной реальности. Пожалуй, тогда я впервые по-настоящему осознала, что больше никогда не увижу Адама.

Знаю, бывают вещи и хуже этой и с многими из них мне только предстояло встретиться. Тем не менее, я чувствовала себя слишком грустной и старой для того, чтобы жить спокойной жизнью. К тому же, мой мозг почему-то непрестанно идеализировал образ бывшего возлюбленного, всячески отрицая его негативные качества. Тогда я только начала подозревать о своём психическом расстройстве, которое в юношеские годы казалось обыденными перепадами настроения. Знаете, подросткам свойственно горевать из-за мелочей, но я уже не была подростком, а депрессия становилась все более затяжной.

Я и прежде слышала о биполярном расстройстве, но никогда не относила эти истории к себе, пока однажды не прочитала книгу, где вскользь упоминались присущие заболеванию симптомы. К тому моменту мне уже исполнилось двадцать, и я, наконец, нашла объяснения большей части тех вещей, что происходили со мной на протяжении многих лет.

Весь последующий год я посещала различных докторов, принимала лекарства и проходила терапии. В общем и целом, я сменила около восьми психологов и шести психотерапевтов, одному из которых все же удалось упечь меня в лечебницу. Ни один док толком не мог мне помочь. Рано или поздно становилось ясно: о моем расстройстве местным эскулапам известно меньше, чем мне. Я читала слишком много философов, с детства интересовалась психиатрией, имела толерантность к препаратам и, в отличие от докторов, владела еще и практическими знаниями. Так что с появлением диагноза все это сложилось в один мрачный пазл под названием «Внутри Елизаветы Васляевой», собрать и разобрать который могла только я.

Один терапевт, все же, смог мне помочь. Его звали Сиддхартха Гаутама, а для друзей -- просто Будда.

Все это время меня окружали разные люди, и многие мужчины одаривали меня недвусмысленным вниманием. На протяжении пары летних месяцев я даже виделась со своей школьной любовью. Мы пили мятные настойки, будучи абсолютно голыми, и рассказывали друг другу забавные истории, которые произошли с нами за прошедшие годы. Как и я, Эрик очень изменился внешне, но в душе оставался все тем же беспечным шалопаем. Общаясь с ним, я едва ли могла представить этого молодого человека в серьезной связи. Эрик, который тогда только закончил самые долгие в своей жизни отношения, тоже не имел на меня серьезных видов. Было весело.

И, все же, я хранила обещание, данное себе в девятнадцать лет и не желала строить отношения с тем, к кому не испытывала хоть сколько-нибудь сильных чувств.

«Я буду вместе лишь с тем мужчиной, которого смогу полюбить сильнее, чем Адама» -- однажды написала я в своём дневнике.

Так и получилось.

Так вот, ни алкоголь, ни стремительно растущий круг знакомых, ни впечатляющее разнообразие прописанных докторами таблеток не смогли помочь мне. Бывают в жизни моменты отчаяния, когда ты тупо сидишь на заднице. Уставший, удрученный и побитый судьбой, просто сидишь и ждешь, когда кто-нибудь войдет в дверь, чтобы наладить твою жизнь. В таком состоянии я написала лучшую в своем творчестве трагикомедию, но никто не появился в дверном проеме. Не протянул мне руку и не кинулся решать мои проблемы. Тоска по Адаму, которого я не видела уже больше двух лет, отчаянно не хотела меня покидать. Тогда-то и произошло чудо, по внезапности своей сравнимое лишь со вторым пришествием. За дело взялся единственный человек, способный мне помочь. Как ни странно, этим человеком оказалась я сама.

Случилось это в феврале пятнадцатого, на той самой крыше, о которой я говорила. Не знаю, было здесь дело в травке, буддизме, или рассвете, -- а может мне просто вдруг стало тошно от собственных печалей -- но я внезапно поняла, как должен быть устроен мой внутренний мир. Я больше не стремилась отыскать истинную любовь и не хотела вернуть расположение своего бывшего. Все, чего мне хотелось, так это просто стать счастливой. Тогда-то я и осознала, что счастье не должно быть изменчивым. Какой бы ценной мне не казалась любовь, человек должен уметь быть счастливым вне зависимости от того, состоит он в отношениях или нет. Первое, что должно тебя волновать, -- это отношения с самим собой, второе -- с миром вокруг. Лишь добившись внутренней гармонии можно стать по-настоящему счастливым. В конце концов, счастье -- это не какая-нибудь шаткая переменная. Счастье -- это константа, понимание себя и внутреннее спокойствие, которое находятся с тобой все время. Только приняв себя со всеми изъянами и недостатками можно построить крепкие отношения с другим человеком.

\emph{Прежде, чем ладить с другим, научись ладить с собой}, шепнул мне внутренний голос.

Увы, даже высокая самооценка не является залогом внутренней гармонии. На самом деле, ее наличие довольно легко проверить. Для этого нужно лишь запереться в комнате, отказавшись от социальных сетей и прочих иллюзий чьего-либо присутствия.

Вот так, выдыхая смешанный с марихуаной табак на фоне поднимающегося зимнего солнца, я внезапно стала собственной аудиторией. Мне больше не требовалось литературное признание, верность накануне предавшей меня подруги и наличие спутника жизни. Я вспомнила Адама, его добрые глаза, очаровательные ямочки на щеках, сладкий запах туалетной воды, бархатный голос, кожаную куртку и волосы, ниспадающие на плечи. Его родинку над верхней губой, постоянную щетину, плавно переходящую в бороду, пошлый взгляд и семейки с коняшками. Вспомнила все до мельчайших деталей и мысленно попрощалась с этим человеком.

\hypertarget{chapter-27}{%
\chapter{~}\label{chapter-27}}

Сколько воды утекло прежде, чем я обнаружила себя здесь, на смотровой площадке Севастополя, расположенной напротив памятника затопленным кораблям. Стоя под руку с Вениамином, я внезапно вспомнила о том, каким потерянным было мое состояние до того, как я удосужилась откорректировать собственной мировоззрение. Более того, глядя на своего жениха, я не без шока отметила, что он, видимо, и есть тот цепляющий взгляд ламберсексуал, о котором я грезила во времена депрессии. Это, конечно же, удивило меня, но не расстроило. Просто еще одна деталь громадной мозаики, которая поможет в очередной раз переосмыслить общую картину. Еще тысяча таких вот деталей всплывет прежде, чем я стану прахом. И каждый раз конечный результат будет все более и более удивительным. Наверное, в этом и кроется прелесть жизни. Ее смысл в том, что смысла нет.

Передо мной лоснилось море. Оно шумело и смеялось, унося все тревоги, скопившиеся в головах прогуливающихся по бульвару людей. Я выросла в местах, где море видела почти так же часто как небо. И, все-таки, здесь и сейчас оно казалось мне исключительно прекрасным. Стоя у воды, я жадно впитывала бриз и наслаждалась открывающимся пейзажем с видом молодого лорда Байрона, созерцающего берега родной Англии.

Ничего на свете больше не имело значения. Были лишь солнце и волны, скалы да море, морская пена и тот особенный аромат, который источают исключительно крымские берега. А еще была я и Вениамин, стоявший рядом. Солнце по привычке играло в его волосах, превращая их из пшеничных в ярко золотые, а я была гораздо счастливей, чем могла когда-нибудь вообразить. Стоя здесь, в объятьях любимого человека, я знала, что однажды все это исчезнет. Не будет ни меня, ни моих проблем. Останутся лишь берега и долины, обрамляющие море. Сколько бы лет ни прошло, люди все так же будут приходить к морю в моменты радости и печали. Просто потому, что эта чистота волн и непоколебимость морских глубин помогают нам многое переосмыслить.

Да что тут и говорить: мир построен на одном лишь переосмыслении. Мой так точно.

\hypertarget{chapter-28}{%
\chapter{~}\label{chapter-28}}

-- Взгляни на это! -- возбужденно воскликнула я, наблюдая невиданных размеров солнце, что стремилось к небосводу.

-- Все это, конечно, чудесно, но нам нужно в аптеку, -- напомнил Веня, пристыженно поглядывая на мой глаз.

Должна сказать, последний и впрямь был представлен не в лучшем свете. За десять минут до этого мой женишок в порыве страсти не смог совладать с собственными силами и нечаянно спустил прямиков мне в око. Он проделал это с точностью, достойной зависти оперативника, так что теперь глаз покраснел и невероятно опух. Что еще хуже, из него постоянно текли слезы. Я толком ничего не видела этим глазом. Для тех, чей зрительный орган ни разу не контактировал со спермой, замечу, что чувство это наиприятнейшее. Однажды навстречу мне дул яростный ветер, и кусочек горящей сигаретной головки занесло все в тот же многострадальный глаз. Нынешняя боль отдаленно напоминала тот ожог, только вот чувствовалось это так, словно весь минувший день Вениамин накапливал не сперму, а настоящую кислоту.

\begin{verbatim}
*Что говорить о мужчинах, если даже их семя настолько разрушительное?*
\end{verbatim}

Сперва мы оба хохотали над ситуацией, -- ох, это прекрасное пятисекундное неведение -- а затем меня настигла неописуемая боль. Глаз стремительно начал опухать. Вот вам еще одно доказательство того, что мужчины -- ядовитые создания. Осознав плачевность сложившихся обстоятельств, Веня спохватился и позвонил Валерчику, чья девушка работала медсестрой. После неизбежной минутки смеха нас снарядили списком жидкостей, названия которых я благополучно забыла, как только глаз перестал печь. И вот, мы уже были в нескольких метрах от торгового центра.

На улице стояла внезапная прохлада, более характерная для середины осени. На мне был плащ Вениамина и куча другой, спонтанно попавшейся под руку одежды. При этом я, кажется, впервые за долгое время выскочила на улицу без макияжа и прочих сопутствующих прихорашиванию штук. Пострадавшим глазом оказался левый. Это было не так уж и плохо потому, что я по-прежнему носила косой пробор и ниспадающие на лицо пряди находились как раз таки с левой стороны.

Несмотря на начинавшую набирать обороты боль, на моих губах играла улыбка. Словом, я не теряла бодрого настроя и даже успела умилиться огромному солнечному диску, застывшему над нашими головами.

-- Прости глупого Веньку, -- в который раз повторил мой любимый.

Особой нужды в извинениях не было, но с того дня Вениамин навсегда сделался для меня просто Венькой. Конечно, были еще Венечка, Веня, и на худой конец даже Венюша, но именно эта ласкательно-уменьшительная форма имени показалась мне самой теплой. Сама не знаю, почему.

Внутренне разрываясь между состоянием крайней неловкости и желанием засмеяться на все помещение, мы вошли в аптеку. Веня что-то рассказывал фармацевту, подкрепляя это полученным накануне списком лекарств. Я стояла в сторонке, усердно имитируя неслыханный доселе интерес к детскому питанию.

-- А что с глазом? -- раскатистым басом спросил аптекарь.

Он произнес это настолько громко, что в нашу сторону уже смотрело большинство покупателей, стоявших на кассах напротив аптеки. Было как-то неловко. Я повернулась спиной к посторонним и приподняла волосы, демонстрируя боевую рану. Увидев ее, фармацевт деловито прищурил глаза. Он резко повернулся к Вене и одарил моего мужчину до неприличия недоверчивым взглядом.

-- Аллергия или инфекция? -- продолжил спрашивать мужчина.

Мы ответили секунда в секунду.

-- Аллергия, -- с каменным лицом сказала я.

-- Инфекция, -- не менее серьезно сказал Вениамин.

По всей видимости, наш фармацевт перечитал криминальных сводок. В частности тех, в которых рассказывается о домашнем насилии. Не веря ни единому слову Вени, мужчина продолжал задавать какие-то неуместные вопросы.

Тем временем глаз продолжал печь. Покраснение также росло в размерах.

-- Дайте уже эти хреновы капли! -- вспылил Вениамин, увидев, как на моем лице проступает страдальческая гримаса времен Данте.

Он стукнул кулаком по прилавку, что, наконец, подтолкнуло продавца к единственно правильному действию. Через пару секунд я уже заливала глаз каплями. К тому моменту как мы добрались до парковки боль начала притупляться, и окрыленная этим, я снова принялась любоваться солнцем. Порой у меня такое чувство, что большую часть жизни я именно этим и занимаюсь. Зависаю в моменте, впитываю каждую его деталь, а затем выкладываю увиденное на бумаге. Не самый плохой вариант, на самом деле.

-- Он явно подумал, что я бью тебя, -- сказал Веня, когда мы оказались наедине.

-- Мне тоже так показалось, -- я улыбнулась. -- Пожалуй, наши БДСМ-ные игры не в счет?

-- Пожалуй, что нет.

И мы принялись обмениваться сопутствующими теме шутками, что одна за другой слетали с губ.

-- Ты знаешь, -- произнес Веня немного погодя. -- А ведь моя бывшая рассказывала нашим знакомым, что я ее избиваю.

-- Ты это уже говорил.

-- Да, но этот противный мужик в аптеке\ldots{} От него мне стало как-то не по себе. Ты видела, как он на меня смотрел? Да он был уверен, что это я тебя ударил! И как ему такое только в голову пришло? Чтобы я бил свою деточку -- это, блядь, невообразимо! -- он резко притормозил посреди улицы. -- Какого хрена вообще? Какого хрена вообще он позволил себе так на нас смотреть, Лиза?! Знаешь что! Сейчас я вернусь. Ты стой здесь, а я сейчас вернусь и скажу ему все, что думаю!

Никогда прежде я не видела жениха таким разъяренным, но тесный контакт с различными психами, одним из которых являлась и я сама, научили меня правильному поведению в подобных ситуациях.

\emph{Tiefe Wasser sind nicht still}, напомнил внутренний голос.

\emph{Ой, так вы у нас из Германии!}

Я знала, что делать, дабы избежать конфликта. Нужно было всего лишь отвлечь Вениамина, переключить его внимание на другую тему, и я как бы невзначай спросила:

-- Так и что эти друзья, они верили твоей бывшей?

Мужчина пожал плечами.

-- Некоторые верили. Только в самом начале. Затем и они прекратили.

-- У нее были синяки?

Вениамин взглянул на меня с удивлением.

-- Конечно, не было!

-- Тогда я не понимаю.

-- Я тоже. Как-то раз она пришла в гости к моим друзьям и принялась рассказывать им, что я колотил ее всю ночь. Я в это время был на балконе и толком не слышал, что она говорит. Но в тот же вечер мне об этом рассказали. И рассказали это именно потому, что никто ей не поверил.

Затем Вениамин сказал, что подобное поведение длилось около года. Я показательно вздохнула, обдумывая услышанное.

-- Жертв насилия всегда очень легко опознать. Особенно если речь идет о домашних побоях. Ты знаешь, меня удивляет даже не то, что она врала о тебе такие гадости, а то, что она продолжала при этом с тобой жить.

Веня не нашел, что на это ответить.

Вскоре он забыл о желании поговорить с аптекарем, и мое внимание опять переключилось на небесные красоты. Через дорогу от нас находился детский городок, а рядом тянулась широкая дорога, что спиралью поднималась вверх, уводя автомобили в центр города. На возвышенностях по обе стороны трассы виднелись небольшие домики. За ними то и садилось солнце. Небо сделалось ярко розовым. Лишь кое-где мелькали лиловые просветы. Я смотрела на немыслимых размеров солнечный диск, на деревья, крыши и фонари. Смотрела и как никогда ярко осознавала, как же сильно я люблю этот раскинувшийся в горах городок.

Все это вдруг напомнило мне о старых добрых временах. О тех днях, когда у меня еще была Мишель, которую даже после разрыва продолжала считать своим самым лучшим другом. Возможно, даже единственным.

***

Это были первые недели две тысячи четырнадцатого года. Над городом и за его пределами стояла холодная зима. Кучи блестящего, никем не тронутого снега обволакивали загородную трассу со всех сторон. Миновав блокпост, установленный тут с осени прошлого года, -- именно тогда конфликт Украины с одной близлежащей страной стал наиболее явным -- мы выехали на извечно пустующую в это время года трассу. Справа поворот на Одессу, слева -- моя дача и соседствующий с ней загородный домик подруги. Туда мы и направлялись.

Не прошло и трех месяцев после расставания с Адамом, отчего я чувствовала необратимую печаль каждый божий день. Наверное, это нормально, когда тебе всего двадцать, но мое сердце болело, подкрепленное душевным расстройством, о наличии которого я к тому моменту не имела ни малейшего понятия. Внезапно для всех я превратилась в одновременно очень печального и очень веселого человека. И чем печальнее я себя чувствовала, тем энергичнее я становилась. Как все бывшие возлюбленные и кем-то брошенные люди, сидя в компании, я постоянно перебивала кого-то на полуслове для того, чтобы с несчастным видом сказать:

-- Такой же одеколон был у Адама!

-- Такая же погода стояла, когда мы встретились!

-- Эта музыка играла, когда он впервые поцеловал меня!

-- Именно этот виски мы и пили той ночью!

И так далее и тому подобное по списку. Никогда бы не подумала, что умею быть такой абсурдно грустной. Апогеем стало мой отказ смотреть «Идальго» потому, что однажды на Адаме были те самые трусы с лошадками. Видимо, очень памятные моему переживающему не лучшие времена сердцу.

Обычно после парочки таких заявлений я погружалась в молчание. После чего вскакивала из-за стола или с любого другого места, в котором я находилась, и принималась сходить с ума. Петь песни посреди пешеходного перехода, отплясывать на барной стойке, прыгать с разгона в кусты и творить еще целую кучу вещей, которыми обычно занимаются наркоманы и прочие нетрезвенники.

Вот только я была трезвой.

Конечно же, все эти шалости сопровождались неиссякаемым запасом шуток, за которые друзья меня обожали. Ну, что я могу сказать? Каждый по-разному переживает депрессию.

И вот, на пути к дому Миши, мне в очередной раз сделалось тоскливо. То ли небо было того же оттенка, что и на третий день наших отношений с Адамом, то ли ветер также дул с запада, я уже толком и не помню. Но помню, что от этих мыслей природа за окном вдруг стала серой и совсем непривлекательной. Как и моя жизнь.

Утопая в этой бесперспективности бытия, я сделала музыку громче.

Мы обсуждали любимые сериалы и пели малоизвестные песни Битлз на протяжении всего пути. Где-то там впереди, за облысевшими ветвями деревьев начинало опускаться солнце. Это был ярко оранжевый диск громадных размеров, который, по мере нашего приближения, делался все больше и ярче. Я опустила окно и подкурила сигарету. Мишель тем временем напевала ту самую песню, из-за которой получила свое прозвище.

Я взглянула на свою подругу, ее зеленые глаза и пухлые пальцы, то и дело барабанящие по рулю в такт музыке. Взглянула и подумала, как же здорово, что она есть в моей жизни. Большинство моих знакомых имели огромные круги общения. Я же могла похвастаться таким количеством друзей, разве что, в школьные годы. Со временем рамки моего доверия стремительно сужались и, в итоге, я поняла, что доверять стоит лишь паре-тройке человек. Однажды, еще когда я только готовилась к выпускному классу, Эрик сказал мне, что друг может быть только один. Остальные это просто знакомые. Теперь же я начинала понимать, что в этом вопросе предмет моих школьных грез не был далек от правды.

\emph{И, все-таки, хорошо, когда у тебя есть всего один, но очень близкий и любимый друг, который не то, что тебя с полуслова понимает, а иногда словно сам вкладывает тебе в голову эти самые слова}, думала я.

Потому, что в моменты затяжной депрессии нет ничего лучше, чем мчаться по дороге с близким человеком и слушать любимую музыку, наблюдая впереди, возможно, самый красивый закат в твоей жизни.

В те дни прошедшие годы казались мне старыми добрыми временами. Теперь же, воспоминания пронеслись передо мной яркими вспышками. Я поняла, что сегодня именно те холодные дни, проведенные в обители Миши, стали для меня старыми и добрыми. Пожалуй, осознание важности чего-то невозвратного -- это процесс необратимый.

Колесо Сансары сделало свой оборот.

***

-- Снова ты со своими закатами, -- проследив мой взгляд, с любовью в голосе произнес Веня.

Я подумала, что наверняка однажды буду стоять где-нибудь у окна с видом на исландские долины, пить крепкий черный чай и думать о том, какими славными были дни, проведенные в советской севастопольской квартирке.

С Исландией это, конечно, было не точно, но вот на счет чая я могла быть полностью уверена.

Я виновато улыбнулась.

-- Солнце скоро сядет за холм. Дай полюбоваться еще минутку.

Вдруг Веня взял мою руку и, весело подмигнув, принялся пересекать улицу, само собой, утаскивая меня следом. Мы бежали квартала два, а я, в общем-то, и не знала куда. И вот, когда я, будучи абсолютно неспортивным человеком, уже почувствовала, что в боку сейчас что-то явно сломается, мы так же внезапно остановились.

-- Вот, -- гордо объяснил ситуацию Веня.

С трудом отдышавшись, я обернулась. Мы стояли на тот самом холме, за которым еще пару минут назад собиралось скрыться солнце. На лбу моего Сусанина проступил пот и, усевшись на траву, он принялся вытирать лицо краем рубашки. Я опустилась рядом и уложила голову на плече Вениамина.

Так мы провели самый потрясающий из всех крымских закатов.

\hypertarget{chapter-29}{%
\chapter{~}\label{chapter-29}}

Близился день моего отъезда. Как и всегда, конец отпуска подкрадывался незаметно, но очень стремительно. Так за бесконечными разговорами, прогулками по городу, распитием вина, проводами закатов и бессонными ночами пролетел остаток весны. Лето уже вовсю давало о себе знать, а значит, пришла пора ехать домой -- к родным, собакам и бару. Последний особенно нуждался во мне: изо дня в день коллеги и приятели заваливали меня вопросами о том, когда же я, наконец, вернусь.

На протяжении первых месяцев две тысячи шестнадцатого года я была единственным барменом в ресторане, если не считать мимолетных стажеров. Юные и напуганные предстоящей работой, они исчезали так же быстро, как и появлялись. Я говорю «юные», хотя многие из них были на пару-тройку лет старше меня.

И, все-таки, в моих глазах они были юными, не привыкшими к тяжелой работе молодыми людьми. Некоторые не выдерживали и пары часов, тогда как самые отважные простаивали пару смен. Стоит ли говорить, что каждый из них был со странностями.

«Мы все странные, -- любил говаривать Веня. -- А кто не странный, тот странный.»

Однако, странности моих потенциальных сменщиков таили в себе качества едва ли совместимые с родом их будущей деятельности. К примеру, был один айтишник. Занудный, инфантильный, но, в общем-то, ответственный парень. Проблема крылась в том, что он не мог никому отказать. Угощал всех сотрудников бесплатным кофе и, стоило им только попросить, наливал чего покрепче. Ввиду то ли врожденного, то ли приобретенного в бывшей сфере деятельности задротства, которое нынче принято называть стеснительностью, он не мог сказать «нет». Так, даже самый жалкий и захудалый сотрудник «Штиля» приобретал в глазах моего временного коллеги небывалый авторитет. Что уж говорить о барменах, работающих в клубе? По вечерам они ленились ходить на склад и непрестанно поднимались в ресторан, дабы перед предстоящей ночью запастись алкогольными и безалкогольными напитками.

Как и все представители клуба, помощники барменов и официанты были существами грубыми, не особо обремененными вежливостью. Первые месяцы моей работы в «Штиле» прошли в этом чертовом клубе. Меня не страшило ни отсутствие чаевых, ни график с восьми вечера до шести утра, ни смехотворный оклад, духота и толпы пьяных отбросов, снующие туда-сюда и кричащие что-то невразумительное мне, себе и друг другу. Откровенно говоря, меня не смущали и стриптизеры с танцовщицами, которые периодически подрабатывали проститутками. Даже тот факт, что стоя на рабочем месте, ты извечно видишь либо задницы, либо пироги, обтянутые узкими стрингами и танцующие прямиком на стойке всю ночь напролет, не мог испортить мне настроение. Говоря о клубе, меня напрягали лишь две вещи: громкая, лишенная вкуса и смысла музыка да неадекватный персонал. Включили бы Пинк Флойд -- другое дело!

Порой у меня складывалось впечатление, что во время собеседования будущих сотрудников развлекательного комплекса проверяли на вменяемость, а затем, всех, кто провалил тест, ссылали вниз, в клуб. Если не считать парочки старых, дружелюбных барменов, которых все уважали за количество лет, проведенных в «Штиле», персонал клуба представлял собой сборище неуравновешенных, крикливых и злых людей. В основном, девиц. Особенно ярко эти качества проявлялись среди кассиров и официантов, в то время как стриптизерши оказались самыми тихими и воспитанными представительницами коллектива.

Подгоняемые какой-то неведомой здравому уму силой, работники нижнего этажа «Штиля» ни с кем не здоровались, толкались, топтали друг другу ноги и разговаривали исключительно при помощи крика. Они терпеть не могли друг друга и все вокруг. Даже рабочая запара не была в силах сплотить этих идеальных пациенток Карла Юнга. Единственное, что могло заставить их держаться вместе -- появление нового человека в коллективе.

Официанты ненавидели барменов за то, что те были более эрудированными и выполняли, по их мнению, легкую работу, которая не предполагала беготни среди танцующей и изрядно выпившей толпы. А также за то, что те частенько давали им различного рода поручения, (к примеру, унести подносы с грязной посудой, принести с мойки чистые стаканы или пополнить запасы льда) которые официантам приходилось выполнять. Конечно же, они в равной степени ненавидели и стриптизерш и танцовщиц по вполне предсказуемым причинам. Другие фицы и, в частности, новоприбывшие лица также незамедлительно подвергались моральному насилию. На самом деле, проще сказать, кого не ненавидели официанты цокольного этажа.

Как бы там ни было, осенью две тысячи пятнадцатого предметом их ненависти стала я. Сложно сказать, почему это произошло. В основном, они просто злились на то, что их зарплата была на пару сотен меньше моей. К тому же, на все заведение я оказалась единственной девушкой, занимавшей должность бармена. Среди официанток клуба не было ни одного мужчины, и, гонимые слегка ограниченной логикой, они считали, что я нарушаю какие-то невидимые рамки. В те дни у меня был пирсинг -- над губой и в брови -- и несколько татуировок, кокетливо виднеющихся из-под рабочей формы. Они также добавляли масла в огонь.

Пожалуй, все это не имело особого смысла потому, как официанток ничуть не меньше раздражала моя шевелюра, а происходило это даже в те ночи, когда я собирала ее в объемный пучок или две идеально зачесанных гульки.

***

Половину времени я работала рядом с Антоном -- главным барменом «Штиля». Это был высокий, молчаливый мужчина с грустными синими глазами. Мы познакомились за полгода до того как я попала в развлекательный комплекс. Антон был руководителем школы барменов, где я стала одной из первых учениц. Он отвечал за практическую часть обучения и проводил трехчасовые тренировки по флейрингу, в котором стал общепризнанным мастером и многократным чемпионом

Несмотря на всегда галантную одежду, приятную внешность и неприкрытый талант, Антон сперва мне не понравился. Слишком уж холодными казались мне его глаза, а взгляд часто был каким-то отсутствующим. Добавьте к этому дефицит эмоций, и получите мое первое впечатление о будущем начальнике.

Определенно, в первые дни обучения Антон Качанников казался мне не очень приятной личностью: я никогда не была склонна доверять тихим, безэмоциональным людям. Порою мне даже делалось не по себе от одного его присутствия. Так было до того дня, пока учитель не собрал всех в тренировочном зале, где я вновь оказалась единственной представительницей слабого пола.

-- Почему вы решили стать барменами? -- спросил Антон.

Казалось, вся серьезность этого бренного мира собралась в его вопросе, и стоявшие напротив ученики принялись отвечать с не менее солидным видом.

-- Я считаю, что эта работа открывает широкие перспективы.

-- Мне с детских лет хотелось стать барменом.

-- По-моему, работа барменом дает кучу возможностей\ldots{}

И так далее. Кто-то даже сказал, что ему очень хочется работать с людьми.

Все это время лицо Антона оставалось бесстрастным. Он лишь кивал, услышав очередной ответ, и молча переводил взгляд на следующего ученика.

-- Мне кажется, это так интересно, -- ответил стоявший рядом со мной парень.

-- Ну, а я просто люблю бухать, -- сказала я. -- Люблю я алкоголь и хочу знать о нем все.

-- Правда? -- тень улыбки скользнула по губам Антона.

-- Ага. Так всегда бывает, если я что-то люблю. И вообще я вовсе не собираюсь становиться барменом.

На мгновение огромные глаза нашего тренера стали еще больше. А затем его лицо впервые на моей памяти освятила широчайшая из улыбок. Тут-то я и поняла, что мы, наверное, подружимся.

\hypertarget{chapter-30}{%
\chapter{~}\label{chapter-30}}

Да, я действительно любила алкоголь, и на то был целый ряд причин, о содержании которого вам расскажет любой более-менее вменяемый алкоголик. Впервые мысль стать барменом посетила меня в начале две тысячи пятнадцатого.

Тем вечером я привычно шагала мимо широких полок супермаркета, что были вдоль и поперек уставлены стеклянными бутылками. Виски, коньяк, ром, абсент, бренди, текила, джин, ликеры, бальзамы, настойки и, конечно же, вино -- все это я употребляла с завидной регулярностью и в отношении каждого из перечисленных выше напитков у меня были любимчики.

Не любила я только водку и то лишь благодаря детской травме.

***

Мне было лет семь. Одним жарким летним днем я примчалась с речки и пулей влетела в дачный домик, ну просто умирая от жажды. Сколько себя помню, я всегда пила воду в огромных количествах. Или, по крайней мере, так было прежде, чем я переключилась на алкоголь. Так, я могла залпом опустошить литровую бутылку и не напиться. Смеясь, родители называли меня водохлебом.

Так произошло и в то лето. Я вбежала на кухню, распахнула холодильник, схватила стоявшую внутри бутылку минералки и, прежде, чем мама успела мне что-либо сказать, жадно к ней приложилась. Волею судьбы, бутылка минералки оказалась бутылкой из-под минералки, в которую утром налили содержимое кем-то надбитой бутылки водки. Естественно, всего этого я знать не могла. Сложно сказать, когда именно я поняла, что что-то пошло не так. Жидкость была ледяной и пила я ее очень быстро. Большими глотками. Взахлеб. Бутылка опустела миллилитров на двести, а лицо матушки сделалось совсем испуганным, когда мой мозг, наконец, распознал подмену.

Ротовую полость словно обдало кипятком, а язык мгновенно онемел. Но все это меркло и бледнело по сравнению с тем, что происходило ниже. Внутри все горело, и я вдруг с ужасом поняла, что не могу дышать. Пошатываясь, я неуклюже привалилась к стене и принялась тупо таращиться на бутылку. Та по-прежнему оставалась в моих руках. Судорожно пытаясь втянуть в себя хоть глоток воздуха, я принялась кашлять. По лицу беспрерывно текли слезы. Сначала от боли, затем от обиды, ну, и в конце из-за страха. Перед глазами все плыло, а голова раскалывалась почти так же сильно, как раскалывалась голова одного из античных богов, о котором я читала тем летом. Вскоре из черепа его появилась дочь, а я вот просто стояла и задыхалась. При этом продолжала с ужасом смотреть на пластиковую бутылку -- источник всех моих мучений. Тогда я была уверена, что по какой-то страшной ошибке выпила самый настоящий яд.

Так я впервые попробовала водку. В первый и в последний раз. И так я поняла, что, несмотря на то, что обещанное похмелье, почему-то обошло меня стороной, водку я терпеть не могу. Словом, вопреки ярому пристрастию к алкоголю, я всячески избегала этот продукт. В коктейлях -- без проблем, но только попробуйте подсунуть мне это пойло в чистом виде!

\emph{Ну все}, обреченно подумалось семилетней мне. \emph{Меня отравили.}

Испуганное лицо матери сделалось расплывчатым силуэтом, к которому вскоре присоединился силуэт примчавшегося на ее крик отца.

\emph{Я их достала и они меня отравили}, решила я прежде чем шлепнуться лицом в пол.

***

Уже в двадцатилетнем возрасте, плавно разгуливая вдоль алкогольных прилавков, я вдруг осознала, что совсем не разбираюсь в алкоголе. Чертовски хотелось выпить чего-нибудь эдакого, но я понятия не имела, как это эдакое должно выглядеть. Я где-то слышала о том, что текст, написанный на обороте бутылки, не имеет никакого значения. Верить можно лишь передней части этикетки, а ее, как бы странно это не прозвучало, нужно уметь читать.

Вот я повертела в руках сначала одну бутылку, затем другую\ldots{} Увы, чтение этикетки говорило мне не больше, чем тексты Die Antwoord. Меня это разозлило, ведь в годы новообретенной взрослости мне почему-то казалось, что нет ничего хуже, чем опростоволоситься перед самой собой.

Тем вечером я ушла домой с пустыми руками, а с приходом весны уже числилась в недавно открывшейся школе барменов. В итоге я оказалась не только единственной, кто честно ответил на вопрос Антона, но и единственной, кто сдал все экзамены на отлично и с первого раза получил диплом.

-- Это для той, которая не хочет быть барменом и просто любит бухать, -- с гордостью произнес Кочанчиков, протягивая мне документ. -- Аплодисменты, товарищи!

Тренер принялся хлопать в ладоши, а вслед за ним и собравшиеся на вручение дипломов ученики.

-- Теперь я сертифицированный алкоголик! -- с довольным лицом ответила я, тем же вечером закинула диплом в какой-то ящик и с тех пор его не видела.

Как я уже сказала, мое недоверие к Антону улетучилось. Я искренне восхищалась волшебством, которое он проделывал с бутылками, шейкерами, стаканами и прочими предметами во время флейринг-шоу. Сказать больше, я даже научилась нескольким связкам и с гордостью демонстрировала новое умение по поводу и без.

Спустя несколько недель я познакомилась с семьей своего учителя -- красавицей женой и славным мальчуганом, которому вскоре предстояло пойти в первый класс. А под конец обучения я узнала, что у жены Антона был рак. Ей требовалось срочное лечение в столичной клинике, которое стоило не один десяток местных зарплат. Услышав об этом, я тут же поняла причины всех этих затуманенных взглядов Антона, его отрешенности и отсутствующего вида. Поняла и внезапно устыдилась того, что когда-то испытывала неприязнь к этому трудолюбивому и несчастному человеку. Еще более неловко мне стало, когда ученики школы принялись собирать средства на помощь семье Качанниковых. Они, конечно, скидывались кто сколько мог, но ни один не дал меньше трех сотен. Я же к тому моменту уже второй год была на мели, собирая мелочь на сигареты, а последние сбережения ушли на оплату обучения.

В конце концов, я успокоила себя, дав обещание, что помогу Антону при первой же возможности.

\hypertarget{chapter-31}{%
\chapter{~}\label{chapter-31}}

Спустя четыре месяца мне представилась такая возможность. Антон позвонил мне в первые дни сентября две тысячи пятнадцатого. Он рассказал о нехватке рук и очень просил прийти на стажировку в «Штиль». Тем летом я побывала за обратной стороной стойки в парочке заведений, но ни в одном из них так и не задержалась. Постоянный шум, ненормированный график, низкая оплата и куча невоспитанных и нетрезвых клиентов вызывали во мне отвращение. И, все же, я вспомнила о данном себе обещании и не смогла отклонить просьбу бывшего учителя. А еще мне нужно было что-то кушать.

И вот мы стояли за клубной стойкой. До этого дня я не бывала в «Штиле», но не без удивления отметила, что все здесь выглядело именно так, как я себе это представляла. Зал заполняло человек двести. В основном они делились на три категории: пьяные мужики, выплясывающие в компании себе подобных и периодически прославляя ВДВ, что были не прочь склеить каждую встречную-поперечную, одинокие алкоголики, танцующие сами по себе, словно в другом космическом измерении, и лишь изредка пытавшиеся войти в контакт с другими представителями своего вида и девицы, пришедшие сюда, чтобы подцепить папиков. Последние выглядели куда вульгарней местных проституток. Они просто становились перед обрамляющими зал зеркалами и без устали крутили бедрами, закусывали губы и запускали руки в волосы, рассматривая собственное отражение. За этих мне было особенно стыдно.

В календаре значилась пятница, что принесло клубу переполненный зал и миллион заказов. На стойке то и дело накапливалась гора грязной посуды. Рядом виднелись груды скомканных салфеток и переполненные пепельницы. В то же время чистой посуды катастрофически не хватало, но всем срочно от меня что-то требовалось. Классическое завершение рабочей недели.

Сначала я разливала пиво в вышедшие из пользования бокалы с надписями уже не существующих в «Штиле» марок. Затем подавала Текилу Санрайз в двойном хайболе, водку в рюмке для абсента, а американо в чайной чашке. Официанты, то и дело снующие вдоль стойки, всячески игнорировали мои просьбы. Они отмахивались, когда я просила сходить за чистой посудой, а то и вовсе делали вид, что не слышат меня, стоило мне попросить очистить барную от грязных стаканов и заменить пепельницы. Ситуация накалилась, когда за баром закончились чайники. К тому моменту официанты вокруг тоже закончились. Скрепя зубами, я отправилась на мойку.

На пути мне встретилась горстка тех самых официантов, лениво сидевших в курилке. Картина эта меня разозлила. Я всегда была лояльной к подобного рода отдыху хотя бы потому, что не расставалась с сигаретой с шестнадцатилетнего возраста. Проблема была в том, что из всех сидевших около лифта барышень, лишь одна была курильщицей. И, все-таки, я ничего им не сказала. Проходя мимо, я только с грустью подумала о том, что мой последний перекур был три часа назад.

Тут следует сказать, что я всегда была человеком эмоциональным и прямолинейным. Порою даже слишком. Я никому не позволяла разговаривать с собой на повышенных тонах и не стеснялась в выражениях, когда меня что-то не устраивало. По этой простой причине любой из тех, кто знал меня в юности, усомнился бы в своем зрении, завидев то, с каким спокойствием я прохожу мимо развалившихся в проходе официанток. На самом деле, череда несбывшихся надежд, разрушительная школьная любовь, меланхолия, которая длилась не один год и ситуация с Адамом слегка умерили мой пыл. От всего этого я сделалась грустной и, когда мне стукнуло двадцать, впала в затяжную депрессию.

Моряк не злится на рыбок за то, что те его замечают.

\emph{Особенно, моряк, который решил утопиться.}

Как бы смешно это не звучало, уныние и впрямь сделало меня рассудительной, понимающей и терпеливой. Но даже самое душещипательное расставание не сделало бы меня святой. Я по-прежнему воспламенялась при виде несправедливости, но теперь уже далеко не сразу. Обычно у оппонента было всего две попытки для того, чтобы поступить правильно. Во время этих двух попыток мною владело полнейшее безразличие, -- как внешне, так и внутренне -- а затем я выходила из себя. Случалось это столь бурно и особенно внезапно для незнающего человека.

-- Мне нужны винные бокалы, -- произнесла я уже ближе к концу смены.

Стоявшая рядом официантка и бровью не повела. К счастью, рядом находился Антон. Он громко повторил мой запрос, и официантка нехотя отправилась его выполнять. Вскоре вымытые мной чайники вновь подошли к концу. Не хватало также коньячных рюмок, мартинок и еще уймы всякой всячины, а заказы все сыпались и сыпались. Размышляя об этом, я поставила на раздачу последнюю чашку для эспрессо и в замешательстве посмотрела на экран. Там виднелся новый заказ: четыре эспрессо, бутылка коньяка с посудой на четверых и двенадцать мохито. Помимо всего прочего, для последних не было ни льда, ни стаканов, ни даже мяты.

Окинув взглядом этот список, я со спокойной душой отправилась на первый за ночь перекур. Долгожданная сигарета отказалась приятной, но даже она не смогла подарить мне то удовольствие, которое я испытала по возвращению за стойку. По другую сторону перегородки стояла все та же официантка, внезапно вспомнившая о моем существовании. Девицу уже вовсю распирало от ярости. Ее длинный крючковатый нос то и дело трясся, раздувая широкие ноздри.

-- Где кофе? Где коньяк? -- неистово верещала она. -- Где Мохито? ГДЕ МОХИТО?

Так кричал бы Хемингуэй, вошедший в любимый бар после очень длинного дня, проведенного на солнце.

-- Где посуда? -- спокойно спросила я.

-- Да иди ты нахуй со своей посудой! Я тебя спрашиваю, где мой заказ?

Меня не было минут десять. Все это время весьма нетрезвые клиенты донимали ее аналогичными вопросами и, я полагала, что заданы они были не в самой вежливой форме.

Я развела руками.

-- Мне не в чем его готовить.

К тому моменту около нее появилось еще несколько официантов. Все ждали свои заказы, которые я и не думала делать.

-- Да мне поебать, что тебе не в чем его готовить! Могла бы уже давно поднять свою жирную сраку и принести сюда посуду! Нехуй становиться барменом, если ты нихуя не хочешь делать!

-- Нужно отнести грязную посуду на мойку и принести мне чистую.

Официантка продолжала орать и разошлась еще сильнее, когда другие официанты принялись ей вторить.

Я глубоко вздохнула и вопросительно посмотрела на Антона. Тот медленно кивнул, вероятно читая мои мысли. Он то был в курсе моего взрывного характера.

Получив молчаливое разрешение начальства, я не без радости повернулась к все еще не закрывающим рот фицам. Обвела взглядом их покрасневшие от воплей лица и принялась орать в ответ. Да так, что дремавший за стойкой мужик, которого полночи тщетно пыталось разбудить сразу трое охранников, с хриплым визгом подскочил со своего места.

Той ночью я переорала их всех. И переорала бы еще десятерых, потребуйся мне это, но так и не поняла, для чего это нужно.

Не припомню, дабы после этого хоть одна из клубных девиц обращалась ко мне в неуважительной форме. Прошло полтора месяца, и, вняв моим просьбам, руководство перевело меня в ресторан. А уже в конце зимы я начала стажировать нескончаемый поток молодых людей, пришедших работать в «Штиль». К тому моменту мне исполнилось двадцать два.

Одним из стажеров, что продержались дольше пары смен, оказался двадцатилетний компьютерщик. Я уже говорила о его стеснительности, которая временами граничила со слабоумием. Бармены снизу иногда являлись для того, чтобы взять какой-нибудь товар, но чаще всего для этого посылались официантки. Они-то и наводили на паренька кромешный кошмар.

Все такие же кричащие и обозленные на окружающий мир, они выхватывали из рук моего новоиспеченного сменщика нужные бутылки и мчались обратно в клуб, расталкивая всех на своем пути. Иногда вместо официанток являлась кассир. Она не была особенно старой, но вела себя с наглостью николаевских старух. Карина была полной женщиной очень низкого роста, что не мешало ей разгуливать по «Штилю» с вздернутой вверх головой и гордо выпяченной вперед грудью, которая плавно переходила в живот и коротенькие ножки. Сотрудники называли ее не иначе как Гномиком. Пожалуй, она была единственным из знакомых мне людей, чьи параметры представляли собой идеальный квадрат.

В любом случае, ни Карина, ни официанты, пришедшие из клуба, никогда не вносили в журнал забранный товар. Тем временем мой напуганный сменщик боялся и слово лишнее вставить, а про журнал и думать забыл. Само собой, все это привело к недостачам, которые влетели ему в стоимость заработной платы, и парнишка поспешил ретироваться обратно, в мир растворимого кофе и умных технологий.

В итоге, мне вновь пришлось работать за двоих.

Откровенно говоря, я вовсе не винила скоропостижно сбежавшего сменщика. Проведя несколько лет в сфере IT, я прекрасно понимала причины, побудившие парнишку вернуться к старому делу: короткий рабочий день, комфортные условия, возможность отвлекаться на любимые сериалы и оклад, в десятки раз превышающий зарплату бармена. И, хотя однажды я и дала себе слово, никогда больше не работать на нелюбимой работе, порой мысли о добровольно утраченной зоне комфорта тоской отзывались в моей душе.

\hypertarget{chapter-32}{%
\chapter{~}\label{chapter-32}}

Работа без выходных -- само по себе занятие не из приятных. Еще менее приятным его делал тот факт, что работала я не меньше двенадцати часов. Порой клиенты засиживались до самого утра, делая мой рабочий день, скажем, шестнадцатичасовым. После таких смен я понимала, что ехать домой -- так себе затея, и укладывалась в подсобке, чтобы проспать заслуженные четыре часа.

Никто не мог понять, с какой радости я стала таким трудоголиком, памятуя о моем хроническом упрямстве, идущим под руку с выдающимся самомнением, которое в итоге вызывало у меня отвращение перед всем, что представляло собой «бесполезную работу, которую нужно работать, находясь на работе». Конечно, все как-то с этим живут, но я слишком ценила свое время, дабы добровольно отдавать его на подобного рода нужды.

При этом нельзя сказать, что я слыла ленивым или безответственным сотрудником. Напротив, я всегда старалась подобрать более-менее интересную для себя вакансию и с энтузиазмом приступила к делу. Но, что уж таить, весь этот энтузиазм тут же улетучивался в ту самую минуту, когда я понимала, что обзавелась достаточным багажом знаний и опыта и что текущая работа едва ли сможет научить меня чему-то новому. По этой простой причине я не единожды меняла сферу деятельности. Ведь я могла разбираться в чем угодно, но в действительности считала, что единственное мое ремесло -- это писательство. Так что, со временем тоска по перу брала надо мной верх. Я незамедлительно увольнялась без малейших зазрений совести. Так происходило каждый раз, и, конечно же, «Штиль» не стал исключением.

***

Первый тревожный звоночек раздался за месяц до того, как я встретила Вениамина. Южноукраинская весна уже вовсю давала знать о стремительно приближающемся лете. В воздухе не было прохлады, не было влаги и свежести, из-за которых мне так нравилось бродить по улицам в эту пору года, как не было и клиентов в любом из залов «Штиля». Ресторан пустовал уже который день. Единственными посетителями были редкие парочки, пришедшие поиграть в боулинг.

Стрелка часов уже давно перевалила за полдень, когда мы с Линой уселись на заднем дворе развлекательного комплекса. Привычным жестом я забросила уставшие ноги на некогда пивной столик, и протянула подруге наполненный льдом стакан холодного чая. К тому моменту мы обе бросили курить, (как всегда, временно) так что теперь выходили на воздух просто для того, чтобы им дышать.

Лина была лучшим из того, что дала мне работа в «Штиле». И, пожалуй, единственным положительным моментом.

Мы познакомились минувшей зимой. Тогда Лина еще работала в гардеробе клуба и осветляла волосы, что делало их чересчур неестественными. Родом она была из соседнего ПГТ и при первом же знакомстве, объявила мне, что она -- «проста як дверi». Так оно и было, но, все же, Лина оказалась так не похожа не всех тех грубых, несдержанных и прямо-таки оголтелых сельских девиц, которых я встречала прежде. Она была веселой, прямолинейной и рассудительной, чем мне и понравилась. Было в ней еще кое-что примечательное: спокойный, но пристальный взгляд, которому так не хватало проницательности. Глаза Лины говорили о том, что в ее жизни было немало трагических событий, а также о том, что она смогла их пережить. И это мне понравилось еще больше.

Так вот, тем днем мы сидели на заднем дворике «Штиля» и чего только здесь не было! Вокруг виднелись самые неожиданные предметы: топоры, тележки и супермаркета, дрова, автомобильные запчасти\ldots{} На груде бревен красовались бутафорные березы, выполненные в розовых тонах. За ними грудились пласты никому не нужного искусственного снега, а рядом блестел на солнце побитый жизнью штурвал.

-- Что я здесь делаю? -- монотонно произнесла я, разглядывая покосившиеся ветви искусственных деревьев.

-- Сидишь. Бросаешь курить. Дышишь воздухом, -- Лина улыбнулась и, отхлебнув моего холодного чая, спросила: -- Слух, а это точно чай?

-- Да, это Лонг-айленд.

И я повторила свой вопрос.

-- Та шо ты от меня хочешь? Я не пойму тебя. Ты спрашиваешь это уже четвертый день подряд.

-- Но я имею в виду, что я делаю здесь\ldots{}

-- В «Штиле»?

-- В «Штиле», вне «Штиля». Такое чувство, что теперь моя жизнь делится лишь на два временных пространства. А я все так же не понимаю, что я в этой жизни делаю.

Внезапно мне очень захотелось закурить. Впервые с начала года.

-- Лиза, -- Лина громко выдохнула, от чего ее пухлые губы сделались огромными, -- я задаю себе этот вопрос уже много лет. Если бы я знала на него ответ, я бы тебе его сказала. Только ты не забывай, шо мне скоро тридцать, а тебе только двадцать два.

И в самом деле, я частенько забывала о том, что ей не двадцать. Лина вовсе не выглядела на свой возраст. Как, в общем-то, и я.

-- Тебе двадцать семь, -- напомнила я. -- Не тридцать.

-- И шо?

-- А то, что ты тоже можешь изменить свою жизнь.

Лина окинула меня взглядом усталого родителя, который смотрит на надоедливого ребенка, но так ничего и не сказала. Слишком многое ей пришлось пережить и еще больше было того, что Лине пришлось забыть.

Подруге было около двадцати, когда ее жених погиб в автокатастрофе. После его смерти Лина так и не смогла заставить себя завести отношения с мужчиной. Да, в общем-то, ей это и не требовалось: Лина обнаружила, что выступает за обе команды. Теперь она была по девушкам. Тем не менее, вслед за катастрофой ее настигли крайне болезненные отношения. Эта связь продлилась четыре года и буквально выпила из Лины все соки. Добавьте к этому смерть младшего брата, бабушку, которая, не одно десятилетие лежала парализованная, тяжелый сельский труд и отца-алкоголика, обрекшего семью на голодное существование, а затем и вовсе завел любовницу и заделал той ребенка.

И, все же, Лина продолжала жить. Она сидела напротив меня, потягивая ледяной коктейль, и мечтательно улыбалась, рассказывая о предстоящем пикнике. Я слушала о том, что в деревне скоро зарежут молочного поросенка, о том, как правильно мариновать мясо, чтобы шашлык получился мягким\ldots{} Видела, с какой легкостью она радуется мелочам, и не могла не уважать свою подругу.

\emph{Я считаю себя сильным человеком}, однажды сказал мне Адам. \emph{Потому я подсознательно тянусь к таким же сильным людям. }

Услышав это, я вдруг осознала, что Адам, сам того не ведая, в паре слов описал всю суть моих взаимоотношений с другими людьми. С детских лет я была мизантропом. Мне претило общество с его глупыми устоями и двуличными стереотипами. По этой простой причине в отношении друзей я была крайне избирательна, но если уж встречала своего человека, то до конца оставалась верной нашей дружбе.

-- Ты сильная, -- настойчиво повторяла Лина каждый раз, когда слезы угрожали выступить на моих щеках.

Когда я ходила по острию ножа депрессии подруга продолжала говорить о моей силе духа и о том, что я самый крепкий и целеустремленный человек из тех, что она знала. Словом, я не могла ее подвести. Вместо этого рассказала ей о своем психическом расстройстве. Лина отнеслась к нему с пониманием, что было так несвойственно жителям моего родного города.

Мне хотелось плакать, чертовски хотелось, но я не могла позволить себе расклеиться. Стоило слезам подступить к глазам, а предательскому кому образоваться в горле, и мне вдруг становилось очень стыдно за свою слабость. Понемногу желание это начинало улетучиваться, а затем и вовсе сходило на нет.

Уже которую неделю подряд мы выползали на солнышко, закидывали ноги на покрывшийся ржавчиной пивной стол и говорили обо всем на свете.

-- Я словно от сердца отрываю каждый час, который зачем-то просиживаю за барной. Посетителей нет, банкетами и не пахнет\ldots{} Черт, так ведь даже на дом никто ничего не заказывает! И на кой хрен тогда нам тут торчать? Просто я слишком ценю свое время, я тебе это уже говорила. Наверное, в этом и есть моя проблема. На протяжении последних лет я измеряю свое время не событиями, эмоциями и планами, а лишь объемом писательской работы, которую могла бы проделать, если бы не страдала хуйней вроде ежедневных походов на не писательскую работу. Вот даже сейчас я торчу здесь и понимаю, что за это время могла бы написать не одну главу.

Лина посмотрела на меня с сочувствием. Вроде как хотела дать совет, но в последний момент передумала.

-- Выпей-ка лучше своего «чаю», -- сказала она, протягивая мне стакан.

***

Прошла еще одна рабочая неделька. На город обрушились непрекращающиеся ливни, и клиенты вновь вернулись в стены «Штиля». Как и следовало ожидать, даже новый наплыв работы не мог заставить меня забыть об авторских амбициях. Процесс осознания неправильности происходящего был запущен. Перо истосковалось по мне, и я то и дело слышала его зов.

-- Ну, так и зачем же ты тут сидишь? -- спросила Лина.

Вопреки тому, что я так часто задавалась подобными вопросами, я, все же, продолжала отдавать работе подавляющую часть собственного времени. Ответ был прост. Однажды я обнаружила, что многочасовой и выматывающий труд положительно влияет на мое душевное состояние. Это работало ничуть не хуже буддизма, который я стремилась постичь с начала прошлого года.

Наполненные суетой, рутиной и усталостью дни летели один за другим. Из-за постоянной беготни и сомнительного отдыха мои ноги непрестанно болели. Происходило это так часто, что вскоре я и вовсе прекратила обращать внимание на боль, сковывающую ступни и плавно поднимающуюся вверх по икрам. Вместе с болью ушло еще кое-что -- я с искренним удивлением обнаружила, что вдруг избавилась от многолетней депрессии. Сложно сказать, когда именно это произошло. Я была слишком увлечена работой да попытками задержаться на этом свете, но факт оставался фактом. Физически измотанная, я внезапно поняла, что моральная усталость покинула меня. Не было бессонных ночей, среди которых я бесцельно бродила по дому и его окрестностям. Не было слез, которые прежде то и дело выступали на моих глазах, стоило мне остаться наедине с удушающими воспоминаниями. Не было и мыслей о самоубийстве.

-- Уже незачем, -- чуть слышно ответила я.

Перепрыгнула через барную стойку и принялась радостно тискать подругу.

-- Чего-чего?

-- Мне больше не нужно здесь быть, Лина! Миссия выполнена.

-- Я нихера не понимаю, но я за тебя очень рада, -- ответила девушка.

-- Будешь кофе?

Но Лина не успела ответить: зазвонил телефон, и кассир поспешила снять трубку.

-- Это по твою душу, -- сказала Лина, протягивая мне аппарат.

В подземелье хотели пятнадцать детских мохито и четыре взрослых.

\emph{Блядские мохито. Ну кто вообще справляет детские дни рождения в клубе?}

-- И побыстрее! Мелюзга уже вовсю вопит! -- голос хриплый, чуть ли не потусторонний, и прямо-таки гавкает, а не говорит.

Когда я положила трубку, Вера уже стояла на раздаче, обещая вскоре обоссаться.

-- Я бы тебе помогла, но начальство велело даже в туалет не выходить. Я в зале одна осталась.

-- Ничего, я схожу вниз\ldots{} А ты пока принеси стаканы. Слушай, а кто звонил?

-- Ну, Рыжая.

-- Ты уверена? -- я подняла бровь.

-- Еще б мне не быть уверенной. Здесь только мы и есть. По одной на этаж. А чё такое?

-- Да голос, блин, какой-то жуткий. Как с того света, честное слово. Это точно Рыжая? Как-то на нее не похоже.

Но внизу меня действительно ждала Тоня, а в телефоне входящее сообщение от незнакомца.

«Как жизнь молодая?» -- спрашивал будущий муж.

\hypertarget{chapter-33}{%
\chapter{~}\label{chapter-33}}

Вскоре незнакомец превратился в жениха, и вот теперь «Штиль» -- с его крошечными подсобками, барными стойками, пыльными полками, абсурдным задним двориком, морем алкоголя и уставшими сотрудниками -- ждал меня где-то там далеко, за несколько сотен километров от солнечного Севастополя.

Вдохновленная вновь обретенной любовью не только к Вениамину, но и к жизни в целом, я оставалась спокойной. Эту полезную черту я переняла у Вени. Теперь, казалось, ничто не могло омрачить мое существование или хотя бы испортить настроение. Всей душой, всем сердцем, всем своим многострадальным естеством я чувствовала, что все будет хорошо.

Последний день отпуска подходил к концу, плавно перевоплощаясь в теплый крымский вечер. Несмотря на близящийся отъезд, на душе у меня было так же как и за окном -- легко и тихо, чего, увы, нельзя было сказать о Вениамине. На протяжении всего дня он мотался по городу на своей видавшей виды «Таврии». Севастопольский трафик никогда не отличался медлительностью, но сегодня Веня вжимал педаль газа с такой силой, словно с самого утра опаздывал на тот свет. Впервые за долгое время он был молчалив и крайне задумчив. Глядя на эти прищуренные зеленые глаза и широкие светлые брови, намертво сошедшиеся на переносице, можно было бы подумать, что прямо сейчас судьба всего мира решается в голове моего жениха.

Все это время я сидела около Вениамина. Читала книгу, любовалась пейзажами и еще больше своим спутником -- первой и единственной моей сбывшейся мечтой. Стоит ли говорить, я прекрасно понимала, о чем сейчас думает Веня? Даже если бы не было работы, которая приносила ему немалый доход, он не смог бы поехать со мной. В сущности, после того как Крым официально стал частью Российской Федерации, Веня вообще не мог покинуть его пределы.

Прошлой зимой Вениамину исполнилось двадцать пять. Его украинский паспорт требовал обновления, без которого был недействительным. Само собой, обновить его теперь было негде, кроме как в Украине, куда Вениамину -- молодому человеку призывного возраста, добровольно принявшему русское гражданство, -- едва ли следовало въезжать. Мы оба понимали, что в лучшем случае его прямиком на границе заберут на фронт. И мы оба этого не хотели. Вместе с тем, русский паспорт Вени, как и все недавно выданные крымчанам паспорта, также был недействительным во всех городах и странах, не считая самой России.

Короче говоря, Веня не мог поехать со мной, даже если бы от этого зависела его жизнь. Однако, моего возлюбленного пугала другая сторона медали. Он вдруг осознал, что не сумеет поехать и за мной в случае ссоры или конфликта. Или моего желания разорвать наши отношения. Как ни крути, но даже при всей глубине любви ко мне, Веня оставался алкоголиком со страхом одиночества. Находясь в трезвом состоянии, он прекрасно понимал, какие минусы влечет за собой подобное пристрастие. И, все-таки, даже не думал от него отказываться.

Наконец, за окнами уже совсем стемнело. Жара спала, и я смогла распустить огромных размеров пучок, в котором весь день покоились мои волосы. Веня припарковал «Таврию» вблизи дома, и вскоре, позабыв об ужине, мы плюхнулись в постель.

По никому не известным причинам, Вениамин терпеть не мог обсуждать чьих бы то ни были бывших. Той ночью он впервые спросил у меня об Адаме. Затем об Эрике и лишь узнав, как долго я оставалась им верной, более-менее успокоился.

Мы не покидали объятья друг друга до самой полуночи. Теперь до автобуса оставалось меньше девяти часов, и мы вроде как понимали, что неплохо было бы поспать. Тем не менее, ни мне, ни Вениамину не удалось сомкнуть глаз. Под покровом ночи наша комнатка выглядела серой и холодной.

-- Ты вернешься ко мне? -- спросил Вениамин.

Спросил так грустно, словно я вовсе и не собиралась возвращаться. Мне вспомнилось наше прощание с Адамом.

***

За ночь до поезда мы лежали в постели, крепко держась за руки. В комнате было тихо, как и за ее пределами. Пространство вокруг освещал лишь крохотный светильник. Он отбрасывал на стены причудливые тени и те плавно танцевали под сопровождение завывающего за окном ветра.

-- Я хочу, чтобы ты знала, -- ласково произнес Адам, и я в который раз умилилась то ли тембру его голоса, то ли произношению, то ли всему и вместе, -- я никогда этого не забуду.

-- Я тоже, сладкий, -- пообещала я. -- Как бы ни сложилась наша жизнь в дальнейшем, я этого не забуду.

Мне хотелось добавить: «Ведь я люблю тебя!», но я вовремя сумела сдержать этот лирический порыв. Думаю, он и так об этом знал.

Адам лежал напротив меня, прижимаясь щекой к подушке. Его большие карие глаза смотрели на меня тем же печальным взглядом, какой позже будет у Вениамина в ночь нашего прощания.

Возлюбленный молчал так долго, что мне стало не по себе. При этом в сердце поселилась такая тоска, что я не могла и слова вымолвить: боялась, как бы мой голос не дрогнул. Не без усилий, но мне удалось прогнать слезы, что, то и дело, подступали к глазам. Я как раз пыталась подобрать нужные слова, когда Адам вдруг отвернулся. Его могучие плечи беззвучно дернулись, и я с ужасом поняла, что он плачет.

Наступившим утром мы отправились на вокзал. Это был конец октября, самая что ни на есть золотая осень. Последние теплые деньки подходили к концу. Казалось, природа решила вложить в них всю свои красоту -- так очаровательны были скверы, улочки и аллеи Москвы тем уютным утром. Что ни говори, природа явно пребывала в гармонии с погодой. Наблюдая эта красоту, я жадно впитывала каждую деталь. Боялась что-нибудь упустить. Сама не знаю, почему. Наверное, я просто не хотела думать о предстоящей разлуке, которая беспощадно приближалась с каждой минутой.

Так длилось до тех пор, пока нога моя не коснулась Киевского вокзала. Поезд уже ждал на перроне. Отправление через двадцать четыре минуты. Напоследок оглянувшись по сторонам, я глубоко вздохнула. Пожалуй, мне было бы легче, увидь я вокруг себя грязные, залитые дождем улицы, сломанные ветви и прочие признаки надвигающегося шторма.

Это утро было слишком красивым для прощаний.

Адам подхватил мой чемодан. Я взяла его под руку и вскоре мы скрылись в вокзальном павильоне. Чем меньше времени у нас оставалось, тем сложнее мне было сдерживать нахлынувший поток эмоций. С каждым новым шагом я чувствовала, как сердце подскакивает в моей груди. Однако, внешне я оставалось спокойной и даже смогла искренне улыбнуться, когда Адам сказал, что при нашей следующей встрече разница в возрасте уже будет менее заметной. Через месяц мне исполнялось двадцать.

Я знала, что стоит мне позволить себе хотя бы одну слезинку, и я тут же расплачусь. А мне ужасно не хотелось, чтоб Адам запомнил меня такой, слабой и с заплаканными глазами.

-- А ты боялась, что история нашего знакомства будет не романтичной, -- заметил Адам. -- Ты только посмотри на это, Елизаветушка! Поезд, перрон, последние дни золотой осени\ldots{}

Он произнес это с привычным весельем в голосе. Если бы не печаль, скрытая в его глазах, я бы могла поверить в то, что Адам и в самом деле не ощущает одолевающей меня грусти. Мои вещи уже давно были уложены на полку поезда. До отправления оставались считанные минуты, и мы оказались последней парочкой, душевно прощающейся на перроне.

-- Как в старых черно-белых фильмах, -- заметила я и улыбнулась.

Стоявший на ступенях поезда проводник настойчиво повторил, что отправление через одну минуту. Улыбка у меня получилось грустной. Я поняла это по выражению лица своего любимого, и тут же поспешила добавить:

-- А теперь поцелуй меня. Как персонаж чертового черно-белого фильма.

Было это три года назад. С того дня я больше никогда не видела Адама, но подвернись мне шанс что-нибудь изменить, я бы едва ли смогла придумать для нас более подходящее прощание.

\hypertarget{chapter-34}{%
\chapter{~}\label{chapter-34}}

-- Ты вернешься ко мне? -- спрашивал Вениамин.

Он лежал напротив, подложив локоть под голову. Несмотря на поздний час, на улице по-прежнему было жарко. Дверь на балкон оставалась распахнутой, и ночной ветерок то и дело играл с прозрачной занавесью. Сквозь нее в комнату проникал лунный свет, освещая постель.

Я поспешно перевела взгляд на окно, стараясь избавиться от навязчивых воспоминаний давно ушедших лет. Да и какой смысл был в том, чтобы их хранить? В конце концов, теперь они не навевали ничего кроме тоски. Зачастую проблема воспоминаний как раз и состоит в том, что это счастливые моменты, о которых говорят исключительно в прошедшем времени. Чьи это слова? Я напрягла свое сонное сознание, пытаясь вспомнить какого-то забытого автора, когда вдруг поняла, чьи эти слова -- мои. Я написала их несколько зим назад и только сейчас, кажется, в действительности осознала смысл написанного.

Повернувшись к жениху, я увидела, что тот все так же смотрит на меня в ожидании ответа. Взяла его за руку, и изумилась тому, какой горячей она была.

-- Вернусь, -- пообещала я, стараясь вложить в свой голос максимальную порцию спокойствия.

На самом же деле, к концу дня молчаливая меланхолия Вениамина распространилась и на меня. Подкрепленная усталостью, горьким опытом и алкоголем, она достигла немалых размеров. Внутри меня все горело, и только вновь обретенная любовь и обещание сохранять внутреннюю гармонию в любой непонятной ситуации, что я однажды дала себе, удерживали меня от слез. Стоило мне вспомнить об исходе предыдущих романов, о прощаниях, что, казалось бы, сулили новые встречи, о любви, -- сильнейшем и упрямейшем из всех чувств, на которое было способно мое агоническое сердце -- восторге и сладостных переживаниях, которые приводили к бесконечному списку надежд. Надежд на моральный покой, взаимопонимание и обретение долгих счастливых лет совместной жизни. Ничему из этого не было суждено сбыться, и я знала, что стоит допустить хоть одну печальную мысль, стоит хоть разочек себя пожалеть, и мои глаза тут же наполнятся слезами. А я не могла позволить себе расплакаться. Теперь вовсе не потому, что переживала о том, какой запомнит меня Веня в случае, если мы больше не увидимся.

Мне нельзя было плакать просто потому, что я понимала: стоит только начать, и я уже не смогу остановиться. На этот раз по-настоящему.

Боялась ли я вновь остаться с разбитым сердцем? Боялась ли я опять не распознать обман и двуличие, которые всегда были коньком моих любимых? Боялась ли я снова довериться не тому человеку? Чертовски, блядь, боялась! Но я уже впустила Вениамина в свое сердце, так что у меня попросту не было выбора. И пускай отношения у меня всю жизнь не клеились, в любви я кое-что понимала.

Хотя, наверное, и в любви я как раз таки и понимала просто потому, что слишком часто бывала из-за нее несчастной. Наблюдая за окружающими меня людьми, я как-то вдруг поняла, что тот, кто всегда был счастлив, не прикладывая усилий, едва ли может противостоять серьезным проблемам. Несчастья -- это, конечно, плохо, но они нужны взрослому человеку так же, как ребенку нужна школа. Никто не хочет подвергаться хулиганству и оскорблениям, но только с их помощью можно научиться давать отпор.

\emph{И понять как не нужно делать}, подумала я, вспоминая Марту.

***

Мне шел двадцать первый год, когда Марта вернулась со Львова и самоотверженно заявила, что влюбилась. Подруга была старше меня всего на год, но никогда прежде не состояла ни в серьезных, ни в несерьезных отношениях. Дело здесь было вовсе не в том, что с Мартой было что-то не так. Отнюдь. Она была высокой, стройной и хорошо образованной. Просо всю свою жизнь эта девушка посвящала учебе, карьере, поездкам, покупкам и саморазвитию. На любовь у нее просто не оставалось времени. За полтора года до того, как она встретила Семена, пытаясь поймать такси, мы сидели в «Перекрестке» -- тусклом, плохо отапливаемом баре, который так прекрасно вписывался в общий вид Николаева.

К тому моменту прошло меньше трех месяцев после нашего с Адамом прощания. Февраль, четырнадцатое число -- мой наиненавистнейший из всех ненавистных дней в году. Само собой, город заполняли влюбленные парочки. Они были на скамейках, в магазинах, на дорогах и, конечно же, в заведениях. Взявшись за руки, сотни тысяч людей гуляли под непрекращающимся вторую неделю снегом.

И без того грустная, теперь я принялась с ума сходить от одиночества. Хотела запереться дома, но и здесь, в четырех стенах, мне все напоминало о бывшем. Тогда, наспех одевшись, я вылетела на улицу. На мне было лишь легкое пальто, наброшенное поверх домашней одежды -- привычный прикид для затянувшейся депрессии. Никакой снегопад не мог помешать мне появиться в таком виде на людях. Не зная, куда еще податься, я спряталась в этом унылом, забытом богом и николаевцами баре.

Спустя несколько стопок текилы ко мне присоединилась Марта. Она заметила, что нынче вид у меня был еще более отстраненный, чем обычно, спросила, в чем дело, и тогда я впервые рассказала ей о своих чувствах к Адаму. Выложила все подчистую.

-- Я думала, тебе на него уже давно наплевать! -- изумилась подруга.

Все так думали, хотя не прошло и трех месяцев. Я носила макияж и каблуки, от которых потом болели суставы, старательно укладывала волосы, что к тому моменту покраснели, регулярно бывала в оживленных местах и веселила всякого встречного. Что же касается тех дней, когда я просто не могла скрывать свою боль\ldots{} Что ж, в такие моменты я забивалась в самый дальний угол. Пары пальцев хватило бы на то, чтобы сосчитать людей, знавших о моих душевных терзаниях.

Словом, нельзя было винить общество за то, что оно не было в курсе моих переживаний.

-- Я завидую тебе! -- призналась Марта. -- Завидую твоим романам, твоему разбитому сердцу, твоим чувствам и печалям. Завидую тому, что ты пишешь книги и тому, что у тебя не бывает денег. Тебе, как ты говоришь, не трудно и в бомжах ходить, если это дает время на книги. А я берегу каждую копейку. Ты знаешь, каково это -- любить. А я вот не знаю ничего. Я возвращаюсь домой и захожу на сайт «Розетки», чтобы выбрать какую-то технику. Хоть что-нибудь купить и порадовать себя этим. Это всегда работало, но вот вчера я получила зарплату, и внезапно поняла, так ведь у меня уже все есть! У меня есть телефоны, ноутбуки, планшеты, колонки и еще куча всякой всячины. Я была во всех странах, которые хотела посетить. И я поеду куда-нибудь еще раз, если мне захочется. А мне вот не хочется. Я знаю так много всего, но я понятия не имею, что такое любовь. И от этого у меня чувство, что я не знаю ничего. Так я ведь даже никогда и не влюблялась толком. Никто никогда мне не нравился и я без понятия, что такое страсть.

Она глубоко вздохнула и повторила:

-- Как же я тебе завидую, Лиза! Ох, как бы мне хотелось хоть раз узнать, что такое разбитое сердце.

Я исподлобья взглянула на Марту, смутно представляя, каким должен быть мой ответ. Я была грустным, психически неуравновешенным писателем, у которого не было ни денег, ни литературного признания. В те дни все, что у меня было, сводилось к вышеупомянутому разбитому сердцу и стопке никому не нужных рукописей. Каждое мое утро начиналось с поисков мелочи, в надежде, что ее хватит на самую дешевую пачку сигарет. Но Марта не лицемерила. Я знала, что слова ее были искренними. Это-то меня и пугало.

\hypertarget{chapter-35}{%
\chapter{~}\label{chapter-35}}

Когда, спустя восемнадцать месяцев, Марта сказала мне, что запала чуть ли не на первого встречного, я вовсе не удивилась. Должно быть, она ждала от меня мудрого совета, который пристало иметь каждому, кто хоть раз состоял в бурной романтической связи. Однако, видя ее неистовое воодушевление, слыша дрожащий от волнения голос, сопровождаемый взглядом совсем по-детски наивных глаз, я только и смогла выдавить из себя:

-- Ох, деточка\ldots{}

Мне хотелось сказать ей о том, что ни одна любовь не длится вечно. Рассказать о горечи, что следует за пылкими чувствами. Завести долгий монолог о том, что любовь -- это вообще крайне страшная штука. Она и впрямь походит на наркотики и прочие приносящие вред пристрастия. Виной всему служит тот простой факт, что, как бы плохо тебе не было по окончании отношений, придет день, и ты вновь кого-нибудь полюбишь. И чувства эти будут даже сильнее тех предыдущих, которые, казалось, принесли тебе самую страшную боль.

А затем все закончится, и ты вдруг поймешь, что первая боль и вовсе не была болью в сравнении с тем, что ты имеешь теперь. С каждым новым разом любовь растет в геометрической прогрессии, а вместе с ней растет и уровень тяжести всех вытекающих из любви последствий. Увы, даже осознав это, ты никак не можешь отделаться от этого примитивного желания -- любить и быть любимым. Казалось бы, все просто, но нет! Любовь вызывает привыкание. Она окрыляет, делая самый убогий дом уютным. Так, искренне любящий человек чувствует себя чуть ли не бессмертным. Когда любишь, то и море по колено. Потому-то, даже испытав на себе все любовные горести, ты больше не можешь мириться с одиночеством и подсознательно продолжаешь поиски новой любви. Продолжаешь просто потому, что больше не можешь жить без этого чувства.

Тем днем мне многое хотелось сказать Марте. Думаю, она едва ли предполагала, что за этим невзначай и не без юмора брошенным «Ох, деточка\ldots» могли скрываться такие долгие внутренние диалоги. Конечно, мне хотелось помочь ей. Ну, или хотя бы объяснить, что в лучшем случае принесенная любовью радость будет равна той боли, что обрушится на тебя по завершении отношений.

Но я не смогла. Слишком счастливой была моя подруга. Она вовсе не заслуживала таких грустных разговоров и так напоминала меня саму каких-то пять лет назад. Глядя на это ни с чем не сравнимое счастье первой влюбленности, что прямо-таки струилось из каждой клеточки девушки, я почувствовала, что где-то на задворках моего сердца теплиться надежда. В конце концов, не все ведь обязаны жить с разбитым сердцем, правда? К тому же\ldots{} Порой разрушать чужие иллюзии куда сложнее, чем собственные.

Что же касается меня, слишком уж долго я сторонилась отношений. Слишком часто отказывала мужчинам. Я провела не один год в отношениях с нелюбимыми людьми, отчаянно пытаясь забыть Эрика, свою первую школьную любовь. И хотя однажды мне удалось избавиться от этих настойчивых чувств, я не могла забыть той радости, того сумасшедшего прилива сил, который я испытывала каждый раз, когда Эрик был рядом. Даже когда мы не были парой в традиционном понимании этого слова.

Теперь, конечно, уже сложно сказать, что именно пошло не так, но с моих пятнадцати лет он так и не смог полюбить меня. А я так и не смогла полюбить тех парней, что опрометчиво старались завладеть моим сердцем. Все это делало меня грустной и злой. Дошло до того, что я и вовсе возненавидела отношения. Мне была противна сама мысль о том, чтобы вновь пустить кого-то в свое личное пространство, или того хуже -- в свою душу.

В итоге, пришлось признать: игра эта сильно затянулась. К тому моменту мне уже исполнилось девятнадцать, и я разорвала свои последние бесполезные отношения. С Эриком мы не виделись без малого полтора года, когда я вдруг осознала, что уже и думать о нем забыла. Должно быть, в юности этот человек и впрямь имел для меня немалое значение, но потом\ldots{} Я поняла, что любила его словно по привычке. На самом же деле, Эрик уже давно не был мне нужен. Все, что мне было нужно -- это воодушевляющие чувства, которые я испытывала рядом с ним. Мысль о том, что источником этих чувств являюсь я, а не Эрик, неоднократно наведывалась в мою голову, а я снова и снова гнала ее как прокаженного бездомного, что под покровом ночи пытается проникнуть в старый сарай.

К счастью, Эрик оказался не единственным мужчиной, способным дать мне все это. Незадолго до своих двадцати лет я встретила Адама и, сама того не желая, запала на него. На этот раз чувства были более взрослыми. Естественно, и любовь моя была в разы сильнее той, предыдущей. Как и следовало ожидать, боль расставания оказалась невыносимой.

Мне исполнилось двадцать. Я вновь возненавидела любовь. Я проклинала свое глупое сердце за чувства, которых не должно было в нем быть, и злилась на вселенную за то, что та наградила меня такой упрямой, никому не нужной верностью. Я с легким презрением смотрела на девушек, рассказывающих мне о своих первых чувствах так, словно они и правда верили в то, что отношения эти будут единственными в их жизни. Злилась на любовь потому, что знала, какой двуличной и коварной бывает эта сука, но еще больше потому, что чувство это было моей единственной слабостью. А я ненавидела свои слабости. И ненавидела себя за то, что не могла избавиться от них.

И, все же, я и дня не могла прожить без любви, как не могла проснуться без табака и чашки крепкого черного чая. Это была самая настоящая зависимость, и, как бы я ни старалась от нее избавиться, какой бы циничной и независимой я не делалась, в конце концов, пришлось признать, что любовь нужна мне хотя бы для того, чтобы жить. Быть может, это выглядело чересчур странно, но ведь я писатель, а творческие люди, знаете ли, не самые адекватные.

Я так долго была несчастной и одинокой, что и вовсе не верила в то, что встречу родственную душу. Однажды я уже собрала все свои рукописи в увесистую стопку, уложила их на кухонном столе и готова была застрелиться подле этой унылой картины. Подумать только, мне был всего двадцать один год, а уже в который раз чувствовала себя слишком старой и уставшей для того, чтобы жить. Но мне удалось выйти из депрессии и никого это не удивило больше, как меня саму. Общество почему-то всегда считало меня сильной, тогда как я с каждым годом все меньше и меньше верила в возможность восстановления своего морального состояния. Ситуация значительно упростилась, когда я услышала свой диагноз.

Так или иначе, после семи долгих лет, полных печали, одиночества и неразделенной любви и чего-то там еще, я встретила Вениамина. И подумать не могла, что все еще могу испытывать такие сильные чувства. Но я полюбила его. Как никогда никого не любила. А этот белобрысый дурашка теперь лежал напротив и в самом деле сомневался в том, что я вернусь к нему.

\hypertarget{chapter-36}{%
\chapter{~}\label{chapter-36}}

-- Ты ведь вернешься? -- вновь спрашивал Веня последней ночью моего отпуска.

Спрашивал так, словно у меня был выбор. Словно я вовсе не отдала ему своё сердце несколько недель назад.

Конечно, я вернусь. Разве я могу поступить иначе? И я повторила свой ответ.

Услышав мое обещание во второй раз, Вениамин, казалось, наконец, ему поверил. Я уже начала проваливаться в сон, когда заметила, что мой мужчина плачет. Он не отворачивался и не пытался прикрыть глаза рукой. Веня просто безмолвно плакал, сжимая мою руку с такой силой, что вскоре я и вовсе перестала ее чувствовать.

Лежа рядом, я совершенно не знала что сказать кроме слов любви, которая и без того была очевидной. Смотрела на то, как слезы текут по бородатым щекам этого крепкого мужчины. Было что-то трогательное в его приступе сентиментальности. Слишком трогательное для того, чтобы позволить себе не воспринимать эти чувства всерьез.

Следующим утром я никуда не уехала.

***

Прошло еще немало времени прежде, чем нам удалось уснуть. Казалось, даже во сне я продолжала чувствовать пальцы Вениамина, крепко сжимающие мою руку.

-- Прости, что испортил нашу последнюю ночь, -- краснея, произнес Веня следующим утром.

Я лишь отмахнулась. Безусловно, с отъездом вышла накладка, (теперь у меня не останется времени на сон и буквально придется прыгать за барную стойку прямиком из автобуса) но разве это можно назвать проблемой, учитывая спокойствие, которое, наконец, поселилось в наших сердцах? Прошлой ночью все страхи и опасения вырвались наружу. Теперь мы оба понимали, что прощаться будет гораздо проще.

Автобус по-прежнему уходил в девять утра. На этот раз мы даже купили билет, потому как прекрасно знали, что мне нельзя его упустить. Впереди нас ждал еще один прощальный вечер. Долго не думая, мы решили провести его в «Среде Обитания» -- ламповом баре, что по счастливой случайности находился в двух шагах от дома. Усевшись за столик, мы болтали обо все на свете: о литературе, автомобилях и о том, как правильно готовить шоты, чтобы алкоголь лег красочными слоями, а не превратился в одно бесцветное месиво. Короче говоря, мы говорили о чем угодно, только не о грядущем прощании.

Под конец вечера Вениамин куда-то исчез. Он отсутствовал минут двадцать, а затем появился в дверном проеме, держа в руках охапку крымских роз. Последние были явно похищены из чьего-то огорода. Об этом свидетельствовали не только грубо обрезанные ветви, но и лицо моего жениха, побагровевшее от бега. Один цветок Веня заправил за ленту своей шляпы, а остальные деловито преподнес мне. К тому моменту он был уже чертовски пьян, но когда это мешало нам весело проводить время?

***

Я любила абсент. Меньше чем вино, но больше, чем водку. Хотя, я, наверное, что угодно любила больше, чем водку за исключением бабочек, жары и точных наук. Вениамин же совершенно не понимал абсентной эстетики, но очень любил крохотные коктейли, сделанные на основе этого напитка. Ими-то он и надрался моим последним вечером в Севастополе. В то время как я мирно попивала винишко, жених прикончил одиннадцать шотов, а затем еще вернулся в бар затем, чтобы выпить двенадцатый.

Мы возвращались домой пустынными улочками. В домах по обе стороны от дороги свет был погашен. Жившие в них люди уже видели десятый сон. Единственным источником света оказались разбросанные по улицам фонари. Их мягкое свечение, отсутствие других пешеходов и повисшая в воздухе тишина практически летней ночи предавали пространству вокруг схожесть с киношным павильоном.

Тем временем Вениамин продолжал заниматься самым романтичным видом городского вандализма. Он без стыда срывал каждый цветок, которому не посчастливилось встретиться на нашем пути.

Наконец, мы вернулись домой. По-настоящему последний день отпуска официально подошел к концу. Небо все еще оставалось звездным, но где-то вдалеке, за морем, уже проступали алые полосы. Докуривая сигарету, я вдруг поняла, что близится наш первый совместный рассвет. Вслед за этой мыслью Вениамин, какое-то время сидевший в полной тишине, принялся старательно изучать содержимое своих карманов. Он вытащил пять складных ножей, и теперь внимательно рассматривал их лезвия.

-- Этот! -- сказал Веня, в конце концов, определившись.

Он протянул мне стальной нож с самым что ни на есть спокойным видом. По присущей всем уставшим людям привычке, я сначала взяла протянутый предмет, а уж потом спросила, зачем тот, собственно, мне нужен.

Под влиянием обильного количества алкоголя оттенок глаз моего жениха переменился. Зачастую, это был уже привычный мне лазурный или холодный зеленый. Реже -- ярко голубой. Сейчас же Вениамин смотрел на меня широко распахнутыми глазами, которые едва ли можно было назвать изумрудными. Они вдруг приобрели темный болотный оттенок. Все это, на пару с усталостью и выпивкой, делали венькин взгляд крайне мутным.

Слабо сжимая в руке нож и начиная догадываться о том, зачем мне его вручили, я не отрывала взгляда от лица Вениамина. Было в нем что-то чужое, скрытые черты или привычки, о которых я не знала. Эти перемены нельзя было увидеть невооруженным глазом, настолько они были незначительными. К примеру, речь стала быстрее, тембр голоса был чуточку выше обычного, а сменившие оттенок глаза то и дело оказывались неестественно распахнутыми. Безусловно, взятые отдельно друг от друга, такие вещи наверняка остались бы без внимания.

Но я слишком хорошо знала Вениамина, и отчетливо понимала, что, в совокупности такие детали имеют смысл. Они настойчиво вызвали в моей душе неприятное чувство. Ощущение ноющей тревоги. Не имея видимых мотивов, это опасение периодически увеличивалось в размерах. В какой-то момент, мне даже стало трудно дышать. Это показалось мне таким глупым, ведь я даже не знала, чего боюсь. Тогда я постаралась разобраться, что конкретно меня волнует. Однако, все внутренние тревоги казались довольно абстрактными. Они сводились к единственной мысли, что то и дело раздавалась в разных уголках моего сознания.

\emph{Что-то пошло не так}, гласила мысль.

\emph{Что-то плохое происходит прямо сейчас, и это что-то от тебя умело скрывают.}

Как ни старалась я утихомирить свой внутренний голос, эти фразы настойчиво повторялись. И дело было вовсе не в ноже, о котором я уже и думать забыла. Дело было в самом Вениамине, его взгляде и\ldots{} в чем-то еще.

\emph{Бредовое расстройство}, отрешенно подумала я. \emph{У меня бредовое расстройство.}

ЧТО-ТО ПЛОХОЕ ПРОИСХОДИТ ПРЯМО СЕЙЧАС, И ЭТО ЧТО-ТО ОТ ТЕБЯ УМЕЛО СКРЫВАЮТ!

Предложение это звенело внутри меня с назойливостью августовских мух, доживающих последние дни лета. Длилось это до тех пор, пока мысль не стала предельно громкой. Такой громкой, что от нее разболелась голова. Тогда я в последний раз услышала прогудевшие в моей голове слова, а затем все смолкло. С приходом тишины исчезла и тревога. Совсем непонятный мне страх тоже куда-то испарился, так и не раскрыв своих мотивов.

Сразу после этого я почувствовала прилив угрызений совести.

\hypertarget{chapter-37}{%
\chapter{~}\label{chapter-37}}

Конечно, у маниакально-депрессивного психоза имеется довольно богатый список симптомов. Практически каждый из них я многократно ощущала на собственной шкуре, в том числе и навязчивые идеи, не имеющие особого основания, и удручающие, депрессивные мысли, которые мгновенно пускали корни, старательно отравляя мне жизнь. И первые и последние были поразительно правдоподобными, отчего у меня почти никогда не возникало сомнений на этот счет. Из-за этого я частенько ловила себя на том, что больше не хочу жить. Мысли о смерти и явное отсутствие будущих перспектив сводили меня с ума.

Поначалу я еще как-то могла с ними бороться, но спустя проведя большую часть жизни с биполяркой, о наличии которой я не знала, мое внутренне состояние явно ухудшилось. Череда повторяющихся неудач, обид и потерь привела к нервному срыву, а болезнь его только усугубила. Затем я страдала не только по достойным внимания причинам. Сущие пустяки и мелочи, на которые здоровая версия меня плюнула бы без лишних раздумий, ранили меня ничуть не меньше. Как и все упрямые люди, я продолжала бороться, но чем больше усилий я прикладывала, тем хуже становилась моя жизнь.

В конечном счете, я очень злилась, после чего впадала в долгое отчаянье, которое сменялось лютой печалью, чтобы однажды перерасти в самую настоящую истерику. Все, что шло после гнева, так или иначе вызывало во мне мысли о самоубийстве. Как-то я поняла, что единственный способ остановить этот поток заключался в нанесении себе телесных повреждений. Я уже об этом говорила. Резкая физическая боль была единственным фактором, способным отвлечь меня в такие минуты. И я это знала.

Поэтому на пике очередного приступа, я кое-как умудрялась отыскать в себе последние признаки просветления, и хватала в руки лезвия, ножи, молотки и любые другие вспомогательные предметы. Чаще всего, я делала это машинально и приходила в себя уже после того, как чувствовала первые порезы. Иногда под рукой не оказывалось нужных предметов. В таких случаях, приходилось импровизировать. Помнится, однажды я стояла у края крыши многоэтажного дома, готовая к тому, чтобы сделать последний шаг. Валил сильный снег, а на мне не было ни пальто, ни даже свитера. Режущих или хоть сколько-нибудь травмирующих предметов при мне также не наблюдалось. В итоге пришлось заставить себя хорошенько разогнаться и вписаться головой в бетонную стену трансформаторной будки. Было больно, но я прекрасно понимала: либо это, либо асфальт.

Общество никогда не было в состоянии понять мотивов моих поступков. Окружающие и представления не имели о природе этих проблем. Появление новых увечий, как и вид старых шрамов, вызывало на их лицах выражение неподдельного шока.

-- Ты же такая веселая, -- говорили они.

Словом, люди всегда открыто глазели на меня, желая лучше рассмотреть последствия селфхарма, но никогда не осмеливались спросить о них. Очевидно, задавать подобные вопросы сегодня столь же неприлично как интересоваться количеством половых партнеров. Так что, люди просто пялились, а в разговорах показательно старались не смотреть на место пореза, делая вид, что и вовсе его не заметили.

Никто из окружающих меня людей и понятия не имел о том, что мне приходится причинять себе вред под влиянием инстинкта самосохранения, как бы странно это не звучало. Ведь как бы бодро я не выглядела, большую часть времени мне приходилось бороться с настойчивым желанием наложить на себя руки. Они не могли понять эту простую истину просто потому, что были психически здоровы. А еще потому, что в принципе не были способны думать шире собственных представлений о жизни.

Всем хотелось знать, почему я пыталась свести счеты с жизнью. Кто-то говорил о любви, долгах и наркотиках, тогда как правда была куда более банальной и очевидной. По крайней мере, для меня. Я намеревалась убить себя просто потому, что видела в смерти единственный способ избавиться от одолевавшей меня боли. Она росла с каждым днем, разрывая сердце на куски, и вскоре я уже была готова отдать что угодно, только бы больше это не чувствовать. В том числе, и свою жизнь. Наступил день, когда я уже и сама толком не могла сказать, почему так страстно желаю собственной смерти. Само это желание обратилось каким-то важным пунктом в моем списке дел, и внутренний таймер то и дело присылал напоминания.

Тем не менее, я вновь и вновь была вынуждена вступать в битву с самой навязчивой из своих идей. Должно быть, проблема заключалась в том, что на самом деле я никогда не выходила из этого боя победителем. Я лишь брала тайм-аут и тянула время в надежде, что когда-нибудь смогу найти выход из положения.

В общем, к тому моменту я многое успела повидать и во многом травмироваться, но -- если не считать ночных вылазок, которые я пару раз совершала, переборщив с марихуаной -- я никогда прежде не имела дела с паранойей. Паранойя -- именно так я и подумала о тревожных мыслях, что одолели меня последней перед отъездом ночью. Подобно всем людям, обнаружившим, что у них проблемы с головой, и прикладывающим море усилий для их лечения, я тут же усомнилась в адекватности своего восприятия. Что же касается Вениамина, в остальном он был все тем же. Веня обращался ко мне привычными уху теплыми словами, произнесенными хорошо знакомым тоном. Он улыбался мне, целовал и признавался в любви снова и снова, оставляя на моих губах такой же привычный вкус недавно выпитого спиртного. И, как это всегда бывает с параноиками или влюбленными, вскоре я напрочь забыла о так внезапно нахлынувших на меня опасениях.

\hypertarget{chapter-38}{%
\chapter{~}\label{chapter-38}}

-- Я люблю тебя, как не любил ни одну женщину, -- произнес Вениамин.

Нож по-прежнему оставался в его руке. Веня нажал на кнопку, и в лунном свете мелькнул одинокий блеск лезвия. Не медля ни секунды, мой жених приставил нож к своей руке и сделал надрез чуть выше кисти. Какое-то мгновенье ничего не происходило. Кожа на его руке оставалась все такой же бледной, а затем хлынула кровь. Порез оказался куда глубже, чем мы рассчитывали.

-- Еще один шрам, -- сказала я, проводя пальцами по руке Вениамина. Они тут же окрасились в алый. -- Будет что детям рассказать.

-- На то и расчет, -- улыбнулся Веня.

Он перевернул нож лезвием к себе и протянул его мне.

К тому моменту небо над морем стало гораздо светлее. Тьма начинала рассеиваться, уступая дорогу предрассветным часам. Я взяла нож и какое-то время разглядывала его лезвие. Местами запачканное кровью моего сумасшедшего мужчины, оно было таким же серым, как небо, что как раз начинало алеть над нашими головами.

Я провела краем лезвия по коже правой руки, но это была всего лишь царапина. Тогда я со всей силы надавила на рукоятку, чувствуя, как под ней рвется кожа, и повторила процесс. Еще одна царапина, не больше. И тут я вдруг поняла, что в кои-то веки не хочу себя калечить. Я была абсолютно счастлива.

-- А я ведь тоже тебя люблю.

В моем голосе звучало ничем неприкрытое удивление, словно я впервые узнала о своих чувствах к будущему мужу.

-- Ты не обязана себя резать, -- заметил Вениамин, хотя прекрасно понимал, что без этого его затея не имеет смысла.

Все это время его рана оставалась ничем не прикрытой, и теперь кровь окрасила всю ладонь мужчины.

Я отрицательно покачала головой.

-- Знаю, что не обязана. Но я хочу.

И я протянула нож Вене.

-- У тебя получится лучше моего. Я уже потеряла навыки селфхарма.

Однако, у него ничего не получилось. Это была очередная царапина.

-- Не могу я причинять тебе боль, -- виновато произнес Вениамин. -- Эти твои царапины, они ведь тоже милые. Мы можем просто\ldots{}

Но я уже его не слышала. Просто поставила нож на балконные перила, а затем что было духу ударила по нему наружной стороной руки. На этот раз правой. Подняла ее, чтобы полюбоваться проделанной работой, но так и не успела оценить масштабы травмы. Кровь мигом залила все вокруг.

Прерванный на полуслове Вениамин широко улыбнулся. Он принялся целовать мою руку, затем меня. В общем, вскоре мы оба уже изрядно перепачкались в собственной крови.

-- Любовь моя, -- торжественно начал Веня. -- Я люблю тебя, как не любил ни одну женщину. И я хочу жениться на тебе как можно скорее, деточка. Хочу, чтобы ты стала Елизаветой Ларионовой, и тебе никогда бы не понадобилось уезжать от меня.

Он еще раз поцеловал мою руку, а затем приложил ее к своему порезу и бросился клясться в вечной любви.

-- Ты веришь мне? -- спросил Веня.

-- Верю, и клянусь тебе в том же, -- честно ответила я, а затем так же честно добавила: -- Но, если ты обидишь меня, я возьму этот самый нож и проткну им твое сердце. В этом я тебе тоже могу поклясться.

Это прозвучало гораздо серьезней, чем должно было, и я с опасением взглянула на Вениамина. Тот стоял с крайне беспечным видом, едва ли напуганный моими словами.

-- Проткни, -- ответил он, обнимая меня за плечи. -- Мне все равно не нужна эта жизнь без тебя.

Солнце вот-вот обещало показаться за горизонтом. Мы стояли на балконе, в обнимку, обращенные лицами к уходящей тьме. Всего лишь двое городских сумасшедших в ожидании рассвета.

***

Автобус мы вновь упустили, но особого выбора у меня не было. Пришлось мчаться в конец города, надеясь перехватить рейс там. Как ни странно, нам это удалось. Более того, Веня, (к тому моменту достаточно трезвый и спокойный для того, чтоб сесть за руль своей развалюхи) даже перестарался. Теперь мы остановились на выезде из Севастополя. Недалеко от Инкермана. Предварительные расчеты говорили, что до появления автобуса оставалось около десяти минут. Мы молча смотрели вдаль, готовые в любой момент сорваться с места и, подхватив чемодан, остановить автобус.

И при этом не готовые вовсе.

Пожалуй, тем утром никто из нас не знал, что следует сказать. Все и так было сказано. И не раз. На моей руке красовалось кольцо Вениамина, слишком широкое для носившего его пальца. Чуть выше виднелся новенький шрам. Длиной он был лишь в несколько сантиметров, но почему-то оказался крайне глубоким.

По обе стороны от нас виднелись одни только скалы да небольшие горы. Местами их разбавляла пробивающаяся через камни зелень. «Таврия» стояла на съезде спиралевидного шоссе, из-за чего приближающийся транспорт до последнего не было видно. Вскоре послышался шум двигателя. Спустя несколько секунд мимо нас на бешеной скорости пронесся очередной автомобиль. Еще мгновенье, и машина вновь скрылась за поворотом. Какое-то время до моих ушей доносился удаляющийся шум мотора, но и он быстро исчез. Тогда в воздухе вновь повисла тишина. Та самая неповторимая стеклянная тишина, какая бывает лишь в горах. Время шло, но ни один из нас не решался ее нарушить.

Вениамин подкурил бессчетную сигарету. Я сделала то же, но тут же избавилась от своей сигареты: уровень никотина в моей крови оказалась слишком высоким как для не так давно начавшегося утра. От табачного дыма начало мутить. Время не просто тянулось. Оно двигалось со скоростью трехпалого ленивца, чью шерсть покрывает мох, принимая его за бревно ввиду сидячего образа жизни. Однажды что-то подобное случилось с моим приятелем.

Мой мужчина загадочно и вместе с тем сосредоточенно глядел вдаль. Правда, даль ограничивалась всего парой-тройкой метров, за которыми был поворот вверх по дороге. Воспользовавшись первым из этих фактов, я позволила себе расслабиться. И тут же заплакала. Без звука, как герои немого кино, но Веня сразу же это заметил.

-- Держи, глупышка, -- он протянул мне платок, который в те дни зачем-то все время носил с собой.

Я взяла его, смахнула слезы и постаралась так же элегантно избавиться от мгновенно появившихся соплей.

-- Нет, так дело не пойдет. Давай как я учил.

Терять мне было нечего, так что я сделала глубокий вдох и высморкалась с запалом заядлого астматика. Где-то в середине этого процесса из-за угла выглянул тот самый автобус.

От неожиданности я выронила платок, и сопли, вперемешку со слезами, уже стекали по моему подбородку. Куда менее изящно, чем мне того хотелось. Веня среагировал практически вовремя. Он выскочил на дорогу и уже энергично махал руками. Автобус притормозил в нескольких метрах от нас. Не успев толком остановиться, водитель тут же принялся жать на клаксон. Одной рукой я подхватила чемодан, второй же безуспешно пыталась удержать продолжающие течь сопли, начиная при этом бежать в сторону автобуса. При виде этой картины мой будущий муж расхохотался смехом, что был неприлично детским, учитывая масштабы его бороды.

Мы остановились у самого входа в автобус.

-- Это самое нелепое прощанье в моей жизни, -- сказала я, стараясь прикрыть нижнюю часть лица.

Веня отбросил мою руку. Он крепко поцеловал меня и ответил:

-- Потому, что это не оно.

Водитель нетерпеливо жал на гудок. Я выпустила руку Вениамина из своей и вошла в автобус. К тому моменту мы провели в ожидании более двадцати минут. Математика никогда не была моей сильной стороной. Как и прощания.

\hypertarget{chapter-39}{%
\chapter{~}\label{chapter-39}}

Чем сильнее я отдалялась от Севастополя, тем более заметным становилось ухудшение погоды. Солнечный город -- с его бескрайним морем, горными долинами и залитыми солнцем пляжами -- остался за спиной.

Я сидела у окна автобуса, наблюдая за тем, как стремительно сереет небо. Слева от меня расположилась крашеная блондинка лет тридцати пяти. Она читала электронную книгу и с завидной настойчивостью пыталась поймать мобильную связь каждые пять минут.

Автобус заехал в Армянск и, выйдя из него, я почувствовала на своем лице первые капли дождя. До границы оставалось где-то десять минут езды. При мне был заметно утяжелившийся вином чемодан без малейшего намека на колесики, полноценный макияж и не менее полноценная укладка. Зонта у меня, конечно же, не имелось, так что отсутствующий во всех прогнозах погоды дождь едва ли удивил. Мне было всего двадцать два, несколько лет разделяли меня от того, чтобы отказаться от комплексов и косметики без особого случая, но этот закон подлости я к тому моменту уже успела усвоить: если ливню и суждено меня настичь, то это непременно случится на границе.

\emph{Ох, чувствую, скоро повеселимся!}

Пугающе длинные очереди на шесть контрольных пунктов, (по три с каждой стороны) отсутствие подвесного покрытия, едва ли асфальтированная дорога длиною в несколько километров и целая куча ям на любой вкус -- ну, разве не идеальное место для очередной комической сценки под дождем со мной в главной роли?

Раздумывая об этом, я решила поступить так, как обычно поступаю в любой хоть сколько-нибудь озадачивающей ситуации: отправилась пить кофе.

***

-- Ты знаешь, куда здесь идти? -- спросила блондинка. Та самая, что ехала на соседнем от меня сидении.

Сейчас ее электронная книга и мобильник скрылись в пределах сумки, и женщина, что было духу, пыталась перекричать раскаты грома, доносившиеся со всех сторон. На несчастной были туфли на высоченной шпильке и коротенькое белое платье. Теперь оно стало практически прозрачным под влиянием беспощадного ливня.

Сказать, что лило как из ведра -- в принципе, не сказать ничего. Я простояла в очереди на паспортный контроль всего с четверть часа, но за это время у меня сложилось впечатление, что где-то там наверху боженька с похмелья споткнулся о земной шар, ушиб мизинец и, разозлившись, пнул Землю еще сильнее, отчего та перевернулась, и теперь на меня извергались все воды Антарктического океана.

Очевидно, я слишком долго не отвечала потому, как незнакомка из автобуса повторила свой вопрос.

-- Ты знаешь, куда здесь идти? -- еще громче крикнула она.

Я посмотрела направо: там был высоченный забор с подсветкой и колючей проволокой. Затем посмотрела налево -- абсолютно идентичная картина. На всякий случай я еще взглянула через плечо: там виднелся первый контрольный пункт, который мы как раз прошли, а за ним возвышался автобус, что довез нас до границы.

\emph{А ты вот как думаешь?} -- хотелось ответить мне.

Но я посмотрела на эту до нитки продрогшую барышню на десяток лет старше меня, и мне стало жаль ее. Еще одна любительница нафуфыриться в дорогу.

-- Туда, -- ответила я, указывая в, в общем-то, единственном не прегражденном заборами направлении.

Раз в сотый подряд я отбросила с лица мокрую прядь волос, (она заняла прежнее место ровно через десять секунд) покрепче вцепилась пальцами в замерзшую ручку чемодана, что так и норовила выскользнуть из мокрых рук, и принялась прокладывать путь сквозь болото, лужи и скрывающиеся под ними ямы.

Прошла я метров сто пятьдесят, когда услышала голос все той же блондинки. Теперь ее слова окончательно терялись в гуле непогоды. Кажется, она застряла каблуком в одной из тех самых ямок, что прячутся за лужами. В общем, мне пришлось вернуться.

Спустя каких-то пять минут незнакомка из автобуса была спасена. Правда, за это время я окончательно промокла. Оказавшись на воле, женщина -- она представилась Витой -- радостно заявила мне:

-- А теперь бежим!

До следующего автобуса оставалось практически полтора часа, а ни на одной из нас не осталось сухого места, так что я спросила:

-- А зачем?

И, окутанные пеленой неугомонного ливня, мы зашагали по этим бюджетным декорациям к сериалу «Ходячие мертвецы».

***

Перенесемся в тот момент, когда я вновь проделывала все тот же путь: паспортный контроль, проверка багажа, пограничный контроль и более детальная проверка документов. Все это под открытым небом, без малейшего намека на укрытие.

Выстояв последнюю очередь, я, формально, оказалась на родине. В общем-то, ничего не изменилась. Дождь и не думал утихать. Холодные капли продолжали беспощадно хлестать по лицу. К счастью, к тому моменту я замерзла настолько, что практически этого не чувствовала. До отправления следующего автобуса оставалось еще минут с пятнадцать.

Он стоял в чистом поле, но, устроившийся внутри водитель был против того, чтобы кто-то заходил в салон раньше времени отправления. Я швырнула чемодан на какие-то изломанные дождем ростки то ли пшеницы, то ли кукурузы, умостилась сверху и закурила.

Вита стояла рядом. По-прежнему на шпильках, которые теперь вдоль и поперек были облеплены грязью. Обхватив себя руками, она дрожала и нетерпеливо поглядывала на часы. Сейчас ее довольно миловидное лицо больше походило на древнюю маску с гротескно растянутыми чертами. Платье настолько промокло, что сквозь него без труда просматривалось не только точеная фигурка женщины, но и все то, что должно было скрыть белье. Наблюдая за спутницей, я пообещала себе никогда не надевать стринги под короткое светлое платье. Потом вдруг вспомнила, что с утра и сама была немного при параде. В общем, я решилась достать из сумочки зеркало\ldots{}

Картина открывалась, прямо-таки скажем, пугающая. Мои прекрасные волосы, которые совсем недавно шелковистыми волнами ниспадали на плечи, теперь потускнели, став каким-то безжизненным начесом из восьмидесятых. Лицо вообще представляло собой отдельную тему, касаться которой в приличных кругах явно не следует. Должно быть, именно так смотрятся шлюхи после тяжелой ночной смены. Все выглядело настолько дурно, что, попытайся я это исправить, сделалось бы только хуже, так что я подхватила перепачканный мокрой землей чемодан и с безразличным лицом вошла в автобус.

***

-- Как думаешь, мы когда-нибудь остановимся? -- спросила Вита.

Мы расположились на галерке, получив возможность слегка подсохнуть после продолжительного ливня. Последний, кстати говоря, все так же барабанил в окна автобуса. Прошло около трех часов, а мы ни разу не остановились. Несмотря на царившую за пределами автобуса сырость, в салоне было очень душно. Но открыть окно все равно никто не решался. Было в этой ситуации кое-что позанятней духоты: мы уже битый час ехали через поле. Причины этому никто не знал, но навигатор в телефоне сообщал, что таким образом мы прибудем в пункт назначения на пару часов раньше, так что я особо не возражала.

Автобус остановился лишь в Херсоне, впервые за пять, а может и шесть часов.

-- У вас двадцать минут, -- объявил водитель, который еще в начале пути показался мне подозрительной личностью.

Пассажиры тут же ринулись к выходу: избавляться от старой жидкости и наполняться новой. На протяжении всего пути Вите требовалось в дамскую или уже хоть какую-нибудь комнату с наличием унитаза, так что не успел автобус заехать на стоянку у автовокзала, моя новая знакомка уже стояла около двери.

Я, конечно же, отправилась за кофе. Зашла в миниатюрный бар-кафе, который, к тому же, был еще и шашлычной, и уже даже не удивилась внезапной встрече. За прилавком стояла Тома. Все в той же полосатой футболке. Только ее подстриженные под миллиметр волосы, казалось, стали еще короче. В такие моменты мне думается, что у проекта о моей жизни очень низкие рейтинги и маленький бюджет, которого хватает всего на пару статистов. Девушка сразу же меня узнала, (что так несвойственно для работников подобных привокзальных заведений, особенно когда ты врываешься в них в макияже, поплывшем после проливного дождя) и даже назвала по имени.

-- А я вот с мужем развелась и вернулась в Херсон, -- сказала Тома, протягивая мне кофе. -- А ты там что? Замуж хоть еще не вышла? Видок у тебя какой-то таинственный.

Я подумала о том, что встречать незнакомых или практически незнакомых людей куда приятнее, чем людей из прошлого. Им ты вряд ли сможешь вот так просто открыть душу, не рискуя загадить последнюю. Как раз собиралась ответить на вопрос Томы, когда та вдруг захохотала. Бармен-бариста-официант и теперь, очевидно, грильщик, указывала пальцем в сторону окна.

Проследив за этим жестом, я увидела женщину. Она бежала за стремительно удаляющимся автобусом и энергично размахивала руками.

-- Еще одна, -- сказала Тома. -- Каждый день такое вижу.

-- Еб твою мать, -- ответила я.

Той женщиной была Вита.

\hypertarget{chapter-40}{%
\chapter{~}\label{chapter-40}}

Теперь мы уже бежали вместе: Вита -- все на тех же несчастных шпильках -- и я со стаканом кофе в руках, который, к тому же, зачем-то старалась не расплескать.

-- Там моя сумочка! -- кричала Вера. -- В ней все документы! И деньги. И мамины сережки! И мобилка!

\emph{И электронная книга}, подумала я, но ничего не ответила.

Сложно говорить, когда ты, истерически смеясь умираешь от боли в боку, но продолжаешь бежать под проливным дождем.

Мы срезали напрямик через вокзал и практически догнали автобус. Асфальт под ногами сменился слякотью обочины и мне уже были видны лица сидящих в салоне людей. Это были тучные женщины далеко за сорок с мордами-лопатами, которые просто смотрели за тем, как мы бежим за автобусом и размахиваем руками. За ними сидел одинокий дед. Он оказался куда более приветливым. Дед просто улыбался и махал нам в ответ. Следом шло целое семейство с синдромом Дауна и к ним у меня, в общем-то, нет никаких претензий.

Я наблюдала всех этих людей последние часов десять. Мы вместе ехали в первом автобусе, стояли в очередях и все такое прочее. Кто-то их этих людей даже спрашивал у меня ручку. Но сейчас никто и не думал предложить водителю остановиться. Они просто сидели.

И смотрели.

Возможно, даже делали ставки на то, как скоро кто-нибудь из нас поскользнется в грязи. Где-то там, на выезде из автовокзала, между попытками не задохнуться, я попыталась их возненавидеть. Но вышло так себе.

Почему?

А теперь просто представьте открывающуюся этим людям картину: грязь, какой-то вечный ливень и мы -- фанатки группы Кисс с пугающими начесами и соответствующим макияжем -- настойчиво преследуем общественный транспорт. Так, должно быть, выглядели группи, бежавшие за гастрольным автобусом вышеупомянутой группы.

Будь я старухой с мордой-лопатой, пожалуй, я бы тоже постаралась убедить себя в том, что увиденного не существует. Быть может, что мы с Витой до сих пор им снимся.

***

Мы бежали вдоль обочины, несмотря на то, что автобус стремительно продолжал отдаляться. К тому моменту осваивать йогу я начинала только в мыслях, и моя физическая подготовка по-прежнему оставляла желать лучшего. Левый бок исходил конвульсиями, а легкие пылали так сильно, что, казалось, я вот-вот задохнусь. Наконец, тот факт, что за автобусом нам не угнаться, стал более чем очевиден.

Именно в этот момент из-за поворота появилось такси. В салоне уже сидели пассажиры.

-- Тормози, -- простонала я Вите.

Хотя, на самом деле, вышло что-то куда менее членораздельное. Вита меня не поняла, а времени на объяснения совсем не оставалось. Машина уже была метрах в пяти от нас. Пришлось импровизировать. Я сделала лучшее из того, что только могло придти в мою затуманенную, промерзшую до нитки, уставшую голову -- выскочила на встречную полосу. Таксист остановился в каком-то полуметре от меня, и спустя мгновенье мы уже сидели в салоне авто вместе с недоумевающими пассажирами.

-- Вам\ldots{} куда\ldots? -- не столько спросил, сколько плавно произнес ошарашенный таксист.

Тут у женщины-пассажира вдруг прорезался голос.

-- Шо значит куда? -- завизжала она. -- Вы совсем пооболдевали так на дорогу прыгать? У меня тут все яйца покоцаются!

-- Ну, так не покоцались же, -- ответила я.

Женщина деловито исчезла в немыслимых размерах клетчатой сумке. Не той, которая от Louis Vuitton, разумеется. А в той, которая годов с восьмидесятых, если не раньше, является неотъемлемым атрибутом похода на рынок каждой уважающей себя совдеповки. В общем, дамочка пропала где-то в недрах сумки, но вскоре вновь явила себя миру.

-- И правда, не покоцались\ldots{} -- задумчиво произнесла она.

Люблю подкидывать людям поводы для раздумий.

Водитель откашлялся.

-- Так это\ldots{} Вам куда? -- успевший оправиться от шока, он принялся нас разглядывать. -- Девочки, с вами все окей?

На Виту было больно смотреть. Выглядела она так, словно все это время бежала по полям и огородам от самой границы. Да и я, пожалуй, воодушевляла взгляды не многим больше.

Водитель напустил серьезность и понизил голос.

-- За вами там часом никто не гонится?

-- Нет, -- говорю я. -- Это мы гонимся. За автобусом.

***

Наверное, в этот момент я должна была почувствовать себя одним из братьев МакКормиков, внезапно для самого себя догнавшим автобус Боно. Каких-то секунд пять водитель еще не решался открыть дверь. Он остановил автобус у обочины и теперь просто смотрел на то, как я выхожу из такси со стаканом кофе, который, кстати, я так и не пролила. Вита -- с разбитой коленкой, в испачканном платье и с отсутствующим каблуком -- хромала рядом. Мы подошли к автобусу и теперь тупо смотрели на водителя, который, в свою очередь, продолжал смотреть на нас.

Дождь вроде как все еще лил, но теперь это и впрямь казалось комичным.

Что было сил, моя спутница ударила кулаком по стеклянной двери автобуса, оставив на ней отпечаток перепачканных пальцев. Тут водитель опомнился и дверца отъехала. Вита мигом накинулась на виновника нашей незапланированной пробежки.

-- Ты че, мразь, вообще охуел? -- поинтересовалась она поразительно жутким для такой милой женщины голосом. -- Ты вообще, блядь, мозгами тронулся, дядя? Двадцать минут. ТЫ ЖЕ СКАЗАЛ ДВАДЦАТЬ МИНУТ! ТАК КАКОГО\ldots{}

И так далее, и тому подобное.

Я стояла у входа в автобус, Вита -- на ступеньках. Она все кричала и кричала, пытаясь добиться вменяемого ответа от невменяемого водителя.

Я попробовала кофе. Совсем холодный.

-- Да зачем ты его все это время носишь? -- раздраженно спросила спутница.

Она, наконец, поняла, что разговор с водителем бесполезен, и теперь продолжила подниматься по ступеням. Я последовала за Витой, и остановилась около водительского кресла.

-- Да вот за этим, -- ответила я, и плеснула кофе в лицо водителю.

Он все равно остыл.

***

Знаю-знаю, здесь я как бы должна извиниться и признать, что никакой кофе я ни на кого не выплескивала. Однако, именно так все и было. Сложно сказать, в какой момент своей жизни из шаблонного холерика я превратилась в настоящего меланхолика, который, к тому же, периодически проявляет признаки агрессии в самый неожиданный момент, но это и не важно. Пускай во мне жило несколько личностей, настоящая Лиза была лишь одна, и та вообще являлась сангвиником. Вот только по жизни ее поджидали увлекательнейшие метаморфозы. Когда мне было пятнадцать\ldots{}

Черт, как же много моих историй начинается именно с этой фразы.

Так вот, когда мне было пятнадцать, я безответно влюбилась в парня по имени Эрик. Он был высоким, задумчивым и таинственным. И он стал первым, кому по-настоящему удалось украсть мое сердечко и прочие составляющие супового набора. Вы, конечно, помните Эрика. Как помните и мои последующие еще такие детские страдания.

Мне было семнадцать, когда я впервые осознала, что что-то изменилась. В душе я по-прежнему воспринимала себя как вспыльчивую и очень характерную особу, но толком об этом не задумывалась. Две тысячи одиннадцатый год выдался слишком философским, чтобы тратить его на самоанализ.

Однажды, где-то в самый разгар октября, я пила кофе во дворике университета, когда ко мне подошла Нина, моя бывшая одноклассница и будущая супермодель. Было это в начале второго курса, факультета английской филологии с правом преподавания.

-- Что с тобой произошло? -- спросила Нина.

Вот так вот просто. Без приветствий и прочих социально приемлемых заигрываний в начале диалога. Я всегда ценила подобного рода минимализм, так что просто сказала:

-- Жизнь.

А она ответила:

-- Да ну.

И я заметила:

-- Ну да.

Так мы и стояли еще минуты три, пока я не достала из кармана пачку сигарет. В те дни я курила красный «Лакки» -- сигареты, которые ассоциировались у меня с Эриком. Грустила, кашляла, но продолжала курить.

Нина убедилась в том, что поблизости нет никого из знакомых, и только потом подкурила. Меня всегда поражали такие люди, о чем я тут же и сообщила однокласснице. Ведь если у тебя нет смелости заявить о том, что ты куришь, то как же быть с более серьезными вещами? Неужели одобрение общества до сих пор в тренде? Разве жизнь, прожитая в притворстве, имеет какой-то смысл?

Это было еще до того, как мне стало плевать на все на свете.

До того, как я встретила Адама.

До того, как я вновь встретила Эрика.

До того, как я встретила Веню.

-- Может, и нет, -- ответила Нина. -- Но моя репутация должна быть безупречной.

Кажется, я как раз собиралась поинтересоваться, на кой черт кому-то в двадцать первом веке может понадобиться репутация, но Нина меня опередила. Она перевела тему.

-- Женщина, ты мне лучше скажи, что такое с тобой стряслось?

Я не очень поняла, о чем она, так что вопросительно подняла бровь.

-- Ну, знаешь. Когда я впервые увидела тебя на лекции, я думала это кто-то другой. А потом ты что-то сказала преподу, и я слышу, голос ведь твой. Ну, тут я поворачиваюсь к Василисе\ldots{} Ты же помнишь Василису? Из параллельного класса. Поворачиваюсь и говорю: гляди, там Лиза Васляева в третьем ряду сидит и ведет себя совершенно спокойно. А она мне отвечает: да быть такого не может!

Стоит ли говорить о том, какое прекрасное мнение на окружающих произвела я в школьные годы? Пожалуй, что нет.

-- Мы с ней даже немножко поругались, -- продолжала Нина.

Затем она назвала меня молчаливой, задумчивой и таинственной и вновь поинтересовалась, что со мной произошло. А произошел со мной Эрик, которого я когда-то тоже считала молчаливым, задумчивым и таинственным.

Так я узнала, что официально покинула ряды местных сумасшедших.

Теперь я была среди грустненьких.

Надолго ли?

***

Вернемся же к водителю злополучного, если не сказать хуже, автобуса. И к тому факту, что ему еще понадобилось какое-то время, чтобы вернуться за руль. Пройдя вглубь салона, я увидела, что наши места оккупировало семейство каких-то азиатов с тремя детишками. Они либо умело делали вид, либо действительно ничегошеньки не понимали по-русски, то ли оказались по-славянски наглыми. Короче, вставать ребята не хотели. Я просто вытащила из-под одного из них свой чемодан и со словами «сесе нинь дэ гуанчжоу!» села прямиком в проходе салона.

Младшая из детей взглянула на меня как на дуру.

-- Это на китайском, -- объяснила девочка. -- А мы японцы.

Так и знала, что они меня поняли!

-- А как сказать по-японски «это мое место, засранцы»?

-- Кацира га вата\ldots{}

На этом мать семейства шлепнула дочку по лбу и больше со мной никто не говорил.

\emph{Ну и хрен с ним}, подумала я. \emph{Можно и на полу посидеть}

В конце концов, до Николаева оставался всего час езды.

Я вытянула перед собой ноги, загородив добрую часть прохода, и вскоре уснула.

\hypertarget{chapter-41}{%
\chapter{~}\label{chapter-41}}

Возвращаться в свой родной -- этот откровенно уездный -- город было и грустно и весело. Весело потому, что мне хотелось выпрыгнуть из автобуса и бросаться на шею каждого встречного, рассказывая ему или ей о Вениамине. О том безграничном счастье, что я ощущала и о той любви, что нежданно-негаданно вспыхнула в моем сердце.

Когда слишком долго находишься в депрессии, она становится полноценной частью тебя. Ты и депрессия, вы теперь -- одно целое, так что, почувствовал улучшение, я сразу ощутила некий дискомфорт, вызванный столь неожиданной ментальной стабильностью. Как будто кто-то пытался вытолкнуть меня из мрачного, но такого привычного и безопасного пристанища.

Пришло время показаться на работе. Я шла по залитым теплом, уже практически летним улицам своего родного города. Пение птиц больше не раздражало мой слух, солнечный свет не ослеплял, а наоборот приятно согревал кожу. Он скользил по стенам и играл в листве, создавая причудливые тени.

И все вокруг мне казалось прекрасным.

Даже перевернутые по старой доброй традиции мусорники, вроде бы, не так отравляли воздух. Тут-то я и напряглась. Прежде со мной такое бывало лишь при умопомрачительной влюбленности. А влюблялась я, как вы уже поняли, раз в пятилетку.

И все это время я всерьез полагала, что по-настоящему счастливым можно быть лишь по уши погрязнув в любви. Искренне поражалась людям, которые не состоят в отношениях, но при этом регулярно следят за собой, развлекаются и куда-то ходят. В общем, выглядят так, словно они счастливы самим себе. Я смотрела на них с изумлением, а на деле все страдала, страдала и страдала. В основном от одиночества и неразделенных писательских амбиций.

Так вот, мне понадобилось угробить лучшие годы своей жизни, (или, по крайней мере, те, о которых принято говорить, что они лучшие) чтобы осознать собственную неправоту. И еще какое-то время на то, чтобы с ней смириться. Но когда это, наконец, свершилось, я вдруг оказалась безгранично влюбленной. Незапланированно влюбленной.

Затем почувствовала себя не в своей тарелке. Быть выспавшейся, веселой и жизнерадостной казалось мне очень странным. Я и прежде бывала гиперактивной, но это была нездоровая радость. Ввиду моего диагноза, не более того. А теперь я и впрямь чувствовала, что я в порядке. Это началось еще тогда в «Штиле», до знакомства с Веней, но, почему-то не прекращалось и по сей день. И это было жутко. Как будто кто-то просто забрал все, что у меня было, и подсунул взамен жизнь какой-то другой, менее проблемной девушки.

А была у меня, очевидно, только депрессия.

\hypertarget{chapter-42}{%
\chapter{~}\label{chapter-42}}

Как выяснилось, решение выйти замуж далось мне куда легче, чем перспектива объяснить это самое решение другим людям. Меня поддержали лишь Марта, встречи с которой я ждала со дня на день, да Лина. Но первая все так же не знала негативного жизненного опыта, а вторая\ldots{} Думаю, она сделала это только потому, что слишком меня любила.

-- Ты мне смотри, -- сказала она. -- Если он что тебе сделает, я приеду в Крым и оторву его бородку к чертовой матери. Отак ему и передай.

В прошлом Лина была чемпионом по борьбе. В нынешнем -- лесбиянкой и моей хорошей подругой. Я знала, что, в случае чего, именно так она и поступит.

-- Ты его вообще сколько дней знаешь? -- спросил мой брат и закатил глаза.

На самом деле, Олег был моим кузеном. К тому же, не по крови: дядя усыновил его в раннем детстве. И, все-таки, он был моим братом. Самым близким и любимым из всех моих многочисленных кузенов.

Прошел почти год прежде, чем мы вновь заговорили друг с другом.

Мишель -- самая близкая и самая давняя из моих подруг -- сделала вид, что рада за меня. А я, в свою очередь, сделала вид, что верю ей. Наша дружба дала трещину еще в пятнадцатом, но в двадцать лет я набила татуировку с ее именем, и пока еще не успела придумать, как мне быть дальше.

Короче говоря, чем больше людей узнавало о моей радости, и чем больше реакций я видела, тем сильней сужался круг гостей, которых я бы хотела видеть на своей свадьбе. А, вместе с тем, и круг моего общения. Никто из моих многочисленных друзей, приятелей, коллег и знакомых не был рад тому, что я, в кои-то веки, счастлива. Тогда я и вспомнила то, что успела забыть за годы одиночества: почему я с самого детства терпеть не могла людей -- в них нет эмпатии, только тонны эгоизма.

Было неприятно, но, все же, терпимо.

***

В начале того лета мы сидели с Линой в закрывшемся баре. Дело шло к половине четвертого утра. Я как раз закончила перемывать инвентарь и теперь была рада оказаться по другую сторону стойки. Мы пили Ксенту Супириор. Сил на то, чтобы брать чашу для абсента, вновь доставать стаканы, искать специальную ложку, класть на нее кубик сахара, а затем еще что-то там прокапывать, у меня не наблюдалось.

У Лины же явно не было желания ждать, пока я проделаю весь этот скрупулезный ритуал, а потом напрягу ее мыть посуду. Так что мы просто сидели в темном зале и пили этот дорогущий алкоголь абсолютно варварским способом -- прямиком из горла. Зато именно так в нем чувствовались все обещанные на этикетке семьдесят оборотов и даже больше.

Я отхлебнула обжигающий напиток и протянула Ксенту Лине. Та ее не взяла. Она смотрела не на абсент, а на мою руку, державшую бутылку. В баре было жарко, так что рубашки на мне не было. Только майка.

-- Что у тебя с рукой? -- спросила Лина.

-- Ничего, -- ответила я.

-- Н-е-е-е-т, - протянула подруга и, наконец, взяла из моих рук бутылку. -- У тебя стопудово что-то не так с рукой.

-- Кажись за тобой прилетели феи.

Подруга продолжила настаивать на своем. Она настойчиво тыкала в сторону моей правицы.

-- А это, -- до меня, наконец, дошло. -- Это все Тарзан виноват.

Затем мы выпили, (поочередно, как и полагается бухарикам, не признающим посуду) и я рассказала ей, может, и не лучшую, но первую, из своих бесконечных историй.

***

Незадолго после того, как мне исполнилось четыре года, я сломала руку. Просто посмотрела «Тарзана» по черно-белому ящику и вылезла на шкаф, прихватив с собой пояс от какого-то халата.

Со мной в тот день сидела няня. Не такая, каких обычно показывают в кино. Это была хмурая, тучная женщина, которая должна была гулять со мной, заниматься чтением и чём-то там ещё, а на деле просто закрывала меня в зале со включенным на целый день телевизором, и уходила в соседнюю комнату заниматься своими делами. Звали ее Света.

Мы пробыли вместе не так уж долго. От силы пару месяцев, которые родители пахали без выходных, но за это время я успела выучить наизусть все рекламные ролики, что крутили на украинском телевидении в девяносто седьмом. Я до сих пор их помню.

У моей семьи не было денег на то, чтобы позволить одному из родителей сидеть дома, а отправляться в детский сад я категорически отказывалась. До этого я провела там дней пять и вскоре пришла к выводу, что все вокруг какие-то конченные и находиться в обществе других детей мне ну никак не хочется. Что ж, до определенного возраста я была уверена, что терпеть не могу детей. Затем я выросла, и поняла, что моя неприязнь относится не к детям, а к человечеству в целом.

О том, что такое мизантропия и как с этим жить, я тогда ещё не знала.

Выросла я в семье разведенных, но продолжающих поддерживать дружеские отношения родителей -- выходцев рабочего класса. Они очень много работали, но толку от этого было мало. За окном стояли суровые девяностые. Папа в те дни был следователем и ему, как и многим сотрудникам органов, регулярно задерживали зарплату годика на пол, а то и дольше. А затем выдавали не деньгами, а тушёнкой, кашей и конфетами с мармеладом. С тех пор я видеть не могу ни тушенку, ни мармелад.

Так вот, Света должна была наблюдать за мной до самого вечера, но она практически никогда этого не делала. Одним из таких дней, я вооружилась чужим поясом и каким-то феерическим образом забралась на шкаф, воображая себя, то ли Тарзаном, то ли кем-то из его мира. Пояс должен был послужить мне лианной. О том, что его следовало бы как-то закрепить я, конечно, не подумала. Да и разве об этом принято задумываться, когда тебе четыре?

Словом, я прыгнула. Следующим моим воспоминанием стала довольно красочная картина: я лежу на полу, лампа -- все та же лампа гостиной горе-няньки -- бьет мне в глаза, а справа от меня лужа крови, которая, к тому же, растет в размерах. Я пытаюсь отодвинуться от этой лужи, но, ни рука, ни даже плечо меня не слушаются, и кровь уже забирается в материю моего вязаного свитера. Изображенные на нем Том и Джерри, прежде увлеченные погоней, уже никуда не спешат. Лишь таращатся на алую лужу, что все увеличивается и увеличивается.

Затем они смотрят куда-то ниже. Во взглядах проскальзывает ужас. Я тоже туда смотрю и вижу, что в районе сгиба руки виднеется что-то светлое, но не понимаю, что это что-то -- моя локтевая кость.

Тем временем, вокруг уже бегают санитары скорой помощи. Они пытаются надеть на меня кислородную маску, но я всячески сопротивляюсь. Кричу, отбиваюсь одной рукой, (вторую я не чувствую) топаю ногами и даже кого-то кусаю.

Как я уже говорила, Света частенько оставляла меня наедине с телевизором. Никто не фильтровал поток просматриваемых мной фильмов. Многие из них сегодня явно имеют рейтинг R. В итоге, я знала одно: если на тебя надевают кислородную маску, не жди ничего хорошего. Скорее всего, ты проснешься в холодном подвале, раздетый и прикованный в батарее. Возможно, уже трижды изнасилованный. Весь пол будет залит ледяной водой, смешанной с кровью и дерьмом. На стенах окажется ржавчина, а окон как таковых и вовсе не будет. Рядом с тобой могут быть и другие люди, но не факт, что живые. У кого-нибудь вполне может недоставать глаза или даже половины головы, а затем ты вдруг обнаружишь отсутствие одной-двух конечностей на собственном теле.

В общем, я брыкалась, кусалась и орала. Вскоре в комнату вошли новые санитары. Общими усилиями им удалось надеть на меня маску, и я отключилась.

Картина, которую я увидела после пробуждения, не особо отличалась от описанных выше ужасов: я была в государственной больнице. И я не могла встать с постели или даже пошевелиться. Моя пострадавшая рука оказалась подвешенной к потолку. Со всех сторон ее пронзали какие-то спицы и штифты, чти шрамы останутся со мной на всю жизнь. Левая рука, а также мои ноги, были закреплены кожаными ремнями. Как вы уже поняли, у меня случился открытый перелом. Мама ни один год работала в сфере медицины и была знакома со многими достойными врачами, но Света не стала рассказывать ей, или папе, о произошедшем. Она представилась медикам моей родственницей и меня передали первому попавшемуся травматологу. Родители узнали о случившемся лишь когда я проснулась от наркоза, увидела медсестру и начала требовать позвать маму и папу.

Папу и маму, если быть точнее.

Конечно, к тому моменту меня уже прооперировали. Света, завидев моих родителей, ретировалась в первую попавшуюся дверь и больше ее никто не видел. Как выяснилось позже, с врачом мне повезло не больше, чем с нянькой. Однажды его уволили, но случилось это уже после того, как он сложил мою правую руку. И сделал это неправильно.

Я провела в больничной койке ровно пять недель и ни разу за это время не могла встать на ноги. Когда, наконец, пришло время вынимать все эти средневековые железные крепления, что как минимум достойны упоминания в списке декораций «Восставшего из ада», мне было позволено встать на ноги. Тут-то и выяснилось, что я разучилась ходить. К счастью, это умение вернулось ко мне спустя несколько дней, чего нельзя было сказать о моей правой руке. Горе-доктор обещал, что мне понадобится пара недель на реабилитацию. Затем рука вновь сможет чувствовать, держать предметы и все такое прочее. Короче говоря, функционировать.

-- Не переживайте, ничего делать не надо, -- говорил он взволнованной матушке. -- Само пройдет.

Прошел месяц. Затем еще один. Я стала левшой. Моя сломанная и заново сложенная рука не хотела работать. Хуже того, она также не желала расти вместе со мной. Пальцы становились какими-то скрюченными и очень тонкими. Они практически не разгибались, а сама рука была очень бледной. Нет, мне не ампутировали конечность, ничего такого, ведь это кем надо быть, чтобы заявить о таком лишь в средине книги?

В конечном счете, нашлась талантливая доктор, которая помогла поправить ситуацию и год спустя я вновь стала правшой. Осталась лишь одна проблема, решить которую не смогли даже более удачливые доктора: моя рука не могла согнуться так, как следовало. Вместо этого она проворачивалась в локте и изгибалась каким-то совершенно неестественным образом, но никто не хотел заново ломать руку четырехлетнему ребенку. Да и шансы на то, что это поможет, были малы. С момента заживления прошло слишком много времени.

Так я и осталась: с испорченной рукой и поломанной психикой.

\hypertarget{chapter-43}{%
\chapter{~}\label{chapter-43}}

-- Так и че с садиком? -- спросила Лина.

К тому моменту наш запас алкоголя успел наполовину сократиться. И я ответила, что вскоре мне, все же, пришлось туда пойти.

-- Первые недели три было не сложно. Во время обеденных прогулок и прочих местных игрищ я просто уходила куда подальше, отворачивалась от других и делала вид, что их не существует\ldots{}

Тут Лина залилась звонким смехом.

-- Ты и сейчас так делаешь!

-- \ldots Знаешь, песни там пела и все такое. Но период социально приемлемой адаптации прошел и от меня стали требовать непосредственного общения с другими детьми. А они все так же казались мне идиотами.

-- Кстати об идиотах, -- вдруг вспомнила подруга и икнула почти так же звонко, как смеялась. -- Что там твой Веня, молчит?

Молчит он, или делает чего похуже, я не знала потому, что отключила телефон еще в начале вечера. Пожалуй, здесь стоит объясниться.

***

Вениамин позвонил мне в начале первого, что, само по себе, уже было странным. Он знал, что после полуночи за стойкой начинается самая настоящая запара. Если мы и общались в это время, то только посредством переписки.

Тем не менее, Веня позвонил. Я спряталась в шкафу для алкоголя и взяла трубку со словами:

-- Что стряслось?

-- Им не удалось меня поймать, -- вдруг заявил мой жених.

Голос его был странным. Чужим и отрешенным. Как тогда, на балконе, когда одному из нас пришло в голову провести кельтский колдовской ритуал. И этот голос принялся рассказывать мне совершенно несвязную историю. Она звучала как обрывки фраз, услышанные от разных людей в переполненном помещении. Но, что еще хуже, создавалось впечатление, что этот голос -- а вместе с ним и Веня -- явно куда-то бежит.

-- Где ты? -- было первым, что я спросила.

-- Я\ldots{} во\ldots{} во\ldots{} дворах, -- раздался прерывистый ответ.

Как я и говорила, у меня уже имелся немалый опыт общения с невменяемыми людьми. В какой-то степени, я и сама была одной из них. В общем, я сделала вид, что все нормальненько, и постаралась разобрать всю ту бессвязную ахинею, которую говорил жених. Мой краткий и, можно сказать, авторский перевод гласил следующее. Несколько часов назад (когда точно, было не ясно, ведь здесь мнения объекта повествования здорово так расходились) Вениамин закончил работать. Вместо того, чтобы отправиться домой, он решил заскочить «Среду обитания» -- тот самый бар, где прошел один из моих последних вечеров в Севастополе, как раз перед клятвой на крови. Сперва все шло как обычно, а затем местный бармен настойчиво стал требовать, чтобы Веня убрал свой нож.

-- Нож? -- переспросила я.

-- Нет, не нож.

-- А что тогда?

-- Ножи! -- гордо и как-то обиженно исправил меня Вениамин.

Как будто это в корне меняло дело.

Было это после пятнадцатого или шестнадцатого шота с водкой, от чего становилось ясно, что ножи были нужны Вениамину явно не для того, чтобы вырезать фигурки из бесплатного хлеба, который некоторые заведения все еще держат на барной. Пробелов в истории было еще больше, чем неясностей. Закончилось все тем, что сотрудники заведения вызвали милицию. Когда та приехала, вся стойка была в дырках и царапинах от ножа, зал опустел, и только Вениамин, как ни в чем не бывало, сидел на вращающемся стуле, продолжая кутить в одиночестве. Ножи, кстати говоря, торчали рукоятками вверх вдоль барной стойки.

Так вот, в итоге моему жениху было предъявлено обвинение в ношении холодного оружия, порче чужого имущество, плюс что-то там о разбое. Далеко не впервой, но! Когда пришло время оформлять моего красавца, тот резко поднялся со стула, выскочил в дверь и убежал в темноту как был: без сорочки и с рюмкой руке.

Так, очевидно, он и бегал по близлежащим дворам до того, как позвонил мне.

Спустя два месяца я решилась выпить кофе в «Среде обитания». Бармен узнал меня и, посмеявшись вдоволь, спросил, почему же мой суженый больше к ним не заходит. После чего подтвердил рассказанную выше историю, и я узнала, что в перечне поставивших Вениамина в черный список заведений стало одним баром больше.

-- Им меня не догнать, -- отдышавшись, повторил Вениамин.

Затем он сказал, что ему холодно и что он скучает. Кажется, даже всплакнул. В дверь моего алко-шкафа уже вовсю стучали заждавшиеся заказов официантки, а Веня все спрашивал, когда же я приеду.

-- Если бы я только знала, -- сказала я.

Две недели, которые мне полагалось отработать перед увольнением, давно прошли. Никто не спешил нанимать и обучать нового бармена. А, вместе с тем, и не спешил выдавать мне зарплату. Мое заявление об увольнении по собственному желанию все так же пылилось на ресепшене. Его до сих пор не подписали.

-- У нее там личный разговор! -- вдруг услышала я голос Лины.

Он звучал очень громко, и я поняла, что подруга стоит прямиком у шкафа, в котором я все это время сидела.

-- Мне без тебя жить не хочется, -- тоскливо произнес Веня.

В этот момент дверь шкафа распахнулась. На пороге стояла управляющая. Вид у нее был совсем не дружелюбный. Из-за плеча женщины выглядывали фицы с пустыми подносами.

-- Прости, -- прошептала я в трубку. -- Мне нужно бежать. Напиши мне, когда будешь дома.

И я отключилась.

***

Прошло больше часа. Я выполнила все заждавшиеся меня заказы. А следом за ними еще несколько новых. Затем еще и еще. Количество посетителей ресторана начинало уменьшаться. В конечном счете, остался всего один столик, вальяжно распивающий пятую бутылку коньяка, да местный алкаш, дремлющий за стойкой в компании надпитой рюмки.

Вениамин все не звонил. Его телефон был отключен на протяжении последних сорока минут. Стрелка настенных часов уже успела перевалить за отметку «2». В начале третьего Веня объявился. Как выяснилось, все это время он провел дома у Нели -- той самой подчиненной, заказавшей при мне в KFC восемь бургеров и что-то там еще. Как он туда попал было не ясно, ведь бар находился в часе с лишним езды от дома женщины, а транспорт в Севастополе ходит лишь до полуночи. Зачем он туда попал -- тоже хороший вопрос, ведь «Среда обитания» находилась через дорогу от дома, где жил Вениамин.

-- И что там делаешь? -- спросила я.

-- Пью водку, курю траву\ldots{}

-- Водка закончилась! -- услышала я на фоне раскатистый голос Нели. -- Мы идем в магазин!

Шальной и угашенный как вся моя жизнь.

-- Водка закончилась, -- повторил за ней Вениамин. -- Мы идем в магазин.

-- Постой, Веня! Ты где ночевать-то будешь?

-- Я буду ночевать там, где будет водка! -- крикнул он, рассмеялся и повесил трубку.

Я спрятала телефон.

-- Все нормально? -- спросила Лина.

Она как раз занималась финальным отчетом.

-- Нет, -- ответила я.

И рассмеялась. Смех получился каким-то нервозным. Оно и не удивительно.

-- Поня-я-я-тно, -- протянула подруга. -- Уже через минуту заканчиваю. Неси Егеря.

***

Спустя минут пятнадцать, очевидно, пополнив свои заветные запасы, Вениамин позвонил снова. Чтобы сообщить, что мы переезжаем жить в Симферополь. Кто это «мы» и зачем мы туда переезжаем было не очень ясно. Как и все остальное. Разговор оборвался так же внезапно, как начался, и я решила отключить телефон прежде, чем Веня позвонит еще раз, чтобы рассказать мне о смене пола или, скажем, своем внезапном решении усыновить сиамских близнецов из Башкирии.

-- Если он только хочет переехать, все не так-то плохо, -- прокомментировала Лина. -- Но то, что он ночует у этой бабы\ldots{} Как там ее звать то? Короче, пофиг, ты ее вообще знаешь?

-- Ага, -- ответила я.

Настроения у меня больше не было. Егермейстера к тому моменту тоже. Пришлось идти за абсентом.

\hypertarget{chapter-44}{%
\chapter{~}\label{chapter-44}}

Прошло еще двадцать восемь дней прежде, чем мое заявление об уходе было подписано. Двадцать с лишним дней, подавляющую часть которых я провела за стойкой. Находиться здесь мне совершенно не хотелось. Не только потому, что платили барменам настолько мало, что я даже подумывала вернуться в IT. И не только потому, что Вениамин не мог ко мне приехать. Наверное, к тому моменту мне просто все надоело.

Надоело работать по четырнадцать, а то и шестнадцать часов в сутки, постоянно находиться в шуме, таскать ящики с товаром со склада двумя этажами ниже, строить вежливость и по несколько часов к ряду развлекать сидящих за стойкой алкашей, которые все равно не оставят тебе чаевых. А если вдруг оставят, то этих денег не хватит и на пачку сигарет. Да, каких сигарет, на проезд не хватит! Пожалуй, самое обидное в моей работе было даже не отсутствие чаевых, а то, как мне их преподносили. Какой-то захудалый пьяница оставлял на стойке свои копейки с таким видом, словно оплачивал мне год учебы в Оксфорде.

Середина июня уже давно была позади, а я все так же продолжала торчать здесь, сама не зная зачем. И вот однажды я вошла в двери этой богадельни как, в общем-то, делала каждое утро. На часах было ровно пол одиннадцатого утра. Кассир, сменщица Лины, как раз переодевалась в подсобке. Это была сорокалетняя женщина по имени Галя, работавшая в «Штиле» с самого его открытия вот уж без малого тринадцать лет.

-- Привет! -- раздалось из подсобки.

-- Доброе утро, -- ответила я и начала завязывать волосы.

После жалких чаевых, это был мой самый ненавистный аспект работы барменом.

-- Я говорю, привет! -- настойчиво повторила Галя.

-- Я отвечаю: доброе утро!

От постоянного гула музыки и от криков постояльцев слух Галины явно подкосил. Приходилось практически кричать, чтобы до кассира дошел смысл моих слов. Сама она, кстати, почему-то говорила очень тихо. При всем при этом Галя всегда была в курсе последних новостей. Она слышала различные сплетни, знала, кто кому изменяет, помнила все грандиозные (и не очень) попойки и все такое прочее. По всей видимости, острый слух, все-таки, возвращался к Галине при нужных обстоятельствах. Но, что было еще лучше, все эти внутренние проблемы и откровения мало волновали бывалую кассиршу, так что если она и выдавала какой-нибудь факт, то только по делу.

-- ПРИВЕТ! -- еще громче крикнула кассир.

Я подвязала волосы, надела фартук и подошла к самой двери подсобки. Приоткрыла ее и только тогда завопила:

-- ДОБРОЕ УТРО!

На что Галя ответила мне:

-- О, привет. А я уж было думала, ты меня так и не заметишь.

Иногда складывалось впечатление, что она надо мной издевается.

-- Я спущусь на склад, а потом сразу кофе. Тебе делать? -- я уже сверяла полный список алкогольной продукции с тем, что было в наличии за стойкой.

-- Делай, -- ответила Галя и вдруг заговорческим тоном добавила: -- И долго ты здесь еще будешь?

-- Где?

Кассир обвела взглядом бар, потолки и стены ресторана.

-- Надеюсь, нет, -- произнесла я. -- А что?

Следующую свою реплику Галя начала с моей любимой фразы. Она сказала:

-- Да тут поговаривают\ldots{} -- и тут же замолчала.

Пришлось крайне выразительно поднять брови и на какое-то время задержать их в таком состоянии, чтобы услышать продолжение.

-- Поговаривают, что тебе работа твоя не нравится.

-- Ничего себе проницательность!

Сарказма Галя не оценила. Лишь повторила сказанное.

-- Поговаривают, что тебе работа твоя не нравится.

-- Так это же прекрасно! -- заметила я.

-- Прекрасно? Почему?

-- Может, эта мысль как-то приведет их к моему заявлению. И его хоть кто-то подпишет.

Галина завертела головой.

-- Нет-нет-нет, ты же не знаешь ничего. Поговаривают, что тебе работа твоя не нужна и что ты на нее ходишь от скуки. Что у тебя папа известный в городе адвокат и деньги тебе не нужны. Просто тебе скучно. Вот ты и решила сюда устроиться.

Будь я персонажем старых мультфильмов Диснея, моя челюсть уже болталась бы где-то на уровне пяток.

-- Поэтому начальство не подпишет заявление. Ира сказала, что ты должна лично попросить ее об этом. Они уже округлили твою зарплату в обратную сторону и начислили какой-то штраф.

Что ж, думаю, где-то в этот момент все мое дружелюбие в сторону Антона, гуру бартендинга, окончательно улетучилось. Мне стало все равно, где он будет искать новых сотрудников и почему так трудно это сделать.

-- Понятно, -- говорю я.

Затем традиционно скидываю с себя фартук, бросаю его куда-то в сторону и иду в зал ресторана. Несмотря на то, что до открытия минут десять, за угловым столиком уже сидит первый посетитель. Иностранец в очках-полумесяцах. Он медленно пьет эспрессо и что-то читает на планшете. За столиком у другой стены сидит Ирина Валентиновна, управляющая. О ней тоже поговаривают. Особенно о том, как из зеленой официантки она вдруг стала управляющей целого развлекательного комплекса меньше чем за сезон. Никто так и не понял, кто кому изменил: настолько быстро все произошло. А мне было ка-то не до этих сплетен. Да и, отчего-то, не верилось. Стоит женщине добиться успеха, и о ней уже тут же начинают поговаривать.

Вот иду к Ире, но не затем, чтобы рассказать эту увлекательную историю и свой взгляд на ее карьерный рост. По пути захватываю заявление, забытое всеми на рецепции. Управляющая сидит с опущенной вниз головой. Она увлечена своими записями и до последнего момента совершенно не замечает моего присутствия.

-- Доброе утро! -- звонко здороваюсь я.

Женщина подскакивает, а я сажусь за столик прежде, что Ира успевает что-то ответить. Она явно удивлена.

-- Не помешаю?

Управляющая ничего не отвечает, так что я просто протягиваю ей заявление об уходе, написанное двадцать восемь дней назад.

-- Мне кажется, вы его не заметили, и я решила лично попросить вашей подписи.

-- Зачем такая спешка? -- интересуется Ира. -- Ты же знаешь, у нас нет свободных барменов.

Ее губы растягиваются в самой широкой в мире улыбке.

-- Я согласилась подождать только из уважения к Антону, но свои положенные две недели я отработала\ldots{} -- я задумалась над базовыми подсчетами. -- Две недели назад. Больше я ждать не хочу.

-- Может, тебя у нас что-то не устраивает? -- голос управляющей звучит почти заботливо.

Она продолжала улыбаться, и я улыбнулась в ответ. Своей самой широкой в мире улыбкой. Вот они и встретились -- две фальшивые улыбки. Настолько блестящие, что, кажется, будто свет от ресторанной люстры-штурвала просто не знает, в каком месте ему нужно отразиться.

-- Вы знаете, мне на самом деле не нужны деньги. Мой папа известный в городе адвокат и я нашла эту работу просто от скуки. Я знаю раздел трудовой Кодекса Украины наизусть. Как и то, что штрафы запрещены Законом. Рассчитайте меня сегодня же.

Я все так же улыбаюсь. Управляющая -- нет.

***

Меня рассчитали в тот же день. Поставили отметку в трудовой книжке и почему-то не стали проводить проверку товара. Они даже забыли потребовать сдать форму. Та до сих пор пылиться где-то на даче.

Несмотря на то, что Добби подарили обещанный носок, день выдался слишком насыщенным. Оттого мне так приятно было его прожить. Часы неслись, работа кипела, а у меня даже не было времени присесть на минутку. За пару часов до закрытия в бар вошла Лина. Меня всегда поражали люди, который даже в свой выходной приходят на работу, сидят там весь вечер, общаются с коллегами и листают соцсети. Но Лина была исключением, ведь она приходила, чтобы поболтать со мной.

-- Вот это ты выдала! -- прокомментировала подруга, узнав, что сегодня мой последний рабочий день.

Я уже смешивала ей авторский коктейль под названием «Для Друзей Бармена». Прекрасное сочетание всего, что осталось с банкетов, а также того, что можно списать, не привлекая лишнего внимания.

-- Так, и шо же я теперь тут делать буду? Ты не подумай, я рада твоей личной жизни. И ты наконец-то отстанешь от меня с этим своим «Что-о-о я здесь делаю?»\ldots Но я поднялась работать в ресторан только из-за тебя.

-- В качестве извинения, -- сказала я и протянула Лине коктейль.

-- Одним тут не отделаешься, -- ответила она.

***

Следующие часа три прошли за приканчиванием личных алкогольных запасов. Понятия не имею, как долго это длилось, но, когда мы вышли на ступени «Штиля», за окном все уже выглядело так, словно вот-вот начнет светать. Усталость, наконец, дала о себе знать. Спать хотелось невыносимо, но мы еще какое-то время постояли у крыльца, зная, что таких посиделок больше не будет.

Странная штука работа. Ты словно состоишь с ней в каких-то сложных, чуть ли не абьюзивных отношениях. Ненавидишь, всячески увиливаешь от своих обязанностей и считаешь, что только попусту тратишь время. Но стоит тебе бросить это дело, вдруг понимаешь, что вот оно, еще одна крохотная эра твоей жизни закончилось. Так как прежде уже не будет. И тебе даже становится грустно, но вскоре и это проходит.

Так вот, мы еще немного постояли, вдыхая прохладный утренний воздух. После двадцати часов, проведенных в помещение без окон он казался роскошью. Затем мы взяли такси. Мой телефон издал конвульсию входящего сообщения.

-- Машина подъехала, -- прочитала я.

Как-то подозрительно быстро для такого раннего часа. Дорога перед нами была пустой.

-- Опять счетчик хотят пораньше\ldots{}

Подруга не успела закончить мысль: из-за угла выплыл бежевый лимузин. Из салона доносилась музыка, а окна машины переливались неоновыми гирляндами. Нормальное дело для ночного клуба, скажите вы. Вот только клуб к тому моменту уже закрылся. Водительское окно опустилось. Я увидела пожилого усатого мужчину за рулем.

-- Девочки, такси заказывали? -- спросил тот.

Мы переглянулись, затем синхронно пожали плечами и залезли в авто. Внутри оказалось много гирлянд. Куда больше, чем я предполагала. Они сияли и выразительно отбивались в горчичных глазах моей подруги. У окна стояло шампанское, а из колонок доносилась Staying Alive Би Джиз. Мы с Веней частенько под нее отплясывали.

-- Это что за подарок судьбы? -- спросила я у шофера.

-- Выпускные в Могилянке, -- ответил тот. -- Возвращался вот с заказа домой и решил вас подкинуть.

Могилянку я так и не закончила. Бросила на третьем курсе, когда понадобились деньги, так что и выпускного у меня не было.

Несмотря на то, что дороги пустовали, машина ехала медленно. Бутылка шампанского была открыта. Как и люк, в который я вскоре высунулась вместе с вышеупомянутым напитком. Staying Alive, казалось, звучала еще громче. Мы пели вместе с Барри Гиббом, смеялись и прикладывались к шампанскому. Лимузин как раз выезжал на проспект, когда я увидела своих теперь уже бывших сотрудниц. Тех самых официанток из клуба. Они шли вдоль трассы, навстречу машине, но тут же встали как вкопанные, увидев, чья физиономия возвышается над лимузином в сопровождении Би Джиз и шампанского.

После той ночи мне ни разу не было грустно от того, что я так и не побывала на университетском выпускном. Самое эффектное увольнение в моей жизни.

\hypertarget{chapter-45}{%
\chapter{~}\label{chapter-45}}

Дни тянутся неописуемо медленно, когда ты молод. Другое дело, когда ты загружен работой. Это выручает. Я не видела любовь всей своей жизни практически месяц, но у меня просто не было времени грустить по этому поводу. И вот, наконец, наступил день Х: работы у меня больше не было. До отъезда оставалось лишь три дня, а время вдруг начало нестись как сумасшедшее. Так, словно оно пыталось наверстать все те моменты, когда час был равен вечности.

Перед отъездом меня ждало еще несколько дел. Несколько встреч, посиделок, душевных разговоров и прогулок. И, конечно же, меня ожидало и множество прощаний. Одним погожим утром я проснулась раньше обычного. К тому же, без будильника, что уже само по себе было для меня странным. Меня пробудила мысль, такая ясная в своей внезапности. Я вдруг поняла, что мне, во что бы то ни стало, нужно избавиться от своего прошлого.

Я села на кровати, подкурила сигарету и, запыхтев прямиком в постели, начала думать над тем, как это сделать. Большинство вещей из прошлого, что гложили меня прежде, не были физическими. Биться головой о стену в надежде, что я ушибу именно ту часть коры мозга, где прячутся нежеланные воспоминания, мне как-то не хотелось. Я никогда не была сильна в таких делах. Все свои знания касательно анатомии человека почерпнула с судебной медицины и лекций по патологической анатомии, так что от этой идеи пришлось отказаться.

Заиграла музыка. Комнату наполнили сладкие звуки пост-панка, я прогулялась вдоль подоконника и вновь плюхнулась в постель. На соседней подушке спала малышка Тая, старшая из моих собак. Ей было тринадцать лет, но большинство незнакомых людей до сих пор принимали Таю за щенка. Если не считать парочки седых волосков, взгляд был единственным, что выдавало возраст моей собаки. Он был умным и выразительным. Словно говорил: «Я пожила. Я знаю жизнь.»

Мы всегда были вместе, и я едва ли могла припомнить, какой была моя жизнь до того, как в ней появилась Тая. Как бы смешно не звучало, это крохотное ушастое существо с большими глазами и китичками на ушах не раз спасало мне жизнь. В моменты отчаянья, в те самые дни, когда я фанатично хотела лишить себя жизнь, мою голову посещали не мысли об адских кострах и вечной агонии. Я думала о ней, моей девочке. Один небезызвестный канадский скаут сказал, что связь между человеком и собакой может исчезнуть только с жизнью. А разве я могла ее бросить?

Я обняла свою собаку. Та приоткрыла один глаз, лизнула мой нос и продолжила сопеть. Так мы пролежали еще какое-то время. Тая в своих собачьих грезах, а я в мыслях о том, что нам предстоит расстаться.

***

Мне было восемь, когда мы познакомились. За окном стояла ранняя осень. Третий класс, первые деньки учебного года. Несмотря на хроническую тягу к осени, этот период я всегда недолюбливала: слишком жарко, чтобы ходить в отвратительном зеленом пиджаке, как того требовала школа. Я была импульсивным ребенком, а духота лишь добавляла раздражительности.

Так вот, однажды, в первых числах сентября, возвращаясь с занятий, я по традиции зашла в аптеку, где работала моя мама. Она была занята покупателем и попросила меня перенести какие-то коробки из одного края зала в другой. Обычное дело, Ма частенько давала мне мелкие поручения. Все еще возмущенная погодой, школьной формой и чем-то там ещё, я подняла одну из коробок, -- она оказалась очень лёгкой -- и без особой заботы избавилась от неё. Практически швырнула в указанном направлении.

-- Ты что! -- вдруг крикнула матушка. -- Открой, -- добавила она с улыбкой, завидев мой удивленный взгляд.

Я приоткрыла коробку. Вдруг показались два крохотных уха, а за ними глазки-бусинки. Из картонного ящика выглядывало самое чудесное существо во всем мире. Наши взгляды встретились. Злость, усталость, раздражение -- все это моментально исчезло, как только я встретила свою собаку. Я взяла её на руки, и я вдруг поняла, что с этого момента для меня нет никого дороже.

Да, я мечтала о собаке сколько себя помнила. И вот я держала в руках настоящего щенка, похожего на черного лисенка. Она была совсем малышкой -- пушистая, едва больше моей ладони -- и ни на шаг не отходила от меня на протяжении того дня. Как и все последующие годы.

Мы назвали её Таей. Вечерами собака засыпала на моей подушке, а по утрам радостно носилась вдоль кровати и будила меня своим холодным носом. Знаю, многие говорят, что их собаки умные. Моя же понимала каждое слово и различала интонации. На самом деле, никто в семье не воспринимал ее как собаку. Тая была ребенком, которого все любили и который рос рядом со мной.

Однажды я спросила у мамы:

-- Что же я буду делать, когда ее не станет?

Этот вопрос истязал мой детский ум до тех пор, пока я не выросла и не решила наложить на себя руки. Такое вот простое решение любой донимающей вас проблемы.

***

Вспоминая наше знакомство, я довольно быстро провалилась в сон. Говорю же, ранние пробуждения -- не моя стихия. В какой-то момент я отключилась и видела во сне свое детство.

Вот мы с родителями сидим на берегу. Около реки на окраине города. За окном то ли апрель, то ли май девяносто восьмого, но на нас по-прежнему зимняя одежда. Воздух у реки холодный, а небо такое серое, что больше напоминает асфальт, нависший над водой. Отец собирает хворост вдоль тропинки. Чуть дальше сидит мама. Она заворачивает картофель в фольгу. Кроме нас троих на пляже никого нет.

Со стороны мама выглядит подавленной и ни ее пышные волосы цвета золотого каштана, ни умело наложенный макияж не могут этого скрыть. Перехватив мой взгляд, мама улыбается и машет рукой. Она молода и красива. Явно не выглядит на свои тридцать семь. Но даже это не может скрыть усталость в ее взгляде.

Интуитивно я замечаю, что здесь что-то не чисто, но это, в общем-то, все, что я могу понять. Мне всего пять.

Тогда я поворачиваюсь к папе, надеясь получить ответ. Ведь в мире нет вопроса, ответ на который был бы ему неизвестен. В этом я уверена на все сто. Но папа просто движется по периметру, подбирая нужные ветки. Отец выглядит так, словно его не интересует ничего кроме хвороста. Я пожимаю плечами и отворачиваюсь к реке.

Камень, на котором сижу, расположен вплотную к линии воды. От сильного ветра по реке рассыпаются волны. Они бьются о песок, о берег и о камни с бешеной силой. Брызги разлетаются во все стороны, и мне это очень нравится. Настолько, что я даже не замечаю, как сильно к тому моменту промокла моя обувь. У меня в руках томик Гоголя -- «Вечера на хуторе близ Диканьки». Я читаю «Майскую ночь» и останавливаюсь на самом интересном моменте. Гляжу на воду и представляю, как из нее выплывают утопленницы. Выбираются на холодный песок и начинают водить хороводы.

Наступает обед. Повесть уже давно прочитана. Родители зовут меня к костру. Отец подходит первым. Кладет ладонь на мое плечо и слегка ударяет по нему несколько раз -- так подбадривают старого друга в моменты печали. Я захлопываю книгу, и мы идем к костру, где мама уже раскладывает обед по пластиковым тарелкам.

Мои родители никогда не вели задушевных бесед друг с другом в моем присутствии, но сегодня они молчат больше обычного. А если и говорят, то как-то странно, искусственно. Но я едва ли об этом задумываюсь. В моей детской голове теплится волнующая мысль: я наконец-то знаю, кем хочу стать, когда вырасту и поскорее хочу рассказать об этом своем открытии.

После ужина мама еще долго смотрит на волны. Затем она берет меня за руку и начинает свой монолог. Ее голос звучит вполне спокойно, но как-то грустно. Мама говорит о жизни, о проблемах, о любви и о том, как порой сложно бывает совместить все это и не сойти с ума. Я по-прежнему увлечена собственными мыслями. Преимущественно об утопленниках и о том, кем я хочу стать. Словом, я улавливаю далеко не каждое слово. И еще меньше понимаю.

-- Мы с твоим папой тебя очень любим, -- говорит мама. -- Ты же знаешь, как много лет у нас ничего не получалось. Я так мечтала о том, что у меня будет дочка. Что мы с ней будем как подружки. Я очень боялась разницы в возрасте, но десять лет ушло только на то, чтобы завести ребенка. Мы уже начинали готовить документы на усыновление, когда вдруг мне стало плохо, а потом доктор\ldots{} Он сказал, что у нас получилось.

Звучит еще много слов, много вступлений и отступлений, и еще больше воспоминаний. Наконец, Ма делает паузу, а затем говорит мне то, ради чего затевался весь этот монолог.

-- Лишонька, я знаю, что ты сейчас меня не поймешь, но так больше не могло продолжаться. Я больше не могла так жить. Мы с твоим папой слишком разные люди. Поэтому я подала на развод.

И она замолкает. Последнее слово я слышу отчетливей всего остального. В свои пять лет я прекрасно знаю, что такое развод. Мои друзья во дворе об этом знают, и дети в садике, и в кино все время показывают, как это бывает.

\emph{Развод. }

Это слово эхом отдается у меня в голове. Вдруг все волнения становятся какими-то несущественными. Все, над чем я размышляла целый день, внезапно теряет смысл, но я все равно поворачиваюсь к родителям, чтобы озвучить свой совсем недавно обретенный жизненный план.

-- Мам, пап, -- говорю я. -- Я хочу стать патологоанатомом.

\hypertarget{chapter-46}{%
\chapter{~}\label{chapter-46}}

Оставим девяностые позади, чтобы продвинуться чуть ближе к настоящему.

Пожалуй, это был мой первый дневной выход на улицу за прошедший месяц или вроде того. Лето уже вовсю расправило свои знойные крылья. После привычной предрассветной прохлады, которая встречала меня по завершении рабочей смены, такая погода казалась особенно жаркой.

Ко мне приехала Мишель. Попрощаться перед отъездом, до которого оставалось еще часов сорок. В итоге, этот день стал днем нашей последней встречи, только ни я, ни, даже она тогда об этом не думали.

-- Давай туда, -- я указала рукой на то место, где заканчивался ряд покрывшихся ржавчиной гаражей.

Утром мы нашли бесхозную автомобильную шину и теперь тащили ее через весь двор. Пытались отыскать какой-нибудь укромный угол. На самом деле, шину тащила только я, а Миша просто создавал видимость работы. В физическом смысле она была вдвое больше и гораздо сильнее меня, но колесо оказалось не таким уж и тяжелым, так что я не возражала. Да и как тут возразишь, когда вся затея базировалась на собственной дурости.

-- Приехали? -- спросила меня подруга.

-- Приехали.

Я пнула шину ногой и осмотрелась. С двух сторон нас окружали старые гаражи, а остальное пространство занимали какие-то заросли. Воздух здесь был приятным. Он пах зеленью, свежестью и отчужденностью, как все те заброшенные кладбища, тропинками которых я любила гулять. Место что надо.

Тогда я опустилась на землю около колеса и выпотрошила в него содержимое рюкзака. Это была бумага, очень много бумаги. Тетради, блокноты, записки, альбомные листы, разлинованные листы и листы в клетку. Ровные, аккуратно сложенные, измятые, порванные, заляпанные кофе и бог его знает, чем еще. Были там и чеки, исписанные с обратной стороны, и салфетки, и открытки. Были и какие-то бесхозные клочки бумаги, совершенно не поддающиеся идентификации. Были, конечно, и тексты, напечатанные на компьютере и машинке. Довольно много таких текстов.

Откровенно говоря, я и не думала, что у меня собралось такое количество рукописей. Внутренняя часть шины уже была наполнена доверху, а содержимое рюкзака все не заканчивалось. Все это -- мысли, идеи, планы и полноценные тексты, которые я написала за прошедшие лет десять. Последней на свет явилась «Бессонница». Сомнительной ценности повесть, что я выдала за одну ночь. И, вместе с тем, единственное произведение, нашедшее видимый отклик у аудитории. Случилось это за пару лет до встречи с Адамом.

Я протянула руку к «Бессоннице» и развернула рукопись в первом попавшемся месте.

\emph{Глубина твоего познания равносильна ширине твоих взглядов}, говорил мне текст.

Я перевернула страницу.

\emph{Сторонясь людей и подолгу ни с кем не общаясь, начинаешь влюбляться в каждого встречного}, сообщил мне рассказ.

А ниже:

\emph{Шагаешь по старым улочкам, укрываясь шарфом; стараешься не вглядываться в пустые лица. Закрываешь глаза, заполняешь легкие дымом, чувствуешь глубокий шелест музыки, теряешься в мыслях, изношенный до невозможности, уставший, потрепанный, вывернутый наизнанку. Встречаешь чужие, но такие родные глаза. Понимаешь, что можешь быть счастлива с кем-угодно. Но не будешь. }

\emph{Просто потому, что все мои мысли рассыпаются каждый раз, когда я вижу тебя.}

Иными словами, ода Эрику.

Я перелистывала страницу за страницей. Мне было смешно, мне было грустно. А затем смешно от того, как это грустно и наоборот. Думаю, тем днем я держала в руках худшее из своих произведений. Сколько мне было, когда я это написала? Шестнадцать? Семнадцать?

-- Что ты там так долго? -- спросила Мишель.

Я о ней совсем забыла. Тем временем подруга присмотрелась к титульной странице, что валялась поверх остальной макулатуры, однажды вышедшей из-под моей руки, а я швырнула «Бессонницу» в общую кучу. Мысль о том, что я больше никогда ее не прочту, оказалась самой приятной за сегодняшний день.

-- Ах ты боже ж мой! Это то, о чем я думаю? -- воскликнула Миша. -- Дай-ка посмотреть!

Но я уже чиркнула спичкой. Всего мгновенье и мои рукописи охватило пламя. Тетради, блокноты, дневники, чеки, картонки, салфетки, билеты и прочие клочки бумаги. Мои мечты, мои печали, мои воспоминания. Они сгорали на глазах, и я не могла остановить пламя, даже если бы захотела.

Но я не хотела. Все это -- мое прошлое. И теперь, как и следует прошлому, оно обращалось в пепел.

Огонь потух так же быстро как разгорелся. Я заглянула внутрь шины. Не знаю, что именно я ожидала там увидеть. Может быть, финансовую компенсацию за потраченное впустую время или премию Эдгара Аллана По, присланную мне с того света? Может, записку от Булгакова, удивленного тем, что рукописи горят\ldots{} Понятия не имею.

Внутри ничего не было. Лишь пепел. Теперь он больше походил на прах. Останки моих писательских амбиций, мощи моей прошлой жизни. Я зачерпнула рукой горсть пепла, поднесла к лицу и расправила пальцы. И вот дух моего прошлого уже парит и разлетается в лучах июньского солнца.

А почему собственно нет? Однажды все мы станем лишь чьим-нибудь воспоминанием.

Мишель подходит ближе. Она берет меня за руку и спрашивает:

-- И что ты теперь будешь делать?

Перед нами раскинулись ветви широкого дуба. С одной из них свисает огромных размеров паутина. Солнечные лучи играют в ее узоре. Они блестят и переливаются. Под ногами растет трава самого насыщенного из всех оттенков зеленого цвета. Она отлично сочетается с цветом глаз моей подруги. Над головой раскинулось небо, ясное как поэзия пятиклассника или первые песни Битлз. Никакого намека на тучи.

-- Если не книги, то что? -- спрашивает подруга. -- Что ты теперь будешь делать?

Я вдруг вспоминаю, что с утра так ничего и не выпила и отвечаю то же, что и всегда:

-- Кофе.

***

Я переезжала в Крым. Не потому, что любила море и почти научилась игнорировать врожденное отвращение к жаре. Я переезжала к Вениамину просто потому, что этому сумасшедшему алкоголику каким-то образом удалось пробраться в мое сердце. Когда я думала о нем, мою душу переполняли самые теплые чувства. Чувства, о которых я и не подозревала, что все еще могу их испытывать.

Так началась наша прощальная тусовка с Мартой.

В начале вечера мы сидели в моей старой спальне на шестом этаже. Стены в комнате были черные, а из колонок доносился рок-н-ролл. Мы сидели около распахнутого окна, на огромной кровати с кованой спинкой.

Стоял поздний вечер, но солнце лишь начинало садиться, как это всегда бывает летом. Все тот же индустриальный вид из окна. Серые здания, грязные улицы, одинокие люди и хмурый проспект, что заканчивался разбитым зданием вокзала, оставленным здесь задолго до распада СССР. В общем, бетонная зараза во всей красе. А над ней томно раскинулся закат экспрессиониста. Слишком насыщенный для такого хмурого пейзажа.

Тогда я впервые осознала, что я как тот закат -- принадлежу этому месту. Я ненавидела эту рутину и всю жизнь искала возможность уехать из Николаева. Убежать куда глаза глядят. Но на самом деле нуждалась в этом паршивом городишке как никто другой. Печаль тоска и безысходность уже давно стали неотъемлемой частью меня, и я не была уверена, что до сих пор чего-то стою без них.

-- Мне нужно быть в дерьме, чтобы писать свои книги, -- заявила я.

К тому моменту солнце уже почти скрылось за крышами бесконечных девятиэтажек, а мы с Мартой заканчивали пятилитровую бутыль домашнего вина, заботливо изготовленного отцом.

Подруга смотрела на меня с недоверием.

-- Так это ты что же? Я не понимаю. Ты просто вдруг взяла и передумала ехать? -- наконец, спросила она.

-- Я просто не могу уехать, понимаешь?

Я чувствовала, что нужно открывать вторую бутылку.

-- Извини, конечно, но я вижу это совсем по-другому, -- продолжала подруга.

Я попыталась перебить ее, но это не сработало.

-- Я вижу совсем другое, -- продолжала Марта. На ее губах заиграла улыбка. -- Наша бесстрашная Лизавета испугалась серьезных отношений.

-- Ерунда. Я ничего не боюсь.

-- Кому ты пиздишь, Лиза? Ты только пару месяцев как вышла из многолетней депрессии, а тут встретила Веню и решила пока не поздно войти обратно. Ты так долго была в депрессии, что теперь это твоя зона комфорта.

-- Нет, вот здесь ты не права, -- я отмахнулась.

Конечно же, она была права.

-- Слушай, ты сожгла свои книги, чтобы избавиться от прошлого. Ну, это по твоим словам. А сейчас ты говоришь, что никуда не поедешь для того, чтобы написать новые книги. Ты сколько уже за сегодня выпила, женщина?

Мы рассмеялись и выпили еще по бокальчику.

-- Ладно, -- начала я.

И постаралась убрать улыбку со своего лица. Не так-то просто после количество распитого алкоголя.

-- Дело здесь не совсем в книгах\ldots{} Точнее, оно совсем не в книгах. Просто у меня пару раз было такое нехорошее предчувствие на его счет. Ты же знаешь, я не верю во всю эту херню. Просто у меня хорошо развита интуиция и все такое\ldots{}

-- Предчувствие!

Я уже знала, что Марта ответит. То же, что отвечали все близкие мне люди. Предчувствие -- это как когда тебе казалось, что Мишель тебя ненавидит и очень активно обсуждает это с другой твоей подругой? Предчувствие -- это как когда ты решила, что та другая подруга врет через слово? Или предчувствие -- это как когда ты целый год была уверена, что она пытается переспать с Адамом? Смешно, но, в конечном счете, все эти подозрения оказались правдой. Само собой, тогда я этого не знала, и ожидала, что Марта, как и все остальные, назовет мою интуицию всего лишь бурной фантазией. Но она лишь сказала:

-- М-да уж, подруга, предчувствие -- это дерьмово. И что же оно тебе говорит?

Я вздохнула, обдумывая ответ. А затем допила винишко и выпалила на одном дыхании:

-- Ну, не знаю. Может быть, что у нас все будет идеально, а затем однажды один из нас не проснется, а второй будет в панике бегать вокруг окровавленного тела своей второй половинки и даже не вспомнит, что произошло.

-- И под этим бегающим кем-то ты подразумеваешь не себя?

-- Ну, в общем да. Не себя.

Здесь наступила неловкая пауза. Я почувствовала, что начинаю краснеть и сказала:

-- Слушай, проблема в том, что все называют мою интуицию сумасшествием. То есть, вот когда тридцатилетний бугай работает охранником ночной смены в «Штиле» и он рассказывает, что на палубе корабля-ресторана бродит призрак\ldots{} Что шары для боулинга сами себя бросают, что за столиками включается радио и что кнопка вызова официанта каждую ночь мигает, а на экране показываются номера столиков, которые находятся на той палубе, где ходит призрак -- вот это нормально. И когда взрослая женщина носит на запястье красную нитку с булавкой потому, что боится сглаза -- это тоже ок? Или когда та вторая моя подруга рассказывает, что видела призрака в двери за моей спиной, когда мы говорили в скайпе -- это ничего! А вот если я предчувствую, что кто-то поступит со мной плохо -- так я сразу сумасшедшая, понимаешь? И ведь в итоге всегда оказывается, что я права! Помнишь, как я узнала, что Эрик трахается со своей бывшей в туалете клуба? Вот это и правда было для всех шоком. И меня это пугает.

-- И что говорит твоя интуиция? Вениамин тебе навредит или как это должно произойти?

Я кивнула.

-- Ну, не знаю, Лиза. Вот у нас с Семеном все хорошо. Живем уже второй год, и никто никого не убил.

-- Это вдохновляет.

-- Что тебя вдохновляет? -- Марта взглянула на меня как на ребенка с бурной фантазией. -- Понапридумывала себе херни какой-то и паришься на пустом месте! Вот мы с Семеном очень счастливы.

-- А что такое счастье?

-- Счастье -- это найти свою вторую половинку, войти с ней в симбиоз, употреблять вместе наркотики и путешествовать! -- речитативом выдала подруга.

Мне всегда думалось, что симбиоз для неполноценных личностей. Подобная мартиной формулировка была у меня лет в четырнадцать, но критиковать чужое счастье не хотелось, поэтому я ответила:

-- А ты никогда не думала, что не для всех счастье такое, как для тебя?

-- В плане?

-- Ну, например, кто-то может быть слишком целостным для второй половинки? Что кому-то может быть нужна такая же полноценная личность, как и он сам? Что отношения -- это не предел счастья? Что кому-то может хотеться большего. Или даже совершенно другого. Что симбиоз подавляет личность и не дает человеку реализовать свой потенциал. Вообще отводит от цели, мешает выполнить свое призвание.

-- Так и чего же ты на самом деле хочешь? -- с улыбкой спрашивает Марта.

-- Мир, -- отвечаю я. -- Я хочу перевернуть мир!

-- И ты готова пожертвовать ради этого своим счастьем?

-- У нас разные представления о счастье, чувак. Я ночами не сплю из-за этого мира. Уж слишком долго он вертится не в ту сторону, а я чувствую всю эту боль. Ты хотя бы понимаешь, что они сделали с планетой? Как они относятся к животным? Как они относятся друг к другу, в конце концов. Человечество впало в спячку, так больше не может продолжаться! Кто-то должен его разбудить!

Но Марта только смотрит на меня все с той же улыбкой, выдыхает дым и передает мне люфу со словами:

-- Ну, не знаю, о чем тут можно париться. Нам с Семеном пофиг на проблемы мира. По-моему, ты просто боишься серьезных отношений.

\hypertarget{chapter-47}{%
\chapter{~}\label{chapter-47}}

И вот я снова в дороге. Сижу на переднем пассажирском в автомобиле своего отца и смотрю на прерывистую разделительную полосу. Машина движется быстро, отчего полоса становится сплошной, и вскоре от этой картины начинает мутить. Я вздыхаю и отворачиваюсь к лобовому стеклу.

Мы выехали всего час назад и сейчас движемся по трассе Николаев-Херсон. Знаете, там есть небольшой съезд с видом на широкие поля и высокие деревья, густо высаженные вдоль дороги. Пожалуй, так можно описать большинство придорожных пейзажей моего края, но в этом отрезке пути есть кое-что особенное -- идеально ровная дорога. Глянцевая поверхность без выбоин, мусора и ухабов. Внезапно схватив приступ амнезии, можно подумать, что ты находишься где-нибудь в Европе.

Само собой, длится все это счастье от силы минут десять. Затем ты возвращаешься на землю и опять едешь по пути, напоминавшему поле боя.

-- Ты в порядке? -- спрашивает мама.

Она тоже провожает меня до границы. Мама расположилась на заднем сидении. Старшая из моих собак спит, положив голову ей на коленки. Младшая осталась дома. Тогда Еве было чуть больше года, и она вела себя тревожно во время хоть сколько-нибудь длительных поездок. Ну, а Тая истерически не хотела меня отпускать, а потому теперь едет с нами провожающей.

-- Все нормально, -- отвечаю я, хотя на самом деле это не так.

Мы здорово разругались этим утром. Не то, чтобы мои родители были против Вениамина\ldots{} Они просто не хотели, чтобы я принимала поспешные решения в одиночестве. Тем не менее, если речь шла о любовных делах, именно так я всегда и поступала. Никто не смог бы отговорить меня от решения поехать к Вене, как, в общем-то, и в свое время никто не смог меня отговорить от поездки к Адаму.

Я это знала. Они это знали. Так что, мы просто спорили какое-то время, но вскоре аргументы пошли по кругу, и разговор пришлось замять.

-- Ты просто не представляешь, как это сложно, -- начала мама.

-- Ничего не сложно, -- я подвинулась к окну и подкурила сигарету. -- Я всегда смогу развестись, если что-то пойдет не так.

-- Ты мне, кстати, то же самое про курение говорила! И про татуировки! -- продолжала мама. -- Да если бы все было так просто! Если что-то пойдет не так, ты долго не захочешь выходить замуж. Опять замкнешься в себе и только потеряешь время! А так могла бы встретить достойного человека\ldots{}

-- Но я никогда и не хотела выходить замуж!

-- Так на кой черт ты делаешь это сейчас? -- вмешался папа.

-- Потому, что люблю его, -- ответила я.

Отец фыркнул. Он всегда был рациональным человеком Скажи я ему, что залетела, это был бы куда более веский аргумент.

-- Ты просто не представляешь, как это сложно, -- повторила мама, а затем добавила: -- Жить с кем-то после того как ты уже пожил с кем-то. Это очень трудно.

Отец бросил в ее сторону удивленный взгляд, но промолчал. Он был вторым человеком, с которым развелась Ма.

Хотелось мне этого или нет, но большую часть жизни я была одинока. Конечно, я знала людей, которые начинали жить вместе еще в старшей школе, но никогда не понимала, на кой черт им это нужно. Наверное, я просто слишком ценила свое личное пространство. Несколько недель -- таким был максимум моего совместного проживания.

-- Это что, по-твоему, труднее, чем полюбить кого-то после того как ты уже любил кого-то? -- спросила я.

Мама призадумалась, а затем растерянно покачала головой.

-- Наверное, где-то так же.

-- Ну, в таком случае, я справлюсь.

***

Корпоративная тарахтайка Вениамина окончательно сломалась, так что от границы мне предстояло вновь добираться на автобусах. Морально, я уже начинала себя подготавливать ко всем неприятностям, которые с вероятностью в девяносто процентов обещали случиться в дороге, когда вдруг водитель с работы моего жениха предложил свои услуги. Это произошло в последний момент и было как никогда кстати. Конечно, мне все равно предстояло как-то добраться въезда на Симферополь и стартовать в одиночку нужно было еще в Армянске, но я решила, что это, все же, лучше, чем ничего.

Тем утром на трассе стоял неописуемый зной, а на границе образовалась гигантская пробка. За часы, проведенные в очереди под палящим солнцем, я уже раз триста успела пожалеть о том, что ношу волосы по пояс и, клянусь, будь у меня в тот момент ножницы, я бы обязательно ими воспользовалась. Но проносить подобные острые предметы на территорию блокпостов запрещалось, так что мне и моей шевелюре предстояло страдать дальше.

-- Вы впервые пересекаете эту границу? -- спросил таможенник, когда я, наконец, приблизилась к концу последней очереди.

Это был высокий мужчина с большими карими глазами и такой широкой, нетипичной для пограничников улыбкой.

-- Нет, -- ответила я. -- За последний месяц это как минимум мой третий визит.

-- А до этого когда в последний раз границу проходили?

-- С Россией?

-- Ну да.

-- Году в тринадцатом, -- сказала я. -- Только это было на въезде в Москву.

Пограничник развел руками и принялся просматривать какие-то документы на мониторе.

-- Прикольно, -- спустя пару минут раздумья заявил он.

-- Что именно?

Но мужчина уже принялся громко звать коллегу, что занимался проверкой багажа.

-- КОЛЬКА! -- кричал он. -- Давай быстрей сюда, Колька! Гляди че тут показывают!

Вскоре примчался Колька годами двадцатью старше первого пограничника. Он посмотрел на монитор, затем в мой паспорт, на меня и опять на монитор.

-- Вот это прикол, -- выдал Колька.

-- Прикол, точно!

-- Да-а-а прикол не то слово!

-- Да что там такое?! -- не выдержала я.

-- Ничего-ничего, -- ответил первый пограничник.

Колька тем временем удалился. Я не унималась.

-- Нет, серьезно, что там? Вы знаете, у меня сегодня был очень тяжелый день, а ведь сейчас только без пятнадцати три. Короче, парочка приколов мне точно бы не помешала, так что\ldots{} Ну что там такое то?

Мужчина продолжал улыбаться, чего нельзя было сказать о собравшейся за моей спиной очереди.

-- Так вы говорите, что проходили эту границу дважды?

-- Да, этой весной.

-- А до этого, когда вы в последний раз здесь были?

-- Месяц назад.

-- Нет, это не то.

В этот момент во мне проснулись какие-то утерянные украинские гены. Резко захотелось сала, опрокинуть пару стопок да пуститься в танец, послав окружающих в лучших местных традициях.

-- А что -- то?

-- Вы до этого когда в последний раз в Крыму были?

-- Когда я была здесь до этого, никакой границы и в помине не было. У власти не стоял человек с явной умственной заторможенностью, который разговаривает как Рокки Бальбоа в худшие годы своей жизни. А размышляет и того хуже, так что Крым все еще был частью Украины.

Стоит признать, что прозвучало это совсем не так устрашающе, как мне того хотелось. К тому же, я никогда не умела произносить букву «Р», и такие слова как Крым или Рокки в моем исполнении получились комичными.

-- Ладно вам, девушка, -- выдохнул таможенник.

Кажется, он пытался не рассмеяться.

-- Глядите сюда, -- он повернул ко мне монитор.

Я взглянула на экран. Там было мое старое фото, сделанное в шестнадцать лет. Все как обычно: макияж, легкая ухмылка, какая-то темная одежда. И только волосы окрашены в иссине черный с внезапной прядью блонда впереди.

-- А ведь когда-то мне очень нравилась эта прическа, -- сказала я. -- Я до сих пор иногда подумываю ее вернуть\ldots{}

-- Смотрите ниже.

И я посмотрела. Под фото в графе имя и пол было указано\ldots{}

-- Это\ldots{} Это что еще такое?

Прямиком под моей подростковой фотографией значилось, что пол мужской, а ниже:

Васляев Елисей Валерьевич.

-- Да вы не переживайте так! -- понизив голос, произнес пограничник. -- Кажись, это при передаче документов ошибка произошла. Вы ж не думаете, что после отделения Крыма тут все тихо и спокойно было? Тут такой базар был, вот вас случайно и записали как Елисея. А мы с Колькой потом еще долго на это фото смотрели и думали, какой же привлекательный трансвестит попался. Да Колька?

-- Так точно! -- подтвердил Колька из-за соседней стойки.

Я как-то смутилась.

-- И эта прическа вам очень шла!

-- И правда, шла\ldots{} -- протянула я. -- Ну\ldots{} Я пойду?

На этой прекрасной ноте моя каторга на украино-российской границе завершилась, но почему-то все так же хотелось сала. Кстати говоря, до того дня я его на дух не переносила.

***

Вместо положенного времени, на границе я провела целых четыре часа, благополучно пропустив свой автобус. Опоздала и на следующий, и даже на тот, что был после него. Питер, водитель с работы Вениамина, час назад должен был оказаться в Севастополе.

-- Но я уговорил его тебя дождаться, -- по телефону сообщил Веня. -- Я буду ждать тебя на заправке, которая сразу\ldots{}

На этом мой мобильник сдох.

Долго не думая, я взяла такси до Симферополя и всю поездку пыталась сообразить, на какой именно заправке мне нужно быть.

-- Так, где конкретно вас высадить? -- спросил таксист.

-- На заправке, -- сказала я.

-- На какой?

-- Ну\ldots{} Я все еще работаю над этим вопросом.

\emph{Итак, нам нужна заправка.}

\emph{Которая из заправок?}

\emph{Заправка у вокзала? Заправка на выезде? На въезде? В центре? Обычная заправка? Заправка с хипстерами? Заправка с\ldots{}}

И тут я вспомнила! Однажды жених жаловался, что ему пришлось сорок минут прождать Питера потому, что заправщик отказался обслуживать его машину. А отказался он потому, что у Питера не было с собой никаких документов кроме водительских прав. Насколько я помнила, на заправках никто не требовал показать документы на машину, разве только когда\ldots{}

-- А сколько в Симферополе станций, которые заправляют машины маслом?

-- Маслом? -- с недоверием спросил водитель.

-- Ну да, или чем-то вроде того. Вместо бензина. Они еще обычно отказываются заправлять тачку, если у тебя нет на нее документов.

В свои двадцать два я сидела за рулем всего один раз. Мне было тринадцать и без травм не обошлось, так что за прошедшие девять лет я больше не посягала на место водителя и мало что знала об автомобилях. Удивительно, но таксист меня понял.

-- Это, наверное, ВОГ, -- ответил он.

К двух тысячи шестнадцатому году в Украине было более четырех сотен АЗС с названием WOG, но я решила пока не опускать руки.

-- А в Симферополе их сколько?

-- Да дохрена, -- ответил водитель, а затем добавил: -- Было. Понастроилась целая куча этих ваших ВОГ-ов. Куда не едешь, везде ВОГ и эти, как их там, вокруг толпятся. Они даже без машин, вы представляете! И приходят на заправку как в МакДональдз! А потом берут хот-доги с кофе, еще выбирают стоят, с пенкой без пенки. Я спешу, а они спрашивают какой процент жирности того молока и почему нет соевого. Кокосового им значит мало! А потом еще сидят прямиком на парапете, что и проехать заправиться невозможно. Как их там? Ну эти, как их там, латентные! Мальчики с наколками в огромных очках, клетчатых рубашках и старомодных шляпах с полями. На вид как голубые.

\emph{А на вкус как?}

-- В наше время их принято называть хипстерами, -- заметила я.

-- А еще бывают девочки. Тоже в наколках. И выглядят они как наркоманки\ldots{} Тьху!

При этих словах таксист буквально сплюнул через плечо, и я была рада тому, что, вопреки своей привычке сидеть впереди, решила расположиться на заднем пассажирском. Вместе со своими наколками.

-- Так сколько таких заправок сейчас в Симферополе, вы не знаете?

-- Так ведь их убрали уже!

-- Убрали?!

-- Убрали. Остались только на Объездной да на Севастопольской. Там они теперь и собираются.

Водила звучал так, словно говорил не о подростках, а о собраниях НСДАП.

-- А нам это по пути? -- нетерпеливо спросила я.

Водитель многозначительно развел руками, что было не очень уместно, потому как мы как раз влились в плотный поток машин, образовавшийся перед городом.

-- Не знаю, -- ответил таксист. -- Вы же так и не сказали, куда мы вообще едем.

-- Значит, едем на Севастопольскую, -- решила я.

***

ВОГ на Севастопольской мало чем отличался от других АЗС с таким же названием. Правда, ввиду локации, людей здесь было значительно меньше. В общем, создавалось впечатление, что скоро в Крыму станет еще одной заправкой меньше. В паре метров от входа лежала груда каких бесхозных кирпичей. Рядом стоял согнутый пополам дорожный знак и никаких тебе автомобилей.

-- Кажется, это не то место, -- с грустью произнесла я.

-- Да, здесь приличной девушке совсем делать нечего, -- заметил таксист.

Он уже повернул ключ зажигания и начал отъезжать.

-- А вот еще один остался, -- вдруг сказал мужчина. -- Гляньте-ка на него! Сидит как ни в чем не бывало!

Я машинально подняла глаза, и увидела своего жениха, одиноко сидящего за заправкой на очередной куче камней. Вроде той, что лежала у основного входа. На Вене была его любимая клетчатая рубашка, повязанная вокруг пояса, майка, открывающая отличный вид на татуированные мышцы моего мужчины, те самые неубиваемые ботинки, с которыми я уже давно смирилась, его огромные очки и, конечно же, шляпа. Словом, все то, что так смущало моего водителя.

-- Венька! -- радостно крикнула я, так что таксист аж оторопел.

-- Это ваш что ли? -- удивленно спросил он.

-- Мой! -- все так же не скрывая радость, ответила я, расплатилась с таксистом и вылетела из машины.

Вениамин сидел в наушниках, так что я подкралась к нему сзади и игриво прикрыла глаза жениха ладонями. Не снимая наушников, Веня нащупал свое кольцо на безымянном пальце моей правой руки и воскликнул:

-- Деточка!

Он тут же заключил меня в объятья, так что нам понадобилось еще минут десять, чтобы суметь оторваться друг от друга.

-- Я уж думал, ты решила меня бросить!

Счастливая улыбка не сходила с его лица, но по глазам Вениамина я видела, что тот говорит правду.

-- Ну, что ты, Венька, -- я вновь поцеловала его.

-- Ну а что? Все-таки, два месяца почти прошло, а ты до сих пор со своей старой фамилией.

Я осмотрелась по сторонам. Кроме нас на заправке никого не было, так что как настоящий мастер менять тему, я поинтересовалась:

-- А где Питер?

-- Уехал за новым товаром, но обещал вернуться, если ты приедешь.

-- Ну, вот я здесь.

И я улыбнулась, в надежде, что мой жених не захочет продолжать свою тему. Чувствовала себя слишком уставшей для подобных вопросов. По крайней мере, сегодня.

Но Веня не продолжил. Он лишь улыбнулся мне в ответ со словами:

-- Я вижу.

Затем подхватил меня на руки и прижал к себе.

Питер приехал спустя минут двадцать. К тому моменту мы оба были слишком уставшими для вежливой болтовни, так что Веня скоренько закинул мои пожитки в багажник, и мы упали на заднее сиденье. В салоне было душно и жутко хотелось спать. Я прилегла на плечо Вениамина, и мгновенно ощутила, что в кои-то веки нахожусь именно там, где мне и нужно быть. Я была измучена бессонницей, дорогой, прощаниями, жарой и творческим кризисом, финансовыми трудностями и постоянными предательствами, но вот я вдохнула родной запах рубашки своего мужчины, и все эти вещи вдруг оказались такими далекими.

Из окна автомобиля на нас смотрели величественные горы. Куда ни глянь, на фоне ясного синего неба виднелись их пастельные очертания, а где-то там, впереди, нас уже ждало море. Со всеми его изящными изгибами, игривыми волнами и лазурными берегами.

Таинственное, прекрасное море, каким оно бывает только в Севастополе.

\hypertarget{chapter-48}{%
\chapter{~}\label{chapter-48}}

У каждого автора есть такой «блокнот в винных пятнах». Ты постоянно носишь его с собой и практически никогда не перечитываешь, но если что и записываешь -- всегда в точку. Нет, я не хочу сказать, что писательству нужен алкоголь, как в случае с Буковски, Ремарком, Томпсоном, По, Фитцджеральдом, Хемингуэем, Керуаком, Джойс, Есениным, или даже О.Генри с его извечно позитивным настроем.

В общем, список можно продолжать бесконечно. И, хотя факты говорят об обратном, я никогда не считала, что алкоголь -- обязательная часть писательства. Неважно чем именно был перепачкан блокнот, -- хоть крошками протеинового батончика -- он непременно есть у каждого, кто однажды решил связать свою жизнь с самой требовательной женщиной в мире -- литературой. Свой блокнот я забросила еще в средине лета. Чего никак нельзя было сказать об алкоголе.

-- Ты интересно пишешь. Мне нравится, -- однажды сказал Веня.

И вроде как, это должно было меня порадовать. Увы, он говорил о всего лишь о наших письмах, не о моих книгах.

-- Мне стыдно за свое скудоумие, -- добавил он, как бы извиняясь за то, что до сих пор не знаком с творчеством женщины, на которой решил жениться. -- Я читаю только фантастику.

Не знаю, что меня удивило больше: фантастика, или то, что Веня впервые употребил в разговоре такое слово, как «скудоумие», но я только спросила:

-- Зачем же себя так ограничивать?

-- Беда в том, что больше меня ничего не затягивает.

И, все-таки, кое-что он прочитал. Я узнала об этом, когда вторая неделя моего возвращения подходила к концу. Мы как раз заканчивали ужинать на балконе, и речь зашла об этом.

-- И как тебе?

-- Очень странный этот твой жанр, трагикомедия, -- признался Вениамин. -- Я смог прочитать только третью часть книги, но\ldots{}

-- Но?

-- Я просто не понимаю, как человеку может быть весело, когда его жизнь так печальна? Ты грустишь, когда все плохо и веселишься, когда все хорошо. Не думаю, что в реальной жизни бывает по-другому\ldots{} Хочешь еще картошечки?

-- Нет. Но так ведь в этом то и весь смысл!

-- В чем?

-- В том, чтобы продолжать шутить, когда тебе дерьмово.

-- Разве кто-то так на самом деле делает?

-- Ну, я делаю. И чем мне хуже, тем мне лучше.

Вениамин завис на минутку, после чего повторил:

-- Очень странный жанр.

И он принялся старательно подкладывать еду в мою тарелку.

-- Знаешь, -- не отступала я, -- если бы в реальной жизни все было так, как ты говоришь, меня бы скорей всего здесь не было.

-- Ты бы не рискнула приехать к незнакомому мужчине?

-- Нет, этого у меня уж никак не отнять. Скорей всего, сейчас я бы пыталась соорудить заточку из антистрессовой головоломки. Где-нибудь на Володарского. Или покоилась бы с миром на старом городском кладбище, но это вряд ли потому как захоронения на нем сейчас проводят только в крайних случаях, да и я предпочитаю кремацию.

-- Ну, ты -- это другое, -- не отрываясь от ужина, ответил Веня.

Спокойствие в голосе этого человека было одной из многих причин, по которым я так его любила. Ну, какой еще мужчина станет нормально относиться к упоминанию трупиков за столом?

-- Другое?

-- Ага. С тобой законы физики не работают.

-- Ах, если бы только физики, -- ответила я и подлила себе еще винишка.

Проблема состояла в том, что я практически никогда не пьянела. Алкоголь, как и большинство наркотиков, едва ли мог на меня повлиять. Это была самая дерьмовая суперспособность из всех, что только можно было придумать. Ведь трезвость так ужасна! Для меня же единственный способ избежать ее включал в себя систематическое принятие спиртного (примерно каждые десять минут) и постоянное экспериментирование с крепостью выпитого.

-- Как, черт возьми, тебе удается бухать круглые сутки и так хорошо выглядеть? -- однажды спросила меня Дана, старая подруга из Казани. -- Я выпиваю бокал шампанского, и лицо уже куда-то плывет.

-- Весь секрет в том, что я практически не пьянею, так что мне приходится подливать себе что-то каждые пятнадцать минут. Если я хочу соответствовать окружающей меня публике, конечно.

-- И все что ли? Вот он твой великий совет: больше пить?

-- А еще от алкоголя у меня опухают только губы. От этого я кажусь симпатичней.

На самом деле, я слишком часто слышу этот вопрос, дабы знать на него правильный ответ. Видимо, таков вот мой бесполезный талант номер двадцать три. Прекрасная почва для алкоголизма.

-- А этот пленительный блеск в глазах? -- с недоверием спросила Дана. -- Еще скажи, что и тут алкоголь виноват!

-- О, нет! Блеск в глазах -- это другое.

Я выставила указательный палец и с серьезным видом постучала им в районе виска.

-- Попрошу не путать! Всему виной проблемы с кукуней. У психов всегда глаза блестят, ты разве не замечала?

***

Одним июльским утром я проснулась в подавленном настроении. Это было как раз то самое мерзкое состояние, когда ты сам не знаешь, чего хочешь, и при этом ощущаешь острую нехватку чего бы там ни было. В такие моменты мне сразу вспоминалась одна из тех странных сказок, что папа читал мне в глубоком детстве. Называлась она «Поди туда, не знаю куда». В ходе сюжета лирическому герою было велено было пойти туда -- не знаю, куда и принести то -- не знаю, что.

Именно это мне и хотелось ответить Вениамину, который настойчиво интересовался, чем он может мне помочь.

-- Давай найдем какой-нибудь уютный паб? -- предложила я. -- В Севастополе ведь еще остались заведения, в которых тебе разрешено появляться?

-- Да целая куча!

-- А питейные заведения?

-- Ну, таких поменьше будет, -- признался Вениамин. -- Но они все равно есть.

-- Тогда, вперед! - решила я.

Тем не менее, в тот день мы так ничего и не нашли. Уже с приближением вечера мы бродили крохотной лесопосадкой, что тянулась вдоль дороги. Здесь пахло хвоей и прохладой, а земля была усеяна шишками, желудями и мхом. Если в правильный момент закрыть глаза, создавалось ощущение, что ты находишься на какой-нибудь лесной опушке. Уши, кстати говоря, тоже стоило бы закрыть. В паре метров от нас была находилась трасса, и даже густая стена деревьев не была в силах скрыть ее гул.

Насколько я помнила, любовь настигла меня где-то здесь, на выходе из импровизированной городской чащи. Выходя из дома тем прекрасным майским утром, я всего лишь хотела позавтракать в КФЦ, а вместо этого обнаружила себя по уши встрескавшейся в мужчину, которого я едва знала и который, накануне сделал мне предложение.

Что ни говори, а люди, воюющие с фастфудом, видимо, не так уж и ошибаются.

-- О чем это ты там задумалась? -- поинтересовался Вениамин.

Он курил, опершись на ствол какого-то старого дерева.

-- Возможно, мне и правда нужно перейти на правильное питание, -- ответила я.

-- А я бы сейчас скушал пяток бургеров\ldots{}

-- Пяток?

-- И это далеко не мой рекорд, -- довольно заявил Веня. -- Как-то я прикончил двадцать девять бургеров!

Я остановилась и не без подозрения взглянула на своего спутника.

-- Ты умял двадцать девять бургеров?!

-- Ага, -- Вениамин похлопал костяшками пальцев по прессу, ради которого он и пальцем не шелохнул.

Я всегда считала, что жизнь слишком снисходительная к мужчинам.

-- И что, ты был настолько голоден?

-- Нет, я наелся еще после четвертого. Остальные я съел на спор.

-- Понятно, -- протянула я, пытаясь прикинуть, сможет ли Веня когда-нибудь сам для себя готовить. -- И на что вы спорили?

-- На бургер.

-- То есть ты заплатил за двадцать девять бургеров и съел их просто ради того, чтобы получить еще один?

-- Ну да, тридцатый, -- как ни в чем не бывало ответил мой жених.

В итоге я решила, что нет, готовить для себя Веня точно не способен. Разве что яичницу из дюжины яиц на завтрак.

-- А вот и КФЦ! -- объявил Вениамин, когда мох под нашими ногами сменился асфальтом.

Знаете, есть много романов с вполне посредственной завязкой. Конечно, далеко не все персонажи любовных историй встречаются на сверхважных государственных заданиях или пытаясь укрыться от ядерного взрыва. Однако ни тем прекрасным майским днем, ни, собственно, любым другим днем своей жизни, я никак не думала, что наша история любви начнется в отделении ближайшего фастфуда.

Поразительно.

***

Как я уже сказала, в жизни каждого автора бывают такие дни, когда история не пишется и тебе буквально приходится выдавливать из себя каждое слово. Подобным образом я провела около двух недель. Я была счастлива с Вениамином, но писать у меня, почему-то, не получалось. О том, что эти вещи могут быть связаны, я как-то сразу не подумала.

\emph{Печаль -- хлеб поэта}, напомнил мне внутренний голос.

А ведь так оно и было! Все свои лучшие строки -- все то, что имело для меня хоть какую-то художественную ценность -- я писала в состоянии глубочайшей депрессии, одиночества и эмоциональной подавленности. Большую часть взрослой жизни я провела подобным образом. Стоило мне впасть в отчаянье, и слова лились из меня так, словно они уже где-то существовали. В такие моменты я забывала обо всем на свете и не отрывалась от писанины до тех пор, пока чувствовала, что мне есть что сказать.

-- Тебе не кажется, что это как-то нечестно? -- спросила Марта, когда я поделилась с ней причинами своего творческого кризиса. -- Да и не кризис это вовсе. Просто творческий застой.

-- Да, какая собственно разница? Творческий кризис, творческий застой, творческий ступор\ldots{} Это все писательский блок как его не назови!

-- Какая разница? Спроси у Гёте, Сэлинджера или Флобера! Такое со всеми бывает.

-- Если ты думаешь, что от перечня известных авторов, которые десятилетиями писали свои хиты, мне станет как-то легче\ldots{} -- начала я. -- А хотя знаешь, мне и правда стало легче.

Марта рассмеялась, и мы еще какое-то время болтали о пустяках. Существует бездонный список вещей, о которых можно вести преинтереснейшие разговоры, но работает это лишь в компании близкого друга.

Увы, наш разговор не мог длиться вечно. Марта ушла развлекать своего мужа, а я вновь вернулась к мукам творчества. Гёте, Сэлинджер, Флобер, Толкин, Манн, а также мой наркозависимый однофамилец -- все они давно были мертвы. Отстрадали свое и уже успели стать полноценной частью истории.

-- Вот почему просто нельзя как-то обойти всю эту систему? -- угрюмо размышляла я, разрывая очередной лист бумаги и швыряя его останки в другой конец комнаты. -- Почему просто нельзя перейти к тому моменту, где я уже умерла, а мои книги приобрели литературную ценность?

Еще одна рукописная страница отправилась вслед за своей предшественницей. Я сидела и пялилась в невидимую точку посреди своего новенького неразлинованного блокнота, добрая часть страниц которого уже заполняла пол кухни. Не знаю, сколько часов я провела подобным образом, кто их вообще считает? Но в определенный момент мне стало казаться, что я схожу с ума. Мозг буквально спекся от перенапряжения. Следом за этим появилась острая головная боль.

Все эти классики, имеющие явные проблемы со скоростью написания своих лучших творений, по-прежнему были мертвы и известны, а я все сидела у подоконника и сверлила взглядом очередной белоснежный лист. За окном уже проступили поздние сумерки, но я и не думала двигаться с места. Головная боль, по традиции, переросла в самую настоящую мигрень. С наступлением темноты мысли у меня перепутались настолько, что я уже не понимала, какие из них мои.

\emph{Я сижу у окна. За окном осина}, произнес Бродский своим монотонным голосом.

Это случалось со мной с самого детства. Стоило моему котелку перегреться, или если я просто не спала по несколько суток к ряду, подсознание начинало выдавать рандомную поэзию. Чаще всего это был Пастернак, Бродский, Маяковский или Буковски, но случалось и что-нибудь из средневековья, а то и из античной литературы. Порой я забывалась и озвучивала эти строки.

\emph{Я любил немногих. Однако -- сильно.}

Помнится, на третьем курсе я пришла в университет после ночной смены в техподдержке и что-то не поделила с главой кафедры филологии. Она сказала мне:

-- Васляева, я по-вашему на дуру похожа?!

А я ответила:

-- Неужто зреньем бог меня обидел, чтоб я на небе солнца не увидел?

Так вот, когда Вениамин появился в свете фонарей, я продолжала сидеть у окна, пристально глядя на чистый лист. Наверное, надеялась, что, в конечном счете, он воспламенится, и, как это произошло с кольцом всевластия, из огня появятся какой-нибудь идеально выведенный текст.

Увы, подобного не случилось.

Тем летом я толком ничего и не написала.

\hypertarget{chapter-49}{%
\chapter{~}\label{chapter-49}}

Перенесемся в тот томный летний вечер, когда Вениамин вновь сделал мне предложение. Мы как раз закончили бродить по торговому центру. Единственным, что там купили, были билеты на вечерний сеанс новой ленты Джеймса Вана. По крайней мере, так я думала.

-- Дай мне свою руку, -- загадочно произнес Веня.

В этот момент мы спускались на эскалаторе. Стоял поздний час и кроме нас здесь уже никого не было.

-- Ты ведь не собираешься делать мне предложение в торговом центре? -- я улыбнулась.

-- Нет, -- как ни в чем не бывало ответил жених.

Затем он взял мою руку в свои ладони и поцеловал меня. Только когда наши губы разомкнулись, я увидела новенькое кольцо на безымянном пальце своей правой руки.

-- Я дождусь, когда мы выйдем на улицу, и сделаю предложение там.

Тем вечером в воздухе царила праздничная атмосфера. На парковке почему-то звучали рождественские песни, но это лишь добавляло уюта. На мне было длинное черное платье с сумасшедшим декольте и босоножки на высоченном каблуке, что делали нас с Веней одного роста. Платье было чужим, я одолжила его у Вениной сестры, а вот трепет в выглядывающей из него груди определенно принадлежал мне.

Совсем недавно меня пугала даже сама мысль о том, чтобы добровольно связать себя узами брака. Теперь же мои щеки пылали, а сердце билось чаще, чем когда-либо. Я мельком поглядывала на державшего меня за руку мужчину, а в голове эхом отдавалась лишь одна мысль.

Иногда я все еще вижу нас, идущих через парковку навстречу теплой июльской ночи. Поблизости нет ни души, и только Синатра поет свои бессмертные песни.

\emph{Он собирается сделать мне предложение. Он собирается сделать мне предложение. Он собирается сделать мне предложение.}

Не знаю, сколько десятков раз эта фраза успела прозвучать у меня голове прежде, чем мы миновали парковку. Теперь вокруг не было ничего кроме звезд да деревьев, чьи ветви успокаивающе шелестели от легких прикосновений морского бриза.

Следуя за своим спутником, я сделала еще несколько шагов в темноту и вдруг оказалась в небольшом китайском садике, умело скрытом от чужих глаз живой изгородью. Взглянула на плавающие на воде кувшинки, на Луну, отражающуюся в стеклянной поверхности декоративного озера, на все эти каменные сооружение и растительность, украшенную неоновыми огоньками. Открыла рот и тут же закрыла его обратно, вдруг осознав, что в кои-то веки я просто не знаю, что сказать.

-- Вот мы и на месте, -- тихо произнес Вениамин.

Его ладони были влажными, и я поняла, что Веня нервничает не меньше моего. Он взглянул на часы.

-- Прошло ровно два месяца и прежде чем ты ответишь\ldots{} В общем, я просто хочу сказать, что не прекращу спрашивать до тех пор, пока ты не скажешь «да». Конечно, мне придется прекратить, если ты решишь меня бросить. Тогда я, скорей всего, буду очень много пить и трахать все, что движется в мою сторону, но ты должна знать, что\ldots{}

-- Ты и так очень много пьешь, Вениамин Алексеевич. Задавай уже свой вопрос пока ты не наговорил чего лишнего.

Он улыбнулся, опускаясь на колено.

-- Любовь моя, ты\ldots{}

-- Да!

Вениамин замер с легкой улыбкой на губах. Прядь пшеничных волос кокетливо упала на лицо мужчины, которого я совсем скоро начну называть мужем, и я подумала о том, что еще никогда не видела его таким красивым.

-- Ты не шутишь? -- все тем же дрожащим голосом спросил Вениамин. -- Ты правда выйдешь за меня?

Я не шутила.

***

Сколько себя помню, давнишние кинотеатры входили в список мест, где я могла чувствовать себя комфортно. Знаете, такие здания старого образца: со ступеньками возле экрана и всего одним выходом; они всегда меня очаровывали. Эти крохотные кинозалы казались мне уютными. Кресла -- обычно в алой обивке -- достаточно расшатанные для того, чтобы в них улечься, мягкий свет убаюкивает, а экран в десятки раз меньше IMAX-а, что позволяет развалиться прямиком в первом ряду. В кинорыночной терминологии такие кинотеатры называются моноэкранами потому, что в них всего один экран. Особенно сильно я любила приходить сюда на самый поздний и самый ранний сеансы -- время, гарантирующее, что ты будешь единственным зрителем во всем зале.

Дождливыми, ветряными и пасмурными днями, когда одиночество сильней обычного сводило меня с ума и начинало казаться, что от этой тоски уже никуда не деться, я находила спасение в кино. Просто хватала самый большой стакан кофе или чего покрепче, и полностью погружалась в просмотр. Уединение с кинематографом положительно влияло на мою нервную систему. Когда фильм заканчивался, я подолгу бродила по улицам такого ненавистного мне города, читала книги в пустующем салоне дежурного троллейбуса и думала о том, к чему все это меня приведет.

Что ж, моя любовь к подобному времяпрепровождению была настолько сильна, что в октябре две тысячи двенадцатого один старый кинотеатр официально отменил утренний восьмичасовой показ. Им просто надоело крутить фильмы ради меня одной, так что тем утром меня встретила новенькая вывеска, сообщающая, что впредь кинотеатр будет работать с десяти. Что ж, мне, все-таки, пришлось посетить лекцию по трудовому праву, которую я так отчаянно стремилась прогулять.

Сегодня меня ждал очередной последний сеанс в пустом зале старого кинотеатра. Вот только город, в котором я находилась, не был мне ненавистен. На смену удручающим холодам и извечной сырости моего одиночества пришла сладкая летняя ночь. Воздух Севастополя был наполнен прохладой, в нескольких километрах за моей спиной волновалось море, -- я практически слышала, как волны ударяются о бетонные плиты пустынной набережной, -- а мою ладонь сжимала рука мужчины, которого я любила той самой необъяснимой, сумасшедшей любовью, что вдохновляла Шекспира, воодушевляла Байрона, терзала Есенина и, по всей видимости, стерла Трою с лица Земли.

Могла ли моя жизнь сложиться еще прекрасней?

По-прежнему держась за руки, мы сидели в самом центре традиционно пустующего зала. Кресла обрамляла темная кожа, но, в основном, картина мало чем отличалась от обстановки, в которой прошли мои тинейджерские годы. На экране появился Патрик Уилсон. По всей видимости, лирические отступления стали неотъемлемой частью фильмов ужасов потому, как Патрик взял в руки гитару и принялся исполнять Can't Help Falling In Love Элвиса.

Как это всегда бывает со звуками или запахами, звучание этой хорошо забытой мелодии перенесло меня в прошлое.

***

За окнами барабанит дождь. Обернувшись пледом, я сижу у приоткрытого окна. Вдыхаю прохладный воздух и закуриваю. Мне девятнадцать. На часах начало третьего утра, и я рада, что завтра воскресенье. Иначе мне еще пару часов назад пришлось бы идти на работу, где пять дней в неделю, ровно в полночь, я на ближайшие девять часов превращалась не в тыкву, но в агента технической поддержки одного небезызвестного зарубежного провайдера хостинга.

На мне наушники, из которых раздается Can't Help Falling In Love, так что я не сразу замечаю звонок Адама, но, все же, успеваю поднять трубку. Затем тушу сигарету, думая, что Адам попросту звонит пожелать мне спокойной ночи. В последнее время что-то в наших отношениях явно не заладилось и тому было несколько тревожных звоночков. Однако я уже нахожусь на той стадии влюбленности, когда не представляю своей жизни без Адама, и это мешает мне трезво мыслить.

-- Привет, сладкая, -- говорит Адам.

Он говорит, что устал. Желает мне спокойной ночи, но я слышу странные нотки в голосе любимого и спрашиваю:

-- Что-то не так?

-- Не совсем, -- отвечает Адам.

Он громко вздыхает, затем какое-то время молчит. Вероятно, обдумывает каждое из слов, которые собирается мне сказать. Этой паузы хватает на то, чтобы я подкурила новую сигарету и даже на то, чтобы половина ее превратилась в пепел.

Наконец, Адам нарушает молчание. Он говорит:

-- Я долго думал о нас.

Пауза.

-- О том, к чему все это приведет.

Дальше я слышу монолог, в ходе которого вдруг выясняется, что мы слишком разные люди. Слышу многочисленные аргументы в подтверждение этого заявления. Про себя я успеваю заметить, что еще совсем недавно различия между нами не играли особой роли. В какой-то мере они даже нравились Адаму и лишь подстегивали его, когда я так не хотела серьезных отношений. Но аргументы продолжают сыпаться на меня.

Один за другим, они всплывают монотонно и уверенно. Уровень выкупаемости происходящего растет вместе с количеством этих самых аргументов, и я начинаю понимать, что Адам меня бросает.

-- Это все потому, что я не хочу детей? -- спрашиваю я.

Он отвечает, что это всего лишь одна из причин.

-- Но для меня она очень важна, -- подчеркивает Адам.

\emph{Мне всего девятнадцать, чувак, какие нахрен дети?!}

Мне вдруг становится тошно. Хочется накричать на него, напомнить, что я с самого начала предупреждала его о детях, как предупреждала о депрессии и, скорей всего, предупреждала обо всех остальных «ненормальных» оставляющих себя.

Адам не назвал меня безумной, он был слишком интеллигентен для подобных выпадов, но я знала, что именно это его и спугнуло. Говорят, если в тебя влюблен писатель, ты можешь жить вечно. Что же касается душевнобольных, с такими -- что не день, то сюрприз.

Мужчин всегда влекло мое сумасшествие. Сперва они были им очарованы, а затем ненавидели меня за это. Могла ли я их за это винить? Едва ли, ведь никто не предлагал мне жизни без психического расстройства, а у этих людей такой выбор имеется. Так почему бы им не воспользоваться?

-- Я хочу семью.

-- Я знаю, -- отвечаю я.

После этого мой возлюбленный приводит еще один ряд причин, по которым нам не суждено быть вместе. Теперь он уверен, что если не закончить эти отношения сейчас, мы будем все время ссориться и однажды просто возненавидим друг друга. А он не хочет, чтобы так все заканчивалось.

-- Я не хочу, чтобы так все заканчивалось. Если мы встретимся, скажем, через пять лет, я хочу, чтобы у нас все еще был шанс, понимаешь?

\emph{Потому, что через пять лет я могу захотеть детей, но до этого момента тебе лень со мной возиться?}

Я отвечаю, что понимаю, хотя на самом деле не понимаю нихрена.

Тогда Адам благодарит меня за то, что этой осенью я сделала его счастливым, и спрашивает: что я могу сказать на все это?

Я чувствую себя заключенным, которого ведут сквозь полный зал молчаливых зрителей. Усаживают на электрический стул, закрепляют все нужные ремни и проверяют их, а потом предлагают сказать последнее слово. Мне смешно и стыдно от собственной ранимости, но именно так я себя и ощущаю. Вот только какой в этом смысл?

Мне девятнадцать и я все еще не в курсе, что больна биполяркой, но все это не мешает мне понимать один простой факт: со мной что-то не так. Мое восприятие, ровным счетом, как и мое поведение, мои слова и поступки -- все это заметно отличается от того, как ведет себя общество. Последнее, кстати говоря, мне противно. Я нахожу сегодняшний социум лживым, двуличным, и чертовски гнилым. Мне отвратительны все эти социально приемлемые нормы с их двойными стандартами и лицемерными суждениями. Ведь я вижу их насквозь и по-прежнему поражаюсь тому, как слепы все вокруг.

Потому я несчастна. Чувствую себя невероятно одинокой и никем не понятой.

Пускай я и не знаю тому названия, но в моей голове точно есть что-то лишнее. Отличающееся от стандартного набора. Что-то, с чем я так и не смогла совладать. Я стала расстройством, расстройство стало мной и получать Елизавету Васляеву бывало все труднее и труднее. Разве что порционно, когда она (я) находилась в добром расположении духа. Я не знала, как избавиться от этой напасти, не сводя счетов с жизнью, и уж тем более не смогла бы придумать ни одной причины, которая заставила бы здорового человека желать подобной участи. Так что же, на милость, я должна была ответить Адаму?

Он заметил, что я непривычно молчалива. Затем все тем же низким, мурлыкающим басом, каким мужчины говорят только с любовницами, повторил свой вопрос.

-- Хочешь мне что-то сказать?

На ум вдруг пришли последние слова Питера Мануэля, более известного как Зверь из Биркеншоу.

\emph{Сделайте громче радио, и я спокойно пойду.}

В общем, я сижу у окна и прекрасно понимаю, что, что бы я ни сказала, у этой истории есть только один конец. Адам хочет со мной расстаться, и ничто в моих словах не изменит его решения. Именно поэтому я говорю правду.

-- Не думаю, что все это должно вот так закончиться.

Еще с минуту Адам молчит, а затем переводит разговор в нейтральное русло.

-- И чем ты завтра займешься? -- спрашивает он. -- У тебя же выходной. Пойдешь с подружками на шопинг или вроде того?

-- Я ненавижу шопинг! -- эмоционально отвечаю я, на мгновенье забыв о расставании и искренне удивляясь тому, что мой молодой человек, да вообще любой человек на планете Земля, еще не в курсе того, как сильно я не люблю шопинг.

-- Ну и чем тогда займешься? -- спрашивает Адам.

Он улыбается краешком губ, и я буквально вижу это сквозь тысячи разделяющих нас километров.

-- Это зависит от того, когда я усну, -- тут до меня окончательно доходит вся горечь происходящего.

Я чувствую, как на глаза наворачиваются слезы и стараюсь сделать все возможное для того, чтобы мой голос звучал спокойно.

-- А чем еще ты собираешься там заниматься в три часа ночи, если не спать?

-- Я попытаюсь уснуть.

Голос меня выдает.

-- Слушай, -- ласково говорит Адам. -- Ты ведь позвонишь мне, если тебе будет плохо?

-- Конечно.

Мы желаем друг другу спокойной ночи и разговор заканчивается.

Той ночью я так и не смогла уснуть. Как не смогла уснуть и многими другими ночами, когда мне было чертовски плохо. Но я так ему и не позвонила.

***

Заполнявшие кинозал звуки музыки умолкли, а вместе с ними исчезло и мои грезы об Адаме. И какими же удивительными показались теперь мне эти воспоминания! Такими далекими и несущественными. Я вдруг поняла, каково это: по-настоящему разлюбить кого-то.

Оказалось, что это возможно. Оно того стоило? Этот вопрос я задавала себе миллион раз, конечно же, имея в виду, любовь. Теперь, когда я смотрела на профиль Вениамина, что, то исчезал в темноте, то вновь появлялся по ходу того, как на экране сменялась картинка, я вдруг осознала, каким ребячеством были мои чувства к Адаму. А еще большим ребячеством были те страдания, что последовали за нашим импровизированным романом.

Все бы ничего, вот только колкая игла памяти коснулась моих мыслей. Казалось, что-то подобное уже однажды приходило мне в голову. Когда-то давно я наивно полагала, что нельзя любить сильнее, чем я любила Эрика, а затем я встретила Адама. Похоже, законы логики здесь напрочь не действовали: сколько бы боли не причиняла мне любовь, с каждым новым разом она оказывалась все сильнее.

Пара на экране взялась за руки. Они были вместе не одно десятилетие и по-прежнему выглядели влюбленными. Глядя на них, я думала о Вене. Мне вдруг стало страшно при мысли о том, что этим вечером я сознательно согласилась провести свою жизнь с одним единственным мужчиной. Я доверяла ему больше чем самой себе и это пугало меня еще сильнее.

Затем мой жених обернулся. Наши взгляды встретились, и этот страх развеялся так же быстро, как и появился.

Вениамин обнял меня одной рукой, и вновь повернулся к экрану.

Волшебство вернулось в мою жизнь.

\hypertarget{chapter-50}{%
\chapter{~}\label{chapter-50}}

За окнами по-прежнему стояла невообразимая жара. Лето лениво растянулось вдоль нашей улицы, уходя далеко за ее пределы. Что ж, Крым, вроде как, славится своими морскими красотами. Вот только никто не говорил мне о том, что, несмотря на наличие всех этих водоемов, климат здесь истерически сухой.

Веня сидел в дверном проходе советских времен спаленки, которая в те дни была нашим домом. Крохотное, ободранное помещение с разваливающейся мебелью и облезлыми обоями.

Мое любимое помещение.

Помнится, когда мы только начинали жить вместе, бабуля Вениамина торжественно передала нам древнюю кровать с деревянным дном. Это действо она сопроводила следующими словами:

-- На этой кровати уже трое померло. Думала, четвертой буду, пользуйтесь.

Ну, вот мы и пользовались.

Солнце лишь недавно опустилось за горизонт, и я вышла на незастекленный балкон. Такой же потасканный, как и спальня, которая к нему вела. И, все-таки, он стал местом, где прошли лучшие дни моей жизни.

Что ни говори, а панорама здесь и впрямь была живописной. Подкурив сигарету, я наблюдала за тем, как по всему городу, один за другим, зажигались очаровательные огоньки чьих-то спален, кухонь и гостиных. По правую руку от меня виднелось море. Величественное и черное. Мне всегда нравилось смотреть на него.

Удивительный город -- Севастополь. С одной стороны, он неоспоримо погряз в останках постсоветского пространства, в серости зданий, консервативных идиллиях и чертовски запущенной экономике. И, все же, он неповторим. Я знаю мужчин, которые всерьез утверждают, что любили многих женщин, каждую по-своему. И каждой они оставили частичку своей души. Что ж, у меня сложилась похожая ситуация с городами.

К тому моменту мы жили здесь больше месяца, если не считать моего весеннего отпуска. Естественно, я уже успела сообразить, что к чему: идти до центра было ой как далеко. Дорога занимала около пяти часов. К тому же, шагать почти все время нужно было вверх. Приятного мало, согласитесь.

Тем не менее, картинка из окна выглядела так, словно вот оно море, прямо у тебя под рукой.

Если долго бродить по Севастополю, он начинает напоминать вычурную спираль, что окружает воду и горы. Мы же поселились где-то у ее основания.

-- Деточка, -- послышался голос из спальни. -- Ну, сколько можно курить?

Я сделала еще пару затяжек и покинула балкон.

За пару часов до этого мы с Веней зашли в винную лавку. Продавщица уже давно успела нас запомнить. Не удивительно. Думаю, в этом мире не так много людей, которые приходят в винные лавки за виски, ромом, коньяком, или водкой, а потом покупают все и сразу, прихватив еще парочку бутылок полусладкого. Мол, это же винная лавка, как-никак.

Короче, в магазине была какая-то акция, и вместо намеченных полутора литров Веня взял шесть бутылок рома. Расправившись с первой бутылкой, он принялся просить меня включить какое-нибудь арт-хаусное старье, и я включила «Влюбленного гробовщика».

-- Херня какая-то, -- продекламировал Вениамин спустя четверть часа.

Кинематографический критик.

Здесь стоит признать: Веня никогда не понимал книг, фильмов, музыки и прочих частичек искусства, которые были мне дороги. Стоит ли говорить о чем-то более экзотическом? Тафофилии, буддизме или пост-модерне? Нет, муж определенно всего этого не понимал.

Хуже того, даже не пытался понять. Просто молча наблюдал за происходящим на экране и то не особо внимательно.

***

Незадолго до нового года мы решим устроить себе маленький праздник. Закажем домой огромную пиццу, купим бутылку вина и литра три дешевого портвейна. К тому моменту мы уже переберемся в Симферополь, так что доставщик по традиции полчаса будет нарезать круги вокруг дома, и пицца вновь достанется нам холодной.

Заглянем же на мгновенье в этот зимний день.

Устроившись на полу, мы обнимаемся и потягиваем винишко. Все это чувствуется как долгожданное спокойствие. Я вроде как вновь счастлива. На экране показывают мой любимый момент -- финальную сцену, в которой основные персонажи таки догнали свой поезд. Для этого им пришлось избавиться от чемоданов с вещами покойного отца, которые братья весь фильм таскали за собой. Занятно, что в самом начале фильма была идентичная ситуация, только тогда старшему брату удалось догнать поезд вместе со всеми этими чемоданами. В финале он же и предложил от них избавиться. Такая вот незаметная метафора о принятии прошлого и избавлении от старой боли.

Все это я пытаюсь объяснить Вениамину, который явно не желает искать в фильме метафор. К моему изумлению, его лицо не выражает ничего кроме, быть может, скуки.

Я вздыхаю. Потом спрашиваю:

-- Тебе не понравилось?

-- Понравилось, -- тихо отвечает Веня.

И все же, весь его вид говорит обратное. На лице мужа четко читаются вопросы вроде «зачем мы вообще смотрим на этих людей?» и «почему они весь фильм ничего не делают кроме как куда-то едут?»

-- Ты не понимаешь фильмов Уэса Андерсона, -- с грустью говорю я. -- А я-то ещё удивляюсь тому, что ты не понимаешь меня\ldots{}

Случится это спустя несколько месяцев.

***

Сейчас же мы сидели севастопольской квартирке бабушки Вениамина. Последний как раз выключил очередной из моих любимых фильмов, так толком в него и не всмотревшись. На столе стояло еще четыре полных бутылки рома и одна наполовину выпитая. Или наполовину не выпитая. В принципе, какая разница, если вскоре она тоже опустеет?

-- Ну и чем думаешь заняться? -- спросила я.

Мне нравилось напиваться с Веней как минимум потому, что я любила алкоголь самой неприкрытой любовью. Веню я тоже любила, а пьяный секс с ним казался мне превосходным. На трезвую голову любимый бывал слишком ласковым, когда мне хотелось всех этих ролевых грубостей. Конечно, нежный секс нравился мне ничуть не меньше, но ещё больше нравилось находить ту самую золотую середину.

Вениамин указал на открытую бутылку. Он разом опустошил ее, выпив ром прямиком из горла.

-- Схожу в душ и прилягу к тебе.

Мне хотелось обнять мужа, утащить в сторону постели. И все же, по одному ему известной причине, Веня не очень-то жаловал банные процедуры. Порой приходилось чуть ли не силой заталкивать его в ванную, так что я, само собой, обрадовалась такому внезапному заявлению.

-- Буду тебя ждать, -- ответила я.

Венька лукаво подмигнул и удалился.

Что ж, на этой многообещающей ноте я решила воспользоваться рекламной паузой, и запрыгнула в своё любимое кружевное боди. Это был полупрозрачный корсет чёрного цвета. И тем вечером он как нельзя кстати пришелся в компании чулок с подвязками и пояса для них, который я как раз натягивала. Мне нравилось носить нуарное белье, не знаю почему. А Вене нравилось, как я в нем выгляжу. Да, многим нравилось, на самом деле.

Прошло уже приличное количество времени, а Вениамин все не появлялся. Повертевшись и покрутившись с полчаса на кровати, я провалилась в сон. Сложно сказать, что мне снилось. Да и снилось ли мне вообще что-нибудь, или я просто дремала?

Бессонница, дающая знать о себе изредка, лишь делает людей нервными. Говоря же о систематической бессоннице, она превращает мозг человека в холодец, жидкий и при этом напрочь застывший. Чем меньше ты спишь, тем более сюрреалистичным становится мир вокруг. В итоге, ты проводишь так много часов в попытках сомкнуть глаза, что уже попросту не понимаешь, удалось ли тебе уснуть.

Мне показалось, что я уснула.

\hypertarget{chapter-51}{%
\chapter{~}\label{chapter-51}}

Сквозь сон моих до ушей доносился чужой голос.

-- Когда это было? -- требовательно спрашивал он. -- Когда это было? Когда это было? Когда\ldots{}

И так раз за разом.

Распахнув глаза, я увидела, что голос исходил от Вениамина. Однако он принадлежал явно не тому человеку, за которого я согласилась выйти замуж. Чужими были интонации, манера разговора и даже сам тембр голоса. Обычно меня не так-то просто разбудить, но, сознание мигом отреагировало на новый источник звука.

Я села на кровати.

Передо мной стоял Веня. Он был полностью голым, если не считать полотенца, невзначай накинутого на бедра. Волосы -- все еще влажные -- падали на лицо. В полумраке прикроватного ночника они сливались с густой бородой, скрывая большую часть лица. Исподлобья на меня смотрел, нет, прямо-таки таращился одинокий глаз моего муженька. Но даже он казался мне чужим.

В голове мельком пронеслось какое-то воспоминание. Еще не до конца сформировавшийся флешбек. Это почему-то показалось мне важным, но я так и успела развить эту мысль. Очевидно, вдохновившись моим пробуждением, Веня сменил пластинку.

-- Ты говорила, -- гробовым голосом произнес он.

В первую неделю знакомства Вениамин рассказал, что у него глаза-хамелеоны. Меняются в зависимости от настроения как те дурацкие кольца, что были так популярны в моем детстве. Среди перечисленных цветов был серый. Он подразумевал ярость. Я хорошо запомнила это просто потому, что никогда не видела Веню с серыми глазами и смутно представляла себе эту картину. Сейчас же (без пяти минут) муж глядел на меня с неприкрытой ненавистью. Его глаза были не просто серыми. Они были стальными. Даже при тусклом свете ночника я могла ясно рассмотреть этот нездоровый блеск.

\emph{Oh, you've got green eyes,}

\emph{Oh, you've hot blue eyes,}

\emph{Oh, you've got grey eyes.}

Сон как рукой сняло, но картинка в моей голове по-прежнему была мутной как Ингул во время весенних отливов. Вместе с тем, поведение Вениамина едва ли выручало.

-- Ты чего? -- прохрипела я и прошлась по комнате быстрым взглядом.

-- Ты говорила во сне, -- вновь с ненавистью бросил он.

В исполнении Вениамина эти слова звучали как приговор.

На столе недоставало одной бутылки рома. Теперь она одиноко лежала на полу у кровати. Пустая. Такой расклад мне совсем не понравился. И этот голос. Он звучал неестественно глубоко и как-то слишком медленно. Так звучат поврежденные пластинки.

-- Правда? -- наивно спросила я.

Что ж, временами, в моменты эмоционального и физического истощения, я действительно разговаривала во сне. Причем делала это отнюдь не сонным, а очень даже бодрым и довольно правдоподобным голосом. Эта привычка образовалась у меня с наступлением совершеннолетия. Тогда я устроилась на свою первую официальную работу. Занималась ею ночью, а утром отправлялась в университет. Само собой, на сон при таком графике отводилось всего ничего, так что за неимением срочных звонков и писем я могла вырубиться прямиком на рабочем месте. Именно тогда я и научилась дрыхнуть, не подавая виду.

-- Лизочка, -- глаза сами открывались, как только кто-то из сотрудников называл мое имя.

Стоило менеджеру подойти ко мне, и какая-то часть мозга тут же просыпалась. Она молниеносно реагировала на различную активность и вела вполне конструктивные диалоги с другими людьми, в то время как я продолжала спать. Кстати говоря, я почти никогда этого не помнила и, в принципе, отлично высыпалась. Так что все, что мне известно о своей сонной болтовне, поступило со слов очевидцев. Более того, во всех случаях моих зафиксированных припадков ночной общительности другие люди первыми начинали разговор.

Как ни странно, прошло много лет с тех пор как мне в последний раз рассказывали, что я говорю во сне. По словам Вени, когда он зашел в спальню, я уже общалась с кем-то. Причем называла этого анонима любимым и рассказывала, что мы знакомы много-много лет. Якобы, мы встретились каким-то летним днем, но начали отношения лишь этой осенью. На минуточку, за окном стояло лето двух тысячи шестнадцатого года.

Стоит ли говорить, что у меня не было никакого романа на стороне? Как, ровным счетом, не было и бывшего возлюбленного, подходившего под это описание. И, все-таки, Вениамин упорно отказывался мне верить. Я выползла из постели и попыталась обнять его, но Веня лишь отстранился.

-- Врешь, -- упрямо заявил он.

После чего хмыкнул и окинул меня каким-то оценивающим взглядом. Словно другой человек. В тот момент я была слишком сонной и обескураженной, дабы разозлиться. Прежде мужчины никогда не разговаривали со мной в таком тоне, и я просто не знала, как на это реагировать, несмотря на все эти ваши феминизмы.

\emph{Oh, you've got grey eyes.}

Веня резко подорвался с кровати и направился на балкон, прихватив с собой очередную бутылку. Я поспешила за ним, окончательно позабыв о том, что на мне надето. Если полтора года за стойкой меня чему и научили, так это тому, что спорить с алкоголиками -- дело крайне бесполезное. Словом, я не стала отрицать слов Вениамина и решила посмотреть, что будет дальше.

Дальше оказалось хуже.

Сероглазый двойник моего мужа хранил молчание. Он расправился с пятью сигаретами, не проронив ни слова, и периодически прикладываясь к бутылке. Когда и четверная сигарета отжила свое, Веня вдруг заговорил. Его слова все еще звучали медленно и как-то безэмоционально.

-- Ты до сих пор его любишь.

-- Кого?

-- Адама, -- подозрительно быстро ответил Вениамин.

И я поняла, что именно об этом он так старательно думал на протяжении пяти сигарет, курение которых пришлось на двадцать жутких минут гробовой тишины.

-- Конечно, нет! Я о нем уже давно и думать забыла, Венька.

Но мой жених не унимался. Совершенно внезапно он принялся меня допрашивать. Прежде Веня старательно избегал любых разговоров, что, так или иначе, были связаны с любовью, моими бывшими и прошлыми отношениями. Как и большинству мужчин, которых я знала, Вениамину была противна сама мысль и том, что когда-то его девчонка могла быть с кем-то еще. И, как и всех упомянутых выше мужчин, тот факт, что у меня было впятеро меньше партнеров, чем у него самого, никак не утешало моего жениха.

Сейчас же он сделался невероятно заинтересованным. Веня задавал миллион вопросов о моем бывшем и делал это с такой не свойственной ему внимательностью к деталям.

-- Когда ты его в последний раз видела?

-- Когда мне было девятнадцать.

-- Когда?

-- В две тысячи тринадцатом году.

-- Точно?

Я кивнула.

-- И что, трахается он лучше меня?

-- Вообще нет, -- тут мне даже врать не пришлось.

-- Так, когда ты его видела?

-- Три года назад.

-- Летом! -- чуть ли не ликующе выкрикнул Вениамин.

-- Нет, осенью.

-- И как это произошло? Ты ведь ему что-то сказала на прощанье?

\begin{itemize}
\tightlist
\item
  Нет, -- просто ответила я.
\end{itemize}

-- Почему?

-- Потому, что тогда я не знала, что это наша последняя встреча.

Вениамин подкурил n-нную сигарету и вернулся к допросу.

-- А он?

-- Думаю, он тоже этого не знал.

-- Нет. Что он сказал тебе. Ты помнишь?

Конечно, я помнила. Поезд, опустевший перрон и последние дни золотой осени.

\emph{А ты боялась, что история нашего знакомства будет не романтичной}, говорит мне Адам.

Поезд издает финальный гудок, и Адам улыбается.

-- А теперь поцелуй меня. Как персонаж чертового черно-белого фильма.

Тогда я запрыгнула в поезд и, что было сил, старалась не оглядываться. Боялась, что если увижу Адама еще хоть разок, непременно разревусь. Само собой, я помнила этот день. И дело здесь не только в моей исключительной памяти. Как можно не помнить последних слов, сказанных на прощанье человеком, которого ты когда-то так сильно любил?

-- Я не помню, -- наконец, ответила я.

Тем вечером я впервые соврала Вениамину.

***

-- Поклянись, что любишь меня, -- не столько попросил, сколько потребовал мой жених.

Вечер уже обратился ночью, а мы все так же торчали на балконе. Я -- в белье и с заплаканными глазами, Вениамин -- в явном алкогольном опьянении и с ранее упомянутой бутылкой в левой руке. Правой он коснулся моей щеки и смахнул с нее одинокую слезинку. Его большой палец все еще поглаживал меня по лицу, тогда как остальные крепко обхватили заднюю часть моей шеи.

Вениамин пристально посмотрел на меня, и я нехотя встретилась взглядом с его глазами. Все еще серыми. И такими чужими.

Тогда он повторил последнюю фразу.

-- Поклянись, что любишь меня.

И тут я вспомнила. Алкоголь, балкон, ночь, кельтский колдовской ритуал, чужие глаза и слова любви -- все это уже было. Я поняла, что именно меня так настораживало на протяжении последних пары часов. Далекая мысль, посетившая меня той весенней ночью, когда жениху взбрело в голову дать клятву на крови, вернулась.

-- Я люблю тебя, Венька, -- ответила я, не отводя взгляд.

-- Ах ты, сука! - прорычал Вениамин. -- Могла бы хоть раз сказать мне правду, -- сухо добавил он в ответ на мой удивленный взгляд.

У меня буквально не было слов, так что я сделала то, что стоит делать писателю, когда ему ничего сказать -- удалилась.

\hypertarget{chapter-52}{%
\chapter{~}\label{chapter-52}}

Не буду врать, мне всегда нравился грубый секс. Нравилось, когда таскают за волосы, отвешивают пощечины и оставляют красные отпечатки на моей не самой тощей заднице. Мне нравились грязные словечки, мужское доминирование и еще много всякой всячины. Понятия не имею, откуда это пошло. Кажется, я просто родилась с такой вот странной тягой к насилию в голове. Естественно, та часть моего мозга, что отвечает за адекватность, отлично понимала: насилие -- это ужасно. И, все же, насилие всегда меня возбуждало, и я ничего не могла поделать с этой частью Елизаветы Васляевой.

Тьма за окнами сгущалась. Занавесь колыхнулась, и в спальню проник холодок, пронизывающий до мурашек, какой гуляет исключительно в предрассветные часы. Тот самый, которого с ужасом ждут тяжелобольные пациенты сотен тысяч больниц и клиник, разбросанных по всему свету. Все они страшатся самого темного, самого серого и холодного времени суток -- часа перед рассветом. По никем толком не объяснимой, но регулярно повторяющейся статистике, пик смертности тяжелых больных приходится именно на это время.

Разбросав волосы на подушке, я лежала без сна уже\ldots{} Сколько? Час или два, не знаю. Говорю же, время течет совсем иначе, когда у тебя бессонница.

Кухонные часы отбили четыре утра.

Если исходить из того, что каждую секунду умирает два человека, Земля лишилась как минимум полторы сотни обитателей, пока я бессонно валялась в постели, размышляя о загадке утренней смерти.

Как-то слишком долго не было видно Вени. Он до сих пор находился на балконе, и в какой-то момент, я даже забеспокоилась, но так и не решилась встать с постели. Ни очередного допроса, ни скандала, ни грубости мне, понятное дело, не хотелось. К тому же, я чувствовала себя до невозможности опустошенной. Мне сделалось настолько плохо, что я подозревала, что вставать с постели мне и завтра не захочется.

Как и любые другие состояния, предполагающие неустойчивую психику, депрессивная фаза моего душевного расстройства имела свои триггеры. Рычаги давления, воспользовавшись которыми можно как выйти из депрессии, так и успешно войти обратно. Причем последнего обычно достичь куда проще.

Что ж, той ночью Вениамину это удалось. Хотя я, разумеется, и не думала, что он делал это специально. В конце концов, никто ничего ни от кого не скрывал. Алкоголизм был одной из первых вещей, которые Веня о себе рассказал. Если быть точнее, третьей. Этот факт шел сразу же после его имени и места жительства.

Ну, а я рассказала ему о биполярке.

И ведь даже не докопаешься.

Занавесь вновь шелохнулась. Я натянула покрывало по самые плечи, готовясь к новому потоку холодного воздуха, но его не последовало. Вместо этого на пороге спальни появился Вениамин.

-- Ты не спишь, -- он кивнул.

-- Нет, не сплю. А где твой ром?

Веня широко развел руки и пожал плечами. Этот привычный жест меня слегка успокоил.

Сама того не желая, я тут же вспомнила нашу весну. И наше лето. Вениамина, пожимающего плечами, когда кассир в очередной фастфудной забегаловке с робостью в голосе спросила, как он собирается съесть эту дюжину сэндвичей. Вениамина, озадаченно разводящего руки, когда его тарахтайка встала прямиком посреди оживленной трассы, и стало ясно, что доставить заказ вовремя он сегодня не сможет. Вениамина, сосредоточенно подбирающего мне кольцо в многочисленных ювелирках Севастополя: все обручалки были мне либо малы, либо ужасно велики, и Веня в смятении разводил руками каждый раз, когда какое-нибудь кольцо оказывалось не по размеру. Вениамина, который несколько недель собирался пойти со мной на пляж и решил сделать это в единственный дождливый день месяца; мы еще толком не успели отдалиться от дома, когда на город обрушился настоящий ливень.

-- И что ты мне на это скажешь? -- с предсказуемой досадой в голосе спрашиваю я, глядя на пелену дождя.

Веня широко разводит руки и пожимает плечами. Рубашка на них уже успела насквозь пропитаться дождевой водой.

Я тут же вспомнила Вениамина, вернувшегося из супермаркета с пакетом, полным чем угодно, но только не тем, что было в моем списке. Вениамина, рассказывающего мне о своей дурной репутации, ходившей в его родном городе. Вениамина, принесшего мне пиво вместе сидра, Вениамина, упрямо отказывающегося идти в душ и едва ли способного назвать тому хоть одну разумную причину, Вениамина, который никак не мог поверить, что проехал нужный поворот.

-- Деточка, ну там же не было знака, -- все повторяет он с недоверием пятилетнего ребенка. При этом разводит руками то и дело поглядывает в зеркало дальнего вида. Так, словно ждет, будто чертов знак прямо сейчас вырастет откуда-нибудь из-под земли. -- Скажи мне, что его там не было. Его же не было?

Само собой, знак был на месте. Я заметила его, когда мы возвращались домой. В ответ на мой ехидный взгляд Венька все так же пожал плечами.

\ldots Вениамина, планирующего нашу свадьбу в одной из центральных кофеен Севастополя; он как раз пытался понять, кого именно хочет видеть среди гостей и лишь пожал плечами, когда я спросила про его мать. Вениамина, вернувшегося с работы на шесть часов раньше просто потому, что работать ему в тот день не хотелось. Вениамина, который никак не мог понять, за то я так сильно люблю пост-панк, но почти сразу же согласившегося выгравировать на наших кольцах надпись Love Will Tear Us Apart.

-- А я-то, дурак, думал, правильно говорить: пока смерть не разлучит вас.

-- Нет, -- я улыбаюсь. -- Если что и способно нас разлучить, то только любовь.

-- Я тебя не понимаю, -- честно признается Веня. -- Ты же знаешь, я у тебя глупенький, но если ты этого так хочешь\ldots{}

\ldots Вениамина, рассказывающего мне о том, как он учился владеть мечом, Вениамина, стыдливо признающегося в том, что ему нравится читать исключительно фэнтези, Вениамина, который никак не мог запомнить пароль от моего ноутбука, Вениамина, смутившегося, когда в троллейбусе к нему подсел местный бездомный и попросил рассказать об уходе за бородой, Вениамина, который был так рад меня видеть, что не мог сомкнуть глаз всю ночь, а затем Вениамина, проспавшего следующим утром работу.

Но прежде всего я вспомнила Вениамина, шагающего мне навстречу по пустому залу ожидания местного автовокзала.

***

Вокруг нет ни души. Лишь дождь без устали барабанит по стеклам.

Он поправляет шляпу и останавливается в метре от меня. На губах играет счастливая улыбка, скрыть которую не способна ни одна борода. За плечами традиционно виднеется рюкзак, из которого выглядывает прайс-лист -- последнее напоминание о закончившемся рабочем дне.

Наши взгляды встречаются, отчего улыбка на губах будущего мужа начинает играть с новой силой. Он выглядит бледным, уставшим и чертовски очаровательным. Никто не решается нарушить молчание, так что мы просто глядим друг на друга, не прекращая при этом улыбаться.

Странное зрелище, должно быть, вышло со стороны.

Наконец, Вениамин делает еще шаг в мою сторону. Он не произносит ни слова. Просто широко разводит руки, предлагая мне свои объятья. Я падаю в них без лишних раздумий и тут же осознаю, что, возможно, впервые нахожусь там, где мне на самом деле стоит быть. Только вот, понимаю -- не совсем правильное слово.

Я это чувствую.

Судя по всему, я не одинока в своих наблюдениях. Вениамин отстраняется всего на мгновенье для того, чтобы поцеловать меня. Сначала в лоб, затем в обе щеки. Наконец, он касается моих губ своими и мир вокруг меня кажется таким далеким.

-- Я до последнего не верил, что ты приедешь, -- негромко говорит Веня.

Он прижимает меня к себе так, словно боится, что я вот-вот передумаю, прыгну в первый же автобус до Николаева. И я отвечаю:

-- Я тоже, -- но я уже уткнулась носом в плечо Вениамина, так что слов не разобрать.

Они, в общем-то, и не нужны.

Вениамин обнимает меня двумя руками. Я с удивлением ловлю себя на том, что дрожу и даже не сразу понимаю, что холод здесь совсем не при чем. Мужчина, чьей женой я соглашусь стать спустя каких-то три дня, вновь целует меня, и я буквально слышу, как мое прошлое -- вся моя жизнь ровно до этого самого момента -- с треском сгорает где-то там, позади.
Мне ее ни капельки не жаль.

\hypertarget{chapter-53}{%
\chapter{~}\label{chapter-53}}

-- Это хорошо, что ты не спишь.

Голос незнакомца возвращает меня к реальности. Я не сплю, просто лежу в постели с распахнутыми глазами, находясь где-то далеко-далеко, в самых заветных уголках своей памяти. Со мной такое частенько бывает.

-- Вот как, -- машинально отвечаю я, хотя на самом деле я до сих пор еще где-то там, у себя в голове.

-- Да, вот так.

Его голос уже звучит куда привычней, но я все равно не могу отделаться от этого мерзкого, чувства -- коктейля из досады, обиды и печали с толикой самобичевания и привкусом типичной бабской усталости.

Мне захотелось заплакать. Хотелось разреветься, что есть сил, и плакать до тех пор, пока во мне совсем не останется слез. Порой от этого и правда становится легче. Я попыталась припомнить, когда я в последний раз так плакала, но не смогла. Моя извечная борьба с психическим расстройством не предполагала подобных опций. Я прекрасно помнила, что стоит мне позволить себе хоть всхлипнуть, и вскоре это перерастет в настоящую истерику. А если у меня начнется истерика, то все, пиши пропало.

Внешне я оставалась спокойной, но горький ком уже поднимался к горлу, а перед глазами предательски продолжали всплывать флешбеки из нашего совместного прошлого. Все эти теплые воспоминания мешали мне учувствовать в разговоре.

На самом деле, мне вовсе не хотелось отвечать Вениамину. Если я чему и научилась в «Штиле», так это тому, что с пьяницами нет толку разговаривать, пока они во власти алкогольного делирия. Сказать больше, в большинстве случаев, общаться с людьми в подобном состоянии -- идея не из лучших. Ты словно ходишь по краю нагревающегося вулкана, где каждая сказанная тобой фраза равносильна таким опрометчивым действиям как погружение какой-нибудь конечности в раскаленную лаву.

В итоге, даже если ты вдруг окажешься самым стойким к боли человеком, какой толк мучиться? Во всех твоих страданиях просто не будет никакого смысла. Все равно первую осмысленную фразу алкоголик произнесет только после того, как вдоволь отоспится. К тому же, сейчас я как никогда ощущала: как же сильно я устала. От людей, от их слов, и от их действий, за которые никто потом не хочет нести ответственность. Короче говоря, я продолжала молчать. Видимо, это не на шутку зацепило моего муженька. В ожидании ответа, Вениамин стоял у балконной двери со сложенными на груди руками. Вскорости он, кажется, понял, что разговорить меня не удастся. Тогда Веня мигом преодолел несколько метров, которые нас разделяли, и развалился в кровати.

-- Ты все еще в тех чулках? -- между тем поинтересовался Вениамин и, не дожидаясь ответа, стащил с меня одеяло.

Он принялся плавно водить пальцами по внутренней стороне моего бедра, едва касаясь кожи.

-- У тебя очень красивые ноги, -- вдруг сказал Веня.

-- Странно, что ты только сейчас это заметил, -- отрезала я.

-- И чулки красивые. И подвязки. Тебе идет.

Становилось ясно, что дальше молчать не получится. Я как раз пыталась придумать какой-то нейтральный ответ и в уме уже перебирала все возможные варианты, когда муж спросил:

-- Для него ты их тоже одевала?

-- Надевала, -- не удержалась я.

Я произнесла всего одно слово, но для моего избранника этого оказалось достаточно. Он резко выпрямился, сел на кровати и одарил меня самым презрительным взглядом, на какой только может быть способен человек в начале пятого утра после пяти с лишним бутылок акционного рома.

-- Ты\ldots{} -- начал Веня.

Его, еще недавно затуманенные глаза вновь обрели стальной блеск, а дыхание сделалось громким и тяжелым. Казалось, оно заполнило все пространство вокруг. Мне стало не хватать свежего воздуха.

-- Ты\ldots{} Ты такая же как они все! -- заявил Вениамин. -- Лживая самовлюбленная шлюха с комплексом самоубийцы -- вот ты кто!

Я бы с легкостью могла пропустить все это мимо ушей, вот только его глаза\ldots{} Они продолжали бурить во мне дыры. Не думаю, что прежде хоть одно живое существо смотрело на меня так, как сейчас смотрел Вениамин. От этого взгляда мне сделалось по-настоящему страшно -- столько нескрываемой злобы и ненависти сквозило в глазах мужа.

Тем не менее, Вениамин не останавливался.

-- Ты говоришь, что ты писатель! Ха-ха! Все, что ты делаешь круглыми днями -- это сидишь на диване и отращиваешь жирную задницу.

Мне почему-то подумалось, что еще чуть-чуть, и я потеряю сознание.

-- А потом еще ждешь, что я приду домой и трахну тебя как следует. А ты, сука, вообще в курсе, каково это целый день общаться с людьми? -- он перешел на крик. -- С людьми, которых ты ненавидишь! Но у меня, блядь, нет выбора! Я должен с ними общаться! Я делаю это ради тебя! Потому, что нам нужно на что-то жить! А потом я прихожу домой и слышу твое нытье о том, что сегодня ты ничего не написала! Да кого это вообще ебет?

-- Меня ебет, -- только и успела вставить я.

-- Я провожу с этими мразями целый день! И нахера? Чтобы прийти домой и узнать, что ты любишь какого-то левого мужика! И с хуя ли я должен после этого тебя трахать? Что ты мне на это скажешь, Васляева?

Вениамин продолжал кричать, но я уже его не слушала. Теперь и мое дыхание стало каким-то тяжелым. Я поняла, что время метафор прошло: мне и впрямь становится сложно дышать. В голове уже начинало гудеть, а где-то на фоне пролетела мысль о том, что неплохо бы сейчас выкурить сигаретку.

\emph{С ганжей.}

Последней у меня не было уже месяца два.

В общем, я выбралась из постели, схватила халат и собралась на балкон. Увы, я успела сделать только шаг в сторону выхода. Что-то крепко схватило меня за плечо, а через секунду Вениамин уже швырнул меня обратно в кровать.

-- Мне нужно на воздух, -- собиралась сказать я, но на деле вышло что-то куда менее членораздельное.

-- НЕ СМЕЙ ОТ МЕНЯ УХОДИТЬ!

От его ора меня начинало трясти. Происходящее настолько обескуражило, что я даже не сразу заметила, что плачу. Я пойму это только спустя несколько минут, когда коснусь лицом намокшей от слез подушки.

Я еще раз попыталась подняться. На этот раз не так быстро. Вопреки всем законам физики и, несмотря на количество выпитого спиртного, Вениамин среагировал мгновенно. Он выпрыгнул из одеял, и с бешеной силой сжал мне руку. Запястье тут же взвыло, но в следующий момент это уже не имело никакого значения. Вениамин развернул меня на триста шестьдесят градусов и отвесил тяжелую пощечину.

Это не была пощечина из разряда тех, которыми женщины награждают друг друга во время жарких споров. Это даже не была та пощечина, какой обычно награждают мужчин, уличенных в измене. И уж тем более это не была одна из тех страстных пощечин, которые мне порой так нравились в сексе.

Нет, это был самый настоящий удар, от которого я тут же улетела в сторону прикроватной стены. За этим последовала адская головная боль, сильней которой было разве что острое чувство обиды и непонимания происходящего, мигом возникшее у меня в голове.

Насилие -- всегда страшно. Об этом знает каждый, но\ldots{} Как выяснилось, насилие вдвойне ужасно, если оно исходит от твоего любимого человека. От того, кому ты всецело доверяешь. О того, с чьей стороны ты наоборот подсознательно ждешь защиты. Конечно, я не раз слышала о семьях, где домашнее насилие является чуть ли не полноценной частью каждодневной рутины. Но слышать о подобном и пережить это -- абсолютно несравнимые вещи. Никаких слов не хватит, чтобы описать леденящий душу ужас, что сковал меня вслед за ударом.

Где-то на задворках моего подсознания мелькнула смутно знакомая мысль.

\emph{Что-то плохое происходит прямо сейчас}, шепнула она.

Как бы там ни было, Вениамин не дал углубиться в раздумья. Он тут же схватил меня за волосы. Я попыталась вырваться из этой хватки, но Веня уже с силой швырял меня лицом в постель. Он несколько раз грубо тряхнул меня за волосы и лишь после этого плотно прижал лицом к подушке.

-- ДА\ldots СКОЛЬКО\ldots ТЫ\ldots БЛЯДЬ\ldots{} ЕЩЕ\ldots БУДЕШЬ\ldots{}

Сероглазый двойник мужчины, которого я любила, продолжал кричать. Он прямо-таки орал, надрывисто выплевывая каждое слово. Правда, голос этот быстро отошел на задний план, уступая место разыгравшейся в моей черепушке мигрени.

Моя щека коснулась чего-то влажного. Вот тут-то я и выяснила, что уже реву.

Возможно, я предприняла еще одну попытку встать. Возможно, нет. Когда с тобой происходят ситуации сродни этой, подобные детали делаются бесполезными.

-- МНЕ ПОЕБАТЬ, ЛИЗА\ldots{}

Поток слез все никак не желал останавливаться, отчего дышать было практически невозможно. Одной рукой Вениамин продолжал держать меня за волосы. Вторая рука мужа скользнула вниз по моей шее и мертвой хваткой застыла у ее основания.

-- ДА Я ТЕБЯ ПОСЛЕ ЭТОГО И КРАЕМ ГЛАЗА ВИДЕТЬ НЕ ХОЧУ!

Продолжая выкрикивать ругательства, разобрать которые я не смогла бы при всем желании, Веня что было силы сжал мне шею, перекрыв и без того слабый поток воздуха.

-- Не смей от меня уходить! -- эта фраза была одной из немногих, что мне удалось разобрать. Вениамин продолжал повторять ее снова и снова.

Понятия не имею, как долго это длилось. Как выяснилось, время замирает не только во время первого поцелуя. Котелок раскалывался, щека горела, кожа головы изнывала от боли, шея ныла, а легкие пекли от невозможности вдохнуть. Стоит ли говорить, что все это меркло и бледнело в сравнении с тем, что творилось в моей душе?

Вениамин еще разок тряхнул меня за волосы и убрал руку с шеи. Вторая рука оставалась на месте, когда ее хозяин склонился надо мной.

-- ЛЕЖИ, БЛЯДЬ, КОГДА Я С ТОБОЙ РАЗГОВАРИВАЮ! -- прошипел он мне на ухо.

Я жадно хватала ртом воздух, чувствуя, что слез на глазах становится все больше и больше. Комната и все ее наполняющие стремительно поплыли перед взглядом, когда Веня поднял меня за волосы. Прижал себе так сильно, что я слышала каждую каплю спирта, выпитую им прошедшим вечером.

-- ТЫ МЕНЯ ПОНЯЛА?

Думаю, будь у меня алкотестер, он бы непременно показал все семь промилле.

Вениамин не обращал никакого внимания на подкрадывающуюся ко мне истерику. Вместо того он продолжил браниться, а затем вновь спросил, поняла ли я его. Суть вопроса, как, в общем-то, и всего происходящего, дошла до меня отнюдь не сразу. Раскрасневшийся от гнева муж нетерпеливо тряхнул меня за плечи.

-- Ты меня поняла?

Только после этого я сообразила кивнуть.

-- ДАЖЕ НЕ ДУМАЙ ОТ МЕНЯ УХОДИТЬ! -- еще раз выкрикнул Вениамин.

После этого он, наконец, убрал вторую руку. Я мертвым грузом упала на сбившиеся от потасовки подушки. Веня лег где-то рядом. Пелена слез, скопившихся в преддверье глаз, скрыла от меня его лицо, и я была за это благодарна. Еще одной встречи с этим испепеляющим и таким чужим взглядом серых глаз я бы точно не пережила.

Щека коснулась чего-то влажного. Подушка подо мной оказалась насквозь промокшей. Я лежала на ней без единого движения, ожидая момента, когда мой сожитель замолчит. Брань его становилась все менее яростной, а голос постепенно затихал, пока Веня совсем не умолк.

Теперь можно, отрешенно подумала я и, позволила слезам вырваться наружу.

\hypertarget{chapter-54}{%
\chapter{~}\label{chapter-54}}

-- Посмотри на меня, -- тоном, не требующим повторения, сказал муж.

На сей раз его голос звучал спокойно, но я уже успела убедиться в непостоянности этого спокойствия. Оно продлилось с четверть часа, после чего Вениамин вдруг решил вернуться к старой теме. Теперь он крепко обнимал меня за талию, продолжая задавать все те же вопросы и выкрикивая ругательства, что едва ли отличались оригинальностью. Веня держал меня всего одной рукой, но делал это с завидной для пьяницы ловкостью.

Я лежала на боку, спиной к своему возлюбленному, и изо всех сил пыталась притвориться спящей.

-- Посмотри на меня, деточка.

Пальцы жениха скользнули вверх и тут же обосновались на моем предплечье. Малейшее движение делалось от этой хватки невозможным. До этого момента, я и не подозревала, насколько он сильный. Даже нашла в себе силы искренне этому удивиться.

-- Если ты еще хоть раз попытаешься уйти от меня во время разговора\ldots{} ТЫ МЕНЯ СЛЫШИШЬ, ВАСЛЯЕВА?! ЕСЛИ ТЫ ЕЩЕ ХОТЬ РАЗ, ХОТЬ РАЗ ПОПЫТАЕШЬСЯ ОТ МЕНЯ СБЕЖАТЬ, Я, БЛЯДЬ, НЕ ЗНАЮ, ЧТО С ТОБОЙ СДЕЛАЮ!

Выкрикнув это, Вениамин ослабил хватку, а затем без лишних церемоний развернул меня лицом к себе. Слезы обжигали глаза. Мне совершенно не хотелось их открывать. Я чувствовала на лице горячее дыхание любимого, но все, чего мне сейчас хотелось -- это бежать. Вот так, в чем есть. Только халат сверху накинуть, да и то не обязательно. Бежать в ночь, в волны, в неизвестность. Может быть, сбить ноги о камни или даже сброситься со скалы -- не важно.

Что угодно, лишь бы только вновь не проживать этот момент.

-- ТЫ МЕНЯ ПОНЯЛА? -- с этими словами пальцы делившего со мной постель мужчины коснулись моих волос, чтобы мигом исчезнуть в их волнах.

Мне стоило немалых усилий открыть глаза. Конечно, сначала все вокруг было очень мутным из-за пелены слез, что накрыла мое лицо. Силуэт Вениамина грозно нависал надо мной. В тусклом свете одинокой настольной лампы он выглядел крайне устрашающе и больше походил на проклятье старого призрака, какими детей пугают перед сном.

Не проронив ни слова, я наблюдала эту картину еще с полминуты, пока в глазах у меня окончательно не прояснилось. Видимо, Веня тоже этого ждал потому, как он тут же зашевелился и придвинулся еще ближе.

-- Ты меня поняла? -- повторил он.

На этот раз без крика.

Его рука до сих пор была в моих волосах. Пальцы судорожно сжимались и разжимались. Руки Вениамина все так же продолжали дрожать, но это меня не удивило. Я и сама дрожала. Тряслась как осиновый лист впервые за двадцать два года.

-- Деточка? -- напомнил о себе Вениамин.

\emph{Деточка.} От упоминания самого этого слова мне захотелось плюнуть ему в лицо.

Веня опустил руку, и я заметила, что несколько прядей моих волос остались торчать зажатыми между его пальцев. Вероятно, они запутались в кольцах, но это не очень меня волновало, чего нельзя было сказать о Вене. Несколько секунд он с удивлением рассматривал оторванные частички моей шевелюры с полным недоверием на лице.

\emph{Он не помнит}, поняла я.

Эта мысль напугала меня гораздо сильнее всех вместе взятых криков, пощечин и вырванных прядей. Тем временем, Веня продолжал разглядывать клок волос, намотанный вокруг объемного серебряного кольца. Того самого, которое прошлой весной он надел на безымянный палец моей правой руки.

-- Это я сделал? -- уже с меньшим недоверием спросил Веня.

Вопрос вполне мог сойти за риторический, так что я не стала на него отвечать.

-- Прости, что вырвал тебе волосы, -- неожиданно произнес Вениамин. -- Я этого не хотел.

Откровенно говоря, я понятия не имела, что следует говорить в таких ситуациях. Я вообще, вроде как, никогда не планировала в них попадать, так что я просто кивнула.

Вениамин продолжал за мной наблюдать. Возможно, ждал, когда я сорвусь и наброшусь на него с обвинениями. А, может быть, просто мысленно оценивал нанесенный ущерб, одновременно пытаясь припомнить, каким именно образом он этот самый ущерб нанес.

-- Ты вся дрожишь, -- заметил Веня после продолжительной паузы.

\emph{Да ладно! Скажите, Шерлок, откуда такие методы? Это дедукция или индукция?}

Он попытался меня обнять, но я инстинктивно отстранилась. Вениамин моментально изменился в лице.

-- Ты что, боишься меня? -- без особого доверия спросил он, но, заглянув в мои глаза, сам понял ответ. -- Ты меня боишься.

Прежде я никогда не испытывала такого чувства. Я вообще сомневалась в том, что могу настолько сильно чего-то бояться. Но это был самый настоящий страх, беспощадный и всеобъемлющий. Дикий ужас, сопровождаемый тупым оцепенением. Теперь я как никогда отчетливо могла ощутить это состояние. Оно распространилось по всему телу, затрагивая каждый нерв, каждую мышцу и каждую клеточку.

-- Нет, -- едва слышно ответила я.

Так плохо я еще никогда не врала.

-- Тебе нечего бояться, -- отмахнулся Вениамин, но язык его тела говорил об обратном.

Чем ближе было утро, тем сильнее мой суженый походил на привычного глазу пьянчужку. Постепенно взгляд Вениамина делался мутным, речь становилась медленной, а язык заплетался все сильнее и сильнее. Он вновь положил руку мне на бедро, коснулся губами плеча и несколько раз плавно провел пальцами по разгоряченной коже.

Несмотря на разыгравшуюся ситуацию, -- на страх, который так никуда и не ушел, на слезы, которые еще толком не высохли, на пульсирующую мигрень, прочую физическую боль и обиду, которая без остановки меня грызла -- я чувствовала, как от этих прикосновений по моей коже предательски побежали мурашки. Следом за ними появились и другие. Они плавно накрыли собой каждый миллиметр моей кожи, и я ощутила, как кровь приливает к низу моего живота. Всего один поцелуй, и жар вспыхнул. Меня охватило мгновенное возбуждение. Такое яркое, что я сама себе ужаснулась.

Вениамин тоже это заметил. Он стремительно поднялся выше и принялся стягивать с меня белье. Расстегнуть корсет у него так и не получилось. В конце концов, Веня решил переключиться на те части меня, что не были скрыты под одеждой. Он принялся покрывать поцелуями мои плечи и шею, постепенно приближаясь к декольте.

-- Я люблю тебя, -- не прерываясь произнес Веня.

Его пальцы уже проникли мне под одежду, а я все так же лежала, бесцельно глядя в потолок. Впервые в жизни, находясь в постели с любимым мужчиной, я чувствовала возбуждение, которое не хотела чувствовать. И я сопротивлялась этому как могла, чего нельзя было сказать о моем муже. Казалось, Веня и вовсе забыл, с чего началась эта ночь. Ему, все-таки, удалось сорвать с меня белье. Легким движением руки, Вениамин отшвырнул его в сторону и склонился надо мной.

Склонился точно так же, как когда душил меня в этой самой постели каких-то полтора часа назад. Эта картина моментально всплыла у меня перед глазами, как будет всплывать еще не один раз. Все произошло совсем недавно. Пожалуй, я напоминала сжавшийся комок нервов, который напрягался при малейшем шорохе. Сколько бы Веня не целовал меня, по-прежнему хотелось плакать.

Еще несколько минут он покрывал мое тело поцелуями, повторяя слова любви. Увы, сейчас все эти речи казались мне абсолютно не настоящими, что делало их крайне бесполезными. Разгоревшаяся во мне похоть быстро затмила страх. Лежа в объятьях Вениамина и чувствуя, как пространство между ног становится все мокрее, я внезапно для себя самой я поняла, что злюсь.

Как раз в этот момент Вениамин расправил плечи, закинул на них мои ноги, и вошел в меня. Сделав несколько рывков внутри, он попытался меня поцеловать, но к этому моменту ступор наконец прошел, и я успела слегка осмелеть.

-- Любовь моя, -- сладко проворковал мой жених и это окончательно меня доконало.

В общем, я собралась духом и оттолкнула его в тот самый момент, когда губы Вениамина вот-вот должны были коснуться моих.

На лице мужа отразилось полнейшее недоумение.

-- Что опять не так?! -- с раздражением спросил он и, не дожидаясь ответа, предпринял новую попытку заняться со мной сексом.

Вениамин окончательно захмелел. Сейчас этого уже нельзя было не заметить. Тем не менее, он все с тем же упорством пытался мной овладеть. В какой-то момент, я даже подумала, что можно просто спокойно лежать и ничего не делать. Судя по всему, его бы вряд ли надолго хватило, а у меня уже просто не было сил сопротивляться.

К тому же, моей нервной системе сегодня нехило досталась. Как выяснилось, вопреки сильному возбуждению, я ничегошеньки не чувствовала внизу живота. Черт возьми, я даже не сразу смогла понять, вошел он в меня или нет -- настолько плохо у меня этим утром обстояли дела с чувствительностью.

-- Я так люблю тебя, -- вновь пробормотал Веня.

***

Поразительно! Как много лет я провела, изнывая от одиночества. Все эти годы я особенно не верила, что смогу полюбить еще раз, полюбить сильнее. Ни на мгновенье не забывала, к чему приводит любовь, а оттого смутно могла представить мужчину, ради которого еще раз соглашусь поддаться ее чарам. Тем не менее, не любить я толком никогда не умела. Без этого чувства в душе делалось пусто, а жизнь казалась до невозможности бессмысленной. Это был тот самый отвратительный момент, когда ты уже познал все беды от любви и ох как от них устал, но все никак не можешь искоренить в себе такое простое, понятное каждому желание -- любить и быть любимым.

Что же касалось любовной идиллии, я убедила себя в том, что это невозможно еще задолго до начала работы в «Штиле». В те дни я страдала больше обычного, а потому и пила больше обычного. И вот, где-то между третьей бутылкой крепленого и ночным походом в магазин за пачкой сигарет, я увидела высокого худого парня и девчонку вдвое ниже его.

Я видела их и прежде. На протяжении последних пары лет они встречались мне то целующимися остановке, то гуляющими за руку по парку, то делающими покупки в ближайшем супермаркете. Вскорости выяснилось, что высокий парень и маленького роста девушка еще и живут вместе. Их балкон выходил в тот же двор, что и мой, и я уже не первый год наблюдала в окне соседского дома эти счастливые лица.

Теперь же они обнимались под своими окнами. Так, словно совсем не живут вместе, а то и вовсе видятся раз в месяц. Пожалуй, для нормального человека это может прозвучать дико, но подобные любовные картины не раз вгоняли меня в депрессию. Все эти мельком встреченные на улицах парочки, целующиеся, обнимающиеся и держащиеся за руки люди, что были счастливы в своей любви, постоянно нагоняли на меня тоску. Парни и девушки, которым, в отличии от меня, было дано любить и быть любимыми. Вряд ли хотя бы половина встреченных мною влюбленных на самом деле знала, как сильно им повезло.

Некоторые распоряжались любовью как чем-то само собой разумеющимся. Чем-то совершенно неважным и незначительным, что не могло меня не злить. В такие моменты мне хотелось подскочить к рассорившимся голубкам и приняться орать на них что есть духу. Мне хотелось кричать обо всем том, что приносит в наши жизни любовь, напомнить, как много они получают в своих отношениях и как глупо растрачивают их на ерундовые скандалы и бытовые конфликты. Короче говоря, мне было больно смотреть, как люди изо дня в день топтали и пинали любовь, ради которой никто, по сути, и пальцем не шелохнул.

Пожалуй, похожие чувства должна испытывать бесплодная женщина. Она мечтает о детях и проклинает небеса каждый раз, когда слышит об абортах.

Не то, чтобы я ненавидела всех этих взаимно любящих друг друга женщин и мужчин, или мужчин и мужчин, или женщин и женщин. Скорей, меня не устраивала сама ситуация. Без сомнений, ссорящееся по пустякам и регулярно расстающиеся друг с другом пары могли вызвать во мне раздражение. Но куда чаще они всего-навсего вгоняли меня в уныние. Когда ты одинок и болен душой, такие вещи воспринимаются чуть ли не на уровне личностной обиды.

И не без причины.

-- У тебя этого никогда не будет, -- негромко произнесла двадцатилетняя я, наблюдая за соседской парой.

Впервые в жизни эта мысль не принесла мне боли. Более того, она меня вполне устроила.

Оказалось, отношения -- это гораздо больше, чем просто дарить тепло и получать его. Если опустить детские глупости и взглянуть на мою взрослую и не очень жизнь, становится ясно, что я никогда не отношалась с человеком, которого не любила. Не строила серьезных отношений с тем, кто бы мне нравился, был симпатичен, или кого мне попросту хотелось затащить в постель. Нет-нет-нет, жизнь не раз показывала мне, чем это дело заканчивается. Отношения несли за собой огромный риск. Здесь не было места легким симпатиям и нереализованным постельным амбициям. Если что и могло меня втянуть в подобное дерьмо, так это любовь.

\emph{Love.}

\emph{Love will tear us apart.}

\emph{Again.}

Я слишком долго страдала от неразделенной любви, измен, расставаний и обманов, лжи, недосказанности и поспешно разорванных отношений, стараясь прикрыть раны в сердце выпивкой и новыми знакомствами. Пыталась много работать, да так, чтоб у меня и времени на депрессию не возникало. Ну, и, конечно же, я пару раз пыталась забыть любимого человека, прыгая в постель с другими мужчинами.

Как бы там ни было, в итоге я все равно заканчивала тем, что могла разреветься по практически любому поводу, если тот наводил меня на мысли о собственном одиночестве. Катализаторами, к примеру, могли послужить парные футболки, двойной разъем для наушников, или последний свободный столик в ресторане, который, ох, простите, доступен только для двоих. Черт возьми, даже небрежно выброшенный кем-то из окна автомобиля презерватив, что как-то встретился мне по пути на работу, напоминал об одиночестве.

Словно я когда-то могла о нем забыть!

Пускай тогда мне так и не удалось вытрахать из себя Адама, я поняла кое-что важное. Так уже вышло, что я изнывала от любви слишком долго, и одним разом все это мне просто осточертело. Впредь я больше не пыталась заткнуть свое одиночество. Да и какой смысл в том, чтобы его прятать?

\emph{Пускай звенит}, думала я.

Каждый раз, когда, находясь в эмоционально уязвимом состоянии, я сталкивалась лицом к лицу с каким-нибудь очередным проявлением любовной идиллии, псы в моем сердце уже не лаяли. Они лишь начинали тихонько поскуливать, а я повторяла себе одну и ту же фразу.

-- У тебя этого никогда не будет.

Как ни странно, эти слова меня успокаивали. От правды мне всегда становилось легче. Трудней всего оказалось ее признать.

***

-- Я так люблю тебя, -- горячее дыхание Вениамина обжигало мне шею.

Слушая его, я понимала, что еще каких-то пару лет назад вряд ли могла представить себе ситуацию, при которой выражение чувств со стороны любимого мужчины привело бы меня в ярость.

\emph{Лет?} -- тут же отозвался внутренний голос.

Всегда такой правильный, такой хладнокровный в своих суждениях.

Лет? Пришлось признать, что речь, скорей, идет о месяцах. Если не о неделях. На протяжении всей своей жизни я считала любовь самой важной вещью на свете, но она едва ли отвечала мне взаимностью. Говоря о любви, я имею в виду, не только свои чувства к мужчинам. Я говорю о всяких взаимоотношениях с людьми. Кем бы ни были мои друзья, я любила их ничуть не меньше мужчин. С той же слепой верой и не всегда здоровой самоотдачей. По сути, дружба была для меня так же важна, как и сама любовь. Без них моя жизнь мигом теряла последние краски. Взять хотя бы Мишель.

***

Мы познакомилась так давно, что точной даты уже не припомнить. Произошло это, то ли в конце девяностых, то ли на рассвете двухтысячных. Словом, дело было в раннем детстве, еще до того, как кто-либо из нас пошел в школу. С Мишей мы встретились прямиком у воды, неподалеку от ее загородного дома, где впоследствии пройдут лучшие, а вместе с тем и худшие дни моего отрочества.

Странно звучит, я знаю. Но порой так действительно бывает.

Существуют воспоминания, предысторию которых вспомнить сложно, и ты не можешь с уверенностью утверждать, что было до, или даже после описываемого события, а вот сам момент отпечатывается глубоко на подсознании. Всего одно мгновение, но оно настолько чисто и реально, словно это произошло прошлым вечером. Ты помнишь все, от случайно брошенных реплик, витавших в воздухе запахов -- именно это происходит со мной, когда я думаю о нашем знакомстве.

Так вот, мы встретились жарким летним днем. Должно быть, стоял август, потому как вдоль пляжа уже вовсю цвела амброзия, делая воздух чуть ли не ядом для моих легких. Место действия -- пересечение двух старых дачных кооперативов.

Я сидела у самой воды. Лицом к горизонту, спиной к этим жутким растениям. В те дни здесь все еще ходили катера, а все в близлежащем поселке были жилыми. Местная детвора собиралась на пляже, приводя с собой младших братьев и сестер, которых родители оставили им на попечение. Катер всегда появлялся в одном и том же месте, но, высматривая очередной корабль, ребятишки каждый раз спорили о том, откуда он должен появиться. Они плескали друг в друга водой и кидались песком, отстаивая свою правоту. Так было пока чей-нибудь звонкий голос не крикнул:

-- Идет!

Всего одно слово, и дети -- все как по команде -- сливались на причал, что в те дни только начинал покрываться ржавчиной. С энтузиазмом носились взад-вперед, оставляя за собой шлейф раздражающего топота. Длилось это до тех пор, пока последний пассажир не поднимется на борт. И лишь когда тень катера скрывалась за поворотом, детвора нехотя начинала разбредаться по сторонам.

Идентичная картина открывалась мне каждые выходные. Шумиха сельской детворы уже успела стать неотъемлемой частью пляжа, и я быстро потеряла к ней всяческий интерес. Так что же оставалось?

Вокруг пахло пивом и сушеной рыбой, а вместе с тем покрывшимися потом телами. Пляж, кстати, был маленьким. Метров пятьдесят на тридцать, или около того. С причалом с одной стороны и зарослями бурьяна, что служили отдыхающим туалетом, с другой. Было там еще несколько деревьев и обшарпанного вида сарай для лодок, который, судя по всему, оставался нетронутым еще с советских времен. Что же касается людей\ldots{}

В основном, меня окружали жирные вечно смеющиеся и что-то жующие тетки, абсолютно сумасшедшего вида дети возрастом до десяти лет, а также третьи -- волосатые, ноющие и вечно недовольные существа, не поддающиеся никакой идентификации. Впоследствии выяснилось, что это были всего лишь мужчины.

Сомнительная получалась романтика.

Тем жарким летним днем толстая светловолосая девочка оказалась единственным не издающим шума ребенком. Она уселась на мели и, отвернувшись от солнца, сосредоточенно рыла ямку в песке. В своем процессе девочка выглядела крайне одиноко, но занятного здесь также было мало, так что я как-то вернула свое внимание реке.

Водорослей на виду еще не появилось, но вода, и без того никогда не отличавшаяся чистотой, сделалась заметно мутной. Здесь это всегда происходило раньше положенного срока. Словно река могла предчувствовать приближение бабьего лета. Понятия не имею, почему у меня не было при себе книги. Я просто сидела на берегу, наблюдая за тем, как волны медленно лижут песок, оставляя на нем темные следы.

Помнится, в тот момент я затаила дыхание и изо всех сил пыталась унять зуд. Но тот уже безбожно нарастал под переносицей. Мне было лет шесть, но я уже знала то, что знает каждый аллергик: стоит потерять бдительность, чихнуть хоть разок, и можешь смело запасаться салфетками, потому как в последующие полчаса это произойдет с тобой еще как минимум раз сорок.

Пройдет еще немало лет прежде, чем я пойму, что и со слезами ситуация у меня обстоит подобным образом.

Волны выпрыгивали из реки и тут же исчезали обратно, едва успевая коснуться песка. Так я и сидела, в крайне мрачном расположении духа и с, пожалуй, самым идиотским из всех возможных выражением лица разглядывала реку, когда заметила около себя какое-то движение. Толстая девочка выбралась из воды. Теперь она неуверенной походкой шла в мою сторону, а я до последнего момента делала вид, что не замечаю этого.

-- Привет, я Миша.

Я повернула голову, чтобы увидеть пару зеленых глаз, приветливо улыбающихся мне против солнца.

-- Давай дружить, -- на удивление уверенно произнесла девочка.

Забавно, но эта наивная в своей детскости фраза не несла в себе и малейшего намека на вопрос.

Я посмотрела на воду, затем поглядела на кучкующихся на пляже аборигенов, на зеленые глаза и их обладательницу, которой, по всей видимости, наскучило рыть в песке ямы.

\emph{Миша, так Миша}, подумала я.

Вновь бросила взгляд на реку и кивнула.

На этом мое воспоминание обрывается.

Но у меня есть множество других. О прогулках с собаками, приступах кулинарии, походах по барам, каких-то поездках, совместных чтениях вслух, о бесконечном потоке шуток и танцев и о том, как Мишель доставала меня из петли. Однажды на моей руке появилась татуировка с ее именем. К тому моменту обеим из нас было по двадцать.

Я показала татуировку Мише, еще толком не успев спрыгнуть с велосипеда. Та растрогалась настолько, что заплакала прямиком посреди парковки торгового центра, где мы обычно встречались по субботам.

-- Почему именно я? -- удивилась Мишель. -- Почему из всех своих друзей ты выбрала именно мое имя?

Она все спрашивала и спрашивала, но у меня не было ответа. Я не знала, как объяснить, почему наша дружба так много для меня значит. Мы всегда были вместе, что бы ни случилось, и я просто не смогла найти слов, способных передать это ни с чем несравнимое ощущение родства, которое испытываешь только рядом с лучшим другом.

Потому-то мне было так странно, так больно узнать о предательстве своей лучшей подруги. Услышать все то, что она так ни разу и не решилась сказать мне в лицо, но не стеснялась говорить остальным.

Меня бросали и предавали даже те, кто был рядом не одно десятилетие. Близкие люди отворачивались от меня с наступлением трудных времен. Они брали все, что только можно было взять, а затем с легкостью забывали об этом, как только мне требовалась поддержка. Мне так много врали, изменяли и вставляли палки в колеса те, ради которых я, не задумываясь, отдала бы свою жизнь. Пускай она для меня и не много значила.

Однако, сколько бы печалей и стенаний мне не принесла любовь, сколько бы слез я не проливала бессонными ночами, исход был один. Я по-прежнему оставалась верной идеалам любви и дружбы.

И была по этому поводу беспросветной дурой.

\hypertarget{chapter-55}{%
\chapter{~}\label{chapter-55}}

-- Ты любовь всей моей жизни, -- произнес Вениамин.

Его разогретое спиртом дыхание еще раз коснулось моей шеи, и опустилось ниже. Мне подумалось, что Веня будет признаваться мне в любви до тех пор, пока я не отвечу тем же. Да, быть может еще каких-то пару недель назад я вряд ли могла представить себе ситуацию, при которой выражение чувств со стороны любимого привело бы меня в ярость.

Тем не менее, в свете последних событий, включавших в себя крики, обвинения, брань и побои, о которых я до этого знала разве что из книг да новостей, а также последовавшую за ними пропажу половой чувствительности, которая не могла меня не взволновать, слова Вениамина меня разозлили. Пускай я и чувствовала исходившее от мужа тепло, которому так хотелось поддаться, я никак не могла избавиться от мысли, что рука, поглаживающая меня по лицу, была той самой рукой, которая не так давно таскала меня за волосы, душила и отвешивала пощечины.

\emph{Что-то пошло не так}, подумалось мне.

И я вновь оттолкнула Веню, не говоря ни слова. Во второй раз вышло куда лучше. В стену я, конечно, никого не швыряла, но смогла пнуть мужа достаточно сильно для того, чтобы с меня он слез. Я дождалась, пока муженек окончательно сполз на соседнюю подушку, и тут же выскочила с постели. Затем схватила свой халат и, наспех в него укутавшись, отправилась за сигаретой.

До балкона оставалось всего пару шагов, когда я услышала за спиной какое-то движение. Вениамин медленно приподнялся на локтях. На протяжении нескольких секунд он тупо смотрел на меня затуманенным взглядом, пытаясь понять, что произошло, а потом вдруг пулей вылетел из-под одеяла и ринулся в мою сторону.

\emph{Что-то пошло не так}, заметил отдаленно знакомый. \emph{Что-то плохое происходит прямо сейчас.}

Очевидная, бесполезная мысль. Она настойчиво продолжала гудеть в моей голове, становясь все громче и громче.

Вслед за этим на меня обрушился раскатистый бас Вениамина.

-- СДУРЕЛА ЧТО ЛИ?

\emph{Что-то пошло не так. Что-то плохое происходит прямо сейчас.}

-- ЧТО Я ТЕБЕ ГОВОРИЛ?! -- выкрикнул он. -- ЧТО Я ТЕБЕ ГОВОРИЛ?! НЕ СМЕЙ ОТ МЕНЯ УХОДИТЬ!

\emph{Что-то пошло не так. Что-то плохое происходит прямо сейчас.}

Муж повторял эти слова так много раз, что они уже начали казаться мне до невозможности смешными. А еще мне почему-то подумалось, что, быть может, у Вени с головой совсем плохо, так что он и понятия то не имеет, с чего вдруг я его оттолкнула и куда собственно я направлялась. Возможно, он и впрямь не осознает, что я иду на балкон, и до глубины души уверен в том, что я от него ухожу? Иначе я просто не знала, как еще можно объяснить столь бурную реакцию.

-- НЕ СМЕЙ ОТ МЕНЯ УХОДИТЬ! -- снова сказал Веня.

Я даже знала, какой будет его следующая фраза.

-- ДАЖЕ НЕ ДУМАЙ ОТ МЕНЯ УХОДИТЬ!

Здесь я уже не смогла скрыть улыбку. Зря.

Понятия не имею, как ему это удалось, но через секунду Веня оказался возле меня. Он уже протянул ко мне руки, и я с ужасом осознала, что сейчас все повторится. Спустя мгновенье муж схватит меня за волосы, чтобы швырнуть обратно в постель, зажать шею в мертвую хватку или отвесить мне пару-тройку оплеух. Вероятно, он полагал будто делает это в наказание за то, что мне вздумалось от него уйти.

Все это произошло так быстро. От неожиданности я только и успела, что отшатнуться.

Удивительно, но и этого оказалось достаточно. Без того плохо стоящий на ногах Веня не смог вовремя притормозить. Вместо этого, он окончательно потерял равновесие и растянулся на полу, прямиком у моих босых ног. Само собой, все это только сильнее его разозлило. Бранясь на чем свет стоит, мужчина начал подниматься на ноги. Вот только теперь это давалось ему гораздо хуже. Вениамин то и дело заваливался на спину, сопровождая это действо невнятной руганью, сквозь которую все отчетливей слышалась настигающая его усталость.

-- ЛИЦЕМЕРНАЯ ТЫ МРАЗЬ, ВАСЛЯЕВА!

Меня же охватила полнейшая апатия. Я до сих пор стояла в двух шагах от балкона. Просто не могла заставить себя сдвинуться с места. Ноги вконец отказывались слушаться. За все это время я даже не шелохнулась, если не считать тремора, который и не думал отступать. Мои колени дрожали, а руки тряслись им в такт с удивительной синхронностью. Складывалось впечатление, что эта ночь длится вечность.

В конце концов, до Вениамина начала доходить плачевность собственного положения. Видимо, какая-то не тронутая алкоголем часть мозга моего мужа поняла, что резкий подъем -- не лучшая из его затей. Веня больше не пытался вскочить на ноги. Он медленно встал с пола, а потом так же медленно поднялся на колени. Те тут же затряслись, и двигаться дальше мужчина пока не решался.

К тому моменту царивший в спальне мрак стал понемногу развиваться. Жаль, я не могла сказать того же о своей душе.

Последняя мысль заставила меня опомниться. Сейчас я как никогда остро ощущала нехватку никотина. Стоило мне сделать один единственный шаг назад, и Вениамин мигом сделал очередной рывок. Он так и не смог твердо встать на ноги, и скорее рухнул обратно на пол, обхватив руками мои колени. Его взъерошенная борода едва ли успела коснуться моего живота, когда тишину комнаты нарушил еще один незнакомый голос.

-- ПРОЧЬ ОТ МЕНЯ!

Яростный вопль повторился. Фраза эхом отразилась от стен. Лишь когда крик стал гулять по бетонной поверхности я уловила в нем знакомые загробные нотки. Тогда и поняла, что этот голос принадлежит мне.

-- Но ведь я люблю тебя, -- просто сказал Веня.

Так, словно это что-то меняло.

Он продолжал обнимать меня за ноги, но неожиданно отшатнулся, стоило мне вновь повысить голос.

-- КАТИСЬ ТЫ К ЧЕРТУ СО СВОЕЙ ЛЮБОВЬЮ! -- не без истеричных ноток выкрикнула я.

Затем поспешно вышла на балкон и трясущейся рукой заперла за собой дверь.

***

И вот я стою у края незастекленного балкона, всерьез подумывая с него выброситься. Воздух здесь, как и всегда, пахнет морем. Город еще спит и лишь в паре окон виднеются одинокие искорки света. В это время суток улочки безлюдные и настолько тихие, что можно услышать, шепот морских волны, ударяющихся о камни во время утреннего прибоя.

Тьма окончательно развеялась. Над горизонтом заблестели первые лучи поднимающегося солнца. Вскоре над водой показалась верхушка алого диска, из которого во все стороны струилось теплое сияние. Оно моментально поглотило воду, окрасив ее во все существующие красно-желтые оттенки, а затем принялось игриво подбираться к крышам стоявших вблизи моря домов. Спустя каких-то пару минут солнце поднимется еще выше. Тогда оно сможет дотянуться и до крыши нашего дома, и до соседних домов, и даже до тех, что находятся позади, далеко-далеко за пределами видимости.

Солнечный диск продолжал плавно вырисовываться над линией горизонта и вскоре из него уже струились миллионы янтарных потоков света. Они хаотично разлетались во все стороны, отражались в безмерном количестве поверхностей и игриво переливались в блеске волн, на морской пене да на песке вдоль береговой линии. Вскоре солнце заполнило собой подавляющую часть окружающего меня пространства. Оно лишь наполовину вынырнуло из-за линии горизонта, накрывая город пеленой мерцающего света.

В призме рассветного сияния весь мир казался мне не настоящим, плоским и каким-то двумерным. Да и с настройками контраста явно было что-то не то. Куда ни глянь, уголки Севастополя скорей походили на те идеализированные пейзажи, что с полвека назад было принято печатать на открытках. Знаете, все эти картинки с искусственно усиленной яркостью и заведомо преувеличенными масштабами небесных светил. Однако, в отличие от изображений с винтажных открыток, распростершаяся передо мной картина была настоящей. Солнечные лучи в действительности были такими пестрыми, какими их рисуют, а само солнце и впрямь оказалось огромным. Вскоре его лучики блеснули на каменных ступенях, что каждый день вели меня домой, и теперь скользили по прядям моих волос.

Я смотрела на это невообразимых размеров солнце, зависшее над лазурными волнами, и все размышляла о том, как же это не справедливо: наградить столь паршивое утро таким невероятным закатом. Нет, так не должно было быть.

Но так было.

Наряду с ужасами прошлой ночи происходящее с трудом укладывалось в голове. Затем я подумала о том, что, подстраивайся погода под наше душевное состояние, человечество уже бы давно позабыло о том, как выглядит солнце.

Ведь всегда найдется кто-то, кому еще хуже.

Еще секунд двадцать и солнце окончательно встанет.

Я подкурила то ли вторую, то ли третью за утро сигарету. Охватившая меня ночью дрожь, наконец, начала угасать. Или мне просто этого очень хотелось. В любом случае, пальцы рук, как и мои коленки, периодически подергивались. Из них выскользнула тлеющая сигарета. Она упала на пол и еще какое-то время катилась по балконному полу, чтобы затем исчезнуть где-то в дальнем углу.

Мне страстно захотелось последовать ее примеру.

Все это время истерика то и дело пыталась накрыть меня своими шальными волнами. Чем дольше я с ней боролась, тем сильнее становились эти приступы. Наконец, я опустилась на пол, развернулась спиной к солнцу, и прислонилась к сырой стене. Слишком холодной, как для середины лета. Откровенно говоря, я даже не удосужилась запахнуть полы халата. Просто подтянула колени ко лбу и обхватила их руками -- вот и все, на что меня хватило.

Казалось, я вот-вот заплачу, или даже впаду в истерику, но глаза оставались сухими. Было в этом что-то очень тоскливое. Бесконечная пустота оказалась всем, что я чувствовала. Она проглотила меня целиком и сделала это без лишних церемоний. Я прижималась к холодной цементной стене, пытаясь отыскать внутри себя хоть что-нибудь, способное помочь развеять это чувство -- ощущение полнейшей безысходности.

-- Хоть что-нибудь, -- прошептала я, обращаясь к своей всегда такой славной памяти.

В конечном счете, стало ясно, что внутри меня не осталось ни одной теплой мысли, ни одного лучика света, за который я могла бы ухватиться. Увы, тем утром я так и не смогла отыскать ни одного счастливого воспоминания. Во мне не было ни надежд, ни мечтаний, ни сожалений. Не нашлось ни единой доброй мысли, которая смогла бы помочь остаться на плаву.

Казалось, меня здесь уже тоже не было.

Лишь кромешная пустота, и больше ничего.

Где-то высоко за моей спиной солнце, наконец, достигло своей вершины. Оно величественно парило над водой, отражаясь в морских силуэтах, оставляя на каждом из них частичку себя. Волны шуршали и смеялись, переливаясь всевозможными оттенками лазури. Они словно губка впитывали в себя солнечный свет, чтобы затем унести его зарево в неведомые дали.

Я опустила голову на колени и в полном безразличии смотрела на каменную стену.

А солнце все сияло и блестело, заставляя так же неистово сиять и блестеть все, до чего могли дотянуться его янтарные лучи. Над морем вспыхнули последние огни рассвета, и на какое-то мгновенье весь мир обратился пейзажем Уильяма Тернера.

Тем утром я ни разу не обернулась, чтобы полюбоваться разыгравшимся рассветом. Как выяснилось, мне было на него наплевать.

Что-то определенно пошло не так. И это что-то уже произошло.

\end{document}
