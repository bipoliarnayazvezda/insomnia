% Options for packages loaded elsewhere
\PassOptionsToPackage{unicode}{hyperref}
\PassOptionsToPackage{hyphens}{url}
%
\documentclass[
]{book}
\usepackage{amsmath,amssymb}
\usepackage{lmodern}
\usepackage{iftex}
\ifPDFTeX
  \usepackage[T1]{fontenc}
  \usepackage[utf8]{inputenc}
  \usepackage{textcomp} % provide euro and other symbols
\else % if luatex or xetex
  \usepackage{unicode-math}
  \defaultfontfeatures{Scale=MatchLowercase}
  \defaultfontfeatures[\rmfamily]{Ligatures=TeX,Scale=1}
\fi
% Use upquote if available, for straight quotes in verbatim environments
\IfFileExists{upquote.sty}{\usepackage{upquote}}{}
\IfFileExists{microtype.sty}{% use microtype if available
  \usepackage[]{microtype}
  \UseMicrotypeSet[protrusion]{basicmath} % disable protrusion for tt fonts
}{}
\makeatletter
\@ifundefined{KOMAClassName}{% if non-KOMA class
  \IfFileExists{parskip.sty}{%
    \usepackage{parskip}
  }{% else
    \setlength{\parindent}{0pt}
    \setlength{\parskip}{6pt plus 2pt minus 1pt}}
}{% if KOMA class
  \KOMAoptions{parskip=half}}
\makeatother
\usepackage{xcolor}
\IfFileExists{xurl.sty}{\usepackage{xurl}}{} % add URL line breaks if available
\IfFileExists{bookmark.sty}{\usepackage{bookmark}}{\usepackage{hyperref}}
\hypersetup{
  pdftitle={Бессонница},
  pdfauthor={Александра Булгакова},
  hidelinks,
  pdfcreator={LaTeX via pandoc}}
\urlstyle{same} % disable monospaced font for URLs
\usepackage{longtable,booktabs,array}
\usepackage{calc} % for calculating minipage widths
% Correct order of tables after \paragraph or \subparagraph
\usepackage{etoolbox}
\makeatletter
\patchcmd\longtable{\par}{\if@noskipsec\mbox{}\fi\par}{}{}
\makeatother
% Allow footnotes in longtable head/foot
\IfFileExists{footnotehyper.sty}{\usepackage{footnotehyper}}{\usepackage{footnote}}
\makesavenoteenv{longtable}
\usepackage{graphicx}
\makeatletter
\def\maxwidth{\ifdim\Gin@nat@width>\linewidth\linewidth\else\Gin@nat@width\fi}
\def\maxheight{\ifdim\Gin@nat@height>\textheight\textheight\else\Gin@nat@height\fi}
\makeatother
% Scale images if necessary, so that they will not overflow the page
% margins by default, and it is still possible to overwrite the defaults
% using explicit options in \includegraphics[width, height, ...]{}
\setkeys{Gin}{width=\maxwidth,height=\maxheight,keepaspectratio}
% Set default figure placement to htbp
\makeatletter
\def\fps@figure{htbp}
\makeatother
\setlength{\emergencystretch}{3em} % prevent overfull lines
\providecommand{\tightlist}{%
  \setlength{\itemsep}{0pt}\setlength{\parskip}{0pt}}
\setcounter{secnumdepth}{5}
\usepackage{booktabs}
\ifLuaTeX
  \usepackage{selnolig}  % disable illegal ligatures
\fi
\usepackage[]{natbib}
\bibliographystyle{plainnat}

\title{Бессонница}
\author{Александра Булгакова}
\date{2023-02-09}

\begin{document}
\maketitle

{
\setcounter{tocdepth}{1}
\tableofcontents
}
\hypertarget{ux44dux43fux438ux433ux440ux430ux444}{%
\chapter*{Эпиграф}\label{ux44dux43fux438ux433ux440ux430ux444}}
\addcontentsline{toc}{chapter}{Эпиграф}

\emph{Эта история не о биполярном расстройстве личности, хотя оно здесь есть.}

\emph{Эта история не о любви, хотя она здесь есть.}

\emph{Эта история не о депрессии, хотя она здесь есть.}

\emph{Эта история не о счастье, хотя оно здесь есть.}

\emph{Эта история не о жизни, хотя она здесь есть.}

\emph{Эта история не о смерти, хотя она здесь есть.}

\hypertarget{ux447ux430ux441ux442ux44c-i.-ux431ux435ux441ux441ux43eux43dux43dux438ux446ux430}{%
\chapter*{Часть I. Бессонница}\label{ux447ux430ux441ux442ux44c-i.-ux431ux435ux441ux441ux43eux43dux43dux438ux446ux430}}
\addcontentsline{toc}{chapter}{Часть I. Бессонница}

\hypertarget{chapter-1}{%
\chapter{~}\label{chapter-1}}

Как правило, после внепланового завершения серьезных отношений даже сильные люди впадают в своего рода эмоциональную спячку. Так или иначе, приходится изолироваться от несостоявшейся второй половинки. Перебороть истерики и утихомириться в какой-нибудь тесной квартирке с мусорным ведром под раковиной, микроволновкой для полуфабрикатов, своими немногочисленными пожитками и бутылкой, о которой ты уже толком и не помнишь, что внутри, но к которой все равно продолжаешь периодически прикладываться с крайне отсутствующим видом.

Сколько бы лет тебе не было, -- двадцать пять или пятьдесят, неважно -- вдруг начинаешь вести жизнь завтрашнего пенсионера. Прекращаешь слушать музыку, за едой думаешь исключительно о еде и никогда не выключаешь телевизор. Последний ты даже не смотришь, но все равно оставляешь телек трещать на фоне. Чтобы чувствовать себя менее одиноко.

В моменты как этот всякая деятельность обращается бездействием. С каждым днем ты все больше и больше теряешь интерес к жизни, но все чаще начинаешь задумываться о смерти, которая становится чем-то ожидаемым. Логичным завершением одного тотального фиаско под названием твоя собственная жизнь.

Мать моего отца умерла за месяц до моего рождения. Дедушка же прожил еще полторы декады. Все эти годы он спал со включенным телевизором, который толком не смотрел. Еще дед никогда не смеялся и очень редко улыбался. Разве что мне. После его смерти я нашла коробку со старыми снимками. Практически на каждом из них лицо моего деда освещала широченная улыбка, но отец лишь пожал плечами. Сказал только, что со дня похорон своей жены дед так ни разу не засмеялся.

Он умер хмурым январским утром. Сгорел от внезапно давшего о себе знать рака. Дед пережил голодомор, войну, распад Советского Союза и трех украинских президентов, а потом почувствовал острую боль в желудке и умер спустя месяц. Это вообще легально?

--- Я люблю тебя --- сказала я дедушке, и вышла из палаты.

Знай я, что это будут последние слова, которые он от меня услышит, выбрала бы что-нибудь пооригинальней.

Тем не менее, я до последнего не верила, что рак заберет его так быстро. Мне только исполнилось пятнадцать, дедушке -- восемьдесят три. Он продолжал работать, бегал по утрам и выглядел лет на двадцать моложе своего возраста.

Затем болезнь впервые дала о себе знать. Какие-то две-три недели и недостающие года тут же отразились в чертах его лица.

Тем вечером за окном бушевал ветер. Выходя из палаты, я слышала, как по подоконнику забарабанили первые капли дождя.

--- Надеюсь, это случится не завтра, --- вдруг сказал дедушка. --- Мне бы очень не хотелось умереть в такую погоду.

--- Это не случится завтра, --- спокойно сказала я и наклонилась, чтобы поцеловать дедушкину щеку. --- Я люблю тебя.

И я ушла. Вышла под дождь и медленно зашагала в сторону дома.

Проснувшись следующим утром, я подошла к окну. За ним по-прежнему было темно, и я с удивлением осознала, что впервые в жизни встала раньше будильника. Фактически, утро еще не наступило. До школы оставалось несколько часов, а в душе засело ноющее чувство тревоги. Я взяла книгу -- кажется, это был Лавкрафт -- и устроилась на подоконнике.

Помнится, по радио то и дело передавали штормовое предупреждение, и по мере приближение рассвета мне открывались переполненные водой улицы. Вода и грязь, сопровождаемые возгласами ветра, заполняли собой все вокруг.

Дедушка не умер на следующий день. Он умер той ночью. Думаю, от этого я и проснулась.

Сейчас, спустя десять лет, я с ужасом осознаю, что не могу припомнить, как звучал голос моего деда. Я закрываю глаза и стараюсь расслабить сознание, вспоминая детство, проведенном в его доме. Вижу старый топчан, покрытые пылью ордена и желтую занавесь, что висела у входа в дедушкину спальню. За ними тянется старого образца гостиная. Она ничем не отличается от тех, что можно увидеть в домах других стариков из рабочего класса: раскладной диван, ковер на стене, кресла по обе стороны журнального столика, накрытого плетеной циновкой, и древний телевизор, который никогда не замолкает.

Мне вспоминается крохотная прихожая с обогревателем; я слышу яркий запах цитрусовых, чьи корки дедушка каждую зиму сушил для своей настойки. Вспоминается кухня, пиалка из хрусталя, неизменно полная конфет и гигантская копия наручных часов, висевшая в углу. На столе меня ждет чугунная сковорода времен Феликса Дзержинского, а в ней -- традиционная утренняя яичница с молодой картошкой, вкусней которой и быть не может.

По старой привычке, дедушка ест стоя. Он добавляет в картошку соли и с довольным видом наблюдает за тем, как я уплетаю завтрак. Чайник на плите уже начинает посвистывать, привлекая внимания котов, что всю ночь где-то пропадали, а теперь лениво дремлют на подоконнике.

Лежавший в тарелке завтрак вскоре исчез, как исчезнет и сковорода, и стол, на котором она стоит, кухня, и все остальные комнаты. На месте ветхого домика уже давно стоит другой, но память о нем никуда не делась. Я вспоминаю дедушку, заботливо перемешивающего сахар в моей чашке с чаем. Вижу, как он замечает пустую тарелку, и знаю, что будет дальше. Сейчас дед поставит передо мной чай и предложит добавки. Тогда я делаю глубокий вдох и стараюсь не думать ни о чем другом. Все жду, что голос сам всплывет в памяти.

Но он не всплывает, как не старайся. С пугающей точностью я помню слова, интонации, произношение. Короче говоря, что угодно, то только не то, что хочу вспомнить. И, все-таки, мне кажется, услышь я дедушкин голос хоть на мгновенье, -- случайно, и не подозревая, кому он принадлежит -- я бы обязательно его вспомнила.

Так вот, все эти годы дед жил со включенным телевизором. Прошло еще восемь лет, прежде чем я поняла, почему.

\hypertarget{chapter-2}{%
\chapter{~}\label{chapter-2}}

Приближался день рождения Адама, которого я не видела два с половиной года. Очередной день в баре. К тому времени я уже привыкла находиться по другую сторону стойки и читала что-то от Буковски, периодически поглядывая в сторону поддатых посетителей.

--- Освежи мне! --- заплетающимся языком произнес один из них.

После чего толкнул пивной бокал в обратную от меня сторону. Тот проехал пару метров вдоль барной стойки, ударился о стену и с характерным звоном рассыпался на тысячи блестящих стеклышек.

Мне вдруг очень захотелось стать этим бокалом.

На самом деле, мне даже нравилось работать за барной стойкой. Если не учитывать шестнадцатичасовой рабочий день, мизерную зарплату, постоянные недостачи и полное отсутствие чаевых, а также клиентов, в большинстве своем доводящих до исступления\ldots{} О чем я говорила? Ах, да все не так уж плохо.

Сложно сказать, почему, но, стоило мне надеть фартук и впервые перешагнуть эту заветную черту, отделяющую мир бармена от общества простых смертных, я моментально почувствовала себя в своей тарелке. Разбираться в винных сортах, изучать миксологию, работать над созданием собственных коктейлей и временами баловаться флейрингом -- было в этом что-то особенное. Я без труда могла рассказать о любом из напитков куда больше, чем указано на этикетке, посоветовать вино или удивить гостя чем-нибудь экзотичным. Такая работа и впрямь была мне по душе, но, быть может так скажет любой алкоголик.

Увы, несмотря на престижность заведения, где я работала, (а также на тот факт, что местная стойка считалась самой дорогой во всем городе) платили здесь плачевно мало. Настолько мало, что едва хватало на еду и коммуналку. С другой стороны, работа занимала у меня все время, так что, будь у меня лишние деньги, я все равно не успевала бы их потратить. Вероятно, в этом и был секрет выживания сотрудников «Штиля».

--- Плесни еще на посошок, --- на этот раз горе-метатель казенной посуды обошелся без спецэффектов.

Залпом расправился с последней порцией пива и вскоре исчез, оставив «Штиль» без единого посетителя.

-- Итак, на чем мы остановились\ldots{}

Десять страниц, пятнадцать, двадцать. Ничего не менялось. В зале по-прежнему было пусто.

Как и в моем сердце.

Не то, чтоб я ежеминутно думала о своем бывшем. Эти времена уже прошли. Я месяцев семь как выкарабкалась из затяжной депрессии и всячески старалась загрузить себя работой. Как я уже сказала, платили в «Штиле» смехотворно мало. Ну, кто будет так рвать задницу ради каких-то ста пятидесяти баксов в месяц?

К счастью, работа всецело меня выматывала, что оказалось главным из ее достоинств. После смены на всё про всё оставалось часов шесть, и это без учета дороги, а потом обратно за стойку. Знаете, нелегко быть в депрессии, когда на нее просто не остается времени. Мне чуть ли не заранее приходилось планировать свои нервные срывы.

Так вот, наступило двадцатое апреля две тысячи шестнадцатого года -- день толерантности марихуаны, а по совместительству и день рождения одного небезызвестного немецкого политика еврейского происхождения. День обещал быть куда более радужным, чем карьера последнего. Мне хотелось выглянуть на улицу, но за стойкой, само собой, окон не было.

Не было их и в зале-ресторане. Не было в боулинге, и на кухне тоже ни одного окошка не наблюдалось. Мне это всегда казалось странным. Будто у проектировщиков имелась какая-то тайная нелюбовь к сквозным отверстиям. С другой стороны, в зале ежедневно околачивались стриптизерши, проститутки, эскортницы и прочего рода содержанки, так что что-то здесь явно не вяжется.

В любом случае, за стойкой было темно как в причинном месте. Приходилось читать в свете неоновых вывесок с изображениями Моргана, Бушмилса, Дэниэлса и прочих хорошо мне знакомых парней. Если алое освещение надоест, можно чуток подвинуться влево. Пара шагов вдоль стойки, и алые страницы становятся зелеными в свете лозунгов Егеря, Бехеровки или Ксенты.

Говорят, человеку нужно всего три недели, чтобы привыкнуть к дискомфорту или избавиться от вредной привычки. Увы, я провела в «Штиле» не один год, но так и не смогла свыкнуться с отсутствием окон.

Единственным приятным моментом, помимо доступа к алкоголю, была входная дверь. Сплошь стеклянная, она находилась слева от моего рабочего места и открывала взгляду небольшой клочок улицы. Судя по всему, за пределами «Штиля» и впрямь стояла восхитительная погода. Сияло весеннее солнце; деревья уже позеленели и, движимые легким ветерком, отбрасывали причудливые тени на аллейку. Отсюда мне был виден каменный фонтан да пара скамеек. Брызги воды прямо-таки светились под лучами полуденного солнца. От этой радужной картины в душе сделалось как-то грустно, так что я отправилась варить кофе.

--- Ты будешь делать кофе? --- моментально среагировала одна из официанток, которых в «Штиле» звали просто фицами.

--- Кофе? --- эхом отозвались из зала.

Каким-то мистическим образом они всегда это чувствовали. Девушки по инерции стекались у барной стоило мне лишь подумать о том, чтобы варить кофе.

--- Буду.

--- И мне сделай, Лизочек, --- Вера сняла с полки пустую чашку, чмокнула меня в щеку и побежала обслуживать очередной столик.

Терпеть не могла, когда меня так называли.

За ней оживились и другие сотрудники развлекательного комплекса. Стало ясно, что одной чашкой кофе здесь не обойдется.

Вернувшись на прежнее место, я заметила, что у фонтана появился бездомный. Уже наполовину раздетый, он продолжал смущать прохожих и неспешно стягивал с себя одежду пока, наконец, не остался в чем мать родила. Затем товарищ без лишних раздумий погрузился в фонтан, лениво растянулся, опираясь на бортик, и уже беззаботно потягивал пивасик.

Вот, в принципе, и все, что вам нужно знать о городе, в котором я родилась.

***

Звонил телефон. Кассир сняла трубку и тут же передала ее мне со словами:

--- Это по твою душу.

Этажом ниже просили сделать пятнадцать детских мохито и четыре взрослых. Намечался очередной детский день рождения. С ума сойти. В мое время мы дарили друг другу наклейки и радовались мороженому из МакДональдс, а не арендовали целый этаж в ночном клубе вместе с аниматорами, лайт-джеями и диджеями. А ведь мне едва ли исполнилось двадцать два года.

--- И побыстрее! Мелюзга уже вовсю вопит! --- голос хриплый, чуть ли не потусторонний, и прямо-таки гавкает, а не говорит.

Почти два десятка мохито -- не самый удачный выбор, когда за стойкой нет крашмейкера. Единственный работающий аппарат находился в ночном клубе, так что мне постоянно приходилось бегать туда-сюда за колотым льдом.

Официантка уже стояла на раздаче, когда я положила трубку.

--- Я бы тебе помогла, --- сказала Вера, --- но начальство велело даже в туалет не выходить. Я в зале одна осталась.

--- Ничего, я схожу вниз\ldots{}

--- Пипец. Скоро обоссусь.

--- А ты пока принеси стаканы.

Вероника кивнула и уже собиралась уходить, когда я позвала ее по имени.

--- Слушай, а кто звонил?

--- Ну, Рыжая.

Я вскинула бровь.

\end{document}
