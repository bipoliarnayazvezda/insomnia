% Options for packages loaded elsewhere
\PassOptionsToPackage{unicode}{hyperref}
\PassOptionsToPackage{hyphens}{url}
%
\documentclass[
]{book}
\usepackage{amsmath,amssymb}
\usepackage{lmodern}
\usepackage{iftex}
\ifPDFTeX
  \usepackage[T1]{fontenc}
  \usepackage[utf8]{inputenc}
  \usepackage{textcomp} % provide euro and other symbols
\else % if luatex or xetex
  \usepackage{unicode-math}
  \defaultfontfeatures{Scale=MatchLowercase}
  \defaultfontfeatures[\rmfamily]{Ligatures=TeX,Scale=1}
\fi
% Use upquote if available, for straight quotes in verbatim environments
\IfFileExists{upquote.sty}{\usepackage{upquote}}{}
\IfFileExists{microtype.sty}{% use microtype if available
  \usepackage[]{microtype}
  \UseMicrotypeSet[protrusion]{basicmath} % disable protrusion for tt fonts
}{}
\makeatletter
\@ifundefined{KOMAClassName}{% if non-KOMA class
  \IfFileExists{parskip.sty}{%
    \usepackage{parskip}
  }{% else
    \setlength{\parindent}{0pt}
    \setlength{\parskip}{6pt plus 2pt minus 1pt}}
}{% if KOMA class
  \KOMAoptions{parskip=half}}
\makeatother
\usepackage{xcolor}
\usepackage{longtable,booktabs,array}
\usepackage{calc} % for calculating minipage widths
% Correct order of tables after \paragraph or \subparagraph
\usepackage{etoolbox}
\makeatletter
\patchcmd\longtable{\par}{\if@noskipsec\mbox{}\fi\par}{}{}
\makeatother
% Allow footnotes in longtable head/foot
\IfFileExists{footnotehyper.sty}{\usepackage{footnotehyper}}{\usepackage{footnote}}
\makesavenoteenv{longtable}
\usepackage{graphicx}
\makeatletter
\def\maxwidth{\ifdim\Gin@nat@width>\linewidth\linewidth\else\Gin@nat@width\fi}
\def\maxheight{\ifdim\Gin@nat@height>\textheight\textheight\else\Gin@nat@height\fi}
\makeatother
% Scale images if necessary, so that they will not overflow the page
% margins by default, and it is still possible to overwrite the defaults
% using explicit options in \includegraphics[width, height, ...]{}
\setkeys{Gin}{width=\maxwidth,height=\maxheight,keepaspectratio}
% Set default figure placement to htbp
\makeatletter
\def\fps@figure{htbp}
\makeatother
\setlength{\emergencystretch}{3em} % prevent overfull lines
\providecommand{\tightlist}{%
  \setlength{\itemsep}{0pt}\setlength{\parskip}{0pt}}
\setcounter{secnumdepth}{5}
\usepackage{booktabs}
\ifLuaTeX
  \usepackage{selnolig}  % disable illegal ligatures
\fi
\usepackage[]{natbib}
\bibliographystyle{plainnat}
\IfFileExists{bookmark.sty}{\usepackage{bookmark}}{\usepackage{hyperref}}
\IfFileExists{xurl.sty}{\usepackage{xurl}}{} % add URL line breaks if available
\urlstyle{same} % disable monospaced font for URLs
\hypersetup{
  pdftitle={Бессонница},
  pdfauthor={Александра Булгакова},
  hidelinks,
  pdfcreator={LaTeX via pandoc}}

\title{Бессонница}
\author{Александра Булгакова}
\date{2023-02-11}

\begin{document}
\maketitle

{
\setcounter{tocdepth}{1}
\tableofcontents
}
\hypertarget{ux44dux43fux438ux433ux440ux430ux444}{%
\chapter*{Эпиграф}\label{ux44dux43fux438ux433ux440ux430ux444}}
\addcontentsline{toc}{chapter}{Эпиграф}

\emph{Эта история не о биполярном расстройстве личности, хотя оно здесь есть.}

\emph{Эта история не о любви, хотя она здесь есть.}

\emph{Эта история не о депрессии, хотя она здесь есть.}

\emph{Эта история не о счастье, хотя оно здесь есть.}

\emph{Эта история не о жизни, хотя она здесь есть.}

\emph{Эта история не о смерти, хотя она здесь есть.}

\hypertarget{ux447ux430ux441ux442ux44c-i.-ux431ux435ux441ux441ux43eux43dux43dux438ux446ux430}{%
\chapter*{Часть I. Бессонница}\label{ux447ux430ux441ux442ux44c-i.-ux431ux435ux441ux441ux43eux43dux43dux438ux446ux430}}
\addcontentsline{toc}{chapter}{Часть I. Бессонница}

\hypertarget{chapter-1}{%
\chapter{~}\label{chapter-1}}

Как правило, после внепланового завершения серьезных отношений даже сильные люди впадают в своего рода эмоциональную спячку. Так или иначе, приходится изолироваться от несостоявшейся второй половинки. Перебороть истерики и утихомириться в какой-нибудь тесной квартирке с мусорным ведром под раковиной, микроволновкой для полуфабрикатов, своими немногочисленными пожитками и бутылкой, о которой ты уже толком и не помнишь, что внутри, но к которой все равно продолжаешь периодически прикладываться с крайне отсутствующим видом.

Сколько бы лет тебе не было, -- двадцать пять или пятьдесят, неважно -- вдруг начинаешь вести жизнь завтрашнего пенсионера. Прекращаешь слушать музыку, за едой думаешь исключительно о еде и никогда не выключаешь телевизор. Последний ты даже не смотришь, но все равно оставляешь телек трещать на фоне. Чтобы чувствовать себя менее одиноко.

В моменты как этот всякая деятельность обращается бездействием. С каждым днем ты все больше и больше теряешь интерес к жизни, но все чаще начинаешь задумываться о смерти, которая становится чем-то ожидаемым. Логичным завершением одного тотального фиаско под названием твоя собственная жизнь.

Мать моего отца умерла за месяц до моего рождения. Дедушка же прожил еще полторы декады. Все эти годы он спал со включенным телевизором, который толком не смотрел. Еще дед никогда не смеялся и очень редко улыбался. Разве что мне. После его смерти я нашла коробку со старыми снимками. Практически на каждом из них лицо моего деда освещала широченная улыбка, но отец лишь пожал плечами. Сказал только, что со дня похорон своей жены дед так ни разу не засмеялся.

Он умер хмурым январским утром. Сгорел от внезапно давшего о себе знать рака. Дед пережил голодомор, войну, распад Советского Союза и трех украинских президентов, а потом почувствовал острую боль в желудке и умер спустя месяц. Это вообще легально?

--- Я люблю тебя --- сказала я дедушке, и вышла из палаты.

Знай я, что это будут последние слова, которые он от меня услышит, выбрала бы что-нибудь пооригинальней.

Тем не менее, я до последнего не верила, что рак заберет его так быстро. Мне только исполнилось пятнадцать, дедушке -- восемьдесят три. Он продолжал работать, бегал по утрам и выглядел лет на двадцать моложе своего возраста.

Затем болезнь впервые дала о себе знать. Какие-то две-три недели и недостающие года тут же отразились в чертах его лица.

Тем вечером за окном бушевал ветер. Выходя из палаты, я слышала, как по подоконнику забарабанили первые капли дождя.

--- Надеюсь, это случится не завтра, --- вдруг сказал дедушка. --- Мне бы очень не хотелось умереть в такую погоду.

--- Это не случится завтра, --- спокойно сказала я и наклонилась, чтобы поцеловать дедушкину щеку. --- Я люблю тебя.

И я ушла. Вышла под дождь и медленно зашагала в сторону дома.

Проснувшись следующим утром, я подошла к окну. За ним по-прежнему было темно, и я с удивлением осознала, что впервые в жизни встала раньше будильника. Фактически, утро еще не наступило. До школы оставалось несколько часов, а в душе засело ноющее чувство тревоги. Я взяла книгу -- кажется, это был Лавкрафт -- и устроилась на подоконнике.

Помнится, по радио то и дело передавали штормовое предупреждение, и по мере приближение рассвета мне открывались переполненные водой улицы. Вода и грязь, сопровождаемые возгласами ветра, заполняли собой все вокруг.

Дедушка не умер на следующий день. Он умер той ночью. Думаю, от этого я и проснулась.

Сейчас, спустя десять лет, я с ужасом осознаю, что не могу припомнить, как звучал голос моего деда. Я закрываю глаза и стараюсь расслабить сознание, вспоминая детство, проведенном в его доме. Вижу старый топчан, покрытые пылью ордена и желтую занавесь, что висела у входа в дедушкину спальню. За ними тянется старого образца гостиная. Она ничем не отличается от тех, что можно увидеть в домах других стариков из рабочего класса: раскладной диван, ковер на стене, кресла по обе стороны журнального столика, накрытого плетеной циновкой, и древний телевизор, который никогда не замолкает.

Мне вспоминается крохотная прихожая с обогревателем; я слышу яркий запах цитрусовых, чьи корки дедушка каждую зиму сушил для своей настойки. Вспоминается кухня, пиалка из хрусталя, неизменно полная конфет и гигантская копия наручных часов, висевшая в углу. На столе меня ждет чугунная сковорода времен Феликса Дзержинского, а в ней -- традиционная утренняя яичница с молодой картошкой, вкусней которой и быть не может.

По старой привычке, дедушка ест стоя. Он добавляет в картошку соли и с довольным видом наблюдает за тем, как я уплетаю завтрак. Чайник на плите уже начинает посвистывать, привлекая внимания котов, что всю ночь где-то пропадали, а теперь лениво дремлют на подоконнике.

Лежавший в тарелке завтрак вскоре исчез, как исчезнет и сковорода, и стол, на котором она стоит, кухня, и все остальные комнаты. На месте ветхого домика уже давно стоит другой, но память о нем никуда не делась. Я вспоминаю дедушку, заботливо перемешивающего сахар в моей чашке с чаем. Вижу, как он замечает пустую тарелку, и знаю, что будет дальше. Сейчас дед поставит передо мной чай и предложит добавки. Тогда я делаю глубокий вдох и стараюсь не думать ни о чем другом. Все жду, что голос сам всплывет в памяти.

Но он не всплывает, как не старайся. С пугающей точностью я помню слова, интонации, произношение. Короче говоря, что угодно, то только не то, что хочу вспомнить. И, все-таки, мне кажется, услышь я дедушкин голос хоть на мгновенье, -- случайно, и не подозревая, кому он принадлежит -- я бы обязательно его вспомнила.

Так вот, все эти годы дед жил со включенным телевизором. Прошло еще восемь лет, прежде чем я поняла, почему.

\hypertarget{chapter-2}{%
\chapter{~}\label{chapter-2}}

Приближался день рождения Адама, которого я не видела два с половиной года. Очередной день в баре. К тому времени я уже привыкла находиться по другую сторону стойки и читала что-то от Буковски, периодически поглядывая в сторону поддатых посетителей.

--- Освежи мне! --- заплетающимся языком произнес один из них.

После чего толкнул пивной бокал в обратную от меня сторону. Тот проехал пару метров вдоль барной стойки, ударился о стену и с характерным звоном рассыпался на тысячи блестящих стеклышек.

Мне вдруг очень захотелось стать этим бокалом.

На самом деле, мне даже нравилось работать за барной стойкой. Если не учитывать шестнадцатичасовой рабочий день, мизерную зарплату, постоянные недостачи и полное отсутствие чаевых, а также клиентов, в большинстве своем доводящих до исступления\ldots{} О чем я говорила? Ах, да все не так уж плохо.

Сложно сказать, почему, но, стоило мне надеть фартук и впервые перешагнуть эту заветную черту, отделяющую мир бармена от общества простых смертных, я моментально почувствовала себя в своей тарелке. Разбираться в винных сортах, изучать миксологию, работать над созданием собственных коктейлей и временами баловаться флейрингом -- было в этом что-то особенное. Я без труда могла рассказать о любом из напитков куда больше, чем указано на этикетке, посоветовать вино или удивить гостя чем-нибудь экзотичным. Такая работа и впрямь была мне по душе, но, быть может так скажет любой алкоголик.

Увы, несмотря на престижность заведения, где я работала, (а также на тот факт, что местная стойка считалась самой дорогой во всем городе) платили здесь плачевно мало. Настолько мало, что едва хватало на еду и коммуналку. С другой стороны, работа занимала у меня все время, так что, будь у меня лишние деньги, я все равно не успевала бы их потратить. Вероятно, в этом и был секрет выживания сотрудников «Штиля».

--- Плесни еще на посошок, --- на этот раз горе-метатель казенной посуды обошелся без спецэффектов.

Залпом расправился с последней порцией пива и вскоре исчез, оставив «Штиль» без единого посетителя.

-- Итак, на чем мы остановились\ldots{}

Десять страниц, пятнадцать, двадцать. Ничего не менялось. В зале по-прежнему было пусто.

Как и в моем сердце.

Не то, чтоб я ежеминутно думала о своем бывшем. Эти времена уже прошли. Я месяцев семь как выкарабкалась из затяжной депрессии и всячески старалась загрузить себя работой. Как я уже сказала, платили в «Штиле» смехотворно мало. Ну, кто будет так рвать задницу ради каких-то ста пятидесяти баксов в месяц?

К счастью, работа всецело меня выматывала, что оказалось главным из ее достоинств. После смены на всё про всё оставалось часов шесть, и это без учета дороги, а потом обратно за стойку. Знаете, нелегко быть в депрессии, когда на нее просто не остается времени. Мне чуть ли не заранее приходилось планировать свои нервные срывы.

Так вот, наступило двадцатое апреля две тысячи шестнадцатого года -- день толерантности марихуаны, а по совместительству и день рождения одного небезызвестного немецкого политика еврейского происхождения. День обещал быть куда более радужным, чем карьера последнего. Мне хотелось выглянуть на улицу, но за стойкой, само собой, окон не было.

Не было их и в зале-ресторане. Не было в боулинге, и на кухне тоже ни одного окошка не наблюдалось. Мне это всегда казалось странным. Будто у проектировщиков имелась какая-то тайная нелюбовь к сквозным отверстиям. С другой стороны, в зале ежедневно околачивались стриптизерши, проститутки, эскортницы и прочего рода содержанки, так что что-то здесь явно не вяжется.

В любом случае, за стойкой было темно как в причинном месте. Приходилось читать в свете неоновых вывесок с изображениями Моргана, Бушмилса, Дэниэлса и прочих хорошо мне знакомых парней. Если алое освещение надоест, можно чуток подвинуться влево. Пара шагов вдоль стойки, и алые страницы становятся зелеными в свете лозунгов Егеря, Бехеровки или Ксенты.

Говорят, человеку нужно всего три недели, чтобы привыкнуть к дискомфорту или избавиться от вредной привычки. Увы, я провела в «Штиле» не один год, но так и не смогла свыкнуться с отсутствием окон.

Единственным приятным моментом, помимо доступа к алкоголю, была входная дверь. Сплошь стеклянная, она находилась слева от моего рабочего места и открывала взгляду небольшой клочок улицы. Судя по всему, за пределами «Штиля» и впрямь стояла восхитительная погода. Сияло весеннее солнце; деревья уже позеленели и, движимые легким ветерком, отбрасывали причудливые тени на аллейку. Отсюда мне был виден каменный фонтан да пара скамеек. Брызги воды прямо-таки светились под лучами полуденного солнца. От этой радужной картины в душе сделалось как-то грустно, так что я отправилась варить кофе.

--- Ты будешь делать кофе? --- моментально среагировала одна из официанток, которых в «Штиле» звали просто фицами.

--- Кофе? --- эхом отозвались из зала.

Каким-то мистическим образом они всегда это чувствовали. Девушки по инерции стекались у барной стоило мне лишь подумать о том, чтобы варить кофе.

--- Буду.

--- И мне сделай, Лизочек, --- Вера сняла с полки пустую чашку, чмокнула меня в щеку и побежала обслуживать очередной столик.

Терпеть не могла, когда меня так называли.

За ней оживились и другие сотрудники развлекательного комплекса. Стало ясно, что одной чашкой кофе здесь не обойдется.

Вернувшись на прежнее место, я заметила, что у фонтана появился бездомный. Уже наполовину раздетый, он продолжал смущать прохожих и неспешно стягивал с себя одежду пока, наконец, не остался в чем мать родила. Затем товарищ без лишних раздумий погрузился в фонтан, лениво растянулся, опираясь на бортик, и уже беззаботно потягивал пивасик.

Вот, в принципе, и все, что вам нужно знать о городе, в котором я родилась.

***

Звонил телефон. Кассир сняла трубку и тут же передала ее мне со словами:

--- Это по твою душу.

Этажом ниже просили сделать пятнадцать детских мохито и четыре взрослых. Намечался очередной детский день рождения. С ума сойти. В мое время мы дарили друг другу наклейки и радовались мороженому из МакДональдс, а не арендовали целый этаж в ночном клубе вместе с аниматорами, лайт-джеями и диджеями. А ведь мне едва ли исполнилось двадцать два года.

--- И побыстрее! Мелюзга уже вовсю вопит! --- голос хриплый, чуть ли не потусторонний, и прямо-таки гавкает, а не говорит.

Почти два десятка мохито -- не самый удачный выбор, когда за стойкой нет крашмейкера. Единственный работающий аппарат находился в ночном клубе, так что мне постоянно приходилось бегать туда-сюда за колотым льдом.

Официантка уже стояла на раздаче, когда я положила трубку.

--- Я бы тебе помогла, --- сказала Вера, --- но начальство велело даже в туалет не выходить. Я в зале одна осталась.

--- Ничего, я схожу вниз\ldots{}

--- Пипец. Скоро обоссусь.

--- А ты пока принеси стаканы.

Вероника кивнула и уже собиралась уходить, когда я позвала ее по имени.

--- Слушай, а кто звонил?

--- Ну, Рыжая.

Я вскинула бровь.

-- Ты уверена?

-- Еще б мне не быть уверенной. Здесь только мы и есть. По одной на этаж. А чё такое?

-- Да голос, блин, какой-то жуткий. Как с того света, честное слово. Это точно Рыжая? Как-то на нее не похоже.

-- Чего не похоже? На Рыжую не похоже? Да она опять до шести утра в клубе бухала!

Вера закатила глаза и, прихватив поднос, ушла на мойку. Порой создавалось впечатление, что та завидует коллеге.

Как ни странно, внизу меня действительно ждала Тоня.

-- Привет, Рыжик, как твое похмелье?

Милая девушка, только временами врывается как гром среди ясного неба. Говорила она много, громко и очень быстро. Словом, Тоня с легкостью могла бы стать отличным примером экспрессии в бытовой речи.

-- ЛИЗА, ЛИЗА, ЛИЗА! -- закричала она, перекрывая шумы клубного подземелья. -- Сделай мне чего-нибудь этакого!

-- Этакакого?

-- Ну, чтоб получше стало, а то я щаз точно ласты откину! Вот увидишь, откину! -- она наполнила хайбол льдом и теперь жадно прижимала его к своему лицу. -- Так и шо? Ты сделаешь?

-- Сделаю. Как с заказом закончу.

-- Пожа-а-алуйста, сделай сейчас, а? А я тебе лёд пока замучу.

Ну, как тут откажешь?

До окончания дня оставалось еще немало. Я как раз колдовала над своим противопохмельным отваром, когда телефон в кармане моих джинс завибрировал. На экране меня ждало входящее сообщение от очередного бородатого незнакомца.

Его звали Вениамин.

«Как жизнь молодая?» -- спрашивал он.

Обычно я игнорирую подобные попытки знакомства, но\ldots{} То ли дело здесь было в четыре-двадцать, то ли в том, что мне осточертело стоять за стойкой, а, может, в том, что образ Адама то и дело упрямо выныривал из памяти -- понятия не имею. В общем, я ответила.

\hypertarget{chapter-3}{%
\chapter{~}\label{chapter-3}}

Все та же душная мгла повисла над городом, заботливо обволакивая каждую из его составляющих. До рассвета оставалось еще минимум шесть часов, что вовсе не мешало местной пьяни устраивать под окнами традиционные субботние концерты. Они орали и галдели так, словно эта ночь была последним шансом напиться.

Растянувшись поперек кровати, я пыталась сосредоточиться на книге. В ней говорилось о зимних лесах с их невероятными пейзажами, конных упряжках, размашистых балах и дворцах, одетых в блестящее убранство. А за окном лаяли псы, неслись машины и шумели районные алкоголики, напоминая мне, где на самом деле я нахожусь. Вероятно, дело было в моем душевном расстройстве: маниакальная фаза сменилась депрессивной. Мне вдруг стало неописуемо грустно и обидно за то, какой жизнью приходится жить. Еще и Адам постоянно лез в голову.

\emph{Не думай о нем.}

Чертов Адам с его обаятельной улыбкой да ямочками на щеках.

Захлопнув книгу, я отважилась закрыть окно

(я говорю «отважилась» потому, как термометр показывал под тридцать, а кондиционера в моей квартирке отродясь не водилось)

чтобы хоть немного приглушить раздражающие вопли. Подавляющему большинству людей всегда было плевать на беспредел, творящийся в этом районе. Несколько лет назад в парке под моим домом до смерти забили женщину. Другую изнасиловали и задушили чуть подальше, оставив валяться среди ив, кленов и прочей растительности, что окружала аллею. Нашли ее полностью голую около семи утра. Я прекрасно помню то утро потому, что именно в это время шла через парк поздравить своего отца с юбилеем, а еще потому, что я эйдетик.

Помнится, третью женщину убили как раз-таки на отрезке пути, что вел от одного из этих парков к другому. Она лежала в сквере у онкологического отделения. Снега той зимой были нешуточные, так что несчастную обнаружили лишь в апреле. Вид у нее был, прямо говоря, не очень, отчего распознать черты лица уже не представлялось возможным. Труп опознать не могли, и убитая номер три провела еще года полтора в морге той самой больницы пока дело, наконец, не закрыли.

Есть еще один небольшой парк за детским садом, где я когда-то училась падать с велосипеда. Там тоже кого-то отправили в мир иной. Еще один -- в пяти минутах ходьбы и в нем вообще регулярно находят трупы. Половина из них, конечно, местные торчки, которые не могут правильно рассчитать дозу, но есть и те, кто умер слегка менее приятной и чуть более насильственной смертью. К примеру, начальник районного отделения полиции.

Кстати говоря, по дороге от моего дома в сторону последней из описанных локаций тоже есть парк. Не знаю, что там у них насчет убийств, но однажды у меня закончились сигареты, и я вышла на их поиски. Как раз достигла середины этого крохотного, тянувшегося вдоль проезжей части сквера, когда незнакомый мужчина подбежал ко мне и со всей дури заехал в челюсть.

Я ошалело взглянула на незнакомца и расхохоталась.

Потное, осунувшееся лицо, зрачки как блюдца и челюсть пошла ходуном -- да это тип был явно под белым. Он, не моргая, таращился на меня, очевидно, прикидывая, стоит ли нанести повторный удар. Потом как будто о чем-то вспомнил. Схватился за голову и кинулся прочь.

-- Постой, паровоз! -- кулаки у меня чесались еще с вечера. -- Куда ты погнал?

-- А НУ СТОЯТЬ, ПЕДРИЛА! -- заорала моя мания.

Ей было все равно на то, что тот тип мог уложить меня одной правой.

-- Я не усну, пока не узнаю, зачем ты меня двинул!

Но мужик оказался бывалый. Он перепрыгнул через забор и довольно быстро исчез из поля зрения, так что мне пришлось возобновить поиски никотина.

Короче говоря, парки в этом районе так себе.

Так вот, обычно местный ночной беспредел меня едва ли волновал. Порой я даже сама являлась его частью. Но не сегодня. Этим вечером моя психика была особенно уязвима.

\emph{Не думай о нем.}

Биполярка вообще странная штука. То ты пляшешь, поешь, и трахаешься как под спидами, а затем на полставки подрабатываешь комиком и опять трахаешься. Не можешь спать (слишком перевозбужден) и не можешь есть потому, что чувство голода куда-то улетучивается. Тем не менее, окружающие считают тебя душой компании и вечно зовут на тусовки. Самые отважные умудряются еще и влюбиться.

Затем наступает депрессивный период, и ты опять не можешь спать, потому как слишком печален. Есть, кстати, не хочется по той же причине. Все самые страшные и болезненные воспоминания вырываются наружу, так что истерикам нет предела. Сначала это кратковременная злость, затем пара гордых слезинок, а потом истерика, сменяющаяся полноценной панической атакой. Длится она от пары минут до одного-двух часов, в зависимости от окружения и желания жить в целом. Зачастую, в борьбе с припадками, мне приходилось оставлять на конечностях порезы и ссадины различной глубины. Чтобы хоть как-то угомониться.

По весне завидев шрамы, люди почему-то думают, что я каждый день начинаю с того, что сажусь на край кровати с намерением вскрыть себе вены. Как будто я не в курсе, где эти самые вены находятся. Все дело в том, что временами острая физическая боль становится единственным способ отвлечь себя от печали. Порезы -- это не страшно. Разбитое сердце, душевная боль, одиночество и, в конечном итоге, немое отчаяние -- вот это куда страшнее, ведь именно они вызывают желание распрощаться с жизнью.

Такими днями общаться с людьми совершенно невозможно. Они мало что понимают. Вечно ждут от тебя забавных историй, обилия шуток и прочих признаков веселья. Люди привыкли видеть меня такой. Всегда и везде. Это смешно, но порой, соприкасаясь со мной во время депрессивной фазы, некоторые из них чуть ли не злятся на мою грусть и молчаливость. Капризничают и топают ножками, требуя веселья, к которому так привыкли. Разочаровываются, называют скучной и оскорбляются нежеланием вести бурную ораторскую деятельность. Проверку депрессией проходят единицы, но даже они не способны меня спасти.

В итоге внутри не остается ничего кроме боли. И скоро ты вновь призраком слоняешься по городу в кромешном одиночестве.

Биполярное аффективное расстройство -- вот официальное название того, чем я страдаю. Или наслаждаюсь. Этот вопрос никогда толком не проясняется. Впервые увидев список симптомов БАР-а, я, мягко говоря, охренела. Выглядело все так, словно кто-то просто наблюдал за мной на протяжении двадцати лишним лет, а затем собрал воедино все черты моего характера и назвал это биполярным расстройством. Отвратительное, скажу вам, чувство. Вот живешь ты себе, жалуешься на жару, пьешь вино и пишешь свои книги, а затем вдруг понимаешь, что никакая ты не творческая личность, а всего лишь очередная психбольная. Все твои достоинства, отрицательные качества, привычки и даже так называемые изюминки -- это все лишь признаки психического расстройства, которым болеет 2,4 \% земного населения. Как вам такая новость?

Кухонные часы показывали полночь.

Суббота закончилась. Пьяницы остались.

\hypertarget{chapter-4}{%
\chapter{~}\label{chapter-4}}

Тринадцатое утро мая две тысячи шестнадцатого года встретило меня умеренной погодой -- редкое явление в моих родных краях. За окном обнадеживающе светило солнце, но было довольно прохладно. Самое то для дальней поездки. Дни работы за стойкой приучили меня выползать из постели сразу же после пробуждения. Или вообще в нее не ложиться, если чувствуешь, что не сможешь проснуться вовремя.

Это работало, даже если я уснула с рассветом, так что я в кои-то веки не пропустила свой автобус и даже на него не опаздывала. Сказать больше, мне удалось сделать укладку и нарисовать шикарные стрелки. Вот они, прелести выходного дня.

Выпив кофе, я прихватила небольшой, но довольно увесистый чемодан, и поспешила покинуть пределы родного гетто. Уже на улице остановилась напротив сигаретного киоска, который, казалось, находился здесь всю жизнь.

Я позволила себе слегка забыть о времени, зачем-то разглядывая многочисленные ряды пестрых сигаретных пачек. К тому моменту я не курила уже практически пять месяцев и, как мне думалось, не собиралась. Но что-то заставило меня потянуться в карман за парочкой купюр. Наверное, все дело было в волнении, а точнее, в его отсутствии. Мне всегда думалось, что встречи с незнакомцами тем и прекрасны: ты ничего от них не ждешь. Как и внутри, внешне я оставалась абсолютно спокойной. Только вот внутренний голос откуда-то из глубин подсознания мягко напоминал, что обычно паника подкрадывается ко мне в последний момент. Меньше всего на свете мне хотелось обнаружить себя поддающейся нервному припадку где-нибудь на границе двух внезапно враждующих стран.

-- Пачку Винстона. Синего.

Я без особых зазрений совести расплатилась за сигареты. Так и закончилась моя борьба с никотиновой зависимостью.

Автобус отходил в четверть десятого, и теперь я таки на него опаздывала. В салон я вошла за секунду до отправления, соблюдая старые добрые традиции. Такой вот я человек: ну просто ненавижу ждать, а потому никогда не выхожу из дому заранее. На этот раз мне повезло. Дверь за моей спиной захлопнулась, автобус тронулся и спустя четверть часа я уже наблюдала за бесконечными полями да деревьями, что проносились за окном. Тем днем я читала «Бродяг Дхармы», что очень кстати описывало наступивший период моей жизни. Книга уже подходила к середине, а мы все ехали и ехали. Несколько раз проезжали какие-то населенные пункты, -- ПГТ или вроде того -- но было заметно, что до Крыма по-прежнему как до Луны.

Откровенно говоря, в Крыму я не была лет шесть, да и настоящих отношений у меня уже года три как не было. Словом, я понятия не имела, что меня ждет. Жизненный опыт подсказывал, что надежней всего будет не ждать ничего. В конце концов, рушащиеся надежды -- это всегда неприятно и лучший способ избежать подобной ситуации: попросту их не иметь.

Часов через шесть автобус, наконец, остановился. Водитель объявил, что через полчаса будем на границе, и отправился пить кофе. Внутри меня что-то ёкнуло. Впервые за много-много лет. Это меня обескуражило и я поспешила выйти на свежий воздух.

Одного быстрого взгляда хватило, дабы понять, что я нахожусь прямиком в центре какого-то украинского захолустья. Здесь люди курили под табличками «Не курить!», громко ругались матом при детишках и сидели на асфальте, когда рядом пустовала скамейка. В принципе, ничего необычного. Мы, южане, и не такое видали.

До отправления оставалось минут двадцать пять, так что я заглянула в одну из местных лавчонок. Это крохотное помещение едва превышало размеры туалета хрущевки. Все же, каким-то мистическим образом в него вместилась барная стойка, кофемашина, парочка стульев со столиком, телевизор и даже одно окно. Правда небольшое.

Еще внутри безымянного заведения обнаружились бармен, бариста, повар, продавец, кассир и фиц в одном лице. Девушка так резко выпрыгнула из-за стойки, что я бы обязательно подскочила, не будь я в тот момент где-то далеко в своих мыслях.

-- Посетитель! -- радостно воскликнула девушка. -- Вы -- наш первый посетитель за сегодня! Что будете?

Я слегка удивилась отсутствия акцента, характерного для подобных поселений. Затем взглянула на часы. Дело шло к вечеру. Поздновато как для первого посетителя.

-- Капучино.

-- Но у нас только латте и американо\ldots{}

-- Латте вполне подойдет.

Девушка кивнула и принялась возиться с кофемашиной. Та издавала предсмертные звуки, и я очень надеялась, что успею выпить кофе прежде чем она развалится.

-- Меня зовут Тома, -- представилась девушка. -- Вы недавно у нас?

По всем канонам современного бодипозитивного общества Тома выглядела вполне привлекательно. Это была высокая брюнетка с яркими губами и практически под ноль стриженными волосами. Не толстая и не худая. О таких обычно говорят «в теле». Черты лица -- очень выразительные и чувственные. Все это в совокупности с откровенно мужиковатой одеждой походило на стиль, популярный во времена Твигги.

-- Проездом, -- ответила я и расплатилась за кофе.

Тома окинула меня взглядом.

-- Красивые волосы, -- заметила она. -- Настоящие?

Что ж, дамы и господа, а вот и лидер среди наиболее часто задаваемых мне вопросов.

-- Более чем.

-- А далеко едешь-то?

-- В Севастополь.

-- Ух, далековато. И что там, в Севастополе?

Этот вопрос вызвал в моей голове небезызвестную цитату из «От заката до рассвета», но я как-то удержалась.

-- Мужчина.

-- Хороший?

-- Не исключено, что моей мечты, -- я пожала плечами. -- Но это не точно.

Протяжно вздохнув, Тома облокотилась о стойку.

-- Я тоже была здесь проездом, -- заметила она.

-- И что так?

-- И у меня тоже был мужчина. Как раз возвращалась к нему, когда парня встретила. Влюбилась. Вот разводимся с ним сейчас.

-- С которым из них? -- не поняла я.

Незнакомцы, отчего-то, так любят изливать мне душу. Возможно, как-то ощущают мою эмпатию, но в подобные моменты я всегда чувствую себя неловко, если у меня вдруг все в порядке.

Что ни говори, я в кои-то веки не страдала из-за разбитого сердца, не болталась на дне депрессии, куда обычно приводят творческие амбиции, и даже была немножечко счастлива. Мне сделалось неудобно. Пожалуй, никому не должно быть неловко за отсутствие жизненного дерьма, но мне почему-то стало. Я подумала, что, возможно, именно так себя и чувствовали мои друзья на протяжении парочки последних лет.

-- Да со вторым уже.

Я подняла стаканчик с кофе.

-- За несбывшиеся мечты.

Тома улыбнулась, но как-то грустно. Можно понять.

Я вспомнила о времени. Минута до отправления. Ну, все как обычно.

-- Мне пора. Спасибо за кофе.

-- Как тебя звать-то? -- вспомнила Тома.

-- Елизавета Васляева.

-- Ну, удачи тебе, Елизавета Васляева. Главное, замуж не выходи.

-- Постараюсь, -- уже в дверях ответила я.

\hypertarget{chapter-5}{%
\chapter{~}\label{chapter-5}}

Мы подъезжали к украинско-русской границе. Народ уже толпился у выхода из автобуса. Люди так настойчиво прокладывали себе путь, что я удивилась, как это еще никто не умер. Какая-то масштабная мадам правда завалилась на себе подобную и вскоре обе уже распластались вдоль салона.

Всего нас было тридцать три человека, не считая водителя. Прямо как годиков одному знаменитому божьему сыну. Тридцать человек спорили у входа, пытаясь первыми проникнуть к отделу с багажом. Еще две женщины поднимались с пола. Они выглядели сонными и обессиленными, но по инерции продолжали браниться. Завершая общую картину, я сидела в самом конце салона. Читала книгу и краем глаза поглядывала на происходящее вокруг. Багаж -- если таковым можно назвать чемоданчик размером с ноутбук -- и так был при мне.

Украинско-российская граница, как выяснилось, больше походила на декорации к первому сезоны «Ходячих мертвецов», нежели на хорошо охраняемое место переправы. То тут, то там виднелись полуметровые цементные блоки и целая куча хаотично разбросанной колючей проволоки. Еще одним фактором, производившим странное впечатление, выступали припаркованные повсюду автомобили. Большинство из них были с распахнутыми дверцами и опущенными стеклами, тогда как водителей внутри не наблюдалось. Те растерянно бегали от одной будки к другой, пытаясь правильно заполнить целую стопку бумажек, каждая из которых нужна лишь для получения следующей.

У импровизированного начала границы стояло несколько военных. Ребята были при параде, включая дубинки и огнестрельное. Фонари горели как-то вяло и, хотя солнце еще только подумывало садиться, становилось ясно, что освещаются лишь первые пару метров переправы. Кстати говоря, такой же участок пути имел более-менее вменяемое дорожное покрытие. За ним пешеходов ожидали овраги, трещины, камни, и прочие достопримечательности моего родного края.

Учитывая обстоятельства с ручной кладью, к пункту проверки документов я подошла первая. Вернее, поначалу мне так казалось. Но довольно возрастная женщина внезапно воткнулась прямо передо мной и уже самодовольно махала рукой какому-то не поспевающему за ней мужчинке. По ту сторону границы нас ждал другой автобус. До отправления оставалось больше часа, так что особой нужды в такой спешке не было. Как, видимо, и в вежливости.

-- Куда едете? -- спросил не поддающийся определению по половому признаку голос из будки.

-- Севастополь.

-- Причина визита?

-- Пока под вопросом.

Пауза.

Окошко распахнулось, и на свет божий выползло заспанное личико, немногим старше моего собственного.

-- Назовите причину визита, -- раздраженно повторила барышня.

-- Просто хочу кое-что проверить.

-- Нет, так нельзя.

-- А как можно?

-- Бизнес, туризм, или личная.

Так я совершенно неожиданным для себя образом узнала, как можно. Что бы это ни значило.

-- Что пишем? -- напомнил о себе голос из окошка.

Я призадумалась. Очевидно, что мой визит относился к категории личных. Однако, опыт минувших взаимоотношений с другими представителями земной расы показал, что ни в коем случае нельзя вешать ярлыки. Никакие. Даже если все кажется идеальным. Нет, особенно если все кажется идеальным! Более того, дело здесь не в возможной реакции других людей, а в твоем собственном подсознании. Я имею в виду, как только ты называешь мужчину своим молодым человеком, ты подсознательно начинаешь ждать от него определенную модель поведения. И чем больше его действия походят на те, что ты там себе успела придумать, тем сильнее ты расслабляешься. И тем болезненнее в дальнейшем будет принять суровую реальность.

Словом, в я только-только свыклась с мыслью об отсутствии разбитого сердца в моей грудной клетке и никак не была готова признать связь с Вениамином личной. Поэтому я ответила:

-- Бизнес.

-- Во-о-от как, -- работница таможни подняла бровь с той самой грациозностью, с какой бывалая проститутка поднимает юбку по окончанию тяжелого рабочего дня. Вроде как и лень, но по статусу положено. -- И кем вы работаете?

-- Я писатель.

-- Вот как, -- повторила она. -- И что, прямо пишите?

-- Преимущественно.

-- Вот как.

Мне вдруг стало очень смешно.

-- А работаете-то кем? -- не унималась мадам.

Я положительно не хотела слышать четвертое «вот как», поэтому решила сдаться.

-- Барменом.

-- Так и напишем, -- подытожив, она протянула мне небольшой бланк. -- Распишитесь внизу.

В графе причина въезда был подчеркнут «личный визит». Я выдохнула:

-- Вот как\ldots{}

***

Ох, чего только я не наслушалась перед выездом. Одни рассказывали мне душетрепательные истории и том, что на границе рвут документы. Вторые убеждали, что стоит мне запеть гимн Украины и меня тут же повалят на асфальт. Не то, чтобы я каждый день занималась подобными развлекательными программами, но кто это проверил и зачем? Была еще история про курицу.

Вообще, в моей жизни как-то подозрительно много бесполезных историй, что, так или иначе, связаны с курами. Понятия не имею, что это значит, но, пожалуй, что ничего. Так вот, подруга матери бывшего администратора «Штиля» решила поехать к племяннику, который остался жить на полуострове. Будучи коренной украинской бабулей, она прихватила с собой целую кучу еды, включая сырую курицу огромных размеров. Увы, правила запрещали перевозить продукты. В случае их обнаружения, нарушителя заставляли пройти весь путь обратно, -- от российской границы до украинской -- и оставить еду там, заполнив соответствующие бумаги. Если же ты спешишь на автобус, можно просто отдать еду пограничникам. К чему это я? Легенда гласит, что суровая бабуля съела свою злополучную курицу прямиком при служащих таможни. Сырой.

Стоя в очереди позади лишенной такта женщины я мысленно решила, что она вполне могла бы быть той самой бабулей, и теперь изо всех сил старалась не рассмеяться: перед глазами уже вовсю плясали живописные образы.

Проверка багажа не заняла много времени. Сникерс у меня не отняли и на откупоренную в дороге бутылку виски тоже никто не позарился. Подхватив платье свободной рукой, я шагала по широкой, отвратительно вымощенной дороге, которая с каждым новым метром становилась все хуже. Мелкие камешки то и дело норовили застрять в подошве моих ботинок. По обе стороны от этой недотрассы тянулись небольшие овраги. За ними -- высоченный забор из нержавеющей проволоки. Дальше виднелись бесхозные поля, кустарники и кое-какие деревья.

По мере моего приближения к русской границе забора становилось все меньше, а деревьев -- больше. Вскоре они уже занимали все пространство вдоль дороги. Благодаря размерам своего чемодана, я неплохо оторвалась от других пассажиров. Порой на пути мне встречались люди, идущие в противоположном направлении. Еще реже проезжали автомобили, но в основном я шла одна.

Раскинувшиеся по обе стороны дороги виды создавали иллюзию заброшенности. Казалось, я была единственной, чья нога ступала на эту землю на протяжении последних десятилетий. Тишина, покой и нетронутая природа. Все это напоминало мне пейзажи из документалок о Чикатило, которые я любила смотреть в детстве. Окончательно картину дополнил проржавевший автомобиль, наполовину погрязший в грязи. Стекол на нем не было, как и колес. Сквозь дверцы уже успели прорасти впечатляющих размеров сорняки. Как ни странно, машина удачно вписывалась в обстановку. Думаю, она и сейчас там стоит.

\emph{Самое то для внеплановой медитации}, решила я и забралась на крышу бесхозной машины.

Благодаря эйдетической памяти и своей творческой натуре я без труда могла переместиться куда угодно. Вот только не всегда могла это контролировать. Воображение, на этот раз слишком живое, унесло меня к глубоко похороненным страхам прошлого, так что от медитации пришлось отказаться.

В воздухе запахло морем. Наконец, вдалеке стали видны какие-то светящиеся вышки. Спустя минут семь мне преградил путь шлагбаум в компании молодого солдата. Парень попросил показать паспорт.

-- Наркотики, оружие, спиртное перевозите? -- серьезно спросил он.

-- Целую кучу, -- ответила я.

Чемодан открывать не стали.

-- Дальше по дороге идти нельзя, -- заметил солдат. Я вопросительно подняла брови. -- Такие правила. Пройдите во-о-он туда.

Он указал на начало очередного забора. Подойдя ближе, я ужаснулась. Нет, я, конечно, знала, что в России мало сторонников демократии, но не до такой же степени! В общем, вы в курсе, как выглядят наружные тюремные коридоры? Те, по которым зеков обычно ведут из одного крыла в другое? Такие узенькие площадки максимум метр в ширину, с двух сторон огражденные высоченным забором все с той же колючей проволокой на верхушке. Именно так выглядит начало российской границы.

Оценив ситуацию, я подкурила.

***

Спустя полчаса я, наконец, забралась во второй автобус. До отправления оставалось более двадцати минут, и я понятия не имела, как их провести. В другой ситуации я бы, наверное, уже давно умерла со скуки. Но не сегодня. Сегодня у меня был виски.

Конечно, я в сотый раз повторяла себе, что не стоит ждать ничего серьезного от предстоящей встречи. И, все-таки, я ждала. Вернее, та часть меня, которая заставляла мою внутреннюю феминистку забиться в угол. Предвкушения чего-то хорошего -- вот к чему я питала особую зависимость. Только в подобные волнительные моменты я могла почувствовать себя живой.

Откровенно говоря, и спустя годы, вспоминая вечер на таможне, я все так же чувствую послевкусие той необъяснимой эйфории, которая настигла меня за несколько часов до встречи с Вениамином. Автобус больше не казался мне душным, погода жаркой, а жизнь дерьмовой. Я сидела в конце салона, наблюдая за тем, как уже успевшая полюбиться мне лесополоса принимает солнечные ванны, и чувствовала себя как никогда восхитительно.

Это меня расслабило. Вскоре я прикрыла и позволила памяти ненадолго вернуть меня в прошлое.

\hypertarget{chapter-6}{%
\chapter{~}\label{chapter-6}}

За окном моросил октябрьский дождь две тысячи тринадцатого года выпуска, отчего внутри становилось крайне уютно. Обернувшись одеялом, я сонно смотрела в окно. Передо мной стояла чашка кофе, кола и наполовину пустая бутылка виски. По ту сторону стекла уже начинало темнеть, и я наблюдала, как в окнах соседних домов зажигаются огни. Одно за другим. Прямо как в детстве, но без виски, разумеется. До шестнадцати я пила только вино.

С минуту я разглядывала стоявшие на подоконнике жидкости, после чего потянулась кофе. Любовь пагубно влияла на мой алкоголизм.

За моей спиной скрипнула дверь. Обернувшись, я увидел, стоявшего в дверном проеме Адама. Дождь к тому моменту усилился, и мужчина уже стягивал с себя промокшую до нитки куртку. Несмотря на это, выглядел он как никогда очаровательно. На губах играла мечтательная улыбка, а карие глаза сияли, делая взгляд еще более выразительным. Не говоря ни слова, Адам подошел ко мне и, опустившись напротив, коснулся своими губами моей руки.

До этой минуты я и подумать не могла, что такой банальный и даже старомодный жест способен вызвать во мне целую бурю эмоций. Но он вызвал. Я смущенно улыбнулась, и Адам улыбнулся в ответ. Своей самой широкой улыбкой, открывающей ямочки на щеках. Забыв о кофе, я коснулась рукой темных волос, что вскоре начнут доставать до плеч, и вновь улыбнулась. На этот раз куда шире.

Адам поцеловал меня, и я вдруг осознала, что люблю его.

Той ночью ни один из нас не смог уснуть. Сначала мы были слишком заняты для такого скучного дела, а затем\ldots{}

-- О чем ты думаешь? -- спросил Адам, когда я собрала растрепавшиеся волосы и растянулась на подушке.

Я повременила с ответом.

-- Я думаю о том, что в бутылке подозрительно много виски как для пятничного вечера.

Адам вздохнул.

-- Не хочешь говорить?

-- Пожалуй, нет, -- медленно ответила я, ласково взглянув на своего собеседника.

Он обхватил меня за плечи и крепко поцеловал.

-- А теперь?

Я мужественно молчала.

-- Нет, ну так не честно! Я думал, мы договорились говорить друг другу всю правду, -- Адам обиженно отодвинулся от меня. -- Милая, ты заставляешь меня нервничать.

Так себе аргумент как для двадцатипятилетнего мужчины, но я и правда не знала, что ответить.

-- Не думаю, что сейчас подходящий момент.

Десятисекундная пауза.

-- Так о чем? -- вновь спросил Адам и я сдалась.

-- Думаю, я люблю тебя.

После этой фразы Адам молниеносно изменился в лице. Позже он утверждал, что знал, что я скажу именно это. Более того, он так настаивал именно потому, что думал, что сможет ответить мне тем же. Однако, в последний момент вдруг понял, что не сможет -- вся суть мужчин.

-- Мне нужна сигарета, -- отрезал Адам и покинул постель.

С этого момента наши мимолетные отношения круто переменились. Адам стал первым мужчиной, к которому я питала сильные чувства во взрослом возрасте. Конечно, мне было всего девятнадцать, но неплохой рабочий стаж, несколько написанных книг, вовремя уплаченные налоги и уйма забот позволяли мне считать себя взрослой.

Мы расстались спустя полтора месяца. Тем приятным погожим днем, когда я меньше всего этого ожидала. Без криков и прочих тягостных выяснений отношений. Он ушел, а я проплакала всю ночь. К утру замкнулась в себе и перекрасила волосы в алый.

-- Нельзя говорить человеку о любви на второй день знакомства! -- заявил Адам спустя дня три после этого самого второго дня знакомства. Кстати, формально мы были знакомы куда дольше, но никогда прежде не виделись. -- Это ненормально!

Никогда не забуду того, с каким чувством он произнес последнее слово. Ненормально. Словно это было не одно, а целых четыре отдельно существующих слова. Не-нор-маль-но. Так обычно разговаривают с детьми или умственно отсталыми.

Куда делся тот парень, который уговаривал меня дать шанс этим отношениям? В одной из моих книг персонаж по прозвищу Молния чуть не лишается своей возлюбленной из-за того, что признается ей в любви на второй день знакомства. Никто и не знает, насколько автобиографична эта сцена.

***

Теперь же меня занесло в Крым.

Поездка оказалась невыносимо долгой. В какой-то момент я даже уснула, что редко бывает со мной в общественном транспорте. Точнее, попросту отключилась. Сегодня ведь уже ни для кого не новость, что происходящее во сне проносится в мозге за долю секунды?

Что ж, тем вечером в моей голове пронеслось немало. Проснулась я в тревожном состоянии, и тут же потянулась к часам. Чувствовала себя так, словно проспала часов двадцать. На деле же -- тринадцать минут. Вот уж не думала, что освоенная в студенческие годы практика кратковременного сна однажды мне пригодится.

Так вот, проснулась я в тревожном состоянии. Мне снился Адам. Как всегда, сон оказался чересчур реалистичным, так что я не сразу пришла в себя. И пускай выглядела я абсолютно спокойной, внутри уже начинал мельтешить страх отношений. С одной стороны, я всем сердцем хотела полюбить кого-то сильнее, чем когда-то любила своего бывшего. С другой -- я боялась этого до умопомрачения. Именно поэтому на протяжении последних трех лет я никому не позволяла сблизиться со мной. Даже поцелуи во время секса были запрещены. Мне думалось, что они вызывают слишком интимную связь, как бы абсурдно это не звучало.

И вот я вновь куда-то ехала. На этот раз без планов о чем-то серьезном и с куда большими надеждами одновременно. От подобных мыслей сердце начинало колотиться так сильно, что становилось сложно дышать.

К тому моменту бутылка виски совсем опустела. Автобус проезжал Симферополь. Он показался мне до невозможности скучным и очень напоминал бесконечную улицу Чигрина -- не самый благополучный район моего родного города.

«Спокойно, Васляева, -- сказала я себе. -- Тебе уже не девятнадцать. Ни в кого ты поспешно не влюбишься.»

Вот она, самая распространенная из всех возможных человеческих глупостей: отчего-то полагать, что ты стал умнее.

\hypertarget{chapter-7}{%
\chapter{~}\label{chapter-7}}

Никто понятия не имел, когда должен прийти мой автобус. По предварительным расчетам, время прибытия -- около шести часов вечера. Но зная отечественную пунктуальность, я в этом очень сомневалась. Веня жил в пяти минутах езды от автовокзала, так что мы условились на том, что я поймаю вай-фай где-нибудь на подъезде к Севастополю.

Как ни странно, мы ехали без остановок, если не считать внезапной поломки автобуса. Следующий пришлось ждать посреди какого-то захолустья, так что вай-фаем там и не пахло. В итоге до пункта назначения я добралась лишь когда стрелка на часах стремительно приближалась в десяти вечера.

И вот автобус, наконец, остановился. Я выбралась на улицу и сладко потянулась. Первым, что привлекало внимание, был свежий морской воздух. Вдохнув его, я вдруг поняла, как же сильно истосковалась по бризу.

Солнце, конечно, уже давно село, прихватив с собой тепло весеннего дня. На мне было легкое платье в пол со спущенными плечами, которое теперь мелодично колыхалось в такт ветру. Становилось как-то прохладно. Вдоволь надышавшись морем, я осмотрелась в поисках подходящего заведения. По-прежнему никаких признаков беспроводного интернета. Естественно, местной сим-карты у меня тоже не было, а моя уже часа четыре как перестала быть активной.

В какой-то степени, тот факт, что на вокзале не было Вениамина, сыграл мне на руку. Таким образом, я могла еще раз все обдумать. И, все же, казалось, кто-то уже давно обдумал все за меня. Несмотря на бешеную усталость и практически полное отсутствие сна, я все еще чувствовала себя отлично. Вполне нормальный расклад. В рассвете бессонницы настоящая усталость обычно накрывает день на третий, а то и пятый, когда к ней подключается светомузыка.

К сожалению, здание вокзала не смогло вместить в себя что-либо кроме касс да комнаты ожидания. Пройдя сквозь нее, я вышла на улицу с обратной стороны и обнаружила, что пошел дождь. Стоило сделать пару шагов к проспекту, как ко мне тут же подбежала организованная группировка таксистов. Тут-то я и вспомнила о том, что адрес Вениамина остался в переписке, так что без интернета было не обойтись.

Как вы уже поняли, я никогда не отличалась привычкой планировать наперед. Если дело касалось работы, тут я стратег высшего разряда, но, когда речь заходит о личном комфорте -- только хардкор, только импровизация.

-- Подвести, красавица? -- хором спросило сразу несколько мужчин.

Я улыбнулась их синхронности. Города меняются, а таксисты везде одинаковые.

Вдалеке я заметила заправку. Она находилась в паре светофоров от меня и походила на те, в которых обычно продаются подозрительно дорогие хот-доги. И есть вай-фай. Таксисты еще что-то продолжали спрашивать, когда я ступила на первый пешеходный переход.

-- У вас виски есть?

-- Только кофе с виски.

-- Давайте самый большой.

Если верить табличке на входе, до закрытия оставалось всего-ничего. Но стоявшая за прилавком женщина никуда не спешила.

-- Паспорт с собой? -- спросила она.

-- Паспорт?

Я даже не сразу сообразила, в чем здесь дело.

-- Да, покажите паспорт. Лицам моложе восемнадцати выпивать не положено.

-- Да вы гоните!

Даже не знаю, что меня удивило больше: проверка документов на заправке или то, что кто-то назвал ирландский кофе
выпивкой. Я еще немного повыделывалась прежде, чем предъявить документы.

Здесь и впрямь был беспроводной интернет. Только вот доступ к нему ты мог получить лишь через местную сим-карту, которой у меня, как мы все уже поняли, не было. Еще и виски ненастоящий! С таким же успехом я обошла еще несколько заведений. Дождь заметно усилился и вот-вот обещал обратиться ливнем. Тогда-то на моем пути встретился какой-то местный фастфуд.

Внутри было пусто. Три столика и все свободны. За прилавком стояла полная девушка. Она красила ногти в желтый цвет и, кажется, была не слишком рада моему визиту.

-- Мы закрываемся через шесть минут.

-- Мне нужно иметь сим-карту, чтоб подключиться к вашему вай-фаю? -- на одном дыхании выпалила я.

Девушка взглянула на меня как на умалишенную. Мне, в общем-то, было не привыкать.

-- Нет. Но вам как бы надо бы чёта купить.

-- Латте, -- заказала я. -- Самый большой.

Людей на улицах было поразительно мало, что создавало особенно приятную, тихую атмосферу. Сидя за столиком у окна, я наблюдала за тем, как потоки дождевой воды текут вдоль улицы, и наслаждалась тремя минутами, оставшимися до закрытия заведения. Мне, наконец, удалось подключится к сети, и я ужаснулась количеству входящих сообщения. Кажется, Вениамин не на шутку разволновался. Я опоздала на четыре часа, а внезапно разыгравшийся шторм изрядно добавлял ситуации трагизма.

Успокоив Веню, я пообещала ждать в холле вокзала. Вечер обещал быть интересным.

\hypertarget{chapter-8}{%
\chapter{~}\label{chapter-8}}

Эта сцена не раз мелькала в моих мыслях до нашей первой встречи с Вениамином. Слушая его голос, собираясь на последний перед отпуском рабочий день, стоя за барной, отправляясь ко сну, сидя за рукописями и бегая за постоянно заканчивающимся льдом, я представляла, как впервые увижу его. Но еще чаще я думала о нашем с Веней знакомстве уже после того, как оно состоялось. Когда наполняла бокал вином, готовила ужин, гуляла около моря, ждала свой кофе в какой-нибудь забегаловке, смотрела, как он точит ножи или засыпает прямо посреди комнаты\ldots{}

И особенно сильно я думала о том вечере после очередной ссоры. Как бы ни обстояли дела, несмотря на всю горечь обиды, одно лишь это воспоминание было способно растопить лед, возникший в моем сердце.

Скажи мне кто-нибудь, что писать о собственной жизни будет так сложно -- ни за что бы не поверила. Казалось бы, ничего не нужно, ведь ты и так все знаешь и помнишь до мелочей. Ты часть истории, по отношению к которой твои собственные пожелания, по сути, не имеют никакого отношения. В этом-то и проблема. Произошло то, что произошло и все, что тебе остается -- проанализировать события и отойти в сторонку, притворившись обычным наблюдателем. Есть второй вариант, не подразумевающий сохранения ясности ума, так что мы его опустим.

Говорят, мне повезло с памятью. Я помню все имена, даты, события, диалоги и так далее до подробнейших деталей. Могу без труда вспомнить, во что был одет человек пять лет назад, что он говорил, с каким настроением и почему. Могу сказать, какая за окном стояла погода, что за музыка играла в проезжающей мимо машине и о чем я в тот момент думала. Эйдетизм подразумевает собой особый вид памяти. Совокупность зрительных образов с показаниями остальных счетчиков твоего организма: слуховых, вкусовых, тактильных, обонятельных, двигательных и даже интуитивных. При правильной концентрации можно перенестись в любое место, достаточно лишь разок побывать там, но есть у эйдетики один жирный минус. В справочниках, правда, он подается как большой плюс, но что они в этом смыслят: «эйдетические образы отличаются от образов восприятия тем, что человек как бы продолжает воспринимать предмет в момент его отсутствия».

Отсутствие -- ключевое слово.

***

Звуки и запахи -- это вообще отдельный разговор. Правильное сочетание этих элементов буквально открывает мне окно в прошлое. С такими манипуляциями памяти не мудрено влипнуть в депрессию.

Спустя год после болезненного расставания с Адамом мне нужно было съездить в Черниговскую область. Добиралась я, естественно, таким же разваливающимся поездом, каким когда-то добиралась к своему молодому человеку. Был прохладный осенний вечер сродни тем вечерам, когда я только раздумывала о перспективе отношений с Адамом. Тогда у меня были духи с запахом лаванды. Они пылились на полке с две тысячи тринадцатого, и черт меня дернул воспользоваться ими тогда, оправляясь на железнодорожный вокзал.

Запах парфюма и шум поезда буквально свели меня с ума. Картина перед глазами оставалась прежней, но в какой-то момент, я была уверена в том, что мне все еще девятнадцать, а поезд везет меня в Москву, и уже утром меня встретит туман Киевского вокзала. Из толпы на перроне покажется Адам -- в своей старой кожаной куртке -- и поспешит навстречу, чтобы помочь спустить с поезда тяжелый чемодан.

Не мудрено, что Шульц так ярко описывал приступы шизофрении с примесью самоубийств, возникшие в итоге настойчивых попыток вызвать из памяти эйдетические образы. Но что делать, когда они всплывают против твоей воли?

История закончилась тем, что я вылетела на улицу при первой же возможности, а затем стояла посреди какой-то неизведанной сельской местности и наблюдала за тем, как состав плавно удаляется в сторону заката.

Последнее, кстати, вовсе не метафора. В этот момент солнце действительно опускалось за горизонт, приглушая осенние оттенки. Я смотрела на исчезающий за небосводом алый диск, опавшие листья и разноцветные деревья, колышущиеся на ветру. Смотрела и думала, какая же я дура.

***

Вернемся к относительной реальности.

Первая половина мая шестнадцатого подходила к концу. Закинув ногу на ногу, я сидела в крохотном зале ожидания, который, по совместительству был холлом, а также единственной комнатой в здании севастопольского вокзала. Кроме меня здесь никого не было, так что я позволила себе дважды покурить. Честно говоря, я настолько волновалась, что курнула бы еще не раз и отнюдь не табака, но как-то не хотелось создавать вокруг себя газовую камеру. Под дождь выходить тоже не хотелось, и я просто продолжала сидеть, закинув ногу на ногу. Сперва мне думалось, что я читаю, но спустя минут пять пришлось признать, что я уже раз семнадцатый начинаю один и тот же абзац, но так и не добираюсь до середины.

\emph{Произносить в голове маячащий перед глазами текст и думать о чем-то на фоне -- поразительная, но такая бесполезная способность человеческого мозга}, подумала я, и оставила книгу в покое.

По обе стороны от меня находились большущие старые окна, в которые я то и дело поглядывала. К тому моменту мои каштановые волосы уже были достаточно длинными и объемными, дабы за ними спрятаться. При этом они были вьющимися, что позволяло мне откинуть парочку прядей и незаметно бросать пристальные взгляды то вправо, то влево. Стоит ли говорить, что сидела я при этом прямо и расслабленно, упорно соблюдая видимость чтения?

Вопреки всем моим усилиям, размытые потоки воды да отблески проезжающих по проспекту машин были всем, что я видела. В итоге я решила забить на это неблагодарное дело и попыталась еще раз погрузиться в книгу. К тому же, до окончания «Бродяг» оставалось всего ничего. Только вот я сама уже давно чувствовала себя той еще бродягой, но только сейчас впервые всерьез задалась вопросом: куда эта Дхарма меня приведет?

Я отбросила с лица надоедливую прядь волос -- прядь в действительности крайне надоедлива, так что вы еще не раз прочитаете о том, как я ее отбрасываю -- и склонилась над дорожным романом, когда дверь слева от меня распахнулась. Естественно, я знала, кто сейчас в нее войдет, но не стала поспешно отрываться от чтения.

Знаете, говорят, что женщина может думать миллион мыслей одновременно. Вообще-то, так и есть. За те несколько секунд, которые понадобились Вениамину, чтобы подойти к моему креслу, я задала себе как минимум десять вопросов.

\emph{Подправила ли я помаду? До конца ли высохли мои волосы? Где мой паспорт? Зачем мне сейчас паспорт? Успела ли я втянуть живот? Не прилипло ли платье к груди? А если прилипло, то насколько это сексуально? Достаточно ли непринужденно я сижу? Не слишком ли непринужденно я сижу? Пахну ли я также хорошо, как пять минут назад?}

И так далее, и тому подобное. Не поверите, но я даже успела запомнить номер страницы, на которой остановилась. Сейчас это кажется мне смешным. Как и многое другое.

Позже Веня говорил, что я держалась с ним так, словно мне было глубоко насрать, и это его окончательно очаровало. Такой вот у меня бесполезный талант: выглядеть хладнокровно, когда на самом деле внутри все сжимается от волнения и сердце скачет как курс доллара в Украине с наступлением военного положения. Видимо, по этой простой причине я неплохой игрок в покер.

Так вот, дверь распахнулась. Спустя пару мгновений я позволила себе поднять глаза. Передо мной стоял высокий блондин с зелеными глазами и хемингуэевский бородой. На нем была фланелевая клетчатая рубашка, которую затем я видела так часто, что та успела мне надоесть, и невероятно ласковая улыбка, которую я видела еще чаще. Она мне никогда не надоедала.

Откровенно говоря, я всегда думала, что предпочитаю длинноволосых брюнетов со смуглой кожей. Однако, увидев Вениамина, я поняла, что никогда не встречала мужчину прекраснее. У него была бледная кожа, длинный андеркат и профиль настоящего викинга. Волосы частично скрывала небольшая шляпа с полями.

Веня улыбнулся мне еще шире, и я вдруг осознала, что пялюсь на него уже с добрых полминуты. С другой стороны, он занимался тем же, так что все было в порядке. Когда наши взгляды встретились, Вениамин развел руки, предлагая мне окунуться в его объятья. Хотя, учитывая путь, который я проделала тем днем, правильней было бы сказать «упасть».

Он обнял меня, не говоря ни слова. Так мы и стояли в полнейшей тишине пустующего вокзала. Где-то очень далеко продолжали шуметь автомобили, а дождь играл ночную мелодию на оконных стеклах и тротуаре.

\hypertarget{chapter-9}{%
\chapter{~}\label{chapter-9}}

-- Твой самый большой недостаток? -- спросила я.

Мы шли вдоль стеллажей небольшой винной лавки.

-- Я алкоголик, -- спокойно ответил Веня. -- Алкоголь -- моя единственная проблема. Порой я не ведаю что творю, отчего друзья зовут меня Невероятный Алк. Но в остальном я идеален.

Он ухмыльнулся.

Я обвела взглядом стеклянные бутылки.

-- С вином тоже проблемы?

Он покачал головой.

-- В основном с водкой.

-- Отлично, -- я протянула руку к красному полусладкому. -- Тогда берем две.

Определившись с выпивкой, мы все так же продолжали прогуливаться вдоль винных полок. Веня держал меня за руку. Периодически я ловила на себе его смущенные взгляды. За прошедшие годы мужчины смотрели на меня по-разному. Большинство с желанием. Некоторые как на полнейшую стерву, но все с тем же желанием. Были даже те, в чьем взгляде читалось восхищение, но все это казалось мне поддельным. И вот, разгуливая по алкогольному кварталу, я осознала, что уже очень давно никто не смотрел на меня так.

-- А твой? -- поинтересовался мой спутник.

-- Мой самый большой недостаток?

-- Ага.

-- У меня биполярное расстройство.

-- Что это?

-- Маниакально-депрессивный психоз, -- объяснила я. -- Мне это название даже больше нравится. Оно звучит понятней, но
теперь считается оскорбительным, так что\ldots{}

-- Хорошо, давай поменяем правила, -- предложил Веня. -- Той самый большой плюс.

Теперь пришла моя очередь ухмыльнуться.

-- У меня биполярное расстройство.

-- И в чем здесь проблема?

-- Всегда по-разному.

-- Например? -- не унимался Веня.

Я вздохнула и закатила рукава, демонстрируя выдающуюся коллекцию шрамов с обеих сторон рук.

-- И всего-то? -- еще одна ласковая улыбка.

Я подняла рукав выше.

-- Еще бывает сигареты об себя тушу. Это не симптом, а, скорей, последствия, но, говорят, окружающих напрягает.

Мой спутник вручил мне бутылки и принялся закатывать рукава. Руки под рубашкой украшало куда большее количество шрамов. Они выглядели на удивление красиво, если не считать парочки партаков, одним из которых была эмблема супермена. Вот как нужно производить впечатление на женщину. Спрашиваешь о ее недостатке, а затем даешь понять, что для тебя это всего лишь пустяк.

Я весело рассмеялась. То ли шрамам, то ли татуировке.

-- Серьезно, -- спросила я, вскинув бровь. -- Супермен?

-- Как видишь. Терпеть его не могу.

-- Так что же не перебьешь?

-- Ну, так в этом-то и смысл, -- парировал Вениамин, направляясь в сторону кассы.

***

Звездное небо раскинулось над самым прекрасным из всех городов. Я видела в Севастополе уйму недостатков, но никогда не прекращала любить его. Думаю, с Вениамином было так же.

Очарованные вечером и компанией друг друга, мы сидели на парапете посреди пустой улицы. Несмотря на то, что дождь практически закончился, Вениамин держал надо мной зонтик. Все это время мы то без умолку болтали, то не могли прекратить целоваться. Обе винные бутылки стояли рядом. Никто к ним даже не притронулся.

Первым молчание нарушил Вениамин.

-- Твои волосы еще красивее, чем на фото.

-- А что насчет остальных частей меня? -- я повернулась к (своему?) мужчине, изображая пристальный взгляд.

-- Ты очень красивая, Васляева. Именно такой я тебя и представлял. И ты лучший в мире собеседник. Я окончательно очарован.

Я покраснела. Впервые за долгие годы существу мужского пола удалось меня смутить. Заметив это, Веня лишь крепче прижал меня к себе.

Море находилась совсем недалеко. Ночной ветер доносил до нас его ностальгические нотки, создавая иллюзию оазиса. Я набрала полные легкие свежего воздуха и медленно выдохнула.

-- Люблю я море.

-- А я -- тебя, -- отозвался Веня.

Здесь мне должен был вспомниться Адам, то, как я призналась ему в любви на второй день знакомства и все такое прочее. Но он мне почему-то не вспомнился. Я вообще напрочь забыла о его существовании.

-- Что ты сейчас сказал?

Я в удивлении отстранилась, на что Веня лишь развел руками.

-- Я люблю тебя, Васляева.

-- Секундочку. Ты признаешься мне в любви в первый день знакомства?

-- Так и есть. Ничего не могу с этим поделать. Да и формально мы знакомы гораздо дольше\ldots{}

\emph{Где-то я это уже слышала.}

Казалось, вся моя болтливость иссякла. Я просто сидела, переводя взгляд с Вени на сигарету в его руке и обратно. Затем пришла к самому мудрому в сложившихся обстоятельствах решению, и подкурила.

Тем временем Вениамин взглянул на часы.

-- Совсем забыл о времени, ты ведь устала! -- спохватился он. -- Пойдем.

Так мы стали жить вместе, фактически зная друг друга не больше трех часов.

\end{document}
