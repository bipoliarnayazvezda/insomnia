% Options for packages loaded elsewhere
\PassOptionsToPackage{unicode}{hyperref}
\PassOptionsToPackage{hyphens}{url}
%
\documentclass[
]{book}
\usepackage{amsmath,amssymb}
\usepackage{lmodern}
\usepackage{iftex}
\ifPDFTeX
  \usepackage[T1]{fontenc}
  \usepackage[utf8]{inputenc}
  \usepackage{textcomp} % provide euro and other symbols
\else % if luatex or xetex
  \usepackage{unicode-math}
  \defaultfontfeatures{Scale=MatchLowercase}
  \defaultfontfeatures[\rmfamily]{Ligatures=TeX,Scale=1}
\fi
% Use upquote if available, for straight quotes in verbatim environments
\IfFileExists{upquote.sty}{\usepackage{upquote}}{}
\IfFileExists{microtype.sty}{% use microtype if available
  \usepackage[]{microtype}
  \UseMicrotypeSet[protrusion]{basicmath} % disable protrusion for tt fonts
}{}
\makeatletter
\@ifundefined{KOMAClassName}{% if non-KOMA class
  \IfFileExists{parskip.sty}{%
    \usepackage{parskip}
  }{% else
    \setlength{\parindent}{0pt}
    \setlength{\parskip}{6pt plus 2pt minus 1pt}}
}{% if KOMA class
  \KOMAoptions{parskip=half}}
\makeatother
\usepackage{xcolor}
\IfFileExists{xurl.sty}{\usepackage{xurl}}{} % add URL line breaks if available
\IfFileExists{bookmark.sty}{\usepackage{bookmark}}{\usepackage{hyperref}}
\hypersetup{
  pdftitle={Бессонница},
  pdfauthor={Александра Булгакова},
  hidelinks,
  pdfcreator={LaTeX via pandoc}}
\urlstyle{same} % disable monospaced font for URLs
\usepackage{longtable,booktabs,array}
\usepackage{calc} % for calculating minipage widths
% Correct order of tables after \paragraph or \subparagraph
\usepackage{etoolbox}
\makeatletter
\patchcmd\longtable{\par}{\if@noskipsec\mbox{}\fi\par}{}{}
\makeatother
% Allow footnotes in longtable head/foot
\IfFileExists{footnotehyper.sty}{\usepackage{footnotehyper}}{\usepackage{footnote}}
\makesavenoteenv{longtable}
\usepackage{graphicx}
\makeatletter
\def\maxwidth{\ifdim\Gin@nat@width>\linewidth\linewidth\else\Gin@nat@width\fi}
\def\maxheight{\ifdim\Gin@nat@height>\textheight\textheight\else\Gin@nat@height\fi}
\makeatother
% Scale images if necessary, so that they will not overflow the page
% margins by default, and it is still possible to overwrite the defaults
% using explicit options in \includegraphics[width, height, ...]{}
\setkeys{Gin}{width=\maxwidth,height=\maxheight,keepaspectratio}
% Set default figure placement to htbp
\makeatletter
\def\fps@figure{htbp}
\makeatother
\setlength{\emergencystretch}{3em} % prevent overfull lines
\providecommand{\tightlist}{%
  \setlength{\itemsep}{0pt}\setlength{\parskip}{0pt}}
\setcounter{secnumdepth}{5}
\usepackage{booktabs}
\ifLuaTeX
  \usepackage{selnolig}  % disable illegal ligatures
\fi
\usepackage[]{natbib}
\bibliographystyle{plainnat}

\title{Бессонница}
\author{Александра Булгакова}
\date{2023-02-08}

\begin{document}
\maketitle

{
\setcounter{tocdepth}{1}
\tableofcontents
}
\hypertarget{section}{%
\chapter{~}\label{section}}

После внепланового завершения серьезных отношений сильные люди обычно впадают в своего рода эмоциональную спячку. Так или иначе, приходится изолироваться от твоей несостоявшейся второй половинки. Перебороть истерики и утихомириться в какой-нибудь тесной квартирке с микроволновкой для полуфабрикатов, своими немногочисленными пожитками и бутылкой, о которой ты уже толком и не помнишь, что внутри, но к которой все равно продолжаешь периодически прикладываться с крайне отсутствующим видом. Сколько бы лет тебе не было, (двадцать пять или пятьдесят -- это неважно) вдруг начинаешь жить жизнью завтрашнего пенсионера. Прекращаешь слушать музыку, за едой думаешь исключительно и еде и никогда не выключаешь телевизор, дабы чувствовать себя менее одиноко.

Мать моего отца умерла за месяц до моего рождения. Дедушка же прожил еще пятнадцать лет. Все эти годы он жил со включенным телевизором, который толком никогда не смотрел. Еще дедушка никогда не смеялся и очень редко улыбался, разве что мне. После его смерти я нашла коробку со старыми снимками. Практически на каждом из них лицо моего деда освещала широченная улыбка. Шокировавшись, я показала фотографии отцу, но тот лишь пожал плечами, сказав, что дедушка ни разу не рассмеялся со дня похорон моей бабушки.

Дедушка умер хмурым январским утром. Сгорел от внезапно давшего о себе знать рака. Он пережил голодомор, войну, распад Советского Союза и трех украинских президентов, а затем почувствовал острую боль в желудке и умер спустя месяц. Это вообще легально?

--- Я люблю тебя -- сказала я дедушке, и вышла из палаты.

Думаю, знай я, что это будут последние слова, которые он от меня услышит, выбрала бы что-нибудь пооригинальней. Я до последнего не верила, что рак заберет его так быстро. Мне только исполнилось пятнадцать, дедушке -- восемьдесят три. Он продолжал работать, бегал по утрам и выглядел лет на двадцать младше своего возраста. Затем болезнь впервые дала о себе знать. Какие-то две-три недели и недостающие года тут же отразились в его чертах лица.

\ldots{}

\hypertarget{section-1}{%
\chapter{~}\label{section-1}}

Приближался день рождения Адама, которого я не видела два с половиной года. Очередной день в баре. К тому времени я уже привыкла находиться по другую сторону стойки и читала что-то от Буковски, периодически поглядывая в сторону поддатых посетителей. Если не учитывать шестнадцатичасовой рабочий день, мизерную зарплату, постоянные недостачи и полное отсутствие чаевых, мне нравилось работать за барной стойкой. Разбираться в винных сортах, изучать миксологию и временами работать над созданием собственных коктейлей -- было в этом что-то особенное. Я без труда могла рассказать о любом из напитков куда больше, чем указано на этикетке, правильно посоветовать вино или удивить гостя чем-нибудь экзотичным. Такая работа и впрямь была мне по душе. Увы, несмотря на престижность заведения, в котором я работала, -- к тому же, его стойка считалась самой дорогой во всем городе -- платили здесь плачевно мало. Настолько мало, что едва хватало на еду. С другой стороны, работа занимала у меня все время. Вероятно, в этом и был секрет выживания сотрудников «Шлюза».

Десять страниц, пятнадцать, двадцать. Ничего не менялось. В зале по-прежнему было пусто. Как и в моем сердце. Не то, чтоб я ежеминутно думала о своем бывшее. Эти времена уже прошли. Словом, я полгода как выкарабкалась из затяжной депрессии и старалась всячески загрузить себя работой. Как я уже сказала, платили смехотворно мало. Ну, кто будет так рвать задницу ради 150 баксов в месяц? К счастью, работа всецело меня выматывала. После смены на все про все оставалось часов 6, а потом обратно за стойку. И это с дорогой. Сложно быть в депрессии, когда на нее просто не остается времени. Мне чуть ли не заранее приходилось планировать свои нервные срывы.

\hypertarget{section-2}{%
\section{~}\label{section-2}}

Звонил телефон. Я подняла трубку. Этажом ниже просили сделать 15 детских мохито и 4 взрослых. Намечался очередной детский день рождения. С ума сойти. В мое время мы дарили друг другу наклейки и радовались дешевому мороженному из МакДональзс, а не оарендовали целый этаж в ночном клубе вместе с аниматорами, лайтджеями и диджеями. Прошу заметить, на момент речи мне едва ли исполнилось 22 года.

--- И побыстрее! Мелкие уже вовсю капризничают! -- голос хриплый, чуть ли не потусторонний, и прямо таки гавкает, а не говорит.

\hypertarget{section-3}{%
\chapter{~}\label{section-3}}

Эта сцена не раз мелькала в моих мыслях до нашей первой встречи с Вениамином. Слушая его голос, собираясь на последний перед отпуском рабочий день, стоя за барной, отправляясь ко сну, сидя за рукописями и бегая за постоянно заканчивающимся льдом, я представляла, как впервые увижу его. Однако, еще чаще я думала о нашем с Веней знакомстве после того, как оно состоялось. Когда готовила ужин, гуляла около моря, ждала свой кофе в какой-нибудь забегаловке, смотрела, как он засыпает прямо посреди комнаты\ldots{} И особенно сильно я думала о том вечере, когда мы ссорились. Как бы ни обстояли дела, несмотря на всю горечь обиды, одно это воспоминание было способно растопить лёд, закравшийся в мое сердце.

Скажи мне кто-нибудь, что писать о собственной жизни будет так сложно -- ни за что бы не поверила. Казалось бы, ничего не нужно, ведь ты и так все знаешь и помнишь до мелочей. Ты часть истории, по отношению к которой твои собственные пожелания, по сути, не имеют никакого отношения. Произошло то, что произошло и все, что тебе остается -- проанализировать события и отойти в сторонку, притворившись обычным наблюдателем. Есть второй вариант, не подразумевающий сохранения ясности ума, так что мы его опустим.

Говорят, мне повезло с память. Я помню все имена, даты, события, диалоги и так далее до подробнейших деталей. Могу без труда вспомнить, во что был одет человек пять лет назад, что он говорил, в каком настроении был и почему. Могу сказать, какая за окном стояла погода, что за музыка играла в проезжающей мимо машине и о чем я в тот момент думала. Звуки и запахи -- это вообще отдельный разговор. Правильное сочетание этих элементов буквально открывает мне окно в прошлое. Спустя год после болезненного расставания с Адамом мне нужно было съездить в Черниговскую область. Добиралась я, естественно, таким же разваливающимся поездом, каким когда-то добиралась к своему мужчине. Был прохладный осенний вечер сродни тем вечерам, когда я только раздумывала о перспективе отношений с Адамом. Тогда у меня были духи с запахом лаванды. Они пылились на полке с моих двадцати лет, и черт меня дернул воспользоваться ими спустя год, оправляясь на железнодорожный вокзал. Запах парфюма и шум поезда буквально свели меня с ума. История закончилась тем, что я вылетела на улицу при первой же возможности, а затем стояла посреди какой-то неизведанной сельской местности, наблюдая за тем, как состав отдаляется в сторону заката.

Последнее, кстати, вовсе не метафора. В этот момент солнце действительно опускалось за горизонт, приглушая осенние оттенки. Я смотрела на исчезающий за небосводом алый диск, опавшие листья и разноцветные деревья, колышущиеся на ветру. Смотрела и думала, какая же я дура.

\end{document}
